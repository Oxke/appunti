%! TEX program = lualatex
\input{../preamble_appunti_report.tex}

\title{Appunti di Geometria 2}
\author{Github Repository:
\href{https://github.com/Oxke/appunti/tree/main/Geo2}{\texttt{Oxke/appunti/Geo2}}}
\date{Secondo semestre, 2024 \-- 2025, prof. Lidia Stoppino e Leone Slavich}

\begin{document}

\begin{titlingpage}
\maketitle

\vspace{1cm}
Il corso è diviso in due parti: geometria differenziale (tenuto da Slavich) e
gruppo fondamentale (tenuto dalla Stoppino). Ci sono esercitazioni di geometria
differenziale.
Ci sono vari libri consigliati:
\begin{itemize}[label = --]
    \item Abate \-- Tovena, \emph{Curve e superfici}, Springer
    \item M. D. Do Carmo, \emph{Differential Geometry of Curves and Surfaces},
        Prentice Hall
    \item E. Sernesi, \emph{Geometria 2}, Bollati Boringhieri
    \item Dispense del prof. Ghigi (sulle quali è stato disegnato il corso,
        almeno per la parte di geometria differenziale)
\end{itemize}

\end{titlingpage}

\chapter{Geometria Differenziale}
Il corso di geometria differenziale studierà curve e superfici in
\(\mathbb{R}^3\) definiti \textbf{analiticamente} tramite funzione
\(C^{\infty}\) \emph{(lisce)}. Studiamo la \textbf{geometria locale} e infine
(verso la fine del corso) la \textbf{geometria globale}, in particolare il
teorema di Gauss-Bonnet.

\section{Definizioni e proprietà iniziali}
\subsection{Funzioni lisce}

Sia \(I = {(a,b)} \subseteq \mathbb{R} \) intervallo aperto (anche possibilmente
\(a = -\infty\) o \(b = +\infty\)). Sia 
\[
    C^{0}{(I)} = \{f : I \to \mathbb{R} : f \text{ è continua\footnote{si veda
    Geometria 1 per definizione e studio della continuità} su }I\}
\]
\begin{definition}{Derivabile}
    Diciamo che \(f \in C^{0}{(I)}\) è derivabile se \(\forall x_{0} \in I\),  
    \[
        \lim_{x \to x_{0}} \frac{f{(x)} - f{(x_{0})}}{x - x_{0}} = c =:
        f'{(x_{0})} \quad ; \quad c \in \mathbb{R}
    \]
\end{definition}
Ora procediamo definendo ricorsivamente le funzioni \(C^{k}{(I)}\).
\begin{definition}{Classe \(C^{k}\) }
    Per ogni \(k\ge 1\), diciamo che \(f \in C^{k}{(I)}\) se \(f\) è derivabile
    e \(f' \in C^{k-1}{(I)}\)
\end{definition}
Dunque, ad esempio \(f \in C^{1}{(I)}\) se \(f\) è derivabile su \(I\) e la sua
derivata \(f'\) è continua su \(I\).
Detto più colloquialmente, una funzione \(f \in C^{k}{(I)}\) è una funzione
derivabile (almeno) \(k\) volte, e tale che la sua derivata \(i\)-esima
\(f^{{(i)}}\) è continua per ogni \(i = 0, \dots, k\).
\begin{remark}
    \[
        C_{0} \supset C^{1} \supset C^{2} \supset \dots \supset C^{i} \supset
        C^{i+1} \supseteq  \dots
    \]
\end{remark}
\begin{definition}{funzioni lisce}
    \[
        C^{\infty}{(I)} = \bigcap_{k=0}^{\infty} C^{k}{(I)} = \text{insieme
        delle \textbf{funzioni lisce}}
    \]
\end{definition}

\begin{theorem}[Proprietà delle classi \(C^{k}\)]\label{thm:proprieta_Ck}
    Sia \(k \in \mathbb{N} \cup \{+\infty\} \). Se \(f, g \in C^{k}{(I)}\) e
    \(\lambda \in \mathbb{R}\), allora
\begin{enumerate}[label = \arabic*.]
    \item \(f + g \in C^{k}{(I)}\) 
    \item \(\lambda f \in C^{k}{(I)}\)
    \item \(f\cdot g \in C^{k}{(I)}\)
\end{enumerate}
\end{theorem}
\begin{proof}
    1.~e 2.~sono semplici. Per 3.~si procede per induzione su \(k\).

    Nel caso base \(k = 0\) il prodotto di due funzioni continue è anch'esso
    una funzione continua, che è vero.

    Supponiamo ora che 3.~valga per \(k-1\). Siano \(f, g \in C^{k}{(I)}\).
    Allora \((f\cdot g)' = f' \cdot g + f \cdot g'\) che è somma di funzioni
    \(C^{k-1}\) per ipotesi induttiva e perché \(C^{k} \subset C^{k-1}\), e
    dunque \((f\cdot g)' \in C^{k-1}\) da cui segue che \(f\cdot g \in C^{k}\).

    Infine possiamo concludere per \(k = +\infty\) perché vale per tutti i \(k
    \in \mathbb{N}\).
\end{proof}

Dal teorema~\ref{thm:proprieta_Ck} segue che \(C^{k}{(I)}\) è uno spazio
vettoriale (con operazione di somma e moltiplicazione per scalare) e inoltre
\(C^{k} {(I)}\) contiene le funzioni costanti e allora \(C^{k}{(i)}\) con
operazioni di somma e moltiplicazione puntuale è un anello. Da queste due segue
che \(C^{k}{(I)}\) è una \(\mathbb{R}\)-algebra.

\begin{example}\label{ex:liscia_non_analitica}
    Esistono funzioni \textbf{lisce} che \textbf{non} sono \textbf{analitiche}.
    In particolare esistono funzioni lisce che sono nulle su un aperto ma non
    nulle dappertutto (differentemente da quanto succede sulle funzioni
    olomorfe). Un esempio di tale funzione è 
    \begin{align*}
        f : \mathbb{R} &\longrightarrow \mathbb{R} \\
        x &\longmapsto f (x) = \begin{cases}
            0 & x \le 0 \\
            e^{-\frac{1}{x^2}} & x > 0
        \end{cases}
    \end{align*}
    
    Questa è una funzione \(C^{\infty}{(\mathbb{R})}\) che non può essere
    analitica perché contraddirebbe il teorema del prolungamento.
    \begin{figure}[ht]
        \centering
        \begin{tikzpicture}
            \begin{axis}[
                xmin= -3, xmax= 3,
                ymin= -1, ymax = 1.5,
                axis lines = middle,
                width = 0.8\textwidth,
            ]
            \addplot[domain=-3:3, samples=100]{exp(-1/x^2) * (x > 0)};
            \end{axis}
        \end{tikzpicture}
        \caption{Grafico della funzione \(f(x)\)
        dell'esempio~\ref{ex:liscia_non_analitica}}\label{fig:liscia_non_analitica} 
    \end{figure}
    Similmente si possono costruire funzioni costanti in aperti, raccordate in
    modo \(C^{\infty}\) e ovviamente non analitiche.
\end{example}
\begin{proposition}[Composizione]\label{prp:composizione_ck}
    La composizione di funzioni \(C^{\infty}\) è \(C^{\infty}\). Sia \(f : I \to
    \mathbb{R}\) e \(g : J \to \mathbb{R}\). Allora se \(f \in C^{\infty}{(I)}\)
    e \(g \in C^{\infty}{(J)}\) e \(f{(I)} \subseteq J \) (ossia si possono
    comporre), allora \(g \circ f : I \to \mathbb{R}\) è ben definita e 
    \[
        g \circ f \in C^{\infty}{(I)}
    \]
\end{proposition}
\begin{proof}
    Lo dimostriamo per \(k \in \mathbb{N}\) invece che \(k = \infty\), segue
    naturalmente il caso enunciato. Per \(k = 0\) è ovvio.

    Supponiamo che valga per \(k-1\). Allora siano \(f, g \in C^{k}\) e
    tali che \(f{(I)} \subseteq J \). Allora \({(g \circ f)}' = {(g' \circ f)}
    \cdot f'\) che è prodotto di funzioni \(C^{k-1}\) per ipotesi induttiva e
    per il teorema~\ref{thm:proprieta_Ck} segue che \(g \circ f \in C^{k}{(I)}\).
\end{proof}

\subsection{Diffeomorfismi}
\begin{definition}{Diffeomorfismo}
    Un diffeomorfismo è un isomorfismo nella categoria delle funzioni lisce (su
    \(\mathbb{R}\) nel nostro caso). 
\end{definition}
Informalmente, un diffeomorfismo è un omeomorfismo \(C^{\infty}\). 
\begin{definition}{Diffeomorfismo}
    Siano \(I, J \subseteq \mathbb{R} \) intervalli aperti in \(\mathbb{R}\).
    Allora \(f : I \to J\) è un \textbf{diffeomorfismo} se
\begin{enumerate}[label = \arabic*.]
    \item \(f \in C^{\infty}{(I)}\) 
    \item \(f\) è biettiva
    \item \(f^{-1} \in C^{\infty}{(J)}\)
\end{enumerate}
\end{definition}
\begin{remark}
    La terza condizione \textbf{non} è ridondante. Infatti sia \(I = J =
    \mathbb{R}\) e \(f{(x)} = x^{3}\) che è chiaramente \(C^{\infty}\) e
    biunivoca. Tuttavia \(f^{-1}{(x)} = \sqrt[3]{x}\) non è derivabile in \(0\),
poiché \(f'(0) = 0\) e \({f^{-1}}'(y) = \frac{1}{f'{(x)}}\) se \(f{(x)} = y\)
per ogni \(x \in \mathbb{R}\) tale che \({f^{-1}}'{(y)}\) sia ben definita.
\end{remark}
\begin{remark}
    Se \(I\) e \(J\) sono intervalli aperti di \(\mathbb{R}\) e \(f : I\to J\) è
    diffeomorfismo, allora \(f'{(x)} \neq 0\) per ogni \(x \in I\). Infatti
    sappiamo che
    \begin{equation}\label{eq:derivata_inversa}
        {f^{-1}}'(y) = \frac{1}{f'{(x)}} \quad ; \quad f{(x)} = y \quad
        \forall x \in J 
    \end{equation}
    dunque \(f'{(x)}\) non può essere nullo, poiché significherebbe che
    \(f^{-1}\) non è derivabile in \(y = f^{-1}{(x)}\).
\end{remark}
\begin{lemma}
    Sia \(I \subseteq \mathbb{R}\) un intervallo aperto. Sia \(f : I \to
    \mathbb{R}\) una funzione liscia e tale che \(f'{(x)} \neq 0\) per ogni \(x
    \in I\). Allora \(f{(I)} = J\) è un intervallo aperto e \(f : I \to J\) è un
    diffeomorfismo.
\end{lemma}
\begin{proof}
    Sia \(f\) come nell'enunciato. Allora \(f' : I \to \mathbb{R}\) è continua
    su \(I\) e non si annulla mai. Segue che \(f'\) ha segno costante su \(I\)
    (\(f' > 0\) oppure \(f' < 0\)).

    Assumiamo \(f' > 0\) su \(I\). Allora \(f\) è strettamente crescente e
    dunque iniettiva. Allora \(f : I \to f{(I)} =: J\) è biettiva. Inoltre \(J\)
    è un intervallo aperto in \(\mathbb{R}\). Infatti è chiaramente intervallo
    (è connesso) in quanto immagine continua di un connesso (intervallo).  Sia
    ora \(y_{0} \in J\) e sia \(x_{0} \in I\) tale che \(f{(x_{0})} = y_{0}\).
    Sia \(\varepsilon > 0\) tale che \([x_{0} - \varepsilon, x_{0} +
    \varepsilon] \subseteq I \). Poiché \(f\) è strettamente crescente, segue
    che 
    \[
        f{(x_{0}-\varepsilon)} < f{(x_{0})} < f{(x_{0}+\varepsilon)}
    \]
    sapendo già che \(J\) è un intervallo, segue che \[I\supseteq
    [f{(x_{0}-\varepsilon)}, f{(x_{0} + \varepsilon)}] \text{ è un intorno di
\(y_{0}\)}\]
    Rimane solo da vedere che la funzione \(f^{-1} : J \to I\) è
    \(C^{\infty}\). Notiamo intento che \(f^{-1}\) è continua, poiché \(f\) è aperta.
    Inoltre sappiamo che \(f^{-1}\) è derivabile poiché è l'inversa di una
    funzione derivabile con derivata e vale
    l'equazione~\eqref{eq:derivata_inversa} per ogni \(y \in J\).

    Sia \(u : {(0, +\infty)} \to {(0, +\infty)}\) definita da \(u{(x)} = 1/x\) è
    un diffeomorfismo e da~\ref{eq:derivata_inversa} abbiamo che
    \[
        {f^{-1}}' = u \circ f' \circ f^{-1}
    \]
    e quindi se assumiamo induttivamente che se \(f^{-1} \in C^{k}\) allora ne
    consegue dalla proposizione~\ref{prp:composizione_ck} che \({f^{-1}}' \in
    C^{k}\) e dunque \(f^{-1} \in C^{k+1}\) 

\end{proof}

\subsection{Curve}
\begin{definition}[label=def:curva]{Curva parametrizzata}
    Sia \(\alpha : I \to \mathbb{R}^{3}\) una funzione 
    \(t \mapsto (\alpha_{1}(t), \alpha_{2}(t), \alpha_{3}(t)) \) con \(I\)
    intervallo aperto.
    
    Allora se \(\alpha_{1}, \alpha_{2}, \alpha_{3}\) sono funzioni lisce la
    funzione \(\alpha\) è detta \textbf{curva} parametrizzata in \(\mathbb{R}^3\) 
\end{definition}
In generale se una funzione \(\alpha : I \to \mathbb{R}^{n}\) verrà chiamata
\emph{funzione vettoriale} e ha come componenti \(n\) \emph{funzioni scalari}
\(a_{i} : I \to \mathbb{R}\). Con questa terminologia allora una curva
parametrizzata è una funzione vettoriale in \(\mathbb{R}^{3}\) con componenti
\(C^{\infty}{(I)}\).
\begin{example}[Retta in \(\mathbb{R}^{3}\) ]\label{ex:retta}
    Ovviamente in forma parametrica
    \[
        \alpha{(t)} = \mathbf{p}_{0} + t\mathbf{v} \quad ; \quad \mathbf{p}_{0},
        \mathbf{v} \in \mathbb{R}^3 \text{ fissati e \(t \in \mathbb{R}\)}
    \]
    e dunque \(\alpha_{1}{(t)} = p_{0_{1}} + t v_{1}\) e simili per le altre due
    componenti, e sono tutte funzioni lisce.
\end{example}
\begin{example}
    In \(\mathbb{R}^{2}\) prendiamo \(C = \{{(x,y)} \in \mathbb{R}^2 :
    {(x-x_{0})}^2 + {(y-y_{0})}^2 = r^2\} \) è una circonferenza di raggio \(r
    \in \mathbb{R}_{> 0} \) e centro \({(x_{0}, y_{0})} \in \mathbb{R}^2\). Una
    possibile parametrizzazione è
    \[
      \alpha{(t)} = (x_{0} + r\cos{t}, y_{0} + r\sin{t}) \quad ; \quad t \in
      \mathbb{R}
    \]
    In questo caso avremmo potuto prendere anche \(I = [0, 2\pi]\), non è un
    problema che \(\alpha\) non sia iniettiva
\end{example}

La definizione~\ref{def:curva} è molto generale e non richiede che la curva sia
come ci piacerebbe immaginarcela. Infatti anche se la curva è \(C^{\infty}\),
possiamo costruirne una che abbia un punto angoloso, anche se ha
parametrizzazione \(C^{\infty}\). Un esempio è visto
nell'esempio~\ref{ex:liscia_angolosa}. Inoltre vorremmo avere una definizione
più bella di curva, che dipenda meno dalla parametrizzazione scelta.

\begin{definition}{Vettore tangente}
    Data \(\alpha: I \to \mathbb{R}^3\) una curva parametrizzata, fissato un
    punto \(t \in I\), definiamo il \textbf{vettore tangente} ad \(\alpha\) al
    tempo \(t\) come
    \[
      \dot{\alpha}{(t)} = \frac{d \alpha}{dt} {(t)} = \begin{pmatrix}
          \alpha_{1}'{(t)} \\ \alpha_{2}'{(t)} \\ \alpha_{3}'{(t)}
      \end{pmatrix}
    \]
\end{definition}
\begin{remark}
    Intuitivamente (nella visione cinematica della curva parametrizzata), il
    vettore tangente rappresenta la velocità della particella che si muove lungo
    la curva 
\end{remark}
\begin{remark}
    Una retta ha tante parametrizzazioni diverse 
\end{remark}
Fissiamo due diverse parametrizzazioni della stessa retta \(r\):
\begin{equation*}
    \alpha{(t)} = \mathbf{p}_0 + t\mathbf{v} \quad ; \quad \beta{(t)} =
    \mathbf{q}_0 + t\mathbf{w}
\end{equation*}
Allora \(\alpha\) e \(\beta\) definiscono la stessa retta se e solo se \(\mathbf{v} 
\parallel \mathbf{w} \parallel \mathbf{q}_0 - \mathbf{p}_0\) sono paralleli.
Equivalentemente
\[
  \mathbf{q}_0 = \alpha{(t_{0})} \quad e \quad \mathbf{w}  = \lambda \mathbf{v} 
\]
ma allora
\[
  \beta{(s)} = \mathbf{q}_0 + s\mathbf{w}  = \alpha{(t_{0})} +
  s{(\lambda\mathbf{v} )} = \mathbf{p}_0 + t_{0}\mathbf{v} +\lambda s \mathbf{v}
  = \mathbf{p}_0 + {(t_{0}+\lambda s)} \mathbf{v}  = \alpha{(t_{0} + \lambda s)} 
\]
ossia \(\beta = \alpha\circ h\) con \(h{(s)} = t_{0}+\lambda s\) è una funzione
liscia con derivata mai nulla, dunque un diffeomorfismo. Questo motiva la
seguente definizione

\begin{definition}{Riparametrizzazione}
    Sia \(\alpha:I \to \mathbb{R}^3\) una curva parametrizzata e \(h : J\to I\)
    un diffeomorfismo. Allora \(\beta := \alpha\circ h : J \to \mathbb{R}^3\) è
    una \textbf{riparametrizzazione} di \(\alpha\) 
\end{definition}
\begin{remark}
    \(h'= 0\) significa che \(\dot{\beta}{(t)} = 0 \iff \dot{\alpha}{(t)}
    =0\). Se \(h'> 0\) allora la curva viene percorsa nello stesso verso.
\end{remark}
A noi interessano le curve parametrizzate \emph{a meno di riparametrizzazione}.
Questo suggerisce di introdurre una classe di equivalenza sulle curve
parametrizzate
\begin{definition}{Equivalenza tra curve}
    Siano \(\alpha : I \to \mathbb{R}^3\) e \(\beta : J \to \mathbb{R}^3\)
    curve parametrizzate. Allora \(\alpha\) e \(\beta\) sono \textbf{equivalenti}
    se esiste un diffeomorfismo \(h : J \to I\) tale che \(\beta = \alpha \circ
    h\). In altre parole \(\alpha\) e \(\beta\) sono \textbf{equivalenti} se e
    solo se \(\beta\) è una riparametrizzazione di \(\alpha\).

    La notazione che si usa è allora \(\alpha \sim \beta\)
\end{definition}
\begin{note}
    La relazione di equivalenza \(\sim\) è una relazione di equivalenza. Infatti
    è ovviamente simmetrica per il diffeomorfismo \(t \mapsto t\), è simmetrica
    mediante il diffeomorfismo \(h^{-1}\) ed è transitiva perché la composizione
    di due diffeomorfismi è un diffeomorfismo.
\end{note}
\begin{definition}{Curve geometriche}
    L'insieme delle curve geometriche è l'insieme delle classi di equivalenza
    delle curve parametrizzate rispetto alla relazione di equivalenza \(\sim\) 
    di riparametrizzazione.
\end{definition}
Per ogni curva geometrica, vogliamo trovare una curva parametrizzata in
parametrizzazione ``canonica''.
\begin{definition}{Parametrizzazione per lunghezza d'arco}
    Data \(\alpha : I \to \mathbb{R}^3\) una curva parametrizzata, \(\alpha\) è
    detta \textbf{parametrizzata per lunghezza d'arco} (o \emph{parametrizzata
    per ascissa curvilinea}) se 
    \[
      \|\dot{a}{(t)}\| = 1 \quad \forall t \in I
    \]
\end{definition}
Ora la questione è capire se è possibile riparametrizzare una curva in modo che
abbia parametrizzazione per lunghezza d'arco. Per quanto osservato prima è
necessario che \(\dot{\alpha}{(t)} \neq \mathbf{0}\), infatti \(\dot{\beta}{(s)} =
\dot{\alpha}{(t)} \cdot h'{(s)}\). Vogliamo ora mostrare che questa condizione è
sufficiente. 
\begin{definition}{Curva regolare}
    Una curva parametrizzata \(\alpha : I \to \mathbb{R}^3\) si dice
    \textbf{regolare} se \(\alpha{(t)} \neq \mathbf{0} \) per ogni \(t \in I\) 
\end{definition}

\begin{theorem}[regolare \(\iff \exists\) riparam.~per lunghezza
    d'arco]\label{thm:regolare_parametrizzata}
    Sia \(\alpha : I \to \mathbb{R}^3\) una curva parametrizzata regolare.
    Allora \(\exists h : J \to I\) diffeomorfismo tale che \(\beta = \alpha
    \circ h : J \to \mathbb{R}\) è una parametrizzazione per lunghezza d'arco.
\end{theorem}
Prima di procedere alla dimostrazione facciamo una piccola digressione sulle
lunghezze di una curva. Data \(\alpha: I \to \mathbb{R}^3\) una curva
parametrizzata. Fisso \([c,d] \subseteq I \)  intervallo chiuso. Allora
l'\emph{arco di curva} \(\alpha|_{[c, d]} : [c, d] \to \mathbb{R}\) ha lunghezza
che si calcola come
\[
    L{(\alpha|_{[c,d]} )} = \int_{c}^{d} \|\dot{\alpha}{(\tau)}\| \,d \tau
\]
per continuità della norma l'integranda è continua e dunque
integrabile. In particolare le somme di Riemann di questo integrale
corrispondono alle lunghezza delle curve poligonali che approssimano la curva,
il sup di esse è dunque il valore cercato.

\begin{proof}[Dimostrazione del teorema~\ref{thm:regolare_parametrizzata}]
    Sia \(\alpha : I\to \mathbb{R}^3\). Fisso \(t_{0} \in I\) e definisco \(h :
    I \to \mathbb{R}\) tramite
    \[
      h{(t)} = \int_{t_{0}}^{t} \|\dot{\alpha}{(\tau)}\| \,d \tau
    \]
    allora \(h{(I)} = J \subseteq \mathbb{R}  \) è un intervallo aperto. Inoltre
    \(h : I \to J\) è un diffeomorfismo e \(\beta = \alpha \circ h^{-1} : J \to
    \mathbb{R}^3\) è una riparametrizzazione con ascissa curvilinea.

    Infatti:
\begin{enumerate}[label = \arabic*.]
    \item \(h\) è \(C^{\infty}\). Infatti \(t \mapsto \|\dot{\alpha}{(t)}\|\) è
        \(C^{\infty}\) in quanto composizione di funzioni lisce. Infatti è
        radice di un valore che non è mai nullo, dunque la radice è definibile
        da \({(0, +\infty)}\) e allora \(C^{\infty}\).

        Allora per il teorema fondamentale del calcolo integrale, \(h\) è
        \(C^{\infty}\) 
    \item \(h'{(t)} = \|\dot{\alpha}{(t)}\| > 0\), dunque \(h\) è diffeomorfismo
        e \(J\) è aperto.
    \item \(\beta = \alpha \circ h^{-1} : J \to \mathbb{R}^3\) è \(C^{\infty}\)
        e
        \[
            \dot{\beta}{(s)} = \dot{\alpha}{(h^{-1}{(s)})} \cdot
            {(h^{-1})}'{(s)} =
            \frac{\dot{\alpha}{(h^{-1}{(s)})}}{h'{(h^{-1}{(s)})}} =
            \frac{\dot{\alpha}{(h^{-1}{(s)})}}{\|\dot{\alpha}{(h^{-1}{(s)})}\|}
        \]
        e dunque \(\|\dot{\beta}\| = 1\) 
\end{enumerate}
\end{proof}

Il teorema~\ref{thm:regolare_parametrizzata} è la motivazione per cui
sceglieremo di lavorare con curve regolari.

\begin{example}
    \(C = \{{(x, y)} \in \mathbb{R}^2 : x \ge 0, y\ge 0, xy=0\}\) è l'unione dei
    semiassi positivi. Allora due fatti sono veri:
\begin{enumerate}[label = \arabic*.]
    \item È possibile trovare una curva parametrizzata liscia con \(\alpha : I
        \to R^2\) tale che \(\alpha{(I)} = C\) 
    \item Non esiste una curva regolare tale che \(\alpha{(I)} = C\)
\end{enumerate}
dunque le curve regolari sono l'oggetto giusto per studiare la geometria delle
curve che appaiono geometricamente lisce.
\end{example}

\begin{example}[Curva liscia con punto angoloso]\label{ex:liscia_angolosa}
    Sia \(\alpha : I \to \mathbb{R}^2\) definita da
\end{example}
    
\chapter{Geometria algebrica}
\begin{definition}{Omotopia}
    Siano \(X\) e \(Y\) spazi topologici. Siano \(\alpha, \beta : X \to Y\)
    funzioni continue. Diciamo che \(\alpha\) e \(\beta\) sono \textbf{omotope}
    se esiste una funzione continua \(H : X \times I \to Y\) tale che
    \begin{align*}
        H{(x, 0)} &= \alpha{(x)} \\
        H{(x, 1)} &= \beta{(x)}
    \end{align*} 
    denoteremo l'omotopia con la notazione \(\alpha \approx \beta\) 
\end{definition}

\begin{definition}{Omotopia relativa a un sottospazio}
    Sia \(X\) uno spazio topologico, siano \(\alpha, \beta : X \to
    X\) funzioni omotope con omotopia \(H\). Sia \(A \subseteq X \). Se 
    \[
      \forall t \in I, \quad \forall a \in A \quad H(a, t) = a \text{ o
      equivalentemente } H|_{A \times I} = i \circ P_A
    \]
    Con \(P_A : A \times I \to A\) la proiezione \({(a, t)} \mapsto a\) e \(i :
    A \hookrightarrow X\) l'inclusione.
    Allora diremo che \(\alpha\) e \(\beta\) sono \textbf{omotope relativamente
    ad \(A\)} e lo denoteremo con \(\alpha \approx_A \beta\) 
\end{definition}


\begin{proposition}
    Siano \(X, Y, Z\) spazi topologici. Esistano quattro applicazioni \(f_{i} :
    X \to Y\) e \(g_{i}: Y \to Z\) con \(i = 0, 1\) continue tali che \(f_{0}
    \sim f_{1}\) e \(g_{0} \sim g_{1}\). Allora \(g_{0} \circ f_{0} \sim g_{1}
    \circ f_{1}\) 
\end{proposition}
\begin{proof}
    Siano \(F : X \times I \to Y\)  e \(G : Y \times I \to Z\) le
    omotopie rispettivamente tra le \(f_{i}\) e tra le \(g_{i}\). Vogliamo
    trovare \(H : X \times I \to Z\)
    omotopia. La funzione \(H(x, t) = G(F{(x, t)}, t)\) è tale funzione.
    È infatti continua in quanto composizione di funzioni continue e chiaramente
    \begin{align*}
      H(x, 0) = G{(F{(x, 0)}, 0)} = G{(f_{0}{(x)}, 0)} = g_{0}{(f_{0}{(x)})} =
      {(g_{0} \circ f_{0})}{(x)} \\
      H{(x, 1)} = G{(F{(x, 1)}, 1)} = G{(f_{1}{(x)}, 1)} = g_{1}{(f_{1}{(x)})} =
      {(g_{1} \circ f_{1})}{(x)}
    \end{align*}
\end{proof}
\begin{definition}{Spazi omotopicamente equivalenti}
    Siano \(X\) e \(Y\) spazi topologici. Diciamo che \(X\) è omotopicamente
    equivalente a \(Y\) (denotato \(X \approx Y\)) se esistono due applicazioni
    \(\varphi : X \to Y\) e \(\psi : Y \to  X\) continue tali che \(\psi \circ
    \varphi \sim \mathrm{Id}_X\) e \(\varphi \circ \psi \sim \mathrm{Id}_Y\) 
\end{definition}

\begin{remark}
    Chiaramente \(X \overset{omeo}{=} Y \implies X \approx Y\). Infatti preso
    l'omeomorfismo \(\varphi \) e \(\varphi ^{-1}\) come funzioni \(\varphi \) e
    \(\psi\) allora le loro composizioni sono proprio le identità degli spazi.
\end{remark}
\begin{remark}
    L'equivalenza omotopica è una relazione di equivalenza
\end{remark}

\begin{example}
    \(D^2\) e \(\{P\}\) sono omotopicamente equivalenti.
    \(\varphi : D^2 \to \{P\} \) è per forza l'applicazione costante.
    Mentre \(\psi: P \mapsto (0,0)\) (potrei scegliere qualsiasi altro punto per
    convessità, è per comodità che ne scegliamo il centro). Chiaramente non si
    tratta di omeomorfismi ma abbiamo che \(\varphi \circ \psi =
    \mathrm{Id}_{\{P\} } \) dunque ok, mentre \(\psi \circ \varphi : D^2 \to D^2
    \) è l'applicazione costante in \({(0,0)}\). Poiché possiamo prendere
    l'applicazione \(H : D^2 \times I : D^2\) data da \(H(\mathbf{x} , t) = t
    \mathbf{x} \) che è un'omotopia, ne segue che \(D^2 \approx \{P\} \) 
\end{example}

\begin{example}
    Una generalizzazione semplice del precedente esempio. Ogni convesso \(X\) di
    \(\mathbb{R}^{n}\) è omotopicamente equivalente al punto (singoletto). Basta
    infatti prendere come funzione quella tale che ogni \(H(x, \cdot )\) sia un
    segmento da \(x\) a un punto di \(X\).

    Più genericamente si può prendere uno stellato, e l'equivalenza omotopica
    vale soltanto per la funzione costante che manda ogni \(x\) in
    \(\overline{x}\), con \(\overline{x}\) il punto tale per cui \(X\) è
    stellato. 

    Ne consegue che tutti i sottospazi stellati di \(\mathbb{R}^{n}\) sono
    omotopicamente equivalenti.
\end{example}
\begin{definition}{Contraibilità}
    Uno spazio si dice \textbf{contraibile} se è omotopicamente equivalente a un
    punto.
\end{definition}
\begin{note}
    In tutti gli esempi fatti finora di spazi contraibili (convessi e stellati
    di \(\mathbb{R}^{n}\)), l'omotopia tra \(\mathrm{Id}_X\) e
    \(C_{\overline{x}} \) è relativa a \(\overline{x}\), che rimane costante
    lungo l'omotopia.
\end{note}

Una buona parte del corso sarà dimostrare che alcuni spazi \textbf{non} sono
contraibili. Con strumenti molto più potenti arriveremo a dimostrare in modo
veloce, ad esempio, che \(S^{1}\) non è contraibile. 

\begin{example}
    \(C = [-1, 1] \times S^{1}\)  è omotopicamente equivalente a \(S^{1}\).
    Prendiamo \(\varphi : C \to \{0\} \times S^{1}\) come \({(x, y, z)} \mapsto
    (0, y, z)\) invece per \(\psi\) prendiamo l'inclusione. Una composizione è
    l'identità, e l'altra è la proiezione, la cui omotopia è chiaramente \(H(x,
    y, z, t) = (tx, y, z)\). Chiaramente \(S^{1}\) è omeomorfo a \(\{0\} \times
    S_{1}\) come sottospazio di \(C\).
\end{example}

\begin{eser}
    Verificare che il nastro di Möbius è omotopicamente equivalente a \(S^{1}\).

    \tcblower

    % TODO: completare
\end{eser}

\begin{remark}
    Ne consegue dall'esercizio e dall'esempio precedenti che il cilindro e il
    nastro di Möbius sono omotopicamente equivalenti.
\end{remark}

\begin{definition}{Proprietà omotopica}
    Una proprietà di spazi topologici si dice omotopica se è costante sulle
    classi di equivalenza omotopica.

    In altre parole, se \(X \approx Y\) e \(P(X)\) allora \(P(Y)\) 
\end{definition}

\(\mathbb{R}\) è convesso, dunque è omotopicamente equivalente al punto. Poiché
il primo non è compatto ma il secondo sì, la compattezza non è una proprietà
omotopica.

\begin{definition}{lcpa}
    Lo spazio topologico \(X\) si dice \textbf{localmente connesso per archi} se
    per ogni punto \(x \in X\) esiste un sistema fondamentale di intorni aperti
    di \(X\) connessi per archi.
\end{definition}
\begin{note}
    cioè per ogni \(x \in X\) e \(\forall U\) intorno di \(x\) esiste \(A\)
    aperto connesso per archi tale che \(x \in A \subseteq U \) 
\end{note}

\begin{eser}
    Mostrare che cpa non implica lcpa.

    In particolare mostrare che \(X = \{{(x,y)} \in \mathbb{R}^2 : \frac{x}{y} \in
    \mathbb{Q} \lor \frac{y}{x} \in \mathbb{Q}\} \cup \{{(0,0)}\} \) è cpa ma
    non lcpa.
\end{eser}

\begin{proposition}
    Sia \(\alpha : I \to X\) un laccio, dunque \(\alpha{(0)} = \alpha{(1)}\).
    Allora \(\alpha\) induce un'applicazione continua \(\overline{\alpha} :
    S^{1} \to X\).
% https://q.uiver.app/#q=WzAsMyxbMCwwLCJJIl0sWzEsMCwiWCJdLFswLDEsIkkvXFxzZXR7MCwxfSBcXG92ZXJzZXR7XFx0ZXh0e29tZW99fXtcXGFwcHJveH0gU14xIl0sWzAsMSwiXFxhbHBoYSJdLFswLDIsIlxccGkiLDJdLFsyLDEsIlxcb3ZlcmxpbmV7XFxhbHBoYX0iLDIseyJzdHlsZSI6eyJib2R5Ijp7Im5hbWUiOiJkYXNoZWQifX19XV0=
\[\begin{tikzcd}
	I & X \\
	{I/\{0,1\} \overset{\text{omeo}}{\approx} S^1}
	\arrow["\alpha", from=1-1, to=1-2]
	\arrow["\pi"', from=1-1, to=2-1]
	\arrow["{\overline{\alpha}}"', dashed, from=2-1, to=1-2]
\end{tikzcd}\]
Questo per la proprietà universale della topologia quoziente.

Vogliamo \(\overline{F} : S^{1}\times  I \to X\) un omotopia tra
\(\overline{\alpha}\) e \(c_{x_{0}} \) 
\[\begin{tikzcd}
	I \times I & X \\
	{S^1 \times I}
	\arrow["F", from=1-1, to=1-2]
	\arrow["\pi \times \mathrm{Id}_I"', from=1-1, to=2-1]
	\arrow["{\overline{F}}"', dashed, from=2-1, to=1-2]
\end{tikzcd}\]
e di nuovo esiste continua per la proprietà universale.
Per la commutatività dei due diagrammi, abbiamo che \(\overline{F}{(x, 0)} =
\overline{\alpha}{(x)}\) e \(\overline{F}{(x, 1)} \equiv x_{0}\).
\end{proposition}

\begin{proposition}
    Viceversa, 
    se ho \(f: S^{1} \to X\) applicazione e \(p \in S^{1}\) tale che \(f
    \approx_p c_{f{(p)}} \) con \(c_{f{(p)}} \) l'applicazione costante in
    \(f{(p)}\), allora esiste un cammino \(\alpha : I \to X\) tale che \(a{(I)}
    = S^{1}\), \(a{(0)}= a{(1)} = p\) e \(\varepsilon \alpha \sim
    \varepsilon_{f{(p)}} \) 
\end{proposition}

\begin{proposition}
    Sia \(C \subseteq \mathbb{R}^{n} \) un convesso. Allora \(\forall f : X \to
    C\) applicazione continua, con \(X\) uno spazio topologico qualsiasi e
    \(\forall \overline{y} \in C\) allora \(f \approx c_{\overline{y}}  : X \to
    C\). 
\end{proposition}
\begin{proof}
    Costruiamo esplicitamente l'omotopia. 
    \(F{(x, t)} = {(1-t)}f{(x)} + t\overline{y}\). Per convessità sta in \(C\)
    ed è chiaramente continua. Inoltre \(F{(x, 0)} = f{(x)}\) e \(F{(x, 1)} =
    \overline{y}\) 
\end{proof}
Similmente per uno spazio \(S\) stellato, esiste un punto \(\overline{y} \in S\)
tale che la stessa formula funzioni.

Ne consegue una potente proprietà:
\begin{lemma}
    Sia \(X\) uno spazio topologico e \(S \subseteq \mathbb{R}^{n} \) uno
    stellato con centro \(\overline{y}\). Sia \(f : X \to S\) continua. Allora
    \(f\) è omotopa a \(c_{\overline{y}} \).
\end{lemma}
\begin{corollary}
    Prendendo \(X = S\) e \(f = \mathrm{Id}_S\) otteniamo che \(S\) è
    contraibile
\end{corollary}

\begin{eser}[\(\star\)]
\begin{enumerate}[label = \arabic*.]
    \item Mostrare che la contraibilità è una proprietà topologica.
    \item Mostrare che se \(f: \mathbb{R}^{n} \to \mathbb{R}^{m}\) è una funzione
    continua, allora \(\Gamma_f = \{{(x, y)} \in \mathbb{R}^{n}\times
    \mathbb{R}^{m}: y = f{(x)}\}  \) è contraibile
\end{enumerate}
\end{eser}

\begin{eser}
    Mostrare che il nastro di Möbius è omotopicamente equivalente a \(S^{1}\) 
    \tcblower
    \(M = [-1,1]^2 / \sim \) con \({(x,y)}\sim {(x', y')} \) se e solo se
    \({(x,y)} = {(x',y')}\) oppure \(\{x, x'\} = \{1, -1\}  \) e \(y = -y'\).

    Scegliamo \(F : [-1,1]^2 \times  I \to [-1,1]^2\) data da \(F{(x, y, t)} =
    {(x, ty)}\) che è un'omotopia tra \(\mathrm{Id}_{[-1,1]^2} \) e \(r : [-1,
    1]^2 \to [-1, 1] \times \{0\} \), \(r{(x,y)} = {(x, 0)}\). 

    Vogliamo ora vedere \(\pi \circ F\) è costante sulle fibre di \(\pi \times
    \mathrm{Id}_I\) .% https://q.uiver.app/#q=WzAsNCxbMCwwLCJbLTEsMV1eMiBcXHRpbWVzIEkiXSxbMSwwLCJbLTEsMV1eMiJdLFsxLDEsIk0iXSxbMCwxLCJNXFx0aW1lcyBJIl0sWzAsMywiXFxwaSBcXHRpbWVzIFxcbWF0aHJte0lkfSIsMl0sWzAsMiwiXFxwaSBcXGNpcmMgRiIsMV0sWzMsMiwiXFxvdmVybGluZXtGfSIsMl0sWzAsMSwiRiJdLFsxLDIsIlxccGkiLDJdXQ==
\[\begin{tikzcd}
	{[-1,1]^2 \times I} & {[-1,1]^2} \\
	{M\times I} & M
	\arrow["F", from=1-1, to=1-2]
	\arrow["{\pi \times \mathrm{Id}}"', from=1-1, to=2-1]
	\arrow["{\pi \circ F}"{description}, from=1-1, to=2-2]
	\arrow["\pi"', from=1-2, to=2-2]
	\arrow["{\overline{F}}"', from=2-1, to=2-2]
\end{tikzcd}\]
\end{eser}

\begin{proposition}\label{prop:lcpa_compcpa_aperte}
    Se uno spazio topologico \(X\) è localmente connesso per archi allora le
    sue componenti connesse per archi sono aperte.
\end{proposition}
\begin{proof}\( \)
\begin{itemize}
    \item[\(\implies \)] Sia \(X\) lcpa. Allora sia \(x \in C\), con \(C\) una
        componente connessa per archi di \(X\). Allora esiste un sistema
        fondamentale di intorni aperti di \(x\) connessi per archi.

        Sia \(A\) un intorno aperto cpa di \(x\). Allora per ogni \(y \in A\),
        \(y\) è connesse a \(x\) da un arco. Poiché \(A \subseteq C\), \(C\) è
        aperto.
    \item[\( \)] Se le componenti cpa sono aperte è vero che \(\forall
        x \in X\) esiste \(A\) aperto cpa intorno di \(x\) (basti prendere la
        componente cpa che contiene \(x\)). Non è detto però che \(X\) sia lcpa.
\end{itemize}
\end{proof}
\begin{proposition}\label{prop:lcpa-compcpa_compc}
    Sia \(X\) lcpa. Allora le componenti connesse coincidono con le componenti
    connesse per archi.
\end{proposition}
\begin{proof}
    Sia \(C\) una componente connessa per archi di \(X\). Per la
    proposizione~\ref{prop:lcpa_compcpa_aperte} \(C\) è aperta. Inoltre
    ovviamente è connessa. Poiché è unione disgiunta delle componenti connesse
    per archi, \(C = X \sminus \coprod_{\alpha \in A} C_{\alpha} \) e dunque
    \(C\) è anche chiusa. Ne consegue che \(C\) è componente connessa.
\end{proof}

\begin{theorem}\label{thm:conn-lcpa_cpa}
    Sia \(X\) uno spazio topologico connesso e localmente connesso per archi.
    Allora \(X\) è connesso per archi
\end{theorem}
\begin{proof}
    \(X\) è l'unica componente connessa e per la
    proposizione~\ref{prop:lcpa-compcpa_compc} è l'unica componente connessa per
    archi. Dunque \(X\) è connessa per archi.
\end{proof}

\begin{proposition}\label{prop:contr_cpa}
    Se \(X\) è contraibile, allora è connesso per archi
\end{proposition}
\begin{proof}
    Sia \(X \approx \{P\} \). Allora sia \(\varphi : X \to \{P\} \) e \(\psi :
    \{P\} \to X\) tale che \(\varphi \circ \psi \approx \mathrm{Id}_P\) e \(\psi
    \circ \varphi = \mathrm{Id}_X\). Dunque esiste \(\overline{x} \in X\) tale
    che \(c_{\overline{x}} : X \to X\) è omotopa a \(\mathrm{Id}_X\). Abbiamo
    dunque che esiste \(F : X \times I \to X\) continua tale che \(F{(x, 0)} =
    x\) e \(F{(x, 1)} = \overline{x}\) per ogni \(x \in X \).

    Vogliamo mostrare ora che ogni punto \(x \in X\) è connesso per archi a
    \(\overline{x}\). Possiamo prendere \(\alpha{(t)} := F{(x, t)}\) che è
    esattamente un arco da \(x\) a \(\overline{x}\). 
\end{proof}

\begin{theorem}
    Siano \(X\) e \(Y\) spazi topologici omotopicamente equivalente e sia \(f :
    X \to Y\) un'equivalenza omotopica\footnote{Si dice che \(f\) è un'equivalenza omotopica se rappresenta una delle due
    funzioni \(\varphi \) o \(\psi\) nella definizione di equivalenza
omotopica.}.

    Allora \(f\) induce una biiezione dall'insieme delle componenti connesse per
    archi di \(X\) all'insieme delle componenti connesse per archi di \(Y\).
\end{theorem}
\begin{note}
    Indicata con \(\sim \) la relazione di equivalenza della connessione per
    archi, l'insieme delle componenti connesse per archi di \(X\) si indica
    \(\pi_0{(X)} = X / \sim \) 
\end{note}
    \begin{lemma}\label{help:omotopia_composizione}
        Siano \(X, Y, Z\) spazi topologici e \(f, g\) applicazioni continue.
        Allora \({(g \circ f)}_\# = g_\# \circ f_\#\) 
    \end{lemma}
    \begin{proof}
        Sia \(C\) una componente connessa per archi di \(X\), cioè \(C \in
        \pi_{0}{(X)}\). Allora \(f_\#{(C)}\)  è la componente cpa di \(Y\) che
        contiene \(f{(C)}\).  La componente \(g_\# {(f_\#{(C)})}\) è la
        componente di \(Z\) che contiene l'immagine di \(g\) della componente
        connessa di \(Y\) che contiene \(f{(C)}\). Ma allora se contiene tutta
        la componente connessa necessariamente deve contenere \(g{(f{(C)})}\).
        Ne consegue la tesi.
    \end{proof}
    
    \begin{lemma}\label{help:omotodiesis}
        Siano \(X\) e \(Y\) due spazi topologici. Allora se \(f, g : Z \to W\)
        sono due applicazioni con \(f \approx g\), allora \(f_\# = g_\#\)
    \end{lemma}
    \begin{proof}
        Sia \(F: X \times I \to Y\) l'omotopia, quindi \(F{(x, 0)} = f{(x)}\) e
    \(F{(x, 1)} = g{(x)}\). Sia \(C \in \pi_{0}{(X)}\), allora \(C \times I\) è
    prodotto di connessi per archi e dunque è connesso per archi. Ma allora
    \(f{(C)}, g{(C)}  \subseteq F{(C \times I)} \subseteq Y  \) che è connesso per archi
    ma allora necessariamente \(f_\# {(C)} = F_\#{(C \times I)} =g_\#{(C)}\).
    \end{proof}
\begin{proof}[Dimostrazione del teorema~\ref{thm:conn-lcpa_cpa}]
    Usando \(f\) vogliamo definire un'applicazione \\\(f_\# : \pi_{0}{(X)} \to 
    \pi_{0}{(Y)}\). Sia \(C\) una componente cpa di \(X\). Allora \(f{(C)}
    \subseteq Y \) è un cpa in \(Y\), ed è dunque contenuto in esattamente una 
    componente connessa per archi di \(Y\), che quindi usiamo per definire
    \(f_\# {(C)}\). Vogliamo vedere che \(f_\#\) è iniettiva e suriettiva.
    Sappiamo che esiste \(g : Y \to X\) tale che \(f \circ g \approx
    \mathrm{Id}_Y\) e \(g \circ f \approx \mathrm{Id}_X\).


    Usando i lemmi, abbiamo che \( \mathrm{Id}_{\pi_{0}{(X)}} =
    {(\mathrm{Id}_X)}_\# \overset{\ref{help:omotodiesis}}{=} {(g \circ f)}_\#
    \overset{\ref{help:omotopia_composizione}}{=} g_\# \circ f_\#\) da cui
    necessariamente \(f_\#\) è iniettiva. Prendendo la composizione al
    contrario, abbiamo invece che \(f_\#\) è suriettiva.
\end{proof}

\section{Retrazione}
In molti casi in precedenza, per mostrare una equivalenza omotopica, una delle
funzioni \(\varphi \circ \psi\) o viceversa è una deformazione continua a un
sottoinsieme dello spazio. Questo motiva il concetto di retratto di
deformazione, che vogliamo definire formalmente in questa sezione. 

\begin{definition}{Retrazione}
    Siano \(X\) e \(Y\) spazi topologici. Sia \(A \subseteq X\) un sottoinsieme
    di \(X\). Diciamo che \(r : X \to A\) è una \textbf{retrazione} se è continua
    e \(r{(a)} = a\) per ogni \(a \in A\), ovvero \(r|_A = \mathrm{Id}_A\) 
\end{definition}

\begin{example}
    \(X = D^2\) e \(A = \{{(0,0)}\} \), allora \(r : D^2 \to \{{(0,0)}\} \)
    costante è una retrazione.

    Più genericamente preso un qualsiasi spazio topologico \(X\), per ogni
    \(\overline{x} \in X\) l'applicazione costante \(X \to \{\overline{x}\} \) è
    una retrazione
\end{example}

\begin{example}
    Preso \(X = S^{1} \times I\) e \(r : S^{1} \times  I \to S^{1} \times \{0\}
    \) definita da \(r{(\mathbf{x} , t)} = {(\mathbf{x} , 0)}\) è una retrazione
\end{example}

\begin{definition}{Retratto di deformazione}
    Sia \(A \subseteq X \) con \(X\) spazio topologico. Allora \(A\) si dice
    \textbf{retratto di deformazione} se \(\exists r : X \to A\) retrazione
    (ossia \(r\) è continua e \(r \circ i = \mathrm{Id}_A\)) e \(i\circ r
    \approx \mathrm{Id}_X\) 

    \tcbline

    \(A\) è detto \textbf{retratto \emph{forte} di deformazione} se è un
    retratto di deformazione e \(i \circ r \approx_A \mathrm{Id}_X\) 

    \tcbline

    \(A\) si dice \textbf{retratto \emph{debole} di deformazione} se \(\exists r
    : X \to A\) continua tale che \(r \circ i \approx \mathrm{Id}_A\) e \(i
    \circ r \approx \mathrm{Id}_X\) 
\end{definition}
\begin{remark}
    In tutti i casi di retrazione di deformazione \(i\) e \(r\) sono equivalenze
    omotopiche.
    In particolare se \(A\) è retratto (anche solo debole) di deformazione di
    \(X\) allora \(A\) è omotopicamente equivalente a \(X\).
\end{remark}

\begin{example}
    L'esempio del cilindro \(r : X := S^{1} \times I \to S^{1} \times \{0\} =: A \)
    definito da \(r{(x, y, z)} = {(x, y, 0)}\). Allora \(r \circ i =
    \mathrm{Id}_A\) chiaramente e \(i \circ r {(x, y, z)} = {(x,y,0)} \in
    S^{1}\times I\). Allora \(F(x, y, z, t) = {(x, y, tz)}\) è l'omotopia tra
    \(i\circ r\) e \(\mathrm{Id}_X\). Inoltre \(F|_{A \times I}{(x, y, 0, t)} =
    {(x, y, 0)}\) e quindi per ogni \(t\), \(F|_{A \times \{t\} } =
    \mathrm{Id}_A\) 
\end{example}

\begin{example}
    L'insieme \(X \subseteq \mathbb{R}^2 \) dell'unione delle rette \(ax + by =
    0\) con \(a, b \in \mathbb{Z}\) non entrambi nulli è contraibile. In
    particolare \({(0,0)}\) è un suo retratto di deformazione forte, infatti \(r
    : X \to \{(0,0)\}\) è una retrazione forte, con omotopia \(i \circ r
    \approx_{{(0,0)}} 
    \mathrm{Id}_X\) data da \(F{(x, y, t)} = t{(x,y)}\).

    Lo stesso ragionamento si può applicare a qualsiasi stellato, per il punto
    di origine.

    Se invece di prendere \({(0,0)}\) si prende un altro punto
    \({(\overline{x},\overline{y})} \in X\) abbiamo che anch'esso è retratto di
    deformazione, infatti (consideriamo ora \(r : X \to \{(\overline{x},
    \overline{y})\}\) che è chiaramente una retrazione) e presa l'applicazione 
    \[
      H{(x, y, t)} = \begin{cases}
          (1 - 2t){(x, y)} & t \le \frac{1}{2} \\
          {(2t - 1)}{(\overline{x}, \overline{y})} & t \ge \frac{1}{2}
      \end{cases}
    \]
    questa è continua e \(H{(\cdot , \cdot , 0)} = \mathrm{Id}_X\) e \(H{(\cdot,
    \cdot , 1)} = i \circ r\). Tuttavia non mostra una retrazione forte, infatti
    \(H{(\overline{x}, \overline{y}, \frac{1}{2})} = {(0,0)}\) ad esempio.

    Più precisamente \({(\overline{x}, \overline{y})}\) \textbf{non} è retratto
    di deformazione forte di \(X\), se \({(\overline{x},
    \overline{y})} \neq {(0,0)}\) . Sia infatti \(d = \|{(\overline{x},
    \overline{y})}\|\). Sia ora
    \[
      A_k = B_{d\cdot 2^{-k}}{(\overline{x}, \overline{y})}, \quad \forall k \in
      \mathbb{N}_{> 0} 
    \]
    e sia \({(x_{k}, y_{k})} \in A_\kappa\) un punto tale che sia su un diverso
    segmento di retta rispetto a quello di \({(\overline{x}, \overline{y})}\).
    Ogni \(A_k\) è attraversato da infinite rette, quindi questo è possibile.
    Inoltre, poiché \(H{(x_\kappa, y_\kappa, \cdot )}\) è un arco da
    \({(x_\kappa, y_\kappa)}\) a \({(\overline{x}, \overline{y})}\), esiste per
    ogni \(k\) un \(t_k\) tale che \(H{(x_\kappa, y_\kappa, t_\kappa)} =
    {(0,0)}\). Finalmente notiamo che, denotando \(\overline{t} := \limsup_{k
    \to \infty} t_\kappa\) abbiamo che
    \[
      \limsup_{k \to \infty} {(x_\kappa, y_\kappa, t_\kappa)} = {(\overline{x},
      \overline{y}, \overline{t})}
    \]
    e dunque
    \[
      \lim_{k \to \infty} H{(x_k, y_k, t_k)} = {(0,0)} \neq {(\overline{x},
      \overline{y})} = H{(\overline{x}, \overline{y}, \overline{t})} =
      H{(\limsup_{k \to \infty} {(x_k, y_k, t_k)})}
    \]
\end{example}

Ovviamente non tutte le retrazioni portano a un retratto di deformazione.
Infatti se così fosse tutti i punti di tutti gli spazi sarebbero retratti di
deformazione, e quindi tutti gli spazi sarebbero contraibili, cosa che non è
vera, come vedremo. In particolare mostreremo che \(S^{1}\) non è contraibile, e
quindi nessun suo punto è retratto di deformazione. 

\begin{example}
    Sia \(X = \mathbb{R}^2 \sminus \{{(0,0)}\} \). Sia \(A = S^{1}\). Allora la
    retrazione che ``dimentica la norma del vettore e tiene solo la fase'' è \(r
    : \mathbb{R}^2 \sminus \{\mathbf{0} \} \to S^{1}\) data da \(r{(x, y)} = {(x,y)} /
    \|{(x,y)}\|\). Allora chiaramente \(r \circ i = \mathrm{Id}_{S^{1}} \)  e
    \[
        (i \circ r){(x, y)} = \frac{(x,y)}{\|{(x,y)}\|} \in \mathbb{R}^2 \sminus
        \{\mathbf{0} \} 
    \]
    che è omotopa a \(\mathrm{Id}_X\) tramite il segmento
    \[
      F{(\mathbf{x} ,t)} = {(1-t)}\frac{\mathbf{x} }{\|\mathbf{x} \|} +
      t\mathbf{x}  = {\mathbf{x} }{\left(
      {t + \frac{1-t}{\|\mathbf{x} \|}}\right)} 
    \]
    che fissa \(S^{1}\) 
\end{example}


\begin{example}
    Sia \(X = \mathbb{R}^2 - \{P, Q\} \) con \(P\) e \(Q\) due punti distinti,
    ad esempio sia \(P = {(-1, 0)}\) e \(Q = {(1, 0)}\).
    \[
      r{(x,y)} = \begin{cases}
      \begin{cases}
          \displaystyle{\frac{(x-1, y)}{\|{(x-1, y)}\|} + {(1,0)}} & \{x \ge 1\} \cup
          \overline{B_{1}}{(1,0)} \\
          \displaystyle\left(x, \mathrm{sgn} {(y)} \sqrt{x{(2-x)}}\right) & \{x \le 1\} \sminus
          B_{1}{(1,0)} \\
      \end{cases}
       & \text{se }x \ge 0 \\
          {(-1, 1)} \cdot  r{(-x, y)} & \text{se }x \le 0
      \end{cases}
  \]
  Anche se avessimo scelto come retratto altri sottospazi, avrebbe funzionato.
  In particolare il pillolone e il peso manuale per scarsi sono entrambi
  retratti di deformazione, e quindi i tre sono omotopicamente equivalenti (come
  esercizio si può mostrare che non sono omeomorfi).
\end{example}

\begin{proposition}
    Sia \(X\) spazio topologico \(T_{2}\). Se \(A\) è un suo retratto allora
    \(A\) è chiuso.
\end{proposition}
\begin{proof}
    Sia \(r : X \to A\) una retrazione, quindi \(r|_A = \mathrm{Id}_A\).
    Consideriamo \(\Gamma  = {(i \circ r, \mathrm{Id}_X)} : X \to X \times X)\).
    Allora \(\Gamma^{-1} {(\Delta)} = \{x \in X : x = r{(x)}\} = A\) con
    \(\Delta \subseteq X \times X \) la diagonale. Allora poiché \(X\) è
    \(T_{2}\), \(\Delta\) è chiusa, e dunque \(A\) pure.
\end{proof}
\begin{example}
    \(B_{1}{(0,0)}\) non è retratto di \(\mathbb{R}^2\) perché non è chiuso,
    anche se è omeomorfo.
\end{example}

\begin{definition}
    Sia \(X\) uno spazio topologico e \(x_{0} \in X\). Allora definiamo il
    \textbf{gruppo fondamentale} di \(X\) in \(x_{0}\) come
    \[
      \pi_1{(X, x_{0})} = \{\alpha : I \to X \text{ arco } x_{0} \to
      x_{0}\} / \sim
    \]
    con \(\sim\) la relazione di equivalenza di cammini \(\alpha \sim \beta\) se
    \(\alpha \approx_{[0,1]} \beta \) 
\end{definition}
\begin{proposition}
    La precedente è una buona definizione, ossia \(\pi_{1}{(X, x_{0})}\) è un
    gruppo.
\end{proposition}
\begin{proof}
    La composizione di due cammini è un cammino ed è associativa, l'inverso di
    un cammino è un cammino, la costante \(x_{0}\) è un cammino.
\end{proof}

\begin{proposition}
Se abbiamo uno spazio topologico \(X\)  con \(x_{0}\) e \(x_{1}\) connesse da un
arco \(\gamma\) allora esiste un isomorfismo di gruppi
\[
  \mu_{\gamma} : \pi_{1}{(X, x_{0})} \to \pi_{1}{(X, x_{1})}
\]
\end{proposition}
Specifichiamo che l'isomorfismo non è \emph{canonico}, nel senso che ve ne sono
molti possibilmente e non ce n'è uno privilegiato. Tuttavia vale la seguente
\begin{eser}
    Presi \(\gamma, \gamma' : I \to X\) cammini tra \(x_{0}\) e \(x_{1}\) allora
    \(\mu_\gamma = \mu_{\gamma'} \iff [\gamma * \gamma'] \in Z(\pi{(X, x_{0})})\) 
\end{eser}
Ne consegue che se \(\pi{(X, x_{0})}\) con \(X\) cpa è abeliano allora \(\forall
x \in x\) esiste unico \(\mu_x : \pi_{1}{(X, x_{0})} \cong \pi{(x, x)}\) 

\begin{remark}
    sia \(x\) uno spazio topologico, \(x_{0} \in x\). sia \(c\) una componente
    cpa che contiene \(x_{0}\). allora \(\pi_{1}{(x, x_{0})} \cong \pi_{1}{(c,
    x_{0})}\). 
\end{remark}

\subsection{funtorialità di \(\pi_{1}\)}
    Sia \(f : x \to y\) continua tra spazi topologici. Fissiamo \(x_{0} \in x\)
    \begin{theorem}
        \(f\) induce un omomorfismo di gruppi
        \[
          f_{*} : \pi_{1}{(X, x_{0})} \to \pi_{1}{(Y, f{(x_{0})})}
        \]
    \end{theorem}
    
    \begin{proof}
    \[\begin{tikzcd}
        & X \\
        I & Y
        \arrow["f", from=1-2, to=2-2]
        \arrow["\alpha", from=2-1, to=1-2]
        \arrow["{f \circ \alpha}"', from=2-1, to=2-2]
    \end{tikzcd}\]
    Prendiamo \([\alpha] \in \pi_{1}{(X, x_{0})}\). Definiamo \(f_*{([\alpha])}
    := [f \circ \alpha]\). Effettivamente
    \begin{align*}
      f \circ \alpha : I \to Y \text{ continua } \\
      {(f \circ \alpha)}{(0)} = f{(x_{0})} = {(f \circ \alpha)}{(1)}
    \end{align*}
    dunque \(f \circ \alpha\) è un laccio in \(Y\) con punto base
    \(f{(x_{0})}\). 

    Verifichiamo la buona definizione di \(f_*\). Se \(\beta \sim \alpha\) in
    \(X\), dobbiamo verificare che \(f \circ \beta \sim f \circ \alpha\) in
    \(Y\). Data \(H : I \times I \to X\) l'omotopia tra \(\alpha\) e \(\beta\),
    abbiamo che \(f \circ H : I \times I \to Y\) è omotopia tra \(f \circ
    \alpha\) e \(f\circ \beta\).

    Vediamo ora che \(f_*\) è omomorfismo di gruppi. Siano \([\alpha]\) e
    \([\beta]\) in \(\pi{(X, x_{0})}\) e vogliamo
    \[
      f_*{([\alpha][\beta])} \overset{?}{=} f_*{([\alpha])} \cdot  f_*{([\beta])}
    \]
    ma effettivamnte
    \[
        f_*{([\alpha][\beta])} = [f \circ (\alpha * \beta)] \overset{(1)}{=} [f \circ \alpha *
        f \circ \beta] = [f \circ \alpha] [f \circ \beta] = f_*{([\alpha])} \cdot 
        f_*{([\beta])}
    \]
    dove \({(1)}\) viene semplicemente dal calcolo esplicito, considerando la
    definizione di concatenazione.
    \end{proof}
    \begin{example}
        Sia \(f : X \to Y\) costante. Allora \(f{(X)} = \{\overline{y}\} \) e
        fissato \(x_{0} \in X\) abbiamo 
        \[
          f_* : \pi_{1}{(X, x_{0})} \to \pi_{1}{(Y, \overline{y})}
        \]
        ma preso \([\alpha] \in \pi_{1}{(X, x_{0})}\) allora \(f \circ \alpha :
        I \to Y \) è costante e quindi \(f_*\) è l'omomorfismo banale.
    \end{example}
    
    \begin{theorem}\label{thm:funt_p1}
        Dati \(X,Y,Z\) spazi topologici. Siano \(f : X \to Y\) e \(g : Y \to Z\)
        applicazioni continue, allora fissato \(x_{0} \in X\) abbiamo che
        \[
          \pi_{1}{(X, x_{0})} \overset{f_*}{\to} \pi_{1}{(Y, f{(x_{0})})}
          \overset{g_*}{\to} \pi_{1}{(Z, g{(f{(x_{0})})})}
        \]
        ma vale che 
        \[
          g_* \circ f_* = (g \circ f)_*
        \]
    \end{theorem}
    \begin{proof}
        Sia \([\alpha] \in \pi_{1}{(X, x_{0})}\). Allora
        \[{(g \circ f)}_*{([\alpha])} = [{(g \circ f)} \circ \alpha] = [g \circ
        {(f \circ \alpha)}] = {(g_* \circ f_*)}{([\alpha])}\]
    \end{proof}
    
    \begin{proposition}\label{prop:Id_p1}
        Sia \(X\) uno spazio topologico e \(x_{0} \in X\). Allora
        \((\mathrm{Id}_X)_*\) è l'omomorfismo identico di
        \(\pi{(X, x_{0})}\) 
    \end{proposition}
    \begin{proof}
        \begin{align*}
            {(\mathrm{Id}_X)}_* : \pi_{1}{(X, x_{0})} &\to \pi_{1}{(X, x_{0})} \\
            [\alpha] &\mapsto [\mathrm{Id}_X \circ \alpha] = [\alpha]
        \end{align*}
    \end{proof}
    \begin{proposition}
        Se due spazi topologici sono omeomorfi con \(\varphi : X \to Y \)
        omeomorfismo allora preso \(x_{0} \in X\) 
        \[
          \varphi_* : \pi_{1}{(X, x_{0})} \to \pi_{1}{(Y, \varphi{(x_{0})})}
      \]
      è un isomorfismo di gruppi.
    \end{proposition}
    \begin{proof}
        \(\exists \psi  : Y \to X\) tale che \(\psi \circ \varphi  =
        \mathrm{Id}_X\) e \(\varphi \circ \psi = \mathrm{Id}_Y\). Ora
        \[
            \psi_* \circ \varphi_* \overset{\ref{thm:funt_p1}}{=} {(\psi \circ
            \varphi )}_* = {(\mathrm{Id}_X)}_* \overset{\ref{prop:Id_p1}}{=}
            \mathbb{1}_{\pi_{1}{(X, x_{0})}} 
        \]
        con \(\mathbb{1}\) l'isomorfismo identico.
    \end{proof}
    \begin{remark}
        Per spazi cpa sappiamo che
        \[
          X \sim Y \implies \pi_{1}{(X, x_{0})} \cong \pi_{1}{(Y, y_{0})} \quad
          \forall x_{0} \in X, y_{0} \in Y
        \]
    \end{remark}
    \begin{theorem}\label{thm:omotopia_p1}
        Siano \(X, Y\) spazi topologici e \(f, g : X \to Y\) applicazioni
        continue. Siano \(f \approx g\). Sia \(x_{0} \in X\). Sia \(F : X \times
        I \to Y\) l'omotopia tra \(f\)  e \(g\).
        Notando che \(F{(x_{0}, \cdot )}\) è un cammino in \(Y\) tra \(F{(x_{0},
        0)} = f{(x_{0})}\) e \(F{(x_{0}, 1)} = g{(x_{0})}\), abbiamo che il
        seguente diagramma commuta:
    \[\begin{tikzcd}
        & \pi_{1}{(X, x_{0})} \\
        \pi_{1}{(Y, f{(x_{0})})} && \pi_{1}{(Y, g{(x_{0})})}
        \arrow["{f_*}", from=1-2, to=2-1]
        \arrow["{g_*}", from=1-2, to=2-3]
        \arrow["{\mu_\alpha}", from=2-1, to=2-3]
    \end{tikzcd}\]
    ossia \(g_* = \mu_{\alpha} \circ f_* \) 
    \end{theorem}
    \begin{proof}
        Sia \([\gamma] \in \pi_{1}{(X, x_{0})}\). Vogliamo vedere
        \[
            g_*{([\gamma])} \overset{?}{=} {(\mu_{\alpha}  \circ
            f_*)}{([\gamma])} = \mu_{\alpha} {[f \circ \gamma]} =
            [{(\overline{\alpha} * {(f \circ \gamma)})} * \alpha]
        \]
        Si tratta quindi di verificare che \(f\circ \gamma \approx_{[0,1]}
        \overline{\alpha} * {(f \circ \gamma)} * \alpha \) 
        \[
          H{(t, s)} = \begin{cases}
              \alpha{(1 - 4t)} & t \in [0, \frac{1-s}{4}] \\
              F'{(\frac{4t - 1 + s}{3s + 1}, s)} & t \in [\frac{1- s}{4},
              \frac{1 + s}{2}]\\
              \alpha{(2t -1)} & t \in [\frac{1 + s}{2}, 1]
          \end{cases}
        \]
    \end{proof}
    \begin{corollary}
        Se \(\varphi : X \to Y\) è equivalenza omotopica allora \(\varphi_* :
        \pi_{1}{(X, x_{0})} \to \pi_{1}{(Y, \varphi {(x_{0})})}\), fissato
        \(x_{0} \in X\), è un isomorfismo.
    \end{corollary}
    \begin{proof}
        Presa \(\psi\) un'inversa omotopica di \(\varphi \), abbiamo che 
        \[
            \varphi_* \circ \psi_* = {(\varphi \circ \psi )}_*
            \overset{\ref{thm:omotopia_p1}}{=}
            \mu_{\alpha} \circ {(\mathrm{Id}_X)}_* \overset{\ref{prop:Id_p1}}{=} \mu_{\alpha} \circ
        \mathbb{1}_{\pi_{1}{(X, x_{0})}} \]% https://q.uiver.app/#q=WzAsNSxbMCwwLCJBIl0sWzEsMCwiQiJdLFsyLDAsIkMiXSxbMiwxLCJEIl0sWzEsMSwiRiJdLFswLDQsIlxcbXVfXFxiZXRhIl0sWzMsNCwiXFxtdV9cXGJldGEiLDJdLFsyLDEsIlxcbXVfXFxhbHBoYSIsMl0sWzAsMSwiXFxwc2lfKiJdLFsxLDQsIlxcdmFycGhpXyoiXSxbMiwzLCJcXHZhcnBoaV8qIl1d
\[\begin{tikzcd}
	\pi_{1}{(Y, \varphi {(x_{0})})} & \pi_{1}{(X, \psi {(\varphi {(x_{0})})})} &
    \pi_{1}{(X, x_{0})} \\
	& \pi {(Y, \varphi {(\psi {(\varphi {(x_{0})})})})} & \pi_{1}{(Y, \varphi
    {(x_{0})})}
	\arrow["{\psi_*}", from=1-1, to=1-2]
	\arrow["{\mu_\beta}", from=1-1, to=2-2]
	\arrow["{\varphi_*}", from=1-2, to=2-2]
	\arrow["{\mu_\alpha}"', from=1-3, to=1-2]
	\arrow["{\varphi_*}", from=1-3, to=2-3]
	\arrow["{\mu_\beta}"', from=2-3, to=2-2]
\end{tikzcd}\]
    \end{proof}
    
    \subsection{Osservazioni Fondamentali}
\begin{proposition}
    Se ho \(r : X \to A\) retrazione, con \(A \subseteq X \) e \(r \circ i =
    \mathrm{Id}_A\). Fissato \(x_{0} \in A\), abbiamo che
    \[
      r_* \circ i_* = {(r \circ i)}_* = (\mathrm{Id}_A)_* =
      \mathbb{1}_{\pi_{1}{(A, x_{0})}}
    \]
    dunque \(i_*\) è iniettiva e \(r_*\) è suriettiva.
\end{proposition}
\begin{proposition}
    Se \(A \subseteq X \) è un retratto debole di deformazione e \(r : X \to A\)
    applicazione continua tale che \(i \circ r \approx \mathrm{Id}_X\) e \(r
    \circ i \approx \mathrm{Id}_A\) allora
    \[
      {(i \circ r)}_* = u \icrc (\mathrm{Id}_X)_* = u 
    \]
    dove \(u\) è un isomorfismo da \(\pi_{1}{(X, x_{0})}\) a \(\pi_{1}{(X, (i
    \circ r){(x_{0})})}\) 
\end{proposition}
\begin{proposition}
    Se \(r : X \to A\) è una retrazione forte di deformazione, allora \(r_*
    \circ i_* = \mathrm{Id}_{\pi_{1}{(A, x_{0})}} \) e \(i_* \circ r_* =
    \mathrm{Id}_{\pi_{1}{(X, x_{0})}} \), dunque \(i_*\) e \(r_*\) sono una
    l'inversa dell'altra.
\end{proposition}
\begin{example}
    Il cilindro e il nastro di Möbius si retraggono di deformazione forte in
    \(S^{1}\), e dunque hanno gruppo fondamentale isomorfo a quello di \(S^{1}\)
    (sono connessi per archi dunque hanno \(\pi_{1}\) isomorfo in ogni punto)
\end{example}
\begin{theorem}
    Siano \(X, Y\) spazi topologici. Fissato \(x_{0} \in X\) e \(y_{0} \in Y\)
    vale che 
    \[
      \pi_{1}{(X \times Y, {(x_{0}, y_{0})})} \cong \pi_{1}{(X, x_{0})} \times
      \pi_{1}{(Y, y_{0})}
    \]
\end{theorem}
\begin{proof}
    Intuitivamente, vogliamo proiettare ogni laccio su \(X\) e su \(Y\).

    Più precisamente, sia \([\alpha] \in \pi_{1}{(X \times Y, {(x_{0},
    y_{0})})}\). Allora \(p \circ \alpha\) è laccio in \(X\) con punto base
    \(x_{0}\) e \(q \circ \alpha\) è laccio in \(Y\) con punto base \(y_{0}\).

    Definiamo dunque \begin{align*}
        \phi: \pi_{1}{(X \times  Y, {(x_{0}, y_{0})})} &\longrightarrow
        \pi_{1}(X, x_{0}) \times  \pi1{(Y, y_{0})} \\
        [\alpha] &\longmapsto {([p \circ \alpha], [q \circ \alpha])}
    \end{align*}

    Dobbiamo verificare la buona definizione di \(\phi\). Se \([\alpha] \sim
    [\beta]\) in \(X \times Y\), allora esiste \(F : I \times  I \to X \times  Y
    \) continua tale che \(F{(t, 0)} = \alpha{(t)} = {({(p \circ \alpha)}{(t)},
    {(q \circ \alpha)}{(t)})}\) e \(F{(t, 1)} = \beta{(t)} =
    {({(p \circ \beta)}{(t)}, {(q \circ \beta)}{(t)})}\). Allora per la
    proprietà universale del prodotto, \(p \circ F\) è continua e \(q \circ F\)
    è continua, e sono le omotopie richieste.

    \(\phi\) è un omomorfismo. Infatti
    \begin{align*}
        \phi{([\alpha)][\beta]}) &= \phi{([\alpha * \beta])} = 
        {([p \circ (\alpha * \beta)], [q \circ (\alpha * \beta)])} = \\
        &= {([p \circ \alpha * p \circ \beta], [q \circ \alpha * q \circ
        \beta])} = \\
        &= {([p \circ \alpha][p \circ \beta], [q \circ \alpha][q \circ \beta])}
        = \phi{([\alpha])} \cdot \phi{([\beta])}
    \end{align*}

    \(\phi\) è suriettivo. Infatti fissato \([\alpha_{1}] \in \pi_1{(X, x_{0})}\) e
    \([\beta_{1}] \in \pi_{1}{(Y, y_{0})}\) abbiamo che, preso \(\alpha{(t)} =
    {(\alpha_1{(t)}, \alpha_2{(t)})}\), evidentemente 
    \[
        \phi{([\alpha])} = {([p \circ \alpha], [q \circ \alpha])} =
        {([\alpha_1], [\alpha_2])}
    \]

    Infine 
    \begin{align*}
    \mathrm{Ker}(\phi) &= \{[\alpha] \in \pi_{1}{(X \times Y, {(x_{0}, y_{0})})}
    : \phi{([\alpha])} = {(\varepsilon_{x_{0}} , \varepsilon_{y_{0}} )}\} = \\
   &= \{[\alpha] : p\circ \alpha \sim \varepsilon_{x_{0}} \text{ in \(X\) e } q
   \circ \alpha \sim \varepsilon_{y_{0}} \text{ in \(Y\)}\} 
    \end{align*}

    Ora presa \(F : I \times  I \to X\) l'omotopia tra \(p \circ \alpha\) e
    \(\varepsilon_{x_{0}} \) relativa a \(\{0, 1\} \) e \(G : I \times I \to Y\)
    quella tra \(q \circ \alpha\) e \(\varepsilon_{y_{0}} \). Allora presa
    \begin{align*}
        H = {(F, G)}: I \times I &\longrightarrow X \times Y \\
        {(t,s)} &\longmapsto H{(t,s)} = {(F{(t,s)}, G{(t, s)})}
    \end{align*}
    che è omotopia tra \(H{(\cdot , 0)} = {(p \circ \alpha, q \circ \alpha)}\) e
    \(H{(\cdot , 1)} = {(\varepsilon_{x_{0}} , \varepsilon_{y_{0}} )} =
    \varepsilon_{{(x_{0}, y_{0})}} \) 
e dunque \(\mathrm{Ker}{(\phi)} = \{[\varepsilon_{x_{0},y_{0}}] \}\) e quindi
\(\phi\) è iniettivo.
\end{proof}

\section{Gruppo fondamentale di \(S^{1}\)}

Intuitivamente presa una circonferenza e un punto base (tipo se la vediamo nel
piano complesso potremmo prendere 1). I lacci sono, eccetto quello costante,
quelli fatti da un numero di giri intero sulla circonferenza stessa.
Effettivamente l'idea è contare ``quante volte'' passo da 1, e dire che tale
numero identifica la classe di equivalenza del laccio. Ne concluderemo dunque
che \(\pi_{1}{(S^{1})} \cong \mathbb{Z}\).

Consideriamo ora \(e : \mathbb{R} \to S^{1}\) la mappa esponenziale \(t \mapsto
e^{2\pi i t}\) in \(\mathbb{C}\) oppure \(t \mapsto {(\cos{(2\pi t)}, \sin{(2
\pi t)})}\) in \(\mathbb{R}^2\). Allora \(e\) è una identificazione aperta, e
per verificarlo basta controllare una base di \(\mathbb{R}\). Effettivamente
preso un aperto \({(a,b)}\), l'immagine è un arco aperto oppure tutto \(S^{1}\),
che sono entrambi aperti. Inoltre \(e^{-1}{(1)} = \mathbb{Z}\) e preso
\(e^{-1}{(e^{i\theta})} = \frac{\theta}{\pi} + \mathbb{Z}\) 
\begin{remark}
    Se prendo \({(a,b)} \subseteq \mathbb{R} \) con \(b-a < 1\) allora
    \(e|_{{(a,b)}} \) è un omeomorfismo su \(e{(a,b)}\) che è un arco aperto di
    \(S^{1}\).
\end{remark}
\begin{remark}
    Al contrario, preso un sottoinsieme connesso e aperto di \(S^{1}\) che non
    sia tutto \(S^{1}\) (un archetto aperto insomma) denotato \(A\), allora
    \(e^{-1}{(A)}\) è un insieme composto da \(\mathbb{Z}\) intervalli aperti.

\end{remark}

\begin{lemma}[Aperti uniformemente rivestiti]\label{lem:unif_rivest}
    Se \(A \subseteq S^{1} \) è aperto allora
    \[
      e^{-1}{(A)} = \coprod_{n \in \mathbb{Z}} U_{i}
    \]
    con \(U_{i} \subseteq \mathbb{R} \) aperti tale che \(e|_{U_{i}} : U_{i} \to
    A\) è un omeomorfismo. 
\end{lemma}
\begin{corollary}
    Se ho una funzione \(f: X \to S^{1}\) continua, con \(X\) spazio topologico.
    Se \(f\) non è suriettiva allora è omotopa a una costante.
\end{corollary}
\begin{proof}
    Non è suriettiva, significa che \(\exists p \in S^{1}\) tale che \(p \not\in
    \mathrm{Im}{(f)}\). Posso vedere \(f\)  come \(f: X \to S^{1} \sminus \{p\}
    \sim (0, 1)\) che è contraibile.
\end{proof}






\end{document}
