%! TEX program = lualatex
\input{../preamble_appunti_report.tex}

\title{Appunti di Geometria 2}
\author{Github Repository:
\href{https://github.com/Oxke/appunti/tree/main/Geo2}{\texttt{Oxke/appunti/Geo2}}}
\date{Secondo semestre, 2024 \-- 2025, prof. Lidia Stoppino e Leone Slavich}

\begin{document}

\begin{titlingpage}
\maketitle

\vspace{1cm}
Il corso è diviso in due parti: geometria differenziale (tenuto da Slavich) e
gruppo fondamentale (tenuto dalla Stoppino). Ci sono esercitazioni di geometria
differenziale.
Ci sono vari libri consigliati:
\begin{itemize}[label = --]
    \item Abate \-- Tovena, \emph{Curve e superfici}, Springer
    \item M. D. Do Carmo, \emph{Differential Geometry of Curves and Surfaces},
        Prentice Hall
    \item E. Sernesi, \emph{Geometria 2}, Bollati Boringhieri
    \item Dispense del prof. Ghigi (sulle quali è stato disegnato il corso,
        almeno per la parte di geometria differenziale)
\end{itemize}

\end{titlingpage}

\chapter{Geometria Differenziale}
Il corso di geometria differenziale studierà curve e superfici in
\(\mathbb{R}^3\) definiti \textbf{analiticamente} tramite funzione
\(C^{\infty}\) \emph{(lisce)}. Studiamo la \textbf{geometria locale} e infine
(verso la fine del corso) la \textbf{geometria globale}, in particolare il
teorema di Gauss-Bonnet.

\section{Definizioni e proprietà iniziali}
\subsection{Funzioni lisce}

Sia \(I = {(a,b)} \subseteq \mathbb{R} \) intervallo aperto (anche possibilmente
\(a = -\infty\) o \(b = +\infty\)). Sia 
\[
    C^{0}{(I)} = \{f : I \to \mathbb{R} : f \text{ è continua\footnote{si veda
    Geometria 1 per definizione e studio della continuità} su }I\}
\]
\begin{definition}{Derivabile}
    Diciamo che \(f \in C^{0}{(I)}\) è derivabile se \(\forall x_{0} \in I\),  
    \[
        \lim_{x \to x_{0}} \frac{f{(x)} - f{(x_{0})}}{x - x_{0}} = c =:
        f'{(x_{0})} \quad ; \quad c \in \mathbb{R}
    \]
\end{definition}
Ora procediamo definendo ricorsivamente le funzioni \(C^{k}{(I)}\).
\begin{definition}{Classe \(C^{k}\) }
    Per ogni \(k\ge 1\), diciamo che \(f \in C^{k}{(I)}\) se \(f\) è derivabile
    e \(f' \in C^{k-1}{(I)}\)
\end{definition}
Dunque, ad esempio \(f \in C^{1}{(I)}\) se \(f\) è derivabile su \(I\) e la sua
derivata \(f'\) è continua su \(I\).
Detto più colloquialmente, una funzione \(f \in C^{k}{(I)}\) è una funzione
derivabile (almeno) \(k\) volte, e tale che la sua derivata \(i\)-esima
\(f^{{(i)}}\) è continua per ogni \(i = 0, \dots, k\).
\begin{remark}
    \[
        C_{0} \supset C^{1} \supset C^{2} \supset \dots \supset C^{i} \supset
        C^{i+1} \supseteq  \dots
    \]
\end{remark}
\begin{definition}{funzioni lisce}
    \[
        C^{\infty}{(I)} = \bigcap_{k=0}^{\infty} C^{k}{(I)} = \text{insieme
        delle \textbf{funzioni lisce}}
    \]
\end{definition}

\begin{theorem}[Proprietà delle classi \(C^{k}\)]\label{thm:proprieta_Ck}
    Sia \(k \in \mathbb{N} \cup \{+\infty\} \). Se \(f, g \in C^{k}{(I)}\) e
    \(\lambda \in \mathbb{R}\), allora
\begin{enumerate}[label = \arabic*.]
    \item \(f + g \in C^{k}{(I)}\) 
    \item \(\lambda f \in C^{k}{(I)}\)
    \item \(f\cdot g \in C^{k}{(I)}\)
\end{enumerate}
\end{theorem}
\begin{proof}
    1.~e 2.~sono semplici. Per 3.~si procede per induzione su \(k\).

    Nel caso base \(k = 0\) il prodotto di due funzioni continue è anch'esso
    una funzione continua, che è vero.

    Supponiamo ora che 3.~valga per \(k-1\). Siano \(f, g \in C^{k}{(I)}\).
    Allora \((f\cdot g)' = f' \cdot g + f \cdot g'\) che è somma di funzioni
    \(C^{k-1}\) per ipotesi induttiva e perché \(C^{k} \subset C^{k-1}\), e
    dunque \((f\cdot g)' \in C^{k-1}\) da cui segue che \(f\cdot g \in C^{k}\).

    Infine possiamo concludere per \(k = +\infty\) perché vale per tutti i \(k
    \in \mathbb{N}\).
\end{proof}

Dal teorema~\ref{thm:proprieta_Ck} segue che \(C^{k}{(I)}\) è uno spazio
vettoriale (con operazione di somma e moltiplicazione per scalare) e inoltre
\(C^{k} {(I)}\) contiene le funzioni costanti e allora \(C^{k}{(i)}\) con
operazioni di somma e moltiplicazione puntuale è un anello. Da queste due segue
che \(C^{k}{(I)}\) è una \(\mathbb{R}\)-algebra.

\begin{example}\label{ex:liscia_non_analitica}
    Esistono funzioni \textbf{lisce} che \textbf{non} sono \textbf{analitiche}.
    In particolare esistono funzioni lisce che sono nulle su un aperto ma non
    nulle dappertutto (differentemente da quanto succede sulle funzioni
    olomorfe). Un esempio di tale funzione è 
    \begin{align*}
        f : \mathbb{R} &\longrightarrow \mathbb{R} \\
        x &\longmapsto f (x) = \begin{cases}
            0 & x \le 0 \\
            e^{-\frac{1}{x^2}} & x > 0
        \end{cases}
    \end{align*}
    
    Questa è una funzione \(C^{\infty}{(\mathbb{R})}\) che non può essere
    analitica perché contraddirebbe il teorema del prolungamento.
    \begin{figure}[ht]
        \centering
        \begin{tikzpicture}
            \begin{axis}[
                xmin= -3, xmax= 3,
                ymin= -1, ymax = 1.5,
                axis lines = middle,
                width = 0.8\textwidth,
            ]
            \addplot[domain=-3:3, samples=100]{exp(-1/x^2) * (x > 0)};
            \end{axis}
        \end{tikzpicture}
        \caption{Grafico della funzione \(f(x)\)
        dell'esempio~\ref{ex:liscia_non_analitica}}\label{fig:liscia_non_analitica} 
    \end{figure}
    Similmente si possono costruire funzioni costanti in aperti, raccordate in
    modo \(C^{\infty}\) e ovviamente non analitiche.
\end{example}
\begin{proposition}[Composizione]\label{prp:composizione_ck}
    La composizione di funzioni \(C^{\infty}\) è \(C^{\infty}\). Sia \(f : I \to
    \mathbb{R}\) e \(g : J \to \mathbb{R}\). Allora se \(f \in C^{\infty}{(I)}\)
    e \(g \in C^{\infty}{(J)}\) e \(f{(I)} \subseteq J \) (ossia si possono
    comporre), allora \(g \circ f : I \to \mathbb{R}\) è ben definita e 
    \[
        g \circ f \in C^{\infty}{(I)}
    \]
\end{proposition}
\begin{proof}
    Lo dimostriamo per \(k \in \mathbb{N}\) invece che \(k = \infty\), segue
    naturalmente il caso enunciato. Per \(k = 0\) è ovvio.

    Supponiamo che valga per \(k-1\). Allora siano \(f, g \in C^{k}\) e
    tali che \(f{(I)} \subseteq J \). Allora \({(g \circ f)}' = {(g' \circ f)}
    \cdot f'\) che è prodotto di funzioni \(C^{k-1}\) per ipotesi induttiva e
    per il teorema~\ref{thm:proprieta_Ck} segue che \(g \circ f \in C^{k}{(I)}\).
\end{proof}

\subsection{Diffeomorfismi}
\begin{definition}{Diffeomorfismo}
    Un diffeomorfismo è un isomorfismo nella categoria delle funzioni lisce (su
    \(\mathbb{R}\) nel nostro caso). 
\end{definition}
Informalmente, un diffeomorfismo è un omeomorfismo \(C^{\infty}\). 
\begin{definition}{Diffeomorfismo}
    Siano \(I, J \subseteq \mathbb{R} \) intervalli aperti in \(\mathbb{R}\).
    Allora \(f : I \to J\) è un \textbf{diffeomorfismo} se
\begin{enumerate}[label = \arabic*.]
    \item \(f \in C^{\infty}{(I)}\) 
    \item \(f\) è biettiva
    \item \(f^{-1} \in C^{\infty}{(J)}\)
\end{enumerate}
\end{definition}
\begin{remark}
    La terza condizione \textbf{non} è ridondante. Infatti sia \(I = J =
    \mathbb{R}\) e \(f{(x)} = x^{3}\) che è chiaramente \(C^{\infty}\) e
    biunivoca. Tuttavia \(f^{-1}{(x)} = \sqrt[3]{x}\) non è derivabile in \(0\),
poiché \(f'(0) = 0\) e \({f^{-1}}'(y) = \frac{1}{f'{(x)}}\) se \(f{(x)} = y\)
per ogni \(x \in \mathbb{R}\) tale che \({f^{-1}}'{(y)}\) sia ben definita.
\end{remark}
\begin{remark}
    Se \(I\) e \(J\) sono intervalli aperti di \(\mathbb{R}\) e \(f : I\to J\) è
    diffeomorfismo, allora \(f'{(x)} \neq 0\) per ogni \(x \in I\). Infatti
    sappiamo che
    \begin{equation}\label{eq:derivata_inversa}
        {f^{-1}}'(y) = \frac{1}{f'{(x)}} \quad ; \quad f{(x)} = y \quad
        \forall x \in J 
    \end{equation}
    dunque \(f'{(x)}\) non può essere nullo, poiché significherebbe che
    \(f^{-1}\) non è derivabile in \(y = f^{-1}{(x)}\).
\end{remark}
\begin{lemma}
    Sia \(I \subseteq \mathbb{R}\) un intervallo aperto. Sia \(f : I \to
    \mathbb{R}\) una funzione liscia e tale che \(f'{(x)} \neq 0\) per ogni \(x
    \in I\). Allora \(f{(I)} = J\) è un intervallo aperto e \(f : I \to J\) è un
    diffeomorfismo.
\end{lemma}
\begin{proof}
    Sia \(f\) come nell'enunciato. Allora \(f' : I \to \mathbb{R}\) è continua
    su \(I\) e non si annulla mai. Segue che \(f'\) ha segno costante su \(I\)
    (\(f' > 0\) oppure \(f' < 0\)).

    Assumiamo \(f' > 0\) su \(I\). Allora \(f\) è strettamente crescente e
    dunque iniettiva. Allora \(f : I \to f{(I)} =: J\) è biettiva. Inoltre \(J\)
    è un intervallo aperto in \(\mathbb{R}\). Infatti è chiaramente intervallo
    (è connesso) in quanto immagine continua di un connesso (intervallo).  Sia
    ora \(y_{0} \in J\) e sia \(x_{0} \in I\) tale che \(f{(x_{0})} = y_{0}\).
    Sia \(\varepsilon > 0\) tale che \([x_{0} - \varepsilon, x_{0} +
    \varepsilon] \subseteq I \). Poiché \(f\) è strettamente crescente, segue
    che 
    \[
        f{(x_{0}-\varepsilon)} < f{(x_{0})} < f{(x_{0}+\varepsilon)}
    \]
    sapendo già che \(J\) è un intervallo, segue che \[I\supseteq
    [f{(x_{0}-\varepsilon)}, f{(x_{0} + \varepsilon)}] \text{ è un intorno di
\(y_{0}\)}\]
    Rimane solo da vedere che la funzione \(f^{-1} : J \to I\) è
    \(C^{\infty}\). Notiamo intento che \(f^{-1}\) è continua, poiché \(f\) è aperta.
    Inoltre sappiamo che \(f^{-1}\) è derivabile poiché è l'inversa di una
    funzione derivabile con derivata e vale
    l'equazione~\eqref{eq:derivata_inversa} per ogni \(y \in J\).

    Sia \(u : {(0, +\infty)} \to {(0, +\infty)}\) definita da \(u{(x)} = 1/x\) è
    un diffeomorfismo e da~\ref{eq:derivata_inversa} abbiamo che
    \[
        {f^{-1}}' = u \circ f' \circ f^{-1}
    \]
    e quindi se assumiamo induttivamente che se \(f^{-1} \in C^{k}\) allora ne
    consegue dalla proposizione~\ref{prp:composizione_ck} che \({f^{-1}}' \in
    C^{k}\) e dunque \(f^{-1} \in C^{k+1}\) 

\end{proof}

\subsection{Curve}
\begin{definition}[label=def:curva]{Curva parametrizzata}
    Sia \(\alpha : I \to \mathbb{R}^{3}\) una funzione 
    \(t \mapsto (\alpha_{1}(t), \alpha_{2}(t), \alpha_{3}(t)) \) con \(I\)
    intervallo aperto.
    
    Allora se \(\alpha_{1}, \alpha_{2}, \alpha_{3}\) sono funzioni lisce la
    funzione \(\alpha\) è detta \textbf{curva} parametrizzata in \(\mathbb{R}^3\) 
\end{definition}
In generale se una funzione \(\alpha : I \to \mathbb{R}^{n}\) verrà chiamata
\emph{funzione vettoriale} e ha come componenti \(n\) \emph{funzioni scalari}
\(a_{i} : I \to \mathbb{R}\). Con questa terminologia allora una curva
parametrizzata è una funzione vettoriale in \(\mathbb{R}^{3}\) con componenti
\(C^{\infty}{(I)}\).
\begin{example}[Retta in \(\mathbb{R}^{3}\) ]\label{ex:retta}
    Ovviamente in forma parametrica
    \[
        \alpha{(t)} = \mathbf{p}_{0} + t\mathbf{v} \quad ; \quad \mathbf{p}_{0},
        \mathbf{v} \in \mathbb{R}^3 \text{ fissati e \(t \in \mathbb{R}\)}
    \]
    e dunque \(\alpha_{1}{(t)} = p_{0_{1}} + t v_{1}\) e simili per le altre due
    componenti, e sono tutte funzioni lisce.
\end{example}
\begin{example}
    In \(\mathbb{R}^{2}\) prendiamo \(C = \{{(x,y)} \in \mathbb{R}^2 :
    {(x-x_{0})}^2 + {(y-y_{0})}^2 = r^2\} \) è una circonferenza di raggio \(r
    \in \mathbb{R}_{> 0} \) e centro \({(x_{0}, y_{0})} \in \mathbb{R}^2\). Una
    possibile parametrizzazione è
    \[
      \alpha{(t)} = (x_{0} + r\cos{t}, y_{0} + r\sin{t}) \quad ; \quad t \in
      \mathbb{R}
    \]
    In questo caso avremmo potuto prendere anche \(I = [0, 2\pi]\), non è un
    problema che \(\alpha\) non sia iniettiva
\end{example}

La definizione~\ref{def:curva} è molto generale e non richiede che la curva sia
come ci piacerebbe immaginarcela. Infatti anche se la curva è \(C^{\infty}\),
possiamo costruirne una che abbia un punto angoloso, anche se ha
parametrizzazione \(C^{\infty}\). Un esempio è visto
nell'esempio~\ref{ex:liscia_angolosa}. Inoltre vorremmo avere una definizione
più bella di curva, che dipenda meno dalla parametrizzazione scelta.

\begin{definition}{Vettore tangente}
    Data \(\alpha: I \to \mathbb{R}^3\) una curva parametrizzata, fissato un
    punto \(t \in I\), definiamo il \textbf{vettore tangente} ad \(\alpha\) al
    tempo \(t\) come
    \[
      \dot{\alpha}{(t)} = \frac{d \alpha}{dt} {(t)} = \begin{pmatrix}
          \alpha_{1}'{(t)} \\ \alpha_{2}'{(t)} \\ \alpha_{3}'{(t)}
      \end{pmatrix}
    \]
\end{definition}
\begin{remark}
    Intuitivamente (nella visione cinematica della curva parametrizzata), il
    vettore tangente rappresenta la velocità della particella che si muove lungo
    la curva 
\end{remark}
\begin{remark}
    Una retta ha tante parametrizzazioni diverse 
\end{remark}
Fissiamo due diverse parametrizzazioni della stessa retta \(r\):
\begin{equation*}
    \alpha{(t)} = \mathbf{p}_0 + t\mathbf{v} \quad ; \quad \beta{(t)} =
    \mathbf{q}_0 + t\mathbf{w}
\end{equation*}
Allora \(\alpha\) e \(\beta\) definiscono la stessa retta se e solo se \(\mathbf{v} 
\parallel \mathbf{w} \parallel \mathbf{q}_0 - \mathbf{p}_0\) sono paralleli.
Equivalentemente
\[
  \mathbf{q}_0 = \alpha{(t_{0})} \quad e \quad \mathbf{w}  = \lambda \mathbf{v} 
\]
ma allora
\[
  \beta{(s)} = \mathbf{q}_0 + s\mathbf{w}  = \alpha{(t_{0})} +
  s{(\lambda\mathbf{v} )} = \mathbf{p}_0 + t_{0}\mathbf{v} +\lambda s \mathbf{v}
  = \mathbf{p}_0 + {(t_{0}+\lambda s)} \mathbf{v}  = \alpha{(t_{0} + \lambda s)} 
\]
ossia \(\beta = \alpha\circ h\) con \(h{(s)} = t_{0}+\lambda s\) è una funzione
liscia con derivata mai nulla, dunque un diffeomorfismo. Questo motiva la
seguente definizione

\begin{definition}{Riparametrizzazione}
    Sia \(\alpha:I \to \mathbb{R}^3\) una curva parametrizzata e \(h : J\to I\)
    un diffeomorfismo. Allora \(\beta := \alpha\circ h : J \to \mathbb{R}^3\) è
    una \textbf{riparametrizzazione} di \(\alpha\) 
\end{definition}
\begin{remark}
    \(h'= 0\) significa che \(\dot{\beta}{(t)} = 0 \iff \dot{\alpha}{(t)}
    =0\). Se \(h'> 0\) allora la curva viene percorsa nello stesso verso.
\end{remark}
A noi interessano le curve parametrizzate \emph{a meno di riparametrizzazione}.
Questo suggerisce di introdurre una classe di equivalenza sulle curve
parametrizzate
\begin{definition}{Equivalenza tra curve}
    Siano \(\alpha : I \to \mathbb{R}^3\) e \(\beta : J \to \mathbb{R}^3\)
    curve parametrizzate. Allora \(\alpha\) e \(\beta\) sono \textbf{equivalenti}
    se esiste un diffeomorfismo \(h : J \to I\) tale che \(\beta = \alpha \circ
    h\). In altre parole \(\alpha\) e \(\beta\) sono \textbf{equivalenti} se e
    solo se \(\beta\) è una riparametrizzazione di \(\alpha\).

    La notazione che si usa è allora \(\alpha \sim \beta\)
\end{definition}
\begin{note}
    La relazione di equivalenza \(\sim\) è una relazione di equivalenza. Infatti
    è ovviamente simmetrica per il diffeomorfismo \(t \mapsto t\), è simmetrica
    mediante il diffeomorfismo \(h^{-1}\) ed è transitiva perché la composizione
    di due diffeomorfismi è un diffeomorfismo.
\end{note}
\begin{definition}{Curve geometriche}
    L'insieme delle curve geometriche è l'insieme delle classi di equivalenza
    delle curve parametrizzate rispetto alla relazione di equivalenza \(\sim\) 
    di riparametrizzazione.
\end{definition}
Per ogni curva geometrica, vogliamo trovare una curva parametrizzata in
parametrizzazione ``canonica''.
\begin{definition}{Parametrizzazione per lunghezza d'arco}
    Data \(\alpha : I \to \mathbb{R}^3\) una curva parametrizzata, \(\alpha\) è
    detta \textbf{parametrizzata per lunghezza d'arco} (o \emph{parametrizzata
    per ascissa curvilinea}) se 
    \[
      \|\dot{a}{(t)}\| = 1 \quad \forall t \in I
    \]
\end{definition}
Ora la questione è capire se è possibile riparametrizzare una curva in modo che
abbia parametrizzazione per lunghezza d'arco. Per quanto osservato prima è
necessario che \(\dot{\alpha}{(t)} \neq \mathbf{0}\), infatti \(\dot{\beta}{(s)} =
\dot{\alpha}{(t)} \cdot h'{(s)}\). Vogliamo ora mostrare che questa condizione è
sufficiente. 
\begin{definition}{Curva regolare}
    Una curva parametrizzata \(\alpha : I \to \mathbb{R}^3\) si dice
    \textbf{regolare} se \(\alpha{(t)} \neq \mathbf{0} \) per ogni \(t \in I\) 
\end{definition}

\begin{theorem}[regolare \(\iff \exists\) riparam.~per lunghezza
    d'arco]\label{thm:regolare_parametrizzata}
    Sia \(\alpha : I \to \mathbb{R}^3\) una curva parametrizzata regolare.
    Allora \(\exists h : J \to I\) diffeomorfismo tale che \(\beta = \alpha
    \circ h : J \to \mathbb{R}\) è una parametrizzazione per lunghezza d'arco.
\end{theorem}
Prima di procedere alla dimostrazione facciamo una piccola digressione sulle
lunghezze di una curva. Data \(\alpha: I \to \mathbb{R}^3\) una curva
parametrizzata. Fisso \([c,d] \subseteq I \)  intervallo chiuso. Allora
l'\emph{arco di curva} \(\alpha|_{[c, d]} : [c, d] \to \mathbb{R}\) ha lunghezza
che si calcola come
\[
    L{(\alpha|_{[c,d]} )} = \int_{c}^{d} \|\dot{\alpha}{(\tau)}\| \,d \tau
\]
per continuità della norma l'integranda è continua e dunque
integrabile. In particolare le somme di Riemann di questo integrale
corrispondono alle lunghezza delle curve poligonali che approssimano la curva,
il sup di esse è dunque il valore cercato.

\begin{proof}[Dimostrazione del teorema~\ref{thm:regolare_parametrizzata}]
    Sia \(\alpha : I\to \mathbb{R}^3\). Fisso \(t_{0} \in I\) e definisco \(h :
    I \to \mathbb{R}\) tramite
    \[
      h{(t)} = \int_{t_{0}}^{t} \|\dot{\alpha}{(\tau)}\| \,d \tau
    \]
    allora \(h{(I)} = J \subseteq \mathbb{R}  \) è un intervallo aperto. Inoltre
    \(h : I \to J\) è un diffeomorfismo e \(\beta = \alpha \circ h^{-1} : J \to
    \mathbb{R}^3\) è una riparametrizzazione con ascissa curvilinea.

    Infatti:
\begin{enumerate}[label = \arabic*.]
    \item \(h\) è \(C^{\infty}\). Infatti \(t \mapsto \|\dot{\alpha}{(t)}\|\) è
        \(C^{\infty}\) in quanto composizione di funzioni lisce. Infatti è
        radice di un valore che non è mai nullo, dunque la radice è definibile
        da \({(0, +\infty)}\) e allora \(C^{\infty}\).

        Allora per il teorema fondamentale del calcolo integrale, \(h\) è
        \(C^{\infty}\) 
    \item \(h'{(t)} = \|\dot{\alpha}{(t)}\| > 0\), dunque \(h\) è diffeomorfismo
        e \(J\) è aperto.
    \item \(\beta = \alpha \circ h^{-1} : J \to \mathbb{R}^3\) è \(C^{\infty}\)
        e
        \[
            \dot{\beta}{(s)} = \dot{\alpha}{(h^{-1}{(s)})} \cdot
            {(h^{-1})}'{(s)} =
            \frac{\dot{\alpha}{(h^{-1}{(s)})}}{h'{(h^{-1}{(s)})}} =
            \frac{\dot{\alpha}{(h^{-1}{(s)})}}{\|\dot{\alpha}{(h^{-1}{(s)})}\|}
        \]
        e dunque \(\|\dot{\beta}\| = 1\) 
\end{enumerate}
\end{proof}

Il teorema~\ref{thm:regolare_parametrizzata} è la motivazione per cui
sceglieremo di lavorare con curve regolari.

\begin{example}
    \(C = \{{(x, y)} \in \mathbb{R}^2 : x \ge 0, y\ge 0, xy=0\}\) è l'unione dei
    semiassi positivi. Allora due fatti sono veri:
\begin{enumerate}[label = \arabic*.]
    \item È possibile trovare una curva parametrizzata liscia con \(\alpha : I
        \to R^2\) tale che \(\alpha{(I)} = C\) 
    \item Non esiste una curva regolare tale che \(\alpha{(I)} = C\)
\end{enumerate}
dunque le curve regolari sono l'oggetto giusto per studiare la geometria delle
curve che appaiono geometricamente lisce.
\end{example}

\begin{example}[Curva liscia con punto angoloso]\label{ex:liscia_angolosa}
    Sia \(\alpha : I \to \mathbb{R}^2\) definita da
\end{example}

\end{document}

