\documentclass{article}
\usepackage{layout}
\usepackage[a4paper, total={5in,9in}]{geometry}
\usepackage[T1]{fontenc}
\usepackage[italian]{babel}
\usepackage{mathtools}
\usepackage{amsthm}
\usepackage[framemethod=TikZ]{mdframed}
\usepackage{amsmath}
\usepackage{amssymb}
\usepackage{cancel}
\usepackage[dvipsnames]{xcolor}
\usepackage{tikz}
\usepackage{tikz-cd}
\usepackage{pgfplots}
\pgfplotsset{compat=1.18}
\usepackage[many]{tcolorbox}
\usepackage{import}
\usepackage{pdfpages}
\usepackage{transparent}
\usepackage{enumitem}
\usepackage[colorlinks]{hyperref}

\newcommand*{\sminus}{\raisebox{1.3pt}{$\smallsetminus$}}

\newcommand*{\transp}[2][-3mu]{\ensuremath{\mskip1mu\prescript{\smash{\mathrm t\mkern#1}}{}{\mathstrut#2}}}%

% newcommand for span with langle and rangle around
\newcommand{\Span}[1]{{\left\langle#1\right\rangle}}

\newcommand{\incfig}[2][1]{%
    \def\svgwidth{#1\columnwidth}
    \import{./figures/}{#2.pdf_tex}
}

\pdfsuppresswarningpagegroup=1

\newcounter{theo}[section]\setcounter{theo}{0}
\renewcommand{\thetheo}{\arabic{section}.\arabic{theo}}

\newcounter{excounter}[section]\setcounter{excounter}{0}
\renewcommand{\theexcounter}{\arabic{section}.\arabic{excounter}}

\numberwithin{equation}{section}

\newenvironment{theorem}[1][]{
    \refstepcounter{theo}
     \ifstrempty{#1}
    {\mdfsetup{
        frametitle={
            \tikz[baseline=(current bounding box.east),outer sep=0pt]
            \node[anchor=east,rectangle,fill=blue!20,rounded corners=5pt]
            {\strut Teorema~\thetheo};}
        }
    }{\mdfsetup{
        frametitle={
            \tikz[baseline=(current bounding box.east),outer sep=0pt]
            \node[anchor=east,rectangle,fill=blue!20,rounded corners=5pt]
            {\strut Teorema~\thetheo:~#1};}
        }
    }
    \mdfsetup{
        roundcorner=10pt,
        innertopmargin=10pt,linecolor=blue!20,
        linewidth=2pt,topline=true,
        frametitleaboveskip=\dimexpr-\ht\strutbox\relax,
        % nobreak=false
    }
\begin{mdframed}[]\relax}{
\end{mdframed}}

% \newenvironment{definition}[1][]{
%     \refstepcounter{theo}
%      \ifstrempty{#1}
%     {\mdfsetup{
%         frametitle={
%             \tikz[baseline=(current bounding box.east),outer sep=0pt]
%             \node[anchor=east,rectangle,fill=violet!20,rounded corners=5pt]
%             {\strut Definizione~\thetheo};}
%         }
%     }{\mdfsetup{
%         frametitle={
%             \tikz[baseline=(current bounding box.east),outer sep=0pt]
%             \node[anchor=east,rectangle,fill=violet!20,rounded corners=5pt]
%             {\strut Definizione~\thetheo:~#1};}
%         }
%     }
%     \mdfsetup{
%         roundcorner=10pt,
%         innertopmargin=10pt,linecolor=violet!20,
%         linewidth=2pt,topline=true,
%         frametitleaboveskip=\dimexpr-\ht\strutbox\relax,
%         nobreak=true
%     }
% \begin{mdframed}[]\relax}{
% \end{mdframed}}

\newtcolorbox[auto counter, number within=section]{definition}[2][]{
    colframe=violet!0,
    coltitle=violet, % Title text color
    fonttitle=\bfseries, % Title font
    title={Definizione~\thetcbcounter\ifstrempty{#2}{}{:~#2}}, % Title format
    sharp corners, % Less rounded corners
    boxrule=0pt, % Line width of the box frame
    toptitle=1mm, % Distance from top to title
    bottomtitle=1mm, % Distance from title to box content
    colbacktitle=violet!5, % Background color of the title bar
    left=0mm, right=0mm, top=1mm, bottom=1mm, % Padding around content
    enhanced, % Enable advanced options
    before skip=10pt, % Space before the box
    after skip=10pt, % Space after the box
    breakable, % Allow box to split across pages
    colback=violet!0,
    borderline west={2pt}{-5pt}{violet!40},
    #1
}

\newenvironment{lemmao}[1][]{
    \refstepcounter{theo}
     \ifstrempty{#1}
    {\mdfsetup{
        frametitle={
            \tikz[baseline=(current bounding box.east),outer sep=0pt]
            \node[anchor=east,rectangle,fill=green!20,rounded corners=5pt]
            {\strut Lemma~\thetheo};}
        }
    }{\mdfsetup{
        frametitle={
            \tikz[baseline=(current bounding box.east),outer sep=0pt]
            \node[anchor=east,rectangle,fill=green!20,rounded corners=5pt]
            {\strut Lemma~\thetheo:~#1};}
        }
    }
    \mdfsetup{
        roundcorner=10pt,
        innertopmargin=10pt,linecolor=green!20,
        linewidth=2pt,topline=true,
        frametitleaboveskip=\dimexpr-\ht\strutbox\relax,
        % nobreak=true
    }
\begin{mdframed}[]\relax}{
\end{mdframed}}

\theoremstyle{plain}
\newtheorem{lemma}[theo]{Lemma}
\newtheorem{corollary}{Corollario}[theo]
\newtheorem{proposition}[theo]{Proposizione}

\theoremstyle{definition}
\newtheorem{example}[excounter]{Esempio}

\theoremstyle{remark}
\newtheorem*{note}{Nota}
\newtheorem*{remark}{Osservazione}

\newtcolorbox{notebox}{
  colback=gray!10,
  colframe=black,
  arc=5pt,
  boxrule=1pt,
  left=15pt,
  right=15pt,
  top=15pt,
  bottom=15pt,
}

\DeclareRobustCommand{\rchi}{{\mathpalette\irchi\relax}} % beautiful chi
\newcommand{\irchi}[2]{\raisebox{\depth}{$#1\chi$}} % inner command, used by \rchi

\newtcolorbox[auto counter, number within=section]{eser}[1][]{
    colframe=black!0,
    coltitle=black!70, % Title text color
    fonttitle=\bfseries\sffamily, % Title font
    title={Esercizio~\thetcbcounter~#1}, % Title format
    sharp corners, % Less rounded corners
    boxrule=0mm, % Line width of the box frame
    toptitle=1mm, % Distance from top to title
    bottomtitle=1mm, % Distance from title to box content
    colbacktitle=black!5, % Background color of the title bar
    left=0mm, right=0mm, top=1mm, bottom=1mm, % Padding around content
    enhanced, % Enable advanced options
    before skip=10pt, % Space before the box
    after skip=10pt, % Space after the box
    breakable, % Allow box to split across pages
    colback=black!0,
    borderline west={1pt}{-5pt}{black!70}, 
    segmentation style={dashed, draw=black!40, line width=1pt} % Dashed dividing line
}
\newcommand{\seminorm}[1]{\left\lvert\hspace{-1 pt}\left\lvert\hspace{-1 pt}\left\lvert#1\right\lvert\hspace{-1 pt}\right\lvert\hspace{-1 pt}\right\lvert}


\title{Appunti di Geometria 2}
\author{Github Repository:
\href{https://github.com/Oxke/appunti/tree/main/Geo2}{\texttt{Oxke/appunti/Geo2}}}
\date{Secondo semestre, 2024 \-- 2025, prof. Lidia Stoppino e Leone Slavich}


\begin{document}

\begin{titlingpage}
\maketitle

\vspace{1cm}
Il corso è diviso in due parti: geometria differenziale (tenuto da Slavich) e
gruppo fondamentale (tenuto dalla Stoppino). Ci sono esercitazioni di geometria
differenziale.
Ci sono vari libri consigliati:
\begin{itemize}[label = --]
    \item Abate \-- Tovena, \emph{Curve e superfici}, Springer
    \item M. D. Do Carmo, \emph{Differential Geometry of Curves and Surfaces},
        Prentice Hall
    \item E. Sernesi, \emph{Geometria 2}, Bollati Boringhieri
    \item Dispense del prof. Ghigi (sulle quali è stato disegnato il corso,
        almeno per la parte di geometria differenziale)
\end{itemize}

\end{titlingpage}

\chapter{Geometria Differenziale}
Il corso di geometria differenziale studierà curve e superfici in
\(\mathbb{R}^3\) definiti \textbf{analiticamente} tramite funzione
\(C^{\infty}\) \emph{(lisce)}. Studiamo la \textbf{geometria locale} e infine
(verso la fine del corso) la \textbf{geometria globale}, in particolare il
teorema di Gauss-Bonnet.

\section{Definizioni e proprietà iniziali}
\subsection{Funzioni lisce}

Sia \(I = {(a,b)} \subseteq \mathbb{R} \) intervallo aperto (anche possibilmente
\(a = -\infty\) o \(b = +\infty\)). Sia 
\[
    C^{0}{(I)} = \{f : I \to \mathbb{R} : f \text{ è continua\footnote{si veda
    Geometria 1 per definizione e studio della continuità} su }I\}
\]
\begin{definition}{Derivabile}
    Diciamo che \(f \in C^{0}{(I)}\) è derivabile se \(\forall x_{0} \in I\),  
    \[
        \lim_{x \to x_{0}} \frac{f{(x)} - f{(x_{0})}}{x - x_{0}} = c =:
        f'{(x_{0})} \quad ; \quad c \in \mathbb{R}
    \]
\end{definition}
Ora procediamo definendo ricorsivamente le funzioni \(C^{k}{(I)}\).
\begin{definition}{Classe \(C^{k}\) }
    Per ogni \(k\ge 1\), diciamo che \(f \in C^{k}{(I)}\) se \(f\) è derivabile
    e \(f' \in C^{k-1}{(I)}\)
\end{definition}
Dunque, ad esempio \(f \in C^{1}{(I)}\) se \(f\) è derivabile su \(I\) e la sua
derivata \(f'\) è continua su \(I\).
Detto più colloquialmente, una funzione \(f \in C^{k}{(I)}\) è una funzione
derivabile (almeno) \(k\) volte, e tale che la sua derivata \(i\)-esima
\(f^{{(i)}}\) è continua per ogni \(i = 0, \dots, k\).
\begin{remark}
    \[
        C_{0} \supset C^{1} \supset C^{2} \supset \dots \supset C^{i} \supset
        C^{i+1} \supseteq  \dots
    \]
\end{remark}
\begin{definition}{funzioni lisce}
    \[
        C^{\infty}{(I)} = \bigcap_{k=0}^{\infty} C^{k}{(I)} = \text{insieme
        delle \textbf{funzioni lisce}}
    \]
\end{definition}

\begin{theorem}[Proprietà delle classi \(C^{k}\)]\label{thm:proprieta_Ck}
    Sia \(k \in \mathbb{N} \cup \{+\infty\} \). Se \(f, g \in C^{k}{(I)}\) e
    \(\lambda \in \mathbb{R}\), allora
\begin{enumerate}[label = \arabic*.]
    \item \(f + g \in C^{k}{(I)}\) 
    \item \(\lambda f \in C^{k}{(I)}\)
    \item \(f\cdot g \in C^{k}{(I)}\)
\end{enumerate}
\end{theorem}
\begin{proof}
    1.~e 2.~sono semplici. Per 3.~si procede per induzione su \(k\).

    Nel caso base \(k = 0\) il prodotto di due funzioni continue è anch'esso
    una funzione continua, che è vero.

    Supponiamo ora che 3.~valga per \(k-1\). Siano \(f, g \in C^{k}{(I)}\).
    Allora \((f\cdot g)' = f' \cdot g + f \cdot g'\) che è somma di funzioni
    \(C^{k-1}\) per ipotesi induttiva e perché \(C^{k} \subset C^{k-1}\), e
    dunque \((f\cdot g)' \in C^{k-1}\) da cui segue che \(f\cdot g \in C^{k}\).

    Infine possiamo concludere per \(k = +\infty\) perché vale per tutti i \(k
    \in \mathbb{N}\).
\end{proof}

Dal teorema~\ref{thm:proprieta_Ck} segue che \(C^{k}{(I)}\) è uno spazio
vettoriale (con operazione di somma e moltiplicazione per scalare) e inoltre
\(C^{k} {(I)}\) contiene le funzioni costanti e allora \(C^{k}{(i)}\) con
operazioni di somma e moltiplicazione puntuale è un anello. Da queste due segue
che \(C^{k}{(I)}\) è una \(\mathbb{R}\)-algebra.

\begin{example}\label{ex:liscia_non_analitica}
    Esistono funzioni \textbf{lisce} che \textbf{non} sono \textbf{analitiche}.
    In particolare esistono funzioni lisce che sono nulle su un aperto ma non
    nulle dappertutto (differentemente da quanto succede sulle funzioni
    olomorfe). Un esempio di tale funzione è 
    \begin{align*}
        f : \mathbb{R} &\longrightarrow \mathbb{R} \\
        x &\longmapsto f (x) = \begin{cases}
            0 & x \le 0 \\
            e^{-\frac{1}{x^2}} & x > 0
        \end{cases}
    \end{align*}
    
    Questa è una funzione \(C^{\infty}{(\mathbb{R})}\) che non può essere
    analitica perché contraddirebbe il teorema del prolungamento.
    \begin{figure}[ht]
        \centering
        \begin{tikzpicture}
            \begin{axis}[
                xmin= -3, xmax= 3,
                ymin= -1, ymax = 1.5,
                axis lines = middle,
                width = 0.8\textwidth,
            ]
            \addplot[domain=-3:3, samples=100]{exp(-1/x^2) * (x > 0)};
            \end{axis}
        \end{tikzpicture}
        \caption{Grafico della funzione \(f(x)\)
        dell'esempio~\ref{ex:liscia_non_analitica}}\label{fig:liscia_non_analitica} 
    \end{figure}
    Similmente si possono costruire funzioni costanti in aperti, raccordate in
    modo \(C^{\infty}\) e ovviamente non analitiche.
\end{example}
\begin{proposition}[Composizione]\label{prp:composizione_ck}
    La composizione di funzioni \(C^{\infty}\) è \(C^{\infty}\). Sia \(f : I \to
    \mathbb{R}\) e \(g : J \to \mathbb{R}\). Allora se \(f \in C^{\infty}{(I)}\)
    e \(g \in C^{\infty}{(J)}\) e \(f{(I)} \subseteq J \) (ossia si possono
    comporre), allora \(g \circ f : I \to \mathbb{R}\) è ben definita e 
    \[
        g \circ f \in C^{\infty}{(I)}
    \]
\end{proposition}
\begin{proof}
    Lo dimostriamo per \(k \in \mathbb{N}\) invece che \(k = \infty\), segue
    naturalmente il caso enunciato. Per \(k = 0\) è ovvio.

    Supponiamo che valga per \(k-1\). Allora siano \(f, g \in C^{k}\) e
    tali che \(f{(I)} \subseteq J \). Allora \({(g \circ f)}' = {(g' \circ f)}
    \cdot f'\) che è prodotto di funzioni \(C^{k-1}\) per ipotesi induttiva e
    per il teorema~\ref{thm:proprieta_Ck} segue che \(g \circ f \in C^{k}{(I)}\).
\end{proof}

\subsection{Diffeomorfismi}
\begin{definition}{Diffeomorfismo}
    Un diffeomorfismo è un isomorfismo nella categoria delle funzioni lisce (su
    \(\mathbb{R}\) nel nostro caso). 
\end{definition}
Informalmente, un diffeomorfismo è un omeomorfismo \(C^{\infty}\). 
\begin{definition}{Diffeomorfismo}
    Siano \(I, J \subseteq \mathbb{R} \) intervalli aperti in \(\mathbb{R}\).
    Allora \(f : I \to J\) è un \textbf{diffeomorfismo} se
\begin{enumerate}[label = \arabic*.]
    \item \(f \in C^{\infty}{(I)}\) 
    \item \(f\) è biettiva
    \item \(f^{-1} \in C^{\infty}{(J)}\)
\end{enumerate}
\end{definition}
\begin{remark}
    La terza condizione \textbf{non} è ridondante. Infatti sia \(I = J =
    \mathbb{R}\) e \(f{(x)} = x^{3}\) che è chiaramente \(C^{\infty}\) e
    biunivoca. Tuttavia \(f^{-1}{(x)} = \sqrt[3]{x}\) non è derivabile in \(0\),
poiché \(f'(0) = 0\) e \({f^{-1}}'(y) = \frac{1}{f'{(x)}}\) se \(f{(x)} = y\)
per ogni \(x \in \mathbb{R}\) tale che \({f^{-1}}'{(y)}\) sia ben definita.
\end{remark}
\begin{remark}
    Se \(I\) e \(J\) sono intervalli aperti di \(\mathbb{R}\) e \(f : I\to J\) è
    diffeomorfismo, allora \(f'{(x)} \neq 0\) per ogni \(x \in I\). Infatti
    sappiamo che
    \begin{equation}\label{eq:derivata_inversa}
        {f^{-1}}'(y) = \frac{1}{f'{(x)}} \quad ; \quad f{(x)} = y \quad
        \forall x \in J 
    \end{equation}
    dunque \(f'{(x)}\) non può essere nullo, poiché significherebbe che
    \(f^{-1}\) non è derivabile in \(y = f^{-1}{(x)}\).
\end{remark}
\begin{lemma}
    Sia \(I \subseteq \mathbb{R}\) un intervallo aperto. Sia \(f : I \to
    \mathbb{R}\) una funzione liscia e tale che \(f'{(x)} \neq 0\) per ogni \(x
    \in I\). Allora \(f{(I)} = J\) è un intervallo aperto e \(f : I \to J\) è un
    diffeomorfismo.
\end{lemma}
\begin{proof}
    Sia \(f\) come nell'enunciato. Allora \(f' : I \to \mathbb{R}\) è continua
    su \(I\) e non si annulla mai. Segue che \(f'\) ha segno costante su \(I\)
    (\(f' > 0\) oppure \(f' < 0\)).

    Assumiamo \(f' > 0\) su \(I\). Allora \(f\) è strettamente crescente e
    dunque iniettiva. Allora \(f : I \to f{(I)} =: J\) è biettiva. Inoltre \(J\)
    è un intervallo aperto in \(\mathbb{R}\). Infatti è chiaramente intervallo
    (è connesso) in quanto immagine continua di un connesso (intervallo).  Sia
    ora \(y_{0} \in J\) e sia \(x_{0} \in I\) tale che \(f{(x_{0})} = y_{0}\).
    Sia \(\varepsilon > 0\) tale che \([x_{0} - \varepsilon, x_{0} +
    \varepsilon] \subseteq I \). Poiché \(f\) è strettamente crescente, segue
    che 
    \[
        f{(x_{0}-\varepsilon)} < f{(x_{0})} < f{(x_{0}+\varepsilon)}
    \]
    sapendo già che \(J\) è un intervallo, segue che \[I\supseteq
    [f{(x_{0}-\varepsilon)}, f{(x_{0} + \varepsilon)}] \text{ è un intorno di
\(y_{0}\)}\]
    Rimane solo da vedere che la funzione \(f^{-1} : J \to I\) è
    \(C^{\infty}\). Notiamo intento che \(f^{-1}\) è continua, poiché \(f\) è aperta.
    Inoltre sappiamo che \(f^{-1}\) è derivabile poiché è l'inversa di una
    funzione derivabile con derivata e vale
    l'equazione~\eqref{eq:derivata_inversa} per ogni \(y \in J\).

    Sia \(u : {(0, +\infty)} \to {(0, +\infty)}\) definita da \(u{(x)} = 1/x\) è
    un diffeomorfismo e da~\ref{eq:derivata_inversa} abbiamo che
    \[
        {f^{-1}}' = u \circ f' \circ f^{-1}
    \]
    e quindi se assumiamo induttivamente che se \(f^{-1} \in C^{k}\) allora ne
    consegue dalla proposizione~\ref{prp:composizione_ck} che \({f^{-1}}' \in
    C^{k}\) e dunque \(f^{-1} \in C^{k+1}\) 

\end{proof}

\subsection{Curve}
\begin{definition}{Curva parametrica}
    Sia \(\alpha : I \to \mathbb{R}^{3}\) una funzione 
    \(t \mapsto (\alpha_{1}(t), \alpha_{2}(t), \alpha_{3}(t)) \) con \(I\)
    intervallo aperto.
    
    Allora se \(\alpha_{1}, \alpha_{2}, \alpha_{3}\) sono funzioni lisce la
    funzione \(\alpha\) è detta \textbf{curva} parametrizzata in \(\mathbb{R}^3\) 
\end{definition}
\end{document}

