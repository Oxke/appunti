\documentclass{article}
\usepackage{layout}
\usepackage[a4paper, total={5in,9in}]{geometry}
\usepackage[T1]{fontenc}
\usepackage[italian]{babel}
\usepackage{mathtools}
\usepackage{amsthm}
\usepackage[framemethod=TikZ]{mdframed}
\usepackage{amsmath}
\usepackage{amssymb}
\usepackage{cancel}
\usepackage[dvipsnames]{xcolor}
\usepackage{tikz}
\usepackage{tikz-cd}
\usepackage{pgfplots}
\pgfplotsset{compat=1.18}
\usepackage[many]{tcolorbox}
\usepackage{import}
\usepackage{pdfpages}
\usepackage{transparent}
\usepackage{enumitem}
\usepackage[colorlinks]{hyperref}

\newcommand*{\sminus}{\raisebox{1.3pt}{$\smallsetminus$}}

\newcommand*{\transp}[2][-3mu]{\ensuremath{\mskip1mu\prescript{\smash{\mathrm t\mkern#1}}{}{\mathstrut#2}}}%

% newcommand for span with langle and rangle around
\newcommand{\Span}[1]{{\left\langle#1\right\rangle}}

\newcommand{\incfig}[2][1]{%
    \def\svgwidth{#1\columnwidth}
    \import{./figures/}{#2.pdf_tex}
}

\pdfsuppresswarningpagegroup=1

\newcounter{theo}[section]\setcounter{theo}{0}
\renewcommand{\thetheo}{\arabic{section}.\arabic{theo}}

\newcounter{excounter}[section]\setcounter{excounter}{0}
\renewcommand{\theexcounter}{\arabic{section}.\arabic{excounter}}

\numberwithin{equation}{section}

\newenvironment{theorem}[1][]{
    \refstepcounter{theo}
     \ifstrempty{#1}
    {\mdfsetup{
        frametitle={
            \tikz[baseline=(current bounding box.east),outer sep=0pt]
            \node[anchor=east,rectangle,fill=blue!20,rounded corners=5pt]
            {\strut Teorema~\thetheo};}
        }
    }{\mdfsetup{
        frametitle={
            \tikz[baseline=(current bounding box.east),outer sep=0pt]
            \node[anchor=east,rectangle,fill=blue!20,rounded corners=5pt]
            {\strut Teorema~\thetheo:~#1};}
        }
    }
    \mdfsetup{
        roundcorner=10pt,
        innertopmargin=10pt,linecolor=blue!20,
        linewidth=2pt,topline=true,
        frametitleaboveskip=\dimexpr-\ht\strutbox\relax,
        % nobreak=false
    }
\begin{mdframed}[]\relax}{
\end{mdframed}}

% \newenvironment{definition}[1][]{
%     \refstepcounter{theo}
%      \ifstrempty{#1}
%     {\mdfsetup{
%         frametitle={
%             \tikz[baseline=(current bounding box.east),outer sep=0pt]
%             \node[anchor=east,rectangle,fill=violet!20,rounded corners=5pt]
%             {\strut Definizione~\thetheo};}
%         }
%     }{\mdfsetup{
%         frametitle={
%             \tikz[baseline=(current bounding box.east),outer sep=0pt]
%             \node[anchor=east,rectangle,fill=violet!20,rounded corners=5pt]
%             {\strut Definizione~\thetheo:~#1};}
%         }
%     }
%     \mdfsetup{
%         roundcorner=10pt,
%         innertopmargin=10pt,linecolor=violet!20,
%         linewidth=2pt,topline=true,
%         frametitleaboveskip=\dimexpr-\ht\strutbox\relax,
%         nobreak=true
%     }
% \begin{mdframed}[]\relax}{
% \end{mdframed}}

\newtcolorbox[auto counter, number within=section]{definition}[2][]{
    colframe=violet!0,
    coltitle=violet, % Title text color
    fonttitle=\bfseries, % Title font
    title={Definizione~\thetcbcounter\ifstrempty{#2}{}{:~#2}}, % Title format
    sharp corners, % Less rounded corners
    boxrule=0pt, % Line width of the box frame
    toptitle=1mm, % Distance from top to title
    bottomtitle=1mm, % Distance from title to box content
    colbacktitle=violet!5, % Background color of the title bar
    left=0mm, right=0mm, top=1mm, bottom=1mm, % Padding around content
    enhanced, % Enable advanced options
    before skip=10pt, % Space before the box
    after skip=10pt, % Space after the box
    breakable, % Allow box to split across pages
    colback=violet!0,
    borderline west={2pt}{-5pt}{violet!40},
    #1
}

\newenvironment{lemmao}[1][]{
    \refstepcounter{theo}
     \ifstrempty{#1}
    {\mdfsetup{
        frametitle={
            \tikz[baseline=(current bounding box.east),outer sep=0pt]
            \node[anchor=east,rectangle,fill=green!20,rounded corners=5pt]
            {\strut Lemma~\thetheo};}
        }
    }{\mdfsetup{
        frametitle={
            \tikz[baseline=(current bounding box.east),outer sep=0pt]
            \node[anchor=east,rectangle,fill=green!20,rounded corners=5pt]
            {\strut Lemma~\thetheo:~#1};}
        }
    }
    \mdfsetup{
        roundcorner=10pt,
        innertopmargin=10pt,linecolor=green!20,
        linewidth=2pt,topline=true,
        frametitleaboveskip=\dimexpr-\ht\strutbox\relax,
        % nobreak=true
    }
\begin{mdframed}[]\relax}{
\end{mdframed}}

\theoremstyle{plain}
\newtheorem{lemma}[theo]{Lemma}
\newtheorem{corollary}{Corollario}[theo]
\newtheorem{proposition}[theo]{Proposizione}

\theoremstyle{definition}
\newtheorem{example}[excounter]{Esempio}

\theoremstyle{remark}
\newtheorem*{note}{Nota}
\newtheorem*{remark}{Osservazione}

\newtcolorbox{notebox}{
  colback=gray!10,
  colframe=black,
  arc=5pt,
  boxrule=1pt,
  left=15pt,
  right=15pt,
  top=15pt,
  bottom=15pt,
}

\DeclareRobustCommand{\rchi}{{\mathpalette\irchi\relax}} % beautiful chi
\newcommand{\irchi}[2]{\raisebox{\depth}{$#1\chi$}} % inner command, used by \rchi

\newtcolorbox[auto counter, number within=section]{eser}[1][]{
    colframe=black!0,
    coltitle=black!70, % Title text color
    fonttitle=\bfseries\sffamily, % Title font
    title={Esercizio~\thetcbcounter~#1}, % Title format
    sharp corners, % Less rounded corners
    boxrule=0mm, % Line width of the box frame
    toptitle=1mm, % Distance from top to title
    bottomtitle=1mm, % Distance from title to box content
    colbacktitle=black!5, % Background color of the title bar
    left=0mm, right=0mm, top=1mm, bottom=1mm, % Padding around content
    enhanced, % Enable advanced options
    before skip=10pt, % Space before the box
    after skip=10pt, % Space after the box
    breakable, % Allow box to split across pages
    colback=black!0,
    borderline west={1pt}{-5pt}{black!70}, 
    segmentation style={dashed, draw=black!40, line width=1pt} % Dashed dividing line
}
\newcommand{\seminorm}[1]{\left\lvert\hspace{-1 pt}\left\lvert\hspace{-1 pt}\left\lvert#1\right\lvert\hspace{-1 pt}\right\lvert\hspace{-1 pt}\right\lvert}


\title{appunti di Fondamntenti della Matematica}
\author{Github Repository:
\href{https://github.com/Oxke/appunti/tree/main/FondamentiMatematica}{\texttt{Oxke/appunti/FondamentiMatematica}}}

\date{Secondo semestre, 2024 \-- 2025, prof. Rosso}

\begin{document}

\maketitle

\section{Geometria di Hilbert}

\begin{enumerate}[label = I\arabic*.]
    \item Per due punti \(A, B\) esiste sempre una retta che li contiene
        entrambi.
    \item Per due punti \(A, B\) esiste al più una retta che li contiene
        entrambi.
    \item Su una retta esistono almeno due punti. Esistono almeno tre punti che
        non appartengono alla stessa retta
    \item Per tre punti \(A, B, C\) che non appartengono ad una stessa retta c'è
        sempre un piano \(\alpha\) che li contiene. Per ogni piano esiste un
        punto che gli appartiene.
    \item Per tre punti \(A, B, C\) che non appartengono ad una stessa retta c'è
        al più un piano che li contiene
    \item Se \(A, B, C\) appartengono alla retta \(r\) e \(A, B\) appartengono
        ad un piano \(\alpha\) allora \(r\) è interamente contenuta in \(\alpha\) 
    \item Se due piani \(\alpha\) e \(\beta\) hanno un punto \(A\) in comune
        allora essi hanno almeno un altro punto \(B\) in comune.
    \item Esistono almeno quattro punti che non appartengono ad uno stesso
        piano.
\end{enumerate}

Ne consegue che due rette o hanno un punto in comune o non ne hanno affatto. Due
piani o non hanno punti in comune oppure hanno una retta in comune e nessun
punto al di fuori di essa.

Dati un piano ed una retta che non appartenga al piano, essi o non hanno punti
in comune o ne hanno uno.

Possiamo costruire due sottoinsiemi minimali degli assiomi precedenti 
\subsection{Geometria astratta}
Consideriamo un insieme \(\mathcal{P}\) di punti e un insieme \(\mathcal{R}\) di rette
\begin{enumerate}[label = \arabic*.]
    \item \(\forall \) coppia di punti \(A, B \in \mathcal{P}\) esiste \(r \in
        \mathcal{R}\) tale che \(A \in r\) e \(B \in r\) 
    \item Ogni \(r \in \mathcal{R}\) contiene almeno due punti
\end{enumerate}
Vogliamo ora costruire un modello di tale geometria.
\begin{example}[Geometria sferica]
    Sia \(\mathcal{P} = \{{(x,y,z)} \in \mathbb{R}^3 : x^2 + y^2 +z^2 = 1\} =
    S^2 \) e come rette prendiamo i cerchi massimi, quindi le intersezioni tra
    la sfera e i piani passanti per l'origine: \(r \in \mathcal{R} \iff r =
    \{{(x, y, z)} \in S^2 \} : ax + by : cz = 0 \) per qualche \(a, b, c \in \mathbb{R}\) 

    Quindi non c'è l'unicità della retta passante per due punti, infatti se si
    prendono due punti antipodali infinite rette li contengono
\end{example}
\subsection{Geometria di incidenza}
Una geometria astratta è detta \textbf{geometria di incidenza} se 
\begin{enumerate}[label = \arabic*.]
    \item \(\forall A, B \in \mathcal{P}, \quad \exists ! r \in \mathcal{R}: A,
        B \in r \) 
    \item \(\exists\)ono almeno tre punti \(A, B , C \in \mathcal{P}\) non
        appartenenti ad una stessa retta
\end{enumerate}

\begin{example}[Geometria euclidea]
    Sia \(\mathcal{P}= R^2\) e le rette di forma \(ax + by + q = 0\) per \(a, b,
    q \in \mathbb{R}\) non tutti nulli.
\end{example}

\begin{example}[Geometria iperbolica - semipiano di Poincaré]
    Consideriamo ora \(\mathcal{P} = \{{(x,y)} \in \mathbb{R}^2 : y > 0\} \) e
    \(\mathcal{R} = \mathcal{R}_a \cup \mathcal{R}_{c, r} \) con \(\mathcal{R}_a
    = \{{(x,y)} \in \mathbb{R}^2 : x = a \land y > 0\} \) e \(\mathcal{R}_{c, r}
    = \{{(x,y)} \in R^2 : y > 0 \land {(x-c)}^2 + y^2 = r^2\} \) 

    Quindi in pratica le rette sono o le rette verticali oppure le
    semicirconferenze con centro sull'asse delle ascisse.
\end{example}

\begin{example}
Proviamo a costruire ora un modello finito. Sia \(\mathcal{P}_3 = \{A, B, C\} \),
e \(\mathcal{R}_3 = \{\{A, B\} , \{A, C\} , \{B, C\} \} \). 
Ogni coppia di rette ha intersezione non vuota.

Similmente \(\mathcal{P}_4 = \{A, B, C, D\} \) e \\ \(\mathcal{R}_4 = \{\{A, B\} ,
\{A, C\} , \{A, D\} , \{B, C\} , \{B, D\} , \{C, D\} \} \). 
Esistono coppie di rette che non hanno intersezione, ad esempio \(\{A, B\} \cap
\{C, D\} = \emptyset\), ma ogni retta ha una e solo una retta parallela.

Infine \(\mathcal{P}_5 = \{A, B, C, D\} \) e \(\mathcal{R}_5 =
\binom{\mathcal{P}_5}{2} \) ossia ogni possibile coppia di punti. In tal caso
\(\{A, B\}\) è parallela sia a \(\{C, D\} \) che a \(\{C, E\} \).
\end{example}

Una geometria di incidenza soddisfa la proprietà \textbf{euclidea} (o
parabolica) delle parallele se
presa \(r \in \mathcal{R}\) e \(p \not\in r\) allora 
\[
  \exists  ! s \in \mathcal{R} : P \in s \land r \cap s = \emptyset \quad
  (\text{ es  } \mathcal{P}_3, \mathcal{R}_3)
\]

Una geometria di incidenza soddisfa la proprietà \textbf{ellittica} delle
parallele se \(r \in s\) e \(p \not\in  r\) allora
\[
  \not \exists s \in \mathcal{R} : P \in s \land r \cap s = \emptyset \quad
    (\text{ es  } \mathcal{P}_4, \mathcal{R}_4)
\]

Una geometria di incidenza soddisfa la proprietà \textbf{iperbolica} delle
parallele se \(r \in s\) e \(p \not\in  r\) allora
\[
  \exists \text{ almeno due } s_{1}, s_{2} \in \mathcal{R} : P \in s_{1}, s_{2}
  \land r \cap s = \emptyset = r \cap s_{2} \quad (\text{ es  }\mathcal{P}_5, 
    \mathcal{R}_5)
\]

\begin{eser}
    Mostrare la validità degli assiomi negli esempi proposti
\end{eser}

Per arrivare a \textbf{piano proiettivo} aggiungiamo due assiomi:
\begin{enumerate}[label = \Roman*.]
    \item Unicità retta congiungente due punti
    \item Ogni retta contiene almeno tre punti
\end{enumerate}

Il più piccolo piano proiettivo è il piano di Fano, contenente 7 punti:
\(\mathcal{P} = \{A, B, C, D, E, F, G\} \)  
e 7 rette:
\[\mathcal{R} = \{\{A, B, C\}, \{A, D, E\}, \{A, F, G\}, \{B, D, F\}, \{B, E, G\},
\{C, D, G\}, \{C, E, F\} \} \]

\section{Assiomi di ordinamento}
\begin{proposition}[Pons Asinorum]
    Sia \(ABC\) un triangolo. Se \(AB = AC\) allora \(\hat{B} = \hat{C}\) 
\end{proposition}
\begin{proof}
    Traccio la bisettrice di \(\hat{A}\) che ha piede in \(D\) e determino i due
    triangoli  \(ABD\) e \(ACD\). Hanno due angoli e un lato in comune, quindi
    sono congruenti. Di conseguenza \(\hat{B} = \hat{C}\) in quanto elementi
    corrispondenti di triangoli congruenti.
\end{proof}

Il problema della precedente dimostrazione (eccetto il fatto che non abbiamo
ancora presentato i criteri di congruenza) è che non sappiamo necessariamente
che \(D\) è compreso tra \(B\) e \(C\). Questo motiva gli assiomi di
ordinamento. Useremo la notazione \(A - B - C\) per indicare che \(B\) sta tra
\(A\) e \(C\) (ed è sulla stessa retta di \(A\) e \(C\).

\begin{enumerate}[label = O\arabic*.]
    \item Se \(A - B - C\) allora \(A, B, C\) sono allineati e \(C - B - A\) 
    \item Dati due punti \(B\) e \(D\) esistono due punti \(A, C, E\) su
        \(\overline{BD}\) tali che \(A - B - D\), \(B - C - D\) e \(B - D - E\) 

        In pratica se ho \( B - D\) allora \( A - B - C - D - E\) e posso quindi
        prolungare il segmento \(BD\) sia a sinistra che a destra, e posso
        trovare un punto \(C\) nel mezzo del segmento.
    \item Dati \(A, B, C\) appartenenti ad una stessa retta, allora esiste un
        unico punto che sta tra gli altri due. In altre parola esattamente una
        tra \(A - B - C\), \(A - C - B\), \(B - C - A\) è vera.
\end{enumerate}

Ma cos'è il segmento \(AB\)? È definito come 
\begin{definition}{Segmento e semirette}
    Dati due punti \(A, B \in \mathcal{P}\), 
    \(AB = \{A, B\} \cup \{C : A - C - B\} \)  

    \(\vec{AB} = AB \cup \{C : A - B - C\} \) 

    \(\vec{BA} = AB \cup \{C : C - A - B\} \) 
\end{definition}

Ne consegue che \(\vec{AB} \cap \vec{BA} = AB\) e \(\vec{AB} \cup \vec{BA} =
\overline{AB}\) 




\end{document}


