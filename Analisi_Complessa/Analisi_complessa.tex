\documentclass{article}
\usepackage{layout}
\usepackage[a4paper, total={5in,9in}]{geometry}
\usepackage[T1]{fontenc}
\usepackage[italian]{babel}
\usepackage{mathtools}
\usepackage{amsthm}
\usepackage[framemethod=TikZ]{mdframed}
\usepackage{amsmath}
\usepackage{amssymb}
\usepackage{cancel}
\usepackage[dvipsnames]{xcolor}
\usepackage{tikz}
\usepackage{tikz-cd}
\usepackage{pgfplots}
\pgfplotsset{compat=1.18}
\usepackage[many]{tcolorbox}
\usepackage{import}
\usepackage{pdfpages}
\usepackage{transparent}
\usepackage{enumitem}
\usepackage[colorlinks]{hyperref}

\newcommand*{\sminus}{\raisebox{1.3pt}{$\smallsetminus$}}

\newcommand*{\transp}[2][-3mu]{\ensuremath{\mskip1mu\prescript{\smash{\mathrm t\mkern#1}}{}{\mathstrut#2}}}%

% newcommand for span with langle and rangle around
\newcommand{\Span}[1]{{\left\langle#1\right\rangle}}

\newcommand{\incfig}[2][1]{%
    \def\svgwidth{#1\columnwidth}
    \import{./figures/}{#2.pdf_tex}
}

\pdfsuppresswarningpagegroup=1

\newcounter{theo}[section]\setcounter{theo}{0}
\renewcommand{\thetheo}{\arabic{section}.\arabic{theo}}

\newcounter{excounter}[section]\setcounter{excounter}{0}
\renewcommand{\theexcounter}{\arabic{section}.\arabic{excounter}}

\numberwithin{equation}{section}

\newenvironment{theorem}[1][]{
    \refstepcounter{theo}
     \ifstrempty{#1}
    {\mdfsetup{
        frametitle={
            \tikz[baseline=(current bounding box.east),outer sep=0pt]
            \node[anchor=east,rectangle,fill=blue!20,rounded corners=5pt]
            {\strut Teorema~\thetheo};}
        }
    }{\mdfsetup{
        frametitle={
            \tikz[baseline=(current bounding box.east),outer sep=0pt]
            \node[anchor=east,rectangle,fill=blue!20,rounded corners=5pt]
            {\strut Teorema~\thetheo:~#1};}
        }
    }
    \mdfsetup{
        roundcorner=10pt,
        innertopmargin=10pt,linecolor=blue!20,
        linewidth=2pt,topline=true,
        frametitleaboveskip=\dimexpr-\ht\strutbox\relax,
        % nobreak=false
    }
\begin{mdframed}[]\relax}{
\end{mdframed}}

% \newenvironment{definition}[1][]{
%     \refstepcounter{theo}
%      \ifstrempty{#1}
%     {\mdfsetup{
%         frametitle={
%             \tikz[baseline=(current bounding box.east),outer sep=0pt]
%             \node[anchor=east,rectangle,fill=violet!20,rounded corners=5pt]
%             {\strut Definizione~\thetheo};}
%         }
%     }{\mdfsetup{
%         frametitle={
%             \tikz[baseline=(current bounding box.east),outer sep=0pt]
%             \node[anchor=east,rectangle,fill=violet!20,rounded corners=5pt]
%             {\strut Definizione~\thetheo:~#1};}
%         }
%     }
%     \mdfsetup{
%         roundcorner=10pt,
%         innertopmargin=10pt,linecolor=violet!20,
%         linewidth=2pt,topline=true,
%         frametitleaboveskip=\dimexpr-\ht\strutbox\relax,
%         nobreak=true
%     }
% \begin{mdframed}[]\relax}{
% \end{mdframed}}

\newtcolorbox[auto counter, number within=section]{definition}[2][]{
    colframe=violet!0,
    coltitle=violet, % Title text color
    fonttitle=\bfseries, % Title font
    title={Definizione~\thetcbcounter\ifstrempty{#2}{}{:~#2}}, % Title format
    sharp corners, % Less rounded corners
    boxrule=0pt, % Line width of the box frame
    toptitle=1mm, % Distance from top to title
    bottomtitle=1mm, % Distance from title to box content
    colbacktitle=violet!5, % Background color of the title bar
    left=0mm, right=0mm, top=1mm, bottom=1mm, % Padding around content
    enhanced, % Enable advanced options
    before skip=10pt, % Space before the box
    after skip=10pt, % Space after the box
    breakable, % Allow box to split across pages
    colback=violet!0,
    borderline west={2pt}{-5pt}{violet!40},
    #1
}

\newenvironment{lemmao}[1][]{
    \refstepcounter{theo}
     \ifstrempty{#1}
    {\mdfsetup{
        frametitle={
            \tikz[baseline=(current bounding box.east),outer sep=0pt]
            \node[anchor=east,rectangle,fill=green!20,rounded corners=5pt]
            {\strut Lemma~\thetheo};}
        }
    }{\mdfsetup{
        frametitle={
            \tikz[baseline=(current bounding box.east),outer sep=0pt]
            \node[anchor=east,rectangle,fill=green!20,rounded corners=5pt]
            {\strut Lemma~\thetheo:~#1};}
        }
    }
    \mdfsetup{
        roundcorner=10pt,
        innertopmargin=10pt,linecolor=green!20,
        linewidth=2pt,topline=true,
        frametitleaboveskip=\dimexpr-\ht\strutbox\relax,
        % nobreak=true
    }
\begin{mdframed}[]\relax}{
\end{mdframed}}

\theoremstyle{plain}
\newtheorem{lemma}[theo]{Lemma}
\newtheorem{corollary}{Corollario}[theo]
\newtheorem{proposition}[theo]{Proposizione}

\theoremstyle{definition}
\newtheorem{example}[excounter]{Esempio}

\theoremstyle{remark}
\newtheorem*{note}{Nota}
\newtheorem*{remark}{Osservazione}

\newtcolorbox{notebox}{
  colback=gray!10,
  colframe=black,
  arc=5pt,
  boxrule=1pt,
  left=15pt,
  right=15pt,
  top=15pt,
  bottom=15pt,
}

\DeclareRobustCommand{\rchi}{{\mathpalette\irchi\relax}} % beautiful chi
\newcommand{\irchi}[2]{\raisebox{\depth}{$#1\chi$}} % inner command, used by \rchi

\newtcolorbox[auto counter, number within=section]{eser}[1][]{
    colframe=black!0,
    coltitle=black!70, % Title text color
    fonttitle=\bfseries\sffamily, % Title font
    title={Esercizio~\thetcbcounter~#1}, % Title format
    sharp corners, % Less rounded corners
    boxrule=0mm, % Line width of the box frame
    toptitle=1mm, % Distance from top to title
    bottomtitle=1mm, % Distance from title to box content
    colbacktitle=black!5, % Background color of the title bar
    left=0mm, right=0mm, top=1mm, bottom=1mm, % Padding around content
    enhanced, % Enable advanced options
    before skip=10pt, % Space before the box
    after skip=10pt, % Space after the box
    breakable, % Allow box to split across pages
    colback=black!0,
    borderline west={1pt}{-5pt}{black!70}, 
    segmentation style={dashed, draw=black!40, line width=1pt} % Dashed dividing line
}
\newcommand{\seminorm}[1]{\left\lvert\hspace{-1 pt}\left\lvert\hspace{-1 pt}\left\lvert#1\right\lvert\hspace{-1 pt}\right\lvert\hspace{-1 pt}\right\lvert}


\title{Appunti di Analisi 3 \-- Analisi Complessa}
\author{Osea}
\date{Primo semestre A.A. 2024 \-- 2025, prof. Enrico Vitali}

\begin{document}

\maketitle

\section{Convergenza puntuale e uniforme}
Sia \(E\) un'insieme (non vuoto) e \(\{f_{n}\} \) una successione di funzioni
\(E \to \mathbb{R}\) (o \(E \to \mathbb{R}^{n}\) o \(E \to \mathbb{C}\)). Sia
\(f: E \to \mathbb{R}\).
\begin{definition}{}
    Diciamo che \(\{f_{n}\} \) converge \textbf{puntualmente} ad \(f\) se 
    \[
        \lim_{n \to \infty} f_{n}(x) = f(x) \quad \forall x \in E
    \]
\end{definition}
\begin{example}
    \(E = \mathbb{R}\) e \(f_{n}(x) = \frac{1}{n + x^2}\), \(f_{n} \to 0\) su
    \(\mathbb{R}\) 
\end{example}
\begin{example}
\(f_{n}(x) = {\left( x - \frac{1}{n} \right)}^2 \to x^2\)
\end{example}
\begin{example}
    \(f_{n}(x) = x^2 - \frac{1}{n}\) 
\end{example}
\begin{example}
    \(f_{n}(x) = e^{x - n}\) 
    \(f_{n} \to 0\) 
\end{example}

\begin{example}
    \(E = [0, 1]\), \(f_{n}(x) \) funzione che è a triangolo con vertici
    \((\frac{1}{4n}, 0), (\frac{1}{2n}, 1)\), \( (\frac{1}{n}, 0)\). Allora
    \(f_{n} \to 0\)
\end{example}

In questi esempi l'idea è che per ogni \(\varepsilon\) esiste un
\(n_\varepsilon\) tale che per \(n\ge n_\varepsilon\), \(f_{n}(x) <
\varepsilon\). La domanda è se si riesce a esprimere \(n_\varepsilon\) senza che
dipenda da \(x\).
Nell'esempio di \(f_{n}(x) = \frac{1}{n+x^2}\) si può perché \(f_{n}\) ha un
massimo in \(x = 0\), in tal caso infatti se prendo \(n_\varepsilon \) tale che
\(\frac{1}{n_\varepsilon} < \varepsilon\) allora \(\frac{1}{n+x^2}\le
\frac{1}{n} \le \frac{1}{n_\varepsilon} < \varepsilon\).

Nell'esempio 1.2 invece vogliamo un \(n_\varepsilon\) tale che \(\forall n\ge
n_\varepsilon\), \(|f_{n}(x) - f(x)| \le \varepsilon\) ossia \(|-\frac{2}{n} x +
\frac{1}{n^2} | \le  \varepsilon\). Da questo troviamo che
\[
    \frac{1}{n^2} - \varepsilon \le \frac{2x}{n} \le \frac{1}{n^2} + \varepsilon
\]
Ma è sempre possibile, per qualsiasi \(\frac{1}{n^2} + \varepsilon\) è
possibile trovare un \(x\) tale che sia maggiore, quindi non è possibile non
esprimere \(n_\varepsilon\) anche in funzione di \(x\).

\begin{definition}{}
    Sia \(f\), \(f_{n} : E \to \mathbb{R}\). Diciamo che \(f_{n} \to f\)
    uniformemente in \(E\) se:
    \[
        \forall \varepsilon > 0 \quad \exists n_\varepsilon \in \mathbb{N} :
        \forall n \ge n_\varepsilon, \forall x \in E, \quad |f_{n}(x) - f(x)| <
        \varepsilon
    \]
\end{definition}
\begin{remark}
    La condizione della definizione di convergenza uniforme è equivalente a
    richiedere che \(\sup_{x \in E} |f_n(x) - f(x)| < \varepsilon\). Da questo
    concludiamo che \(f_{n} \to f\) uniformemente se e solo se
    \[
        \lim_{n \to \infty} \sup_{x \in E} |f_{n}(x) - f(x)| <\varepsilon
    \]
\end{remark}
Allora con questa nuova osservazione è facile notare la non convergenza uniforme
dell'esempio 1.2. Infatti se \(f_{n}(x) = {(x-\frac{1}{n})}^2\) e \(f(x) = x^2\)
allora \(\sup_{x \in \mathbb{R}} |f_{n}(x) - f(x)| \ge |f_n(n) - f(n)| = |2 -
\frac{1}{n}| \to 2 > 0\).

Abbiamo però che converge uniformemente sugli insiemi limitati (esercizio).
Similmente nell'esempio 1.4 \(f_{n}\) converge uniformemente sugli insiemi
\((-\infty, a]\) infatti \(0 \le f_{n}(x) \le e^{a-n} \to 0\) per \(n \to +\infty\) 

Geometricamente la convergenza uniforme dice che il grafico di \(f_{n}\) è
contenuta in un intorno tubolare arbitrario di \(f\) per \(n\) sufficientemente
grande.

\begin{proposition}[Criterio di Cauchy / completezza di \(\mathbb{R}\) ]
    Se \(\{a_{n}\} \) è una successione di numeri reali si ha:
    \(a_{n}\) converge se e solo se \(a_{n}\) è una successione di Cauchy, ossia
    se \(\forall \varepsilon > 0 \exists n_\varepsilon\) tale che \(\forall
    n_{1},n_{2} \ge n_\varepsilon\), \(|a_{n_{1}} - a_{n2}| <\varepsilon\) 
\end{proposition}

\begin{theorem}[Criterio di Cauchy per la convergenza uniforme]
    Siano \(f, f_{n} : E \to \mathbb{R}\), con \(f_{n} \to f\) in \(E\). Allora la
    convergenza è uniforme in \(E\) se e solo se 
    \[
        \forall \varepsilon>0 \quad \exists n_\varepsilon \in \mathbb{N} : \forall n,m
        \ge n_\varepsilon \text{ e } \forall x \in E, \quad |f_{n}(x) -
        f_{m}(x)| < \varepsilon
    \]
\end{theorem}
\begin{proof}\( \)
\begin{itemize}
    \item[\(\implies \)] Sia \(f_{n} \to f\) uniformemente in \(E\). Fissato
        \(\varepsilon>0\), sia \(n_\varepsilon\) tale che (convergenza uniforme)
        \(\forall k \ge n_\varepsilon\) e \(\forall x \in E\), \(|f_k(x) - f(x)|
        < \frac{\varepsilon}{2}\) allora presi \(n,m \ge n_\varepsilon\) ho
        che \(|f_n(x) - f_m(x)| \le |f_n(x) - f(x)| + |f_m(x) - f(x)| <
        \frac{\varepsilon}{2} + \frac{\varepsilon}{2} = \varepsilon\) 
    \item[\(\impliedby \)] Valga la condizione di Cauchy. Allora \(\forall x \in
        E\) la successione \(\{f_{n}(x)\} \) è una successione di Cauchy, quindi
        è convergente, quindi \(\exists f : E \to \mathbb{R}\) tale che
        \(f_{n} \to f\). Allora dalla condizione di Cauchy, tenendo \(n\) fisso
        e facendo tendere \(n \to +\infty\) si ottiene esattamente la
        convergenza uniforme.
\end{itemize}
\end{proof}

Fun fact: esistono dei cosiddetti ``Spazi uniformi'', che sono spazi topologici
ma non metrici.

\begin{example}
    Sia \(f_{n} = \frac{n^2 - x}{n^3 + e^{nx}}\). È evidente per \(x \in
    \mathbb{R}\) che \(f_{n}(x) \to 0\).

    C'è convergenza uniforme sui limitati, infatti se \(|x| \le M\) allora
    \(|f_{n}(x)| \le \frac{n^2 + M}{n^3} \to 0\). Consideriamo ora \(x \ge  0\)
    (esercizio). Invece per \(x \le 0\), posso prendere per ogni \(n\) \(x_{n} =
    -n^{4}\) e allora ottengo che \(f_{n}(x_{n}) \to +\infty\) 
\end{example}

\begin{remark}
    Sia \(f_{n}: [a, b) \to  \mathbb{R}\) continua, suppongo che \(\{f_{n}\} \)
    converga uniformemente a \(f\) in \((a, b)\). Allora converge uniformemente
    in \([a, b)\) 
\end{remark}
\begin{proof}
    Per Cauchy
    \[
        \forall \varepsilon > 0 \quad \exists n_\varepsilon \in \mathbb{N} :
        \forall  n, m \ge n_\varepsilon, \forall x \in (a, b), \quad |f_{n}(x) -
        f_{m}(x)| \le \varepsilon
    \]

    Per \(x \to  a\) abbiamo per continuità che \(|f_{n}(a) - f_{m}(a)| \le
    \varepsilon\) per \(n, m \ge \overline{n} \in \mathbb{N}\), quindi preso
    \(\tilde{n} = \max{n_\varepsilon, \overline{n}}\) si ha che \(f_{n}\)
    soddisfa il criterio di Cauchy in \([a, b)\) e quindi converge
    uniformemente.
\end{proof}
Da questa osservazione noto anche che vale il contrapositivo: se \(f_{n}\) non
converge uniformemente in \([a, b)\) non può neanche convergere uniformemente in
\((a, b)\)
\begin{example}
    \(\displaystyle f_{n}(x) = \frac{1}{1 + {n^2 \left( x - \frac{q}{\sqrt{n}}
\right)}^2}\), allora ho che \(\displaystyle f_{n}(0) = \frac{1}{1+n} \to 0\), e
per \(x \neq 0\) pure, infatti
\[
    0 \le \frac{1}{1+ n^2 {\left( x - \frac{1}{\sqrt{n}} \right)}^2}
    \overset{\text{definitivamente}}{\le} \frac{1}{1 + n^2 {\left(
    \frac{x^2}{2} \right) }^2} \to 0
\]
è convergente uniformemente su tutto \(\mathbb{R}\)
\end{example}
Sia \(E\) un insieme non vuoto e sia \(\mathcal{B}(E)\) l'insieme delle funzioni
reali e limitate su \(E\).
\begin{definition}{Norma dell'estremo superiore}
    Sia \(f : E \to \mathbb{R}^{n}\) una funzione. Allora
    \[
        \|f \|_{\infty} := \sup_{x \in E} |f(x)| 
    \]
    è la norma dell'estremo superiore (anche denotata semplicemente \(\|f\|\)).
\end{definition}
\begin{proof}[Buona definizione]
Perché sia una buona definizione, serve che sia una norma. 
\begin{enumerate}[label = \alph*.]
    \item \(\|f\| \ge 0\) e \(\|f\| = 0 \iff f = 0\)
    \item \(\|\lambda f\| = |\lambda| \|f\|\)
    \item \(\|f + g\| \le \|f\| + \|g\|\)
\end{enumerate}
\end{proof}
\begin{proposition}
    \(\mathcal{B}(E)\) è uno spazio metrico normato con la norma dell'estremo
    superiore, e quindi distanza \(d(f, g) = \|f - g\|\) 
\end{proposition}
\begin{proof}
    ovvio
\end{proof}

\subsection{Scambi di limite, derivate, integrali}
\begin{example}
    Dimostrare che se \(f \in C^{0}([a, b] \times [c, d])\) a valori reali e 
    \[
        g(y) = \int_{a}^{b} f(x, y) \, dx \quad y \in [c, d]
    \]
    Allora \(g\) è continua in \([c,d]\) 

    Infatti \(\forall \overline{y} \in [c, d]\) abbiamo che comunque presa
    \(y_{n} \to \overline{y}\) chiaramente \(g(y_{n}) \to g(\overline{y})\).
    Ponendo ora \(f_{n} = f(\cdot , y_{n})\). Allora vogliamo mostrare che
    \(f_{n}(\cdot ) \to f(\cdot , \overline{y})\) uniformemente in \([a, b]\).
    Poiché \(f\) è uniformemente continua in \([a, b] \times [c, d]\) perché
    continua su un compatto, allora \(\forall \varepsilon > 0 \quad \exists
    \delta > 0\) tale che \(\forall x, x' \in [a, b]\) e \(\forall y, y' \in [c,
    d]\) se \(\sqrt{{(x - x')}^2 + {(y - y')}^2} < \delta\) allora \(|f(x, y) -
    f(x', y')| < \varepsilon\). Allora fissato \(\varepsilon>0\) sia \(\delta\)
    come sopra; sia quindi \(n_\varepsilon\) tale che \(n \ge n_\varepsilon
    \implies |y_{n} - \overline{y}| \le \delta\) e quindi, per ogni \(x \in [a,
    b]\) abbiamo \(|(x, y_{n}) - (x,
    \overline{y})| = |y_{n}-y| \le \delta\), da cui \(|f_{n}(x, y_{n}) - f(x,
    \overline{y})|\le \varepsilon\). Abbiamo quindi mostrato l'uniforme
    convergenza.
\end{example}
\begin{proposition}[Derivation under the integral sign]
    Sia \(f \in C^{1}([a,b] \times [c, d])\) e \(g(y) = \int_a^b f(x, y) dx\)
    per \(y \in [c, d]\), allora
    \[
        g \in C^{1}([c, d]) \text{ e } g'(y) = \int_a^b \frac{\partial
        f}{\partial y}(x, y) \, dx
    \]
\end{proposition}
\begin{proof}
    Fissiamo \(\overline{y} \in [c, d]\) e consideriamo 
    \[
        \frac{g(y) - g(\overline{y})}{y - \overline{y}} = \int_a^b
        \frac{f(x, y) - f(x, \overline{y})}{y - \overline{y}} \, dx = \int_a^b 
        \varphi(x, y) dx
    \]
    con \(\varphi(x, y)\) l'integrando.
    Siappiamo che \(\lim_{y \to \overline{y}} \varphi(x, y) = \frac{\partial
    f}{\partial y} (x, \overline{y})\) e vogliamo mostrare che questa
    convergenza è uniforme al variare di \(x\). Per il teorema di Lagrange si ha
    che 
    \[
        \varphi(x, y) = \frac{f(x, y) - f(x, \overline{y})}{y - \overline{y}} =
        \frac{\partial f}{\partial y} (x, \xi_{x, y}) \quad \xi_{x, y} \in
        (\overline{y}, y) \text{ oppure } (\overline{y}, y)
    \]
    Poiché \(\frac{\partial f}{\partial y}\) è uniformemente continua in \([a,
    b] \times [c, d]\) allora
    \begin{align*}
        \forall \varepsilon > 0 \quad \exists \delta > 0 : \forall x, x' \in [a,
        b] \text{ e } \forall y, y' \in [c, d]\\
        |(x, y) - (x', y')| < \delta \implies \left| \frac{\partial f}{\partial y}(x,
        y) - \frac{\partial f}{\partial y}(x', y') \right| < \varepsilon
    \end{align*}
    e ora prendiamo come coppie \((x, \overline{y})\) e \((x, \xi_{x, y})\) e
    abbiamo
    \[
        |(x, \xi_{x, y}) - (x, \overline{y})| = |\xi_{x, y} - \overline{y}| \le
        |y - \overline{y}|
    \]
    Ora come prima ciò dimostra che \(\varphi(x, y) \to \frac{\partial
    f}{\partial y}(x, \overline{y})\) uniformemente in \([a, b]\) e quindi
    \[
        \frac{d}{dy} \int_a^b f(x, y) \, dx = \int_a^b \frac{\partial
        f}{\partial y}(x, y) \, dx
    \]
\end{proof}

\subsection{Serie di funzioni}
I risultati visti per le successioni di funzioni danno luogo ad analoghi
risultati per le serie di funzioni.
Sia quindi \(E\) un insieme \(f_{n} : E \to \mathbb{R}\) (oppure
\(\mathbb{R}^{m}, \mathbb{C}\)) e si considera la serie 
\[
    \sum_{n=1}^{\infty} f_{n}(x)\quad x \in E
\]
che è una serie di funzioni.
\begin{definition}{}
    Diciamo che la serie \(\sum_{n=1}^{\infty} f_{n}(x)\) converge puntualmente
    in \(E\) se la successione delle somme parziali converge puntualmente in
    \(E\), ossia se 
    \[
        \lim_{N \to \infty} \sum_{n=1}^{N} f_{n}(x) = f(x) \quad \forall x \in E
    \]

    Diciamo che la serie \(\sum_{n=1}^{\infty} f_{n}(x)\) converge uniformemente
    in \(E\) se la successione delle somme parziali converge uniformemente in
    \(E\),
\end{definition}
Ne consegue che alcuni risultati hanno rispettivi analoghi, ad esempio
    \[
        \sum_{i=1}^{\infty} f_{n}(x) 
    \]
    converge uniformemente in \(E\) se e solo se 
    \[
        s_N(x) = \sum_{n=1}^{N} f_{n}(x)
    \]
    converge uniformemente in \(E\) (definizione), ossia questo vale se 
    \[
        \forall \varepsilon > 0 \quad \exists n_\varepsilon \in \mathbb{N} :
        \forall N, M \ge n_\varepsilon, \forall x \in E, \quad |s_N(x) - s_M(x)|
        < \varepsilon
    \]
    Ora assumiamo senza perdita di generalità che \(N \le  M\), allora chiamiamo
    \(M = N + p\) e otteniamo che l'ultima eguaglianza si scrive come
    \[
        \left| \sum_{n=N+1}^{N+p} f_{n}(x) \right| < \varepsilon
    \]
Otteniamo
\begin{proposition}[Criterio di Cauchy]
    La serie \(\sum_{n=1}^{\infty} f_{n}(x)\) converge uniformemente in \(E\) se
    e solo se 
    \begin{equation}
        \forall \varepsilon > 0 \quad \exists n_\varepsilon \in \mathbb{N} :
        \forall n, p \ge n_\varepsilon, \forall x \in E, \quad \left|
        \sum_{k=n+1}^{n+p}
        f_{k}(x) \right| < \varepsilon
    \end{equation}
\end{proposition}
\begin{corollary}
    Condizione necessaria affinche la serie \(\sum_{n=1}^{\infty} f_{n}(x)\)
    converga uniformemente in \(E\) è che \(f_{n} \to 0\) uniformemente in \(E\)
\end{corollary}
\begin{proof}
    prendiamo \(p=1\) in (1) e otteniamo
    \(
        \left| f_{n+1}(x) \right| < \varepsilon
    \)
    ossia \(f_{n} \to 0\) uniformemente in \(E\) 
\end{proof}

\begin{example}
Supponiamo ora che esista una successione numerica \({\{a_{n}\}}_{n \in
\mathbb{N}}  \) tale che
\begin{itemize}
    \item \(|f_{n}(x)| \le a_{n}\) per ogni \(x \in E\) 
    \item \(\sum_{n=1}^{\infty} a_{n} < +\infty\)
\end{itemize}
vogliamo mostrare che allora la serie \(\sum_{n=1}^{\infty} f_{n}(x)\) converge
uniformemente in \(E\), usando \((1)\), infatti abbiamo 
\[
    \left| \sum_{k=n+1}^{n+p} f_{k}(x) \right| \le \sum_{k=n+1}^{n+p} |f_{k}(x)|
    \le \sum_{k=n+1}^{n+p} a_{k} < \varepsilon
\]
dove nell'ultima diseguaglianza si è utilizzato il criterio di Cauchy per le
serie numeriche.
\end{example}
\begin{definition}{Convergenza totale}
    Si dice che la serie \(\sum_{n=1}^{\infty} f_{n}(x)\) converge
    \textbf{totalmente} in \(E\) se esiste \(\{a_{n}\} \) in \(\mathbb{R}\) tale
    che 
    \begin{itemize}
        \item \(|f_{n}(x)| \le a_{n}\) per ogni \(x \in E\) 
        \item \(\sum_{n=1}^{\infty} a_{n} < +\infty\)
    \end{itemize}
\end{definition}
Per quanto visto prima quindi
\begin{proposition}
    Convergenza totale implica convergenza uniforme, e notando dalla
    dimostrazione prima abbiamo anche che implica la convergenza assoluta
    uniforme.
\end{proposition}

\begin{example}
Non vale il contrario, un esempio di serie uniformemente convergente ma non
totalmente convergente è
\[
\sum_{n=1}^{\infty} -1 \cdot \frac{{(-1)}^{n}}{n}
\]
dove \(f_{n}(x)\) è costante per ogni \(n\). Allora la serie converge
uniformenenente in \(\mathbb{R}\) ovviamente perché è costante e converge in
quanto a segno alternato, ma non converge totalmente perché la serie armonica
diverge.
\end{example}
\begin{example}
    Sia \(f_{n}(x) = {(-1)}^{n+1} \frac{x^{n}}{n}\), per \(x \in \mathbb{R}\).
    Allora usiamo il criterio della radice ottenendo
    \[
        \lim_{n \to \infty} \sqrt[n]{|f_{n}(x)|} = \lim_{n \to \infty}
        \frac{|x|}{\sqrt[n]{n}} = |x|
    \]
    quindi per \(|x| < 1\) la serie converge assolutamente, per \(|x| > 1\) la
    serie diverge, per \(x = -1\) la serie è la serie armonica che diverge, per
    \(x = 1\) la serie è una serie a segni alterni che converge.

    Concludiamo quindi che la serie converge puntalmente in \((-1, 1]\) e per
    ogni \(0 < \delta < 1\) la serie converge uniformemente in \([-\delta,
    \delta]\), infatti 
    \[
        \left| {(-1)}^{n+1} \frac{x^{n}}{n} \right| \le \delta^{n}{n}
    \]
    la cui serie converge, quindi la serie converge totalmente.

    Naturalmente però la serie non converge totalmente in \([0, 1]\) poiché
\[\max_{x \in [0, 1]} |f_{n}(x)| = \frac{1}{n}\] ma comunque la serie converge
uniformemente in \([0, 1]\), infatti usiamo il criterio di Cauchy.
\[
    \left| \sum_{k=n+1}^{n+p} f_{k}(x) \right| \le |s_{n+p}(x) - s_{n}(x)| <=
    |s_{n+p} - S(x)| + |S(x) - s_{n}(x)| < \frac{x^{n+p}}{n+p} + \frac{x^{n}}{n}
\]
che converge a \(0\) per \(n\to +\infty\)
e si è usato il fatto che se \(S = \sum_{i=1}^{\infty} {(-1)}^{n+1} a_{n} \)
è una serie convergente a segni alterni, con \(a_{n} > 0, a_{n} \to 0\) allora
\(|S - s_{n}| \le a_{n+1} \) 

Procediamo a chiederci se la serie converge uniformemente in \((-1, 0]\).
Utilizziamo allora la seguente osservazione dedotta direttamente dalle
successioni
\begin{remark}
    Sia \(\sum_{i=1}^{\infty} f_{n}(x) \), \(f_{n} \in C^{0}([a, b])\). Se la
    serie converge uniformemente in \((a, b]\) allora converge uniformemente in
    \([a, b]\) (in particolare converge in \(x=a\))
\end{remark}
\begin{proof}
    Per ipotesi \(s_{n}(x)\) converge uniformemente in \((a, b]\) e \(s_{n}\)
    sono funzioni continue in \(x = a\), quindi per il risultato che avevamo già
    per le successioni (in breve basta enunciare il criterio di Cauchy e usare
la continuità in \(x=a\)) otteniamo che la serie converge uniformemente in \([a,
b]\)
\end{proof}
Ne concludiamo che la serie non può convergere uniformemente in \((-1, 0]\)
altrimenti convergerebbe uniformemente in \([-1, 0]\) ma sappiamo che in \(-1\)
non abbiamo neanche convergenza puntuale.
\end{example}

\begin{example}
    Studiare la convergenza puntuale e uniforme di
    \[
        \sum_{n=1}^{\infty} \frac{x}{{(1+x)}^{n}} 
    \]
\end{example}

\subsection{Richiami su limiti e serie}
\begin{proposition}
    Sia \(a_{n}\) una successione di numeri reali positivi. Allora 
    \[
       \liminf_{n \to +\infty} \frac{a_{n+1}}{ a_{n} } \le \liminf_{n \to
       +\infty} \sqrt[n]{a_{n}} \le \limsup_{n \to +\infty} \sqrt[n]{a_{n}} \le
       \limsup_{n \to +\infty} \frac{a_{n+1}}{a_{n}}
    \]
\end{proposition}
\begin{proof}
    Sia \(L = \limsup \frac{a_{n+1}}{a_{n}}\). Se \(L=+\infty\) non c'è nulla da
    dimostrare. Sia allora \( L < +\infty\). Fissato un \(\varepsilon>0\) quindi
    esiste \(n_{\varepsilon} \) tale che \(\forall n \ge n_{\varepsilon}\) si ha 
    \[
        \frac{a_{n+1} }{a_{n}} \le  L + \varepsilon
    \]
    Allora iterando otteniamo
    \[
        a_{n} \le (L + \varepsilon)^{n - n_{\varepsilon}} a_{n_{\varepsilon}}
        \implies \sqrt[n]{a_{n}} \le {(L + \varepsilon)}^{1 -
        \frac{n_{\varepsilon}}{n}} \sqrt[n]{a_{n_{\varepsilon}}} \to L +
        \varepsilon 
    \]
    Per \(n \to \infty\), ora per l'arbitrarietà di \(\varepsilon > 0\)
    otteniamo
    \[
        \limsup_{n \to \infty} \sqrt[n]{a_{n}} \le L = \limsup_{n \to \infty}
        \frac{a_{n+1}}{a_{n}}
    \]
    Similmente si dimostra anche l'altra uguaglianza, quella centrale è ovvia.
\end{proof}
\begin{example}
    Sia \(a_{n} = n\). Allora poiché \(\frac{a_{n+1}}{a_{n}} \to 1\) abbiamo che
    anche \(\sqrt[n]{n} \to 1\).

    Sia \(a_{n} = n!\). Allora \(\frac{a_{n+1}}{a_{n}} = n+1 \to +\infty\) e
    quindi anche \(\sqrt[n]{n!} \to +\infty\)

    Sia \(a_{n} = \frac{n^{n}}{n!}\) allora
    \[
        \frac{a_{n+1}}{a_{n}} = \frac{{(n+1)}^{n+1}}{(n+1)!} \frac{n!}{n^{n}} =
        \frac{{(n+1)}^{n}}{n^{n}} = {\left( 1 + \frac{1}{n} \right)}^{n} \to e
    \]
    e quindi \(\displaystyle \sqrt[n]{\frac{n^{n}}{n!}} =
    \frac{n}{\sqrt[n]{n!}}\to e\).

\end{example}
\begin{remark}
    In realtà (e potremmo vederlo più tardi), l'approssimazione di Stirling ci
    dice
    \[
    n! \sim \sqrt{2\pi n} {\left( \frac{n}{e} \right)}^{n}
    \]
\end{remark}

Ora procediamo vedendo un criterio di convergenza (non assoluta) che sarà il
criterio di convergenza di Abel. Procediamo a passi più piccoli.
\begin{lemmao}[Disuguaglianza di (Brunacci) Abel]
    Siano \(\gamma_0, \gamma_{1}, \dots, \gamma_{\ell} \in \mathbb{C}\) e siano
    \(\zeta_{0}, \zeta_{1}, \dots, \zeta_{\ell} \in \mathbb{C} \). Poniamo ora 
    \[
        w_m = \sum_{i=0}^{m} \zeta_i \quad m = 0, 1, \dots, \ell
    \]
    Sia \(M > 0\) tale che 
    \[
        |w_m| \le M \quad \forall m = 0, 1, \dots, \ell
    \]
    Allora 
    \[
        \left| \sum_{i=0}^{\ell} \gamma_i \zeta_i \right| \le (|\gamma_0 -
        \gamma_{1}| + |\gamma_1 - \gamma_{2}| +~\dots + |\gamma_{\ell-1} -
        \gamma_{\ell}| + |\gamma_{\ell}|)M
    \]
\end{lemmao}
\begin{proof}
    \[
        \sum_{i=0}^{\ell} \gamma_{i} \zeta_{i} = \sum_{i=0}^{\ell} \gamma_{i}
        (w_{i} - w_{i-1}) = \sum_{i=0}^{\ell} (\gamma_{i} - \gamma_{i+1})w_{i}
    \]
    Dove si intende che \(\gamma_{\ell+1} = 0\) e \(w_{-1} = 0\). Ora
    semplicemente per disuguaglianza triangolare e applicando l'ipotesi
    definente \(M\) otteniamo la tesi.
\end{proof}

\begin{theorem}[primo criterio di convergenza di Abel]
    Sia \(\sum_{n=1}^{\infty} c_{n} z_{n}\) una serie numerica. Se 
    \begin{itemize}
        \item \(z_{n} \in \mathbb{C}\) (oppure in \(\mathbb{R}^{N}\))
        \item \(\sum_{n=1}^{\infty} z_{n}\) è una serie le cui somme parziali
            sono limitate
        \item \(\{c_{n}\}\) è una successione di numeri reali non creascente e
            infinitesima
    \end{itemize}
    allora la serie \(\sum_{n=1}^{\infty} c_{n} z_{n}\) converge.
\end{theorem}

\begin{proof}
    Utilizziamo il criterio di convergenza di Cauchy. Fissiamo \(N, p \in
    \mathbb{N}\) e consideriamo
    \[
        \left| \sum_{n=N}^{N+p} c_{n} z^{n} \right| \le 2M\sum_{n=N}^{N+p} |c_{n}
        - c_{n+1}| \text{ (diciamo \(c_{N+p+1} = 0 \) per comodità notazionale) }
    \]
    Dove \(M\) è maggiorante per le somme parziali di \(z_{n}\). Infatti abbiamo
    che 
    \[
        w_m = z_N + z_{N+1} +~\dots + z_{N+m} = \sum_{i=0}^{N+m} z_{n} -
        \sum_{i=0}^{N-1} z_{n}
    \]
    per cui effettivamente \(|w_m| \le 2M\) e possimao applicare la
    disuguaglianza di Abel.
    
    Ora possiamo, sapendo che \(c_k \to 0\) da sopra, ottenere che la serie
    precedentemente trovata è telescopica per \(N\) sufficientemente grande e
    quindi
    \[
        \left| \sum_{n=N}^{N+p} c_{n} z^{n} \right| \le 2M|c_{N}|\to 0 \text{ per
        } N \to +\infty
    \]
    e quindi per il criterio di Cauchy la serie converge.
\end{proof}
\begin{example}
    Consideriamo la serie
    \[
    \sum_{n=1}^{\infty} {(-1)}^{n+1} \frac{z^{n}}{n} \quad z \in \mathbb{C}
    \]
    Allora se \(|z| > 1\) manca la condizione necessaria di convergenza.

    Sia allora \(|z| \le  1\). Consideriamo prima il caso \(|z| < 1\). Sia ha
    allora convergenza assoluta, perché
    \[
        \frac{|z|^{n+1}}{n+1} \cdot \frac{n}{|z|^{n}} =  |z| \frac{n}{n+1} \to 0
    \]

    Consideriamo invine il caso \(|z| = 1\). Se \(z = -1\) la serie non
    converge:
    \[
        \sum_{n=1}^{\infty} \frac{{(-1)}^{n+1} \cdot {(-1)}^{n}}{n} = -
        \sum_{i=1}^{\infty} \frac{1}{n}  \to -\infty
    \]
    Ora consideriamo \(|z|=1\) con \(z\neq-1\) e vogliamo applicare il criterio
    di Abel, con \(c_{n} = \frac{1}{n}\) e \(z_{n} = {(-1)}^{n+1} z^{n}\).
    Chiaramente \(c_{n}\) è infinitesima non crescente reale. Inoltre
    \[
        |z_{1} + z_{2} + \dots + z_N| = \left| \sum_{i=1}^{N} {(-1)}^{n+1} z^{n}
        \right| = \left| \sum_{n=1}^{N} (-z)^{N} \right| = |z| \left| \frac{1 -
        {(-z)}^{N}}{1 - (-z)} \right| \le \frac{z}{|1+z|}
    \]
    Quindi sono soddisfatte le ipotesi del criterio di Abel e la serie converge.

\begin{figure}[ht]
    \centering
    \incfig[.4]{esempio2}
    \caption{Punti del piano \(\mathbb{C}\) tali che la serie dell'esempio 2.2
    converge}\label{fig:esempio2}
\end{figure}
\end{example}
\begin{corollary}[Criterio di Leibniz]
    Se una serie è a segni alterni del tipo 
    \[
        \sum_{n=1}^{\infty} {(-1)}^{n} a_{n}
    \]
    con \(a_{n} \to 0\) e \(a_{n} > 0\) non crescente. Allora abbiamo che
    \(z_{n} = {(-1)}^{n}\) e \(c_{n} = a_{n}\) soddisfano le ipotesi del
    criterio di Abel e quindi la serie converge.
\end{corollary}
\begin{theorem}[Secondo criterio di convergenza di Abel]
    Si consideri la serie \[
        \sum_{n=0}^{\infty} c_{n} z_{n} 
    \]
    con
    \begin{itemize}
        \item \(z_{n} \in \mathbb{C}\) (oppure in \(\mathbb{R}^{N}\))
        \item \(\sum_{n=0}^{\infty} z_{n}\) è una serie convergente
        \item \(\{c_{n}\}\) è una successione monotona e convergente
    \end{itemize}
    Allora la serie \(\sum_{n=0}^{\infty} c_{n} z_{n}\) converge.
\end{theorem}
\begin{proof}
    Supponiamo \(c_{n} \to c\) non crescente. Allora
    \[
        \sum_{n=0}^{N} c_{n} z_{n} = \sum_{n=0}^{N} (c_{n} - c)z_{n} + c
        \sum_{n=0}^{N} z_{n}
    \]
    Ora per \(N \to \infty\) abbiamo che \(c_{n} - c \to 0\) decrescente e le
    somme parziali di \(z_{n}\) sono limitate, perché la serie converge. Quindi
    abbiamo che la prima serie converge per il primo criterio di Abel. 
    Anche la seconda serie converge per ipotesi, quindi la tesi è dimostrata.
\end{proof}

Sappiamo che 
\[
    \left( \sum_{n=0}^{N} a_{n} z^{n}  \right) \left( \sum_{n=0}^{M}
        b_{n} z^{n}
    \right) = \sum_{n=0}^{N+M} \sum_{k=0}^{n} a_{k} b_{n-k}  z^{n}
\]
è il prodtto di polinomi. Quindi formalmente, se \(z = 1\) e \(N, M \to \infty\) otteniamo

\begin{definition}{Serie prodotto alla Cauchy}
\[
    \left( \sum_{n=0}^{\infty} a_{n}  \right) \left( \sum_{n=0}^{\infty} b_{n}
    \right) := \sum_{n=0}^{\infty} \sum_{k=0}^{n} a_{k} b_{n-k}  
\]
che è detta serie prodotto alla Cauchy delle serie \(\sum_{n=0}^{\infty} a_{n}\)
e \(\sum_{n=0}^{\infty} b_{n} \) 
\end{definition}

\begin{theorem}[Mertens + Cauchy]
    Se le serie \(\sum_{n=0}^{\infty} a_{n}\) e \(\sum_{n=0}^{\infty} b_{n} \)
    sono convergenti e almeno una è assolutamente convergente, allora la serie
    prodotto è convergente ed ha per somma il prodotto delle serie.
    
    Se entrambe le serie sono assolutamente convergenti, allora tale è anche la
    serie prodotto.
\end{theorem}

Consideriamo serie della forma 
\[
\sum_{n=0}^{\infty} c_{n} {(z - a)}^{n} \text{ con } c_{n}\in \mathbb{C} \text{
e } z, a \in \mathbb{C}
\]
Abbiamo visto alcuni esempi:
\begin{enumerate}[label = \alph*)]
    \item \(\sum_{n=0}^{\infty} z^{n} \) converge se e solo se \(|z| < 1\)
    \item \(\sum_{n=0}^{\infty} {(-1)}^{n+1} \frac{z^{n}}{n} \), converge se e
        solo se \(|z| \le 1\) e \(z \neq -1\)  
\end{enumerate}
In entrambi i casi (e vedremo in generale) la convergenza è nei punti di un
disco (detto cerchio di convergenza) di centro \(z = a\). Il comportamento sul
bordo del cerchio varia da caso a caso.

\begin{theorem}[Abel]
    Si consideri la seire di potenze 
    \[
    \sum_{n=0}^{\infty} c_{n} {\left( z-a \right)}^{n}
    \]
    Se la serie converge in un punto \(z \in \mathbb{C}\) allora converge
    uniformemente su tutto il segmento di estremi \(a\) e \(z\).
\end{theorem}
\begin{figure}[ht]
    \centering
    \incfig[.3]{abel}
    \caption{abel}
    \label{fig:abel}
\end{figure}
\begin{proof}
    Il teorema è significativo quando \(z_{1} \in \partial D_R(a) \) 
    Non è restrittivo supporre \(a =0\). Consideriamo
    \[
        \sum_{n=0}^{\infty} c_{n} z^{n}_t = \sum_{n=0}^{\infty} c_{n}
        {(tz_{1})}^{n} 
    \]
    utilizziamo il criterio di Cauchy per le convergenze uniformi: fissiamo \(
    \varepsilon >0\),
    vogliamo avere che per un \(n_\varepsilon\) allora per ogni \(N \ge
    n_\varepsilon\), \(p \in \mathbb{N}\) e \(t \in [0, 1]\) si abbia che
    \begin{align*}
        \left|\sum_{m=N}^{N+p} \underbrace{t^{n}}_{\gamma_{m-N} }
        \underbrace{c_{n}z_{1}^{n}}_{\zeta_{m-N}} \right| &< \varepsilon \\
 &\le  M(|\gamma_0 - \gamma_{1}|
        + |\gamma_1 - \gamma_{2}| +~\dots + |\gamma_{p-1} - \gamma_{p}|) 
    \end{align*}
    con \(M\) un maggiorante per le somme parziali di \(c_{n}z_{1}^{n}\). Ora
    poiché per ipotesi tale serie converge, esiste \(n_\varepsilon\) tale per
    cui per ogni \(N \ge n_\varepsilon\) e per ogni \(p \in \mathbb{N}\) si ha
    che
    \[
        \left|\sum_{m=N}^{N+p} c_{m}z_{1}^{m} \right| \le \varepsilon
    \]
    ora poiché \(1 \ge t ^{n}\) per ogni \(n \in \mathbb{N}\) abbiamo che la
    precedente disugaglianza è soddisfatta per \(M = 1\), quindi
    \[
        \left| \sum_{n=N}^{N+p} t ^{n} c_{n} z_{1}^{n} \right| \le t ^{N}
        \varepsilon \le  \varepsilon
    \]
\end{proof}

\section{Analisi complessa}

\subsection{Funzioni analitiche su \(\mathbb{R}\) }
\begin{definition}{Funzione analitica}
    Sia \(I\) un intervallo aperto. Diciamo che una funzione \(f: I \to
    \mathbb{R}\) è analitica se per ogni \(x_{0} \in I\) esiste \(\delta > 0\)
    tale che su \((x_{0} - \delta, x_{0} + \delta)\) la funzione sia esprimibile
    come somma di una serie di potenze 
    \begin{equation}
        f(x) = \sum_{n=0}^{\infty} a_{n} {(x - x_{0})}^{n} 
    \end{equation}
\end{definition}
Sia \(f\) come in \((2)\); Sia \(R\) il raggio di convergenza della serie. Su
ogni intervallo \(J\) tale che \(\overline{J} \subseteq (x_{0} - R, x_{0} + R)
\) sappiamo che la serie converge totalmente. Consideriamo la serie delle
derivate (cioè la \emph{serie derivata})
\[
    \sum_{n=1}^{\infty} n a_{n} {(x - x_{0})}^{n-1} 
\]
è una serie di potenze con raggio di convergenza \(R\), poiché 
\[
    \limsup_{n \to \infty} \sqrt[n]{n|a_{n}|} = \limsup_{n \to \infty}
    \sqrt[n]{|a_{n}|} \cdot \underbrace{\limsup_{n \to \infty} \sqrt[n]{n}}_{=1}
\]
Quindi la Erie delle derivate è uniformemente convergente su ogni compatto di
\(x_{0}-R, x_{0}+R\). 
\begin{lemma}[Teorema di derivazione per Serie]
    Sia \(f = \sum f_{n}\) convergente e \(g = \sum f_{n}' \) uniformemente
    convergente. Allora \(f\) è derivabile e \(f' = g\).
\end{lemma}
Per il teorema di derivazione per serie, \(f\) è derivabile e
\[
    f'(x) = \sum_{n=1}^{\infty} n a_{n} {(x - x_{0})}^{n-1} \quad \forall x \in
    (x_{0} - R, x_{0}+R)
\]
A \(f'\) applichiamo lo stesso ragionamento visto su \(f\): \(f'\) è derivabile
e si ha che 
\[
    f''(x) = \sum_{n=2}^{\infty} n(n-1) a_{n} {(x - x_{0})}^{n-2} \quad \forall
    x \in (x_{0} - R, x_{0}+R)
\]
Procedendo induttivamente otteniamo che \(f \in C^{\infty}(x_{0} - R, x_{0}+R)\)
e
\[
    f^{(k)}(x) = \sum_{n=k}^{\infty} n(n-1) \dots (n-k+1) a_{n} {(x -
    x_{0})}^{n-k}
\]
In particolare abbiamo che \(\displaystyle f^{(k)}(x_{0}) = k! a_k\) e quindi la
serie di potenze è la serie di Taylor di \(f\) centrata in \(x_{0}\). Più
precisamente
\begin{theorem}
    Sia \(\displaystyle f(x) = \sum_{n=0}^{\infty} a_{n} {(x-x_{0})}^{n} \) per
    \(x \in (x_{0} -R, x_{0} + R)\), con \(a_{n} \in \mathbb{R}\) e \(x_{0} \in
    \mathbb{R}\). Allora \(f \in C^{\infty}(x_{0}-R, x_{0}+R)\) e la serie è la
    serie di Taylor di \(f\) centrata in \(x_{0}\).
\end{theorem}
\begin{proof}
    Vedasi sopra.
\end{proof}
Non è vero che ogni funzione \(C^{\infty}\) sia sviluppabile in serie di Taylor.
\begin{example}
    Sia \(f(x) = e^{-\frac{1}{x^2}}\) per \(x \in \mathbb{R} \sminus \{0\} \) e \(f(0)
    = 0\). Allora 
    \[
        f'(x) = \begin{cases}
            \lim_{x \to 0} \frac{f(x)}{x} = \frac{1}{x} e^{-\frac{1}{x^2}} = 0  & x = 0 \\
            \frac{2}{x^3} e^{-\frac{1}{x^2}} & x \neq 0
        \end{cases}
    \]
    eccetera anche per le altre derivate si ha che \(f^{(k)}(0) = 0\). Quindi la
    serie di Taylor centrata in \(0\) è la serie nulla, ma \(f \neq 0\) in alcun
    intorno di \(0\).
    \begin{figure}[ht]
        \centering
        \begin{tikzpicture}
            \begin{axis}[
                xmin= -4, xmax= 4,
                ymin= -1, ymax = 2,
                axis lines = middle,
            ]
            \addplot[domain=-4:4, samples=100]{ e^(-(1/x^2))};
            \end{axis}
        \end{tikzpicture}
        \caption{\(\displaystyle e^{-\frac{1}{x^2}}\)}
    \end{figure}
\end{example}
\begin{theorem}
    Sia \(f \in C^{\infty}(I)\) con \(I \subseteq \mathbb{R} \) un intervallo
    aperto per la quale esistano \(M, L > 0\) tali che per ogni \(k \in \mathbb{N}\) 
    \[
        \forall  x \in I \quad |f^{(k)}(x)| \le M L^{k}
    \]
    Allora \(f\) è analitica.
\end{theorem}
\begin{proof}
    Sia \(x_{0} \in I\) e consideriamo
    \[
        \sum_{n=0}^{\infty} \frac{f^{(k)} (x_{0})}{k!} {(x - x_{0})}^{k} \quad x
        \in I
    \]
    Scriviamo lo sviluppo di Taylor con il resto di Lagrange.
    \[
        f(x) = \sum_{k=0}^{k} \frac{f^{(n)}}{k!}  
        {(x - x_{0})}^{k} + \frac{1}{(n+1)!} f^{(n+1)}(\xi_x) {(x - x_{0})}^{n+1}
    \]
    dove \(\xi_x \in (x_{0}, x)\) è un opportuno punto.
    Mostriamo ora che  
    \[
        \left| \frac{1}{(n+1)!} f^{(n+1)}(\xi_x) {(x - x_{0})}^{n+1} \right| \le
    \frac{M L^{n+1}}{(n+1)!} {(x - x_{0})}^{n+1} \to 0 \text{ per } n \to
    \infty
    \]
\end{proof}
\begin{example}
    Le funzioni \(e^{x}, \sin x, \cos x\) sono analitiche.
\end{example}

\subsection{\(\mathbb{C}\)-differenziabilità}
Sia \(\Omega \subseteq \mathbb{C}\) aperto, \(f : \Omega \to \mathbb{C}\) 
\begin{definition}{\(\mathbb{C}-\)differenziabilità}
    Sia \(a \in \Omega\). Diciamo che \(f\)  è \(\mathbb{C}-\)differenziabile in
    \(z=a\) se esiste
    \begin{equation}
        \lim_{z \to a} \frac{f(z) = f(a)}{z - a} = f'(a)
    \end{equation}
    o equivalentemente
    \begin{align}
        f(z) &= f(a) + f'(a) (z-a) + (\varepsilon(z-a))(z-a) \\
        \lim_{w \to 0} \varepsilon(w) &= 0
    \end{align}
\end{definition}
Se poniamo \(\varepsilon(0) = 0\) allora la \((1)'\) vale per ogni \(z \in
\Omega\), non solo \(z\neq a\) 

Alcune proprietà:
\begin{itemize}[label = --]
    \item Se \(f\) è \(\mathbb{C}\)-differenziabile in \(z = a\) allora è
        continua (da \((1)'\))
    \item \(f, g\) \(\mathbb{C}-\)differenziabile in \(z=a\); allora \(f \pm g\)
        è \(\mathbb{C}-\)differenziabile, \(\lambda f\), con \(\lambda \in
        \mathbb{C}\) è \(\mathbb{C}-\)differenziabile e \(fg\) è
        \(\mathbb{C}-\)differenziabile.

        Se \(g(a) \neq 0\) allora \(\frac{f}{g}\) è
        \(\mathbb{C}-\)differenziabile in \(z=a\) e \({\left( \frac{f}{g}
        \right)}' = \frac{f'g -
        fg'}{g^2}\) 
\end{itemize}
\begin{example}
    \(z \mapsto z\) è \(\mathbb{C}-\)differenziabile in ogni \(z \in
    \mathbb{C}\). Ne consegue dalle proprietà che i polinomi sono
    \(\mathbb{C}-\)differenziabili, e anche le funzioni razionali.
\end{example}
\begin{example}
    \(z \mapsto \overline{z}\) \textbf{non} è \(\mathbb{C}\)-differenziabile.
    Infatti
    \[
        \frac{f(z) - f(a)}{z-a} = \frac{\overline{z} - \overline{a} }{z-a} =
        \frac{\overline{z-a}}{z-a}
    \]
    che non ha limite perché assume valori diversi ad esempio sulla retta \(a +
    \delta \) e \(a + \delta i\) al variare di \(\delta \in \mathbb{R}\).
\end{example}

Una funzione \(f : \Omega \to \mathbb{C}\) può essere vista come \(f : \Omega
\subseteq \mathbb{R}^2 \to \mathbb{R}^2 \) tralasciando la struttura di campo di
\(\mathbb{C}\). Allora possiamo scrivere \(f(x, y) = (u{\left( x, y \right)} ,
v{\left( x, y \right)} ) \in \mathbb{R}^2\), con \(u, v : \Omega \subseteq
\mathbb{R}^2 \to \mathbb{R} \). Si utilizza spesso la scrittura
\[
    f(x, y) = u(x,y) + iv(x,y)
\]
che è una sorta di ``ibrido''.
Possiamo ora scrivere \(\frac{\partial f}{\partial x}\), \(\frac{\partial
f}{\partial y}\), \(\frac{\partial u}{\partial x}\), \(\frac{\partial
v}{\partial y}\) ecc.

Supponiamo ora che \(f\) sia \(\mathbb{C}-\)differenziabile in
\(z=a=x_{0}+iy_{0}\). Esiste quindi
\[
    \lim_{z\to a}  \frac{f(z) - f(a)}{z-a} = f'(a)
\]
Guardiamo ora la retta \(z=a+\delta\), con \(\delta \in \mathbb{R}\), quindi
\[
    f'(a) = \lim_{\delta \to 0} \frac{f(x_{0})+\delta, y_{0}) -
    f(x_{0},y_{0})}{\delta} = \frac{\partial f}{\partial x}(x_{0},y_{0}) =
    \frac{\partial f}{\partial x}(a)
\] 
In maniera analoga, per \(z= a + \delta i\) abbiamo
\[
    f'(a) = \lim_{\delta \to 0} \frac{f(x_{0}, y_{0}+\delta i) -
    f(x_{0},y_{0})}{\delta i} = \frac{1}{i} \frac{\partial f}{\partial
y}f(x_{0},y_{0}) = -i \frac{\partial f}{\partial y}f(a)
\]
In breve abbiamo che deve essere 
\[
    \frac{\partial f}{\partial x}(a) = -i \frac{\partial f}{\partial y}(a)
\]
Che in termini di \(u\) e \(v\) equivale a dire che
\[
    u_x + iv_x = -i {\left( u_y + iv_y \right)} = v_y - iu_y \iff \begin{cases}
        u_x &= v_y \\
        u_y &= -v_x
    \end{cases}
\]
\begin{proposition}[Condizioni necessarie]
    Se \(f\) è \(\mathbb{C}-\)differenziabile in \(z=a\) allora valgono le
    condizioni di \textbf{Cauchy-Riemann}, cioè
    \[
        \frac{\partial f}{\partial x} = -i \frac{\partial f}{\partial y} \text{
        o equivalentemente } \begin{cases}
        u_x &= v_y \\
        u_y &= -v_x
        \end{cases}
    \]
\end{proposition}
\begin{proof}
    Vedasi sopra.
\end{proof}
\begin{proposition}
    Sia \(f\) differenziabile in \(a = (x_{0}, y_{0})\) come funzione \(\mathbb{R}^2
    \supseteq \Omega \to \mathbb{R}^2 \). Se valgono le condizioni di
    Cauchy-Riemann, allora \(f : C \supseteq \Omega \to \mathbb{C} \) è
    \(\mathbb{C}-\)differenziabile in \(z = x_{0} + iy_{0}\) 
\end{proposition}
\begin{proof}
    Per ipotesi (con \(h = (h_{1}, h_{2})\))
    \[
        f(a + h) - f(a) = \frac{\partial f}{\partial x}(a) h_{1} +
        \frac{\partial f}{\partial y}(a) h_{2} + o(h)
    \]
    Poiché \(\frac{\partial f}{\partial y} = i \frac{\partial f}{\partial x}\)
    si ha 
    \begin{align*}
        f(a+h)  - f(a) &= \frac{\partial f}{\partial x}(a) h_{1} +
        i\frac{\partial f}{\partial x}(a) h_{2} + o(h) = \frac{\partial
    f}{\partial x}(a) (h_{1} + ih_{2}) + o(h) \\
        &= \frac{\partial f}{\partial x} (a) h + o(h) = \frac{\partial
        f}{\partial x} (a) (z-a) + o(z-a)
    \end{align*}
\end{proof}
\begin{theorem}[Looman-Menchoff]
    Sia \(f : \Omega \to \mathbb{C}\) continua e dotata di \(\frac{\partial
    f}{\partial x}\), \(\frac{\partial f}{\partial y}\) in \(z=a\). Se valgono
    le condizioni di Cauchy Riemann, allora \(f\) è
    \(\mathbb{C}-\)differenziabile in \(z=a\) 
\end{theorem}
Abbiamo già visto sui reali che analitica implica \(C^{\infty}\). Ora spiace lo
spoiler ma dimostreremo che \(\mathbb{C}\)-differenziabile implica analitica,
quindi \(\mathbb{C}-\)differenziabilità, \(C^{\infty}\), analitica saranno
nozioni equivalenti e gli assegneremo la dicitura di \textbf{olomorfe}.

\begin{definition}{Derivata complessa}
    \begin{align*}
        \frac{\partial f}{\partial z} &= \frac{1}{2} {\left( \frac{\partial
        f}{\partial x} - i \frac{\partial f}{\partial y} \right)} \\
                    \frac{\partial f}{\partial \overline{z}} &= \frac{1}{2}
                    {\left( \frac{\partial f}{\partial x} + i \frac{\partial
                    f}{\partial y} \right)}
    \end{align*}
\end{definition}
Ciò è motivato dal seguente passaggio formale: Sia \(z = x+iy\) con \(x,y \in
\mathbb{R}\), allora \(f(x,y) = f{\left( \frac{z+\overline{z}}{2},
\frac{z-\overline{z}}{2} \right)} \) e quindi si ottiene formalmente il
risultato come sopra definito.
\begin{remark}
    Le condizioni di Cauchy-Riemann diventano
    \( \displaystyle
        \frac{\partial f}{\partial \overline{z}} = 0
    \)
\end{remark}

\subsection{\(\mathbb{C}\)-differenziabilità delle funzioni analitiche}
\begin{theorem}
    Si consideri la serie 
    \[
        \sum_{n=0}^{\infty} c_{n} {\left( z-a \right)}^{n} \quad c_{n} \in \mathbb{C} 
    \]
    con \(R\) il raggio di convergenza.
    Allora la serie derivata 
    \[
        \sum_{n=1}^{\infty} nc_{n} {\left( z-a \right)}^{n-1}
    \] ha lo stesso raggio di convergenza \(R\). Inoltre se \(f(z)\) è la somma
    della serie data e \(g(z)\) la somma della serie derivata, allora avremo che
    \(f\) è \(\mathbb{C}\)-differenziabile e \(f'(z) = g(z)\) per ogni \(z \in
    D_R(a)\)
\end{theorem}
\begin{proof}
    Come nel caso reale,
    \[
        \limsup_{n \to \infty} \sqrt[n]{|nc_{n}|} = \limsup_{n\to \infty}{\sqrt[n]{|c_{n}|}} 
    \]
    quindi i due raggi di convergenza coincidono. Supponiamo \(a=0\). Fissiamo
    \(w \in D_R(0)\) e consideriamo
    \[
        \frac{f(w+h) - f(w)}{h}
    \] con \(h\) tale che \(w+h \in D_R(0)\). Scriviamo, per \(N \in
    \mathbb{N}\),
    \[
        f(z) = S_N(z) + R_N(z) \quad \text{ con } \quad S_N(z) = \sum_{n=0}^{N}
        c_{n} {\left( z - a \right)}^{n}
    \]
    e \(R_N(z)\) il resto della serie. Sappiamo che 
    \[
        \lim_{n \to 0} \frac{S_N(w + h) - S_N(w)}{h} = S'_N(w) \to g(z) \text{
        per \(N\to \infty\)  }
    \]
    Consideriamo il resto
    \[
        \frac{R_N(w +h) - R_N(w)}{h} = \frac{1}{h} \sum_{n=N=1}^{\infty} {\left(
        c_{n} {\left( {\left( w+h \right)}^{n} - w^{n} \right)} \right)}  
    \]
    essendo
    \[
        {\left( w + h \right)}^{n} - w^{n} = {\left( w + h - w \right)} {\left(
        {\left( w+h \right)}^{n-1} + {\left( w+h \right)}^{n-2} w + \dots +
w^{n-1}\right)} 
    \]
    ottengo
    \[
        \left| \frac{R_N(w+h)-R_N(w)}{h} \right| \le \sum_{n=N+1}^{\infty} |c_N| {\left( |w+h|^{n-1}
        +|w+h|^{n-2} |w| +~\dots + |w|^{n-1}\right)} 
    \]
    Ora, per \(h\) tale che \(w+h \in D_\rho(0)\), con \(|w| \le \rho < R\) si
    ha che \(|w+h|^{n-k} |w|^{k-1} \le \rho^{n-1}\) quindi
    \[
        \left| \frac{R_N(w+h) -R_N(w)}{h} \right| \le \sum_{n=N+1}^{\infty}
        |c_n| n \rho^{n-1} \to 0
    \]
    Poiché la serie \(\sum_{n=1}^{\infty} nc_{n} \zeta^{n-1} \) la serie
    derivata converge assolutamente in \(D_R(0)\) in particolare per \(\zeta =
    \rho\). 
    Concludiamo ora 
    \begin{align*}
        \limsup_{h\to 0} \left| \frac{f(w+h) - f(w)}{h} -g(w)\right| \le
        \limsup_{h\to 0} \left| \frac{S_N(w+h) - S_N(w)}{h} - S_N'(w) \right| +
        \\ +
        \limsup_{h \to 0} \left| S_N'(w) - g(w) \right| + \limsup_{h\to 0}
        \left| \frac{R_N(w+h) - R_N(w)}{h} \right| = |S'_N(w) - g(w)| +
        \varepsilon
    \end{align*}
    per \(N\) sufficientemente grande. Si conclude per l'arbitrarietà di \(\varepsilon\) 
\end{proof}

\subsection{Integrazione su Curve}
\begin{definition}{Curva in \(\mathbb{C}\) }
    Diremo \textbf{curva} in \(\mathbb{C}\) ogni funzione continua \(\gamma : [a,b] \to
    \mathbb{C}\). Si dice \textbf{chiusa} se \(\gamma(a) = \gamma(b)\). Il
    \textbf{sostegno} di \(\gamma\) è l'immagine di \(\gamma\), cioè
    \(\gamma([a,b])\). Inoltre \(\gamma\) si dice \(C^{1}\) a tratti se esistono
    \(a=t_{0}<t_{1}<\dots<t_{n}=b\) tali che 
    \[
        \gamma|_{[t_{k-1} , t_k]} \in C^{1}([t_{k-1}, t_k]) \quad \forall k = 1,
        \dots, n
    \]
    Diciamo \textbf{curva opposta} di \(\gamma\) la curva percorsa in
    ``senso opposto'' ossia:
    \[
        -\gamma : [a,b] \to \mathbb{C} \quad -\gamma(t) = \gamma(a+b-t)
    \]
    Chiamiamo \textbf{saldatura} di due curve \(\gamma_1 : [a_{1}, b_{1}] \to
    \mathbb{C}, \gamma_2 [a_{2}, b_{2}] \to \mathbb{C}\), con
    \(\gamma_{1}(b_{1}) = \gamma_{2}(a_{2}). \), la curva 
    \[
        (\gamma_{1} + \gamma_{2})(t) = \begin{cases}
            \gamma_{1}(t) & t \in [a_{1}, b_{1}] \\
            \gamma_{2}(a_{2} + t - b_{1}) & t \in [b_{1}, b_{1} + b_{2} - a_{2}]
        \end{cases}
        \quad \forall t \in [a_{1}, b_{1} + (b_{2}-a_{2})]
    \]
    (Notare che esiste anche la notazione moltiplicativa per saldatura e curva
    opposta).

Siano ora \(\gamma: [a,b] \to \mathbb{C}\) e \(\tilde{\gamma} : [\alpha, \beta]
\to \mathbb{C}\) due curve. Allora diciamo che le due curve sono
\textbf{equivalenti} se esiste \(\varphi : [\alpha, \beta] \to [a,b]\) \(C^{1}\)
a tratti, biettiva, con \(\varphi'>0\), tale che
\(
    \tilde\gamma = \gamma \circ \varphi 
\)
\end{definition}

Per convenzione, se non espressamente specificato diversamente considereremo
curve \(C^{1}\) a tratti.

\begin{definition}{Integrale su curva}
    Sia \(\gamma : [a,b] \to \mathbb{C}\) una curva \(C^{1}\) a tratti e sia
    \(f\) continua a valori in \(\mathbb{C}\) definita (almeno) sul sostegno di
    \(\gamma\). Allora si definisce
    \[
        \int_{\gamma} f(z) dz = \int_{a}^{b} f(\gamma(t)) \gamma'(t) dt
    \]
\end{definition}
Abbiamo le seguenti proprietà:
\begin{itemize}[label = --]
    \item (linearità) \(\displaystyle \int_{\gamma} \left( \lambda f + \mu g
        \right) dz = \lambda \int_{\gamma} f dz + \mu \int_{\gamma} g dz\)
    \item (additività) \(\displaystyle \int_{\gamma_1 + \gamma_2} f dz =
        \int_{\gamma_1} f dz + \int_{\gamma_2} f dz\)
    \item \(\displaystyle \left| \int_{\gamma} f dz \right| \le \text{ lungh
        }(\gamma) \cdot \max_{\text{spt } \gamma } |f|  \) 
    \item \(\displaystyle \int_{-\gamma} f(z) dz = - \int_{\gamma} f(z)dz\) 
\end{itemize}

Sia \(\mathcal{C}\) una curva in \(\mathbb{C}\) assegnata come ``oggetto
geometrico'': circonferenza, retangolo, segmento eccetera. Allora scriveremo
\(
    \int_{\mathcal{C}} f(z) dz
\)
purché il contesto chiarisca il tipo di parametrizzazione. Ad esempio
\(
    \int_{\partial D_R} 
\) o \(
    \int_{\partial R}
\) (rispettivamente integrale su circonferenza e su bordo di un rettangolo) si intenderà a meno di specificare in orientamento antiorario.

\begin{proposition}
    Siano \(\gamma : [a,b] \to \mathbb{C}\) e \(\tilde{\gamma} : [\alpha, \beta]
    \to \mathbb{C}\) due curve equivalenti. Allora
    \[
        \int_{\gamma} f(z) dz = \int_{\tilde{\gamma}} f(z) dz
    \]
\end{proposition}
\begin{proof}
    Sia \(\varphi : [\alpha, \beta] \to [a,b]\) la funzione di equivalenza. Allora
    \begin{align*}
        \int_{\tilde{\gamma}} f(z) dz &= \int_{\alpha}^{\beta} f(\tilde{\gamma}(t))
        \tilde{\gamma}'(t) dt = \int_{\alpha}^{\beta} f(\gamma(\varphi(t)))
        \gamma'(\varphi(t)) \varphi'(t) dt = \\&= \int_{a}^{b} f(\gamma(s)) \gamma'(s)
        ds = \int_{\gamma} f(z) dz
    \end{align*}
\end{proof}

\begin{example}
    Se consideriamo \(\int_{\partial D_R (a)} \) allora la parametrizzazione che
    prendiamo sarà \(\gamma(t) = a + Re^{it}\) con \(t \in [0, 2\pi]\). Quindi
    abbiamo \(\gamma'(t) = iRe^{it}\) e 
    \[
        \int_{\partial D_R(a)} f(z) dz = \int_{0}^{2\pi} f(a + Re^{it}) iRe^{it}
        dt   
    \]
    ad esempio se \(\displaystyle f(z) = \frac{1}{z-a}\) 
    \[
        \int_{\partial D_R(a)} \frac{1}{z-a} dz = \int_{0}^{2\pi} \frac{1}{Re^{it}}
        iRe^{it} dt = \int_{0}^{2\pi} i dt = 2\pi i
    \]
\end{example}
\begin{example}
    Se consideriamo \(R\) rettangolo, \(a \in R\sminus \partial R\). Calcoliamo
    quindi
    \[
        \int_{\partial R} \frac{1}{z-a} dz
    \]
    dove \(z = a + \rho(\theta) e^{i\theta} \) dove \(\theta \in [0, 2\pi]\) e
    \(\rho\) è \(C^{1}\) a tratti. Allora otteniamo che
    \[
        \int_{\partial R} \frac{1}{z-a} dz = \int_{0}^{2\pi} \frac{1}{\rho(\theta)
        e^{i\theta}} i \rho(\theta) e^{i\theta} d\theta = \int_{0}^{2\pi} i d\theta
        = 2\pi i
    \]
    % TODO: fare meglio
\end{example}
\begin{remark}
    Se ho \(F : \Omega \subseteq \mathbb{C} \to \mathbb{C} \), 
    \(\mathbb{C}-\)differenziabile e \(\gamma : [a,b] \to \Omega\) \(C^{1}\) a
    tratti, allora \(\displaystyle \frac{d}{dt}F{(\gamma{(t)})} =
    F'{(\gamma{(t)})}\gamma'{(t)}\). Infatti, fissato \(t_{0}\in [a,b]\)
    Consideriamo
    \[
        \frac{F{(\gamma{(t)})}-F{(\gamma{(t_{0})})}}{t-t_{0}}
    \]
    Ricordiamo che \(F(z) = F{(a)} + F'{(a)}{(z-a)} + {(\varepsilon
    {(z-a)})}{(z-a)}\) con \(\varepsilon {(w)}\) infinitesimo per \(w\to 0\) e
    \(\varepsilon {(0)} = 0\). Allora
    \[
        \frac{F{(\gamma{(t)})}- F{(\gamma{(t_{0})})}}{t-t_{0}} =
        F'{(\gamma{(t_{0})})}\frac{\gamma{(t)}-\gamma{(t_{0})}}{t-t_{0}} +
        \varepsilon{(\gamma{(t)}-\gamma{(t_{0})})}
        \frac{\gamma{(t)}-\gamma{(t_{0})}}{t-t_{0}}
    \]
    e passando al limite otteniamo la tesi.
\end{remark}
\begin{remark}
    \(\int_{\gamma} f(z) dz\) è l'integrale su un intervallo di una funzione
    vettoriale \(f{(\gamma{(t)})}\gamma'{(t)}\). Come tale possiamo applicare i
    risultati visti di passaggio al limite sotto il segno di integrale. Ad
    esempio supponiamo di avere \(\gamma : [a,b] \to \Omega\) e una successione
    e una funzione \(f_{n}, f : \text{spt}\gamma \to \mathbb{C}\) continua e
    \(f_{n} \to f\) uniformemente su \(\text{spt}\gamma\). Allora
    \[
        \int_{\gamma} f_{n}(z) dz \to \int_{\gamma} f(z) dz
    \]
    Infatti 
    per ipotesi sappiamo che
    \[
        \forall \varepsilon>0\,\, \exists n_\varepsilon : \forall n \ge
        n_\varepsilon \,\, \forall z \in \text{spt}\gamma \quad |f_{n}(z) -
        f(z)| < \varepsilon
    \]
    ma quindi anche \(\forall t \in [a,b]\) abbiamo che \(|f_{n}{(\gamma{(t)})}
    - f{(\gamma{(t)})} < \varepsilon\) e quindi
    \[
        \left| f_{n}{(g{(t)})}\gamma'{(t)} - f{(\gamma{(t)})}\gamma'{(t)}\right|
        \le |f_{n}{(\gamma{(t)})} - f{(\gamma{(t)})}| \max_{a\le s\le b}
        |\gamma'{(s)}| < M\varepsilon 
    \]
    cioè \(f_{n}{(\gamma{(\cdot )})}\gamma'{(\cdot )} \to f{(\gamma{(\cdot
    )})}\gamma'{(\cdot )}\) uniformemente.

    In particolare (come successione si consideri la successione delle somme
    parziali di una serie) si ha che se \(\sum_{n=0}^{\infty} f_{n}(z) \)
    converge uniformemente sul supporto di \(\gamma\) allora
    \[
        \int_{\gamma} \sum_{n=0}^{\infty} f_{n}(z) dz = \sum_{n=0}^{\infty}
        \int_{\gamma} f_{n}(z) dz
    \]
\end{remark}

\begin{definition}{Primitiva}
    Sia \(\Omega \subseteq \mathbb{C}\) aperto e \(f : \Omega \to \mathbb{C}\)
    continua. Una funzione \(F : \Omega \to \mathbb{C}\) si dice \textbf{primitiva} di
    \(f\) se \(F\) è \(\mathbb{C}-\)differenziabile e \(F'(z) = f(z)\) per ogni
    \(z \in \Omega\).
\end{definition}
\begin{proposition}
    Sia \(F\) primitiva di \(f\) e \(\gamma:[a,b] \to \Omega\) una curva
    \(C^{1}\) a tratti. Allora 
    \[
        \int_{\gamma} f(z) dz = F{(\gamma{(b)})} - F{(\gamma{(a)})}
    \]
\end{proposition}
\begin{proof}
    \[
        \int_{\gamma} f(z) dz = \int_{a}^{b} f{(\gamma{(t)})}\gamma'(t) dt =
        \int_{a}^{b} F'{(\gamma{(t)})}\gamma'(t) dt = F{(\gamma{(b)})} -
        F{(\gamma{(a)})}
    \]
\end{proof}
\begin{corollary}\label{cor:primitiva_zero}
    Se \(F\) ammette primitiva in \(\Omega\) allora \(\int_{\gamma} f(z) dz =
    0\) per ogni curva chiusa \(\gamma\) in \(\Omega\).
\end{corollary}
\begin{proof}
    ovvia
\end{proof}
\begin{corollary}
    Sia \(\Omega\) un aperto \textbf{connesso}, allora se \(f\) è
    \(\mathbb{C}-\)differenziabile e \(f'=0\) allora \(f\) è costante.
\end{corollary}
\begin{proof}
    Fissiamo \(z_{0}, z_{1} \in \Omega\), allora esiste (connessione per archi)
    una \(\gamma\) \(C^{1}\) a tratti (poligonale) con \(\gamma{(a)}=z_{0}\) e
    \(g{(b)}=z_{1}\) e allora poiché \(f\) è primitiva di \(f'\) abbiamo che 
    \[
        0 = \int_{\gamma} f'(z) dz = f{(\gamma{(b)})} - f{(\gamma{(a)})} =
        f(z_{1}) - f(z_{0})
    \]
\end{proof}
Ricordiamo la notazione ``mista'' per le funzioni \(f: \mathbb{C} \to
\mathbb{C}\), \(f = f(x, y) = u(x, y) + i v {(x, y)}\). Sia ora \(\gamma: [a,b]
\to \Omega\) \(C^{1}\) a tratti e la denotiamo \(\gamma(\cdot) = x{(\cdot )} +
iy(\cdot )\). Allora 
\begin{align*}
    \int_{\gamma} f(z) dz &= \int_{a}^{b} {\left( u(x(t), y(t)) + i v(x(t),y(y))
    \right)} (x'(t) + i y'(t))dt = \\
                          &= \int_{a}^{b} u(x(t), y(t)) x'(t) - v(x(t), y(t))
            y'(t) dt + \\ &+ i \int_{a}^{b} u(x(t), y(t)) y'(t) +
                          v(x(t), y(t)) x'(t) dt
\end{align*}
Se ora poniamo \(\omega_{r}(x,y) = u(x,y)dx - v(x,y)dy \) e \(\omega_{i}(x,y) =
v{(x,y)}dx + u{(x,y)}dy\) allora otteniamo
\[
    \int_{\gamma} f(z) dz = \int_{\gamma} \omega_{r} + i \int_{\gamma}
    \omega_{i}
\]
e anche 
\begin{align*}
    \int_{\gamma} f(z) dz &= \int_{a}^{b} {\left( u(x{(t)}, y{(t)}) + i v(x(t),
            y(t)) \right)} x'(t) dt + \\ &+ i \int_{a}^{b} {\left( u(x{(t)}, y{(t)}) +
            i v(x(t), y(t)) \right)} y'(t) dt = \\ &= \int_{\gamma} f(x,y) dx + i
            \int_{\gamma} f(x,y) dy
\end{align*}

\begin{proposition}\label{prp:forme-funzioni}
    Sia \(\Omega \subseteq \mathbb{C} \) aperto, \(f: \Omega \to \mathbb{C}\). 
\begin{enumerate}[label = \alph*)]
    \item Sia \(f\) continua. Allora \(f\) \emph{ammette primitiva} se e solo se
        \(\omega_r\) e \(\omega_i\) sono \emph{esatte}
    \item Sia \(f \in C^{1}\). Allora \(f\) soddisfa le condizioni di
        \(Cauchy-Riemann\) (cioè \emph{è \(\mathbb{C}\)-differenziabile}) se e solo se
        \(\omega_r\) e \(\omega_i\) sono \emph{chiuse}
\end{enumerate}
\end{proposition}
\begin{proof} \( \) 
\begin{enumerate}[label = \alph*)]
    \item \(f\) ammette primitiva \(F = \varphi + i \psi\); si ha quindi che 
        \[
            u+ iv = F' = F_x = \varphi_x + i \psi_x \quad \text{ e }
            \begin{cases}
                \varphi_x = \psi_y \\
                \varphi_y = -\psi_x
            \end{cases}
        \]
        Allora otteniamo che 
        \[
            \begin{cases}
                u = \varphi_x = \psi_y \\
                v = \psi_x = -\varphi_y
            \end{cases}
        \]
        ne consegue che 
        \[
            \begin{cases}
                \omega_r = udx - vdy = \varphi_x dx - \varphi_y dy = d\varphi \\
                \omega_i = vdx + udy = \psi_x dx + \psi_y dy = d\psi
            \end{cases}
        \]
        sono esatte.

        Viceversa, siano \(\omega_{r}\) e \(\omega_{i}\) esatte, quindi
        \(\omega_r = d\varphi \) e \(\omega_{i} = d\psi\), per opportune
        \(\varphi , \psi \in C^{1}(\Omega)\). Allora 
        \[
            \begin{cases}
                u = \varphi _x \\
                -v = \varphi_y
            \end{cases}
            \quad
            \begin{cases}
                v = \psi_x \\
                u = \psi_y
            \end{cases}
        \]
        Ponendo ora \(F = \varphi + i\psi \in C^{1}\) si ha che 
        \[
            \begin{cases}
                \varphi_x = y = \psi_y \\ 
                \varphi_y = -v = -\psi_x
            \end{cases}
        \]
        che sono esattamente le condizioni di Cauchy-Riemann per \(F\). Allora
        \(F\) è \(\mathbb{C}\)-differenziabile e \(F' = F_x = \varphi_x +
        i\psi_x = u + iv = f\), quindi \(F\) è primitiva di \(f\).
    \item \(f = u + iv\). Le condizioni di Cauchy-Riemann sono 
        \[
            \begin{cases}
                u_x = v_y \\
                u_y = -v_x
            \end{cases}
            \iff 
            \begin{cases}
                w_{i} = vdx + udy \text{ è chiusa} \\
                w_{r} = udx - vdy \text{ è chiusa}
            \end{cases}
        \]
        semplicemente per definizione
\end{enumerate}
\end{proof}

Ricordiamo che vogliamo cercare di invertire il risultato precedente, ossia il
corollario~\ref{cor:primitiva_zero}. Per il viceversa quindi abbiamo che
\(\int_{\gamma} f = 0\) per ogni \(\gamma\) chiusa in \(\Omega\), ma ora poiché
\(\int_{\gamma} f = \int_{\gamma} \omega_{r} + i \int_{\gamma} \omega_{i} \) ne
consegue che 
\[
    \int_{\gamma} \omega_{r} = \int_{\gamma} \omega_{i} = 0 \quad \forall \gamma
    \overset{\text{Teorema }\ref{thm:1_forme}}{\implies } \omega_{r}, \omega_{i}
    \text{ esatte } \overset{\text{Proposizione}}{\implies } f \text{ ammette
    primitiva}
\]
Con questo abbiamo dimostrato
\begin{proposition}
    Sia \(f : \Omega \to \mathbb{C}\) continua, allora \(f\) ammette primitiva
    se e solo se 
    \[
        \int_{\gamma} f(z) dz = 0 \quad \forall \gamma \text{ chiusa in } \Omega
    \]
\end{proposition}
Nella dimostrazione \(f \mathbb{C}-\)differenziabile \(\implies f\) analitica
servirà avere che \(\int_{\gamma} f = 0\) per ogni \(\gamma\) chiusa in
\(\Omega\) semplicemente connesso. Ma non possiamo usare \((b)\) della
proposizione~\ref{prp:forme-funzioni} perché non possiamo assumere che \(f\) sia
\(C^{1}\). Allora mostriamo direttamente che \(\int_{\gamma} f = 0\) in un caso
particolare, usando il seguente lemma

\begin{lemmao}[Cauchy-Goursat]\label{lemma:cauchy_goursat}
    Sia \(f : \Omega \to \mathbb{C}\) \(\mathbb{C}-\)differenziabile. Sia \(R\)
    un rettangolo chiuso, con \(R \subseteq \Omega \). Allora
    \[
        \int_{\partial R} f(z) dz = 0
    \]
\end{lemmao}
\begin{figure}[ht]\label{fig:cauchygoursat}
    \centering
    \incfig[.5]{cauchygoursat}
\end{figure}
\begin{proof}
    Sia \(A = \left| \int_{\partial R} f dz \right| \). Per assurdo supponiamo
    sia \(A > 0\). Ora suddividiamo \(R\) in quattro rettangoli \(R^{1}_1,
    R^{1}_2, R^{1}_3, R^{1}_4\) e abbiamo
    \[
        A = \left|\int_{\partial R} f \right|=\left| \sum_{i=1}^{4}
        \int_{\partial R^{1}_i} f \right| \le \sum_{i=1}^{4} \left|
        \int_{\partial R^{1}_i} f \right|
    \]
    Allora abbiamo che per un qualche \(R^{1}_j\) si ha che 
    \[
        \left| \int_{\partial R^{1}_{j_{1}} } f \right| \ge \frac{A}{4}
    \]
    Procediamo in questo modo suddividendo \(R^{1}_{j_1} \) in quattro rettangoli
    \(R^{2}_i\) per \(i = 1, 2, 3, 4\) e così procedendo si forma una
    successione di rettangoli
    \[
        R^{1}_{j_{1}} \supseteq R^{2}_{j_{2}} \supseteq \dots \supseteq
        R^{n}_{j_{n}}
    \]
    che hanno diametro \(\text{diam} R^{k}_{j_k} = \frac{1}{2^{k}} \text{ diam
    }R \) di lunghezza \(\text{lungh} R^{k}_{j_k} = \frac{1}{2^{k}}
    \text{lungh}\) e tali che
    \[
        \left| \int_{\partial R^{k}_{j_k} } f  \right| \ge \frac{1}{4^{k}}
    \]
    Ora essendo ogni rettangolo compatto, la loro intersezione
    non è vuota e anzi è un solo punto \(\bigcap_{k \in \mathbb{N}}
    \mathbb{R}^{k}_{j_k} = \{a\}\), avendo diametro \(0\). Poiché \(f\) è
    \(\mathbb{C}-\)differenziabile in \(z=a\):
    \[
        f(z) = f(a) + f'(a) (z-a) + \varepsilon(z-a)(z-a) \quad \varepsilon(0) =
        0 \quad \varepsilon(w) \to 0 \text{ per } w \to 0
    \]
    Infine notiamo che
    \[
        \int_{\partial R^{k}_{j_k}} f(z) dz = \int_{\partial R^{k}_{j_k} }
        {\left( f(a) + f'{(a)}{(z-a)} \right)} dz + \int_{\partial R^{k}_{j_k} }
        \varepsilon(z-a)(z-a) dz
    \]
    dove il primo termine è uguale a \(0\) poiché la funzione integranda ammette
    primitiva \(f(a)z + \frac{1}{2} f'(a) {(z-a)}^2\). Ora fissiamo \(\sigma >
    0\). Sia \(\delta >0\) tale che \(|w|<\delta \implies
    |\varepsilon(w)|<\sigma\). Per \(k\) sufficientemente grande abbiamo che
    \(\text{diam}R^{k}_{j_k} < \delta  \) e allora \(|z-a|<\delta\) se \(z \in
    \partial R^{k}_{j_k} \). Allora, per tali \(k\):
    \[
        \left| \int_{\partial R^{k}_{j_k} } \varepsilon(z-a) {(z-a)}  dz \right|
        \le \sigma \text{ diam } R^{k}_{j_k} \text{ lungh } R^{k}_{j_k} = \sigma
        \cdot \frac{1}{2^{k}} \text{ diam } R \cdot \frac{1}{2^{k}} \text{ lungh
        }\partial R
    \]
    Ricordando che 
    \[
        \left| \int_{\partial R^{k}_{j_k} } f  \right| \ge \frac{1}{4^{k}}
    \]
    e mettendo assieme i pezzi otteniamo che \(A = 0\) (per l'arbitrarietà di
    \(\sigma\)), che è assurdo
\end{proof}
Estendiamo ora il risultato
\begin{proposition}\label{prp:cauchygoursat}
    Sia \(f : \Omega \to \mathbb{C}\) continua. Sia \(a \in \Omega\) e
    supponiamo che \(f\) sia \(\mathbb{C}-\)differenziabile in \(\Omega \sminus
    \{a\} \). Allora 
    \[
        \int_{\partial R} f = 0
    \]
    per ogni rettangolo chiuso \(R\) in \(\Omega\) 
\end{proposition}
\begin{proof}
\begin{itemize}[label = --]
    \item Se \(a \not\in R\) allora si usa il lemma di Cauchy-Goursat
    \item Se \(a \in \partial R\) si approssima \(R\) con una successione
        \(R_{n}\) di rettangoli internamente (come in
        figura~\ref{fig:cauchygoursatapprox})
        Risulta poi 
        \[
            0 = \int_{\partial R_{n}} f \to \int_{\partial R} f
        \]
        (possiamo pensare ogni \(\partial R_{n}\) parametrizzato su un
        intervallo fisso \([a,b]\) e c'è convergenza uniforme)
\begin{figure}[ht]
    \centering
    \incfig{cauchygoursatapprox}
    \caption{Approssimazione di \(R\) con \(R_{n}\) per \(a \in \partial R\) e
    decomposizione per \(a \in \mathring{R}\)}
    \label{fig:cauchygoursatapprox}
\end{figure}
    \item \(a \in \mathring{R}\) Scomponendo \(R\) in due rettangoli \(R_{1}\) e
        \(R_{2}\) come in figura, con \(a \in \partial R_{1} \cap \partial
        R_{2}\) si ha che
        \[
            \int_{\partial R } f = \int_{\partial R_{1}} f + \int_{\partial
            R_{2}} f = 0
        \]
        per il caso precedente
\end{itemize}
\end{proof}

\begin{theorem}[Formula di Cauchy per il rettangolo] Sia \(f : \Omega \to
    \mathbb{C}\) \(\mathbb{C}-\)differenziabile. Sia \(R \subseteq \Omega \) un
    rettangolo chiusa. Allora per ogni \(w \in \mathring{R}\) risulta
    \[
        f(w) =\frac{1}{2\pi i} \int_{\partial R} \frac{f(z)}{z-w}dz
    \]
\end{theorem}
\begin{proof}
    Sia 
    \[
        g(z) := \begin{cases}
            \frac{f(z) - f(w)}{z-w} & z \neq w \\
            f'(w) & z = w
        \end{cases}
    \]
    allora poiché \(f\) è \(\mathbb{C}-\)differenziabile, \(g\) è continua
    \(\Omega\). Inoltre \(g\) è \(\mathbb{C}-\)differenziabile in \(\Omega
    \sminus \{w\} \). Allora per la proposizione~\ref{prp:cauchygoursat} si ha
    che
    \[
        0 = \int_{\partial R} g(z) dz = \int_{\partial R} \frac{f(z) -
        f(w)}{z-w} dz = \int_{\partial R} \frac{f(z)}{z-w} dz - f(w)
        \int_{\partial R} \frac{1}{z-w} dz
    \]
    Infine poiché \(\int_{\partial R} \frac{dz}{z-w} = 2\pi i \) per \(w \in
    \mathring{R}\) si ottiene la tesi.
\end{proof}

\begin{theorem}
    Sia \(\Omega \subseteq \mathbb{C} \) aperto e \(f : \Omega \to \mathbb{C}\)
    una funzione \(\mathbb{C}-\)differenziabile. Allora \(f\) è analitica in
    \(\Omega\) 
\end{theorem}
\begin{proof}
    Fissiamo \(a \in \Omega\) e mostriamo che \(f\) è sviluppabile in serie di
    potenze in un intorno di \(a\).
\begin{figure}[ht]
    \centering
    \incfig[.6]{diffanalitica}
    \caption{diffanalitica}
    \label{fig:diffanalitica}
\end{figure}
Sia \(R\) un rettangolo chiuso con \(a \in \mathring{R}\) e \(R \subseteq \Omega
\). Sia \(D_r{(a)}\) con \(\overline{D_r{(a)}} \subseteq \mathring{R} \).
Consideriamo \(z \in D_r{(a)}\). Sappiamo per la formula di Cauchy per il
rettangolo 
\[
    f{(z)} = \frac{1}{2\pi i} \int_{\partial R} \frac{f(\zeta)}{\zeta-z} d\zeta
\]
Ora comunque presi \(z \in D_r{(a)}\) e \(\zeta \in \partial R\) 
\[
    \frac{1}{\zeta - z} = \frac{1}{\zeta - a - (z-a)} = \frac{1}{\zeta - a}
    \cdot \frac{1}{1 - \frac{z-a}{\zeta-a}}
\]
e poiché
\[
    \left| \frac{z-a}{\zeta - a} \right| \le \alpha < 1
\]
per un opportuno \(\alpha\). Allora abbiamo
\[
    \frac{1}{1 - \frac{z-a}{\zeta-a}} = \sum_{n=0}^{\infty} {\left( \frac{z-a}{\zeta
    - a} \right)}^{n}
\]
e quindi
\[
    f{(z)} = \frac{1}{2\pi i} \int_{\partial R} \sum_{n=0}^{\infty}
    \frac{f{(\zeta)}}{{(\zeta - a)}^{n+1}} {(z-a)}^{n} d\zeta
\]
Risulta che
\[
    \left| \frac{f{(\zeta)}}{{(\zeta - a)}^{n+1}} {(z-a)}^{n} \right| \le
    {\left( \max_{\partial R} |f| \right)} \frac{1}{|\zeta - a|} \cdot \alpha
    ^{n}
\]
e quindi poiché \(\alpha < 1\) si ha convergenza globale e si può scambiare il
segno di serie e integrale ottenendo
\[
    f{(z)} = \frac{1}{2\pi i} \sum_{n=0}^{\infty} {\left( \int_{\partial R}
    \frac{f{(\zeta)}}{{(\zeta -a)}^{n+1}} d \zeta \right)} {(z-a)}^{n} =
    \sum_{n=0}^{\infty} c_{n} {(z-a)}^{n}
\]
dove \(\displaystyle c_{n} = \frac{1}{2\pi i} \int_{\partial R}
\frac{f{(\zeta)}}{{(\zeta - a)}^{n+1}} d \zeta \) 
\end{proof}

Abbiamo allora dimostrato che \(f\) è \(\mathbb{C}-\)differenziabile se e solo
se è analitica. Si parla anche di funzioni \textbf{olomorfe} e si indica con \(f
\in \mathcal{H}{(\Omega)}\) 
\begin{remark}
    Se \(f\) è olomorfa allora \(f\) è infinitamente differenziabile in senso
    complesso. Inoltre se guardiamo \(f\) come funzione reale \(f:
    \mathbb{R}^{2} \to \mathbb{R}^{2}\) allora \(f\) è \(C^{\infty}\)
\end{remark}

Sia \(f : \Omega \to \mathbb{C}\). Abbiamo già visto che se per ogni \(\gamma\)
chiusa in \(\Omega\) si ha \(\int_{\gamma} f = 0\) allora \(f\) ammette
primitiva in \(\Omega\), ossia esiste \(F : \Omega \to \mathbb{C}\) tale che
\(F' = f\). In particolare \(F\) è \(\mathbb{C}-\)differenziabile e quindi
olomorfa, ma quindi anche \(f\) è olomorfa.
% TODO: mettere in forma più carina evidenziando la tesi

Ricordando la dimostrazione del teorema~\ref{thm:1_forme} che dice che se
l'integrale su ogni curva chiusa di una forma differenziale è nullo allora la
forma è esatta. Similmente se per ogni curva chiusa \(\gamma\) si ha che
l'integrale su \(\gamma\) di \(f\) è nullo allora \(f\) ammette primitiva,
costruita nello stesso modo, ossia
\[
    F{(z)} = \int_{\gamma_z} f(\zeta) d\zeta
\]
dove \(\gamma_z\) è una curva che unisce \(z_{0}\) a \(z\), con \(z_{0}\)
fissato. Richiedere che l'integrale su ogni curva chiusa sia nullo serve perché
questa funzione sia ben definita.

Supponiamo ora di avere solamente l'ipotesi
\[
    \forall R \subseteq \Omega \quad \int_{\partial R} f = 0
\]
Otteniamo un simile risultato
\begin{theorem}[Morera]
    Sia \(f : \Omega \to \mathbb{C}\) continua e tale che
\[
    \forall R \subseteq \Omega \quad \int_{\partial R} f = 0
\]
Allora \(f\) è olomorfa in \(\Omega\)
\end{theorem}
\begin{proof}
Fissato \(\overline{D}_r{(a)} \subseteq \Omega \), per ogni \(z \in
D_r{(a)}\) costruiamo
\[
    F{(z)} := \int_{\gamma_z} f(\zeta) d\zeta
\]
dove \(\gamma_z\) consiste in due dei lati di un rettangolo con vertici \(a\) e
\(z\). Tecnicamente allora ci sono due curve \(\gamma_z\) e \(\tilde{\gamma}_z\)
con questa proprietà, ma per l'ipotesi posta hanno uguale integrale, quindi
\(F\) è ben posta. Ora come nel caso precedente si dimostra che \(F\) è
\(\mathbb{C}-\)differenziabile e \(F'{(z)} = f{(z)}\) in ogni \(z \in
D_r{(a)}\). Allora \(F\) è \(\mathbb{C}-\)differenziabile in \(D_r{(a)}\). Per
l'arbitrarietà di \(a\) si ha che \(f\) è olomorfa in \(\Omega\).
\end{proof}
\begin{remark}
    Non abbiamo dimostrato in questo caso che \(f\) ammette primitiva su tutto
    \(\Omega\), ma soltanto in un intorno di ogni punto. Questo comunque ci
    permette di mostrare che \(f\) è olomorfa.
\end{remark}

Con quanto appena visto possiamo aggiornare la
Proposizione~\ref{prp:forme-funzioni}. Infatti se \(f\) è olomorfa, in
particolare è \(C^{1}\) e allora \(\omega_{i}\) e \(\omega_{r}\) sono chiuse.
Ora usando il Teorema~\ref{thm:2_forme} vale l'invarianza per omotopia. Allora
\begin{theorem}[Cauchy, forma omotopica]
    Sia \(f \in \mathcal{H}{(\Omega)}\) e \(\gamma_{0}, \gamma_{1}\) curve
    chiuse fra loro omotope in \(\Omega\). Allora
    \[
        \int_{\gamma_{0}} f = \int_{\gamma_{1}} f
    \]
\end{theorem}
\begin{proof}
    vedasi sopra
\end{proof}
Risultato analogo vale per curve omotope rispetto a un'omotopia che fissa gli
estremi.
\begin{corollary}
    Sia \(f \in \mathcal{H}{(\Omega)}\), con \(\Omega\) semplicemente connesso. Allora \(f\) ammette primitiva in \(\Omega\)
\end{corollary}
\begin{remark}
    Segue che \(f \in \mathcal{H}{(\Omega)}\) ammette sempre una primitiva
    locale.
\end{remark}

\begin{theorem}[Formula di Cauchy per il cerchio]
    Sia \(f \in \mathcal{H}{(\Omega)}\) e \(D\) disco aperto con \(\overline{D}
    \subseteq \Omega \). Allora per ogni \(z \in D\)
    \[
        f{(z)} =\frac{1}{2\pi i} \int_{\partial D}
        \frac{f(\zeta)}{\zeta-z}d\zeta
    \]
\end{theorem}
\begin{figure}[ht]
    \centering
    \incfig[.5]{cauchy-disco}
    \caption{cauchy-disco}
    \label{fig:cauchy-disco}
\end{figure}
\begin{proof}
    Sappiamo che
    \[
        f{(z)} = \frac{1}{2\pi i} \int_{\partial R}
        \frac{f(\zeta)}{\zeta-z}d\zeta
    \]
    se \(R\) è un rettangolo chiuso in \(\Omega\), con \(z \in \mathring{R}\).
    Sia \(R \subseteq D \)

    Poiché 
    \[
        \zeta \mapsto \frac{f{(\zeta)}}{\zeta - z}
    \]
    è olomorfa in \(\Omega \sminus \{z\} \) e \(\partial D\) e \(\partial R\)
    sono omotope in \(\Omega \sminus \{z\} \) risulta
    \[
        \int_{\partial D} \frac{f{(\zeta)}}{\zeta - z} d\zeta = \int_{\partial
        R} \frac{f{(\zeta)}}{\zeta - z} d\zeta
    \]
\end{proof}
\begin{remark}
    La formula si estende al caso in cui anziché \(D\) vi è una ``qualunque
    forma'' con bordo omotopo a \(\partial R\) 
\end{remark}

\paragraph{Funzioni olomorfe}
Sia \(f \in \mathcal{H}{(\Omega)}\). Allora se \(a \in \Omega\) sappiamo che in
un intorno di \(z = a\) 
\[
    f{(z)} = \sum_{n=0}^{\infty} c_{n} {(z-a)}^{n}
\]
per opportuni \(\displaystyle c_{n} = \frac{f^{{(n)}}{(a)}}{n!}\) per il teorema
della serie derivata. Domanda naturale è chiedersi quant'è il raggio di
convergenza di tale serie di potenze. 
\begin{proposition}
    La serie \(\sum_{n=0}^{\infty} c_{n} {(z-a)}^{n}\) converge nel più grande
    disco contenuto in \(\Omega\) 
\end{proposition}
\begin{figure}[ht]
    \centering
    \incfig[.4]{radiusconv}
\end{figure}
\begin{proof}
    Sia \(r = d(a, \partial \Omega)\). Fissiamo \(z \in D_{r} {(a)}\) e sia
    \(0< \rho < r\) tale che \(z \in D_{\rho} {(a)}\). Applichiamo la formula di
    Cauchy:
    \[
        f{(z)} = \frac{1}{2\pi i} \int_{\partial D_{\rho} {(a)}}
        \frac{f{(\zeta)}}{\zeta-z}d \zeta
    \]
    e procediamo come nella dimostrazione dell'analiticità delle funzioni
    $\mathbb{C}$-differenziabili. Allora per \(z\) fissato
    \[
        \frac{1}{\zeta - z} = \frac{1}{\zeta - a - {(z-a)}} = \frac{1}{{(\zeta -
        a)}{(1 - \frac{z-a}{\zeta-a})}}
    \]
    e poiché \(\displaystyle \left| \frac{z-a}{\zeta-a} \right| =
    \frac{1}{\rho}|z-a| < 1\) e indipendente da \(\zeta\). Quindi
    \begin{align*}
        f{(z)} &= \frac{1}{2 \pi i} \int_{\partial D_{\rho} {(a)}}
        \sum_{n=0}^{\infty} \frac{f{(\zeta)}}{\zeta -a} {\left( \frac{z-a}{\zeta
    -a} \right)}^{n} d \zeta = \\ &= \sum_{n=0}^{\infty} {\left( \frac{1}{2\pi i}
                \int_{D_\rho{(a)}} \frac{f{(\zeta)}}{{(\zeta-a)}^{n+1}} d
        \zeta\right)} {(z-a)}^{n}
    \end{align*}
    e quindi questa deve essere la serie di taylor
\end{proof}
Dalla dimostrazione scende anche che
\[
    \frac{f^{{(n)}}{(a)}}{n!} = c_{n} = \frac{1}{2\pi i} \int_{\partial D_{\rho}
    {(a)}} \frac{f{(\zeta)}}{{(\zeta - a)}^{n+1}} d \zeta
\]
e poiché la funzione integranda è olomorfa in \(\Omega \sminus \{a\} \) e le
curve \(\partial D\) e \(\partial D_{rho}{(a)} \) sono omotope, per \(D\)
qualsiasi \(a \in D \subseteq \overline{D} \subseteq \Omega  \)  si ottiene il seguente corollario
\begin{corollary}
    Sia \(\overline{D} \subseteq \Omega \) disco chiuso. Allora per ogni \(z \in
    D\) 
    \begin{equation}\label{eq:cauchy-derivata}
        f^{{(n)}}{(z)} = \frac{n!}{2\pi i} \int_{\partial D} \frac{f{(\zeta)}}{{(\zeta
        - z)}^{n+1}} d \zeta
    \end{equation}
\end{corollary}
\begin{remark}
    Per \(n = 0\) si trova proprio la formula di Cauchy per il cerchio.
\end{remark}
\begin{remark}
    Il corollario può essere ottenuto dalla formula di Cauchy per il cerchio per
    derivazione sotto il segno di integrale
    \[
        \frac{d}{dz} \frac{1}{\zeta - z} = \frac{1}{{(\zeta - z)}^2}
    \]
\end{remark}
\begin{proposition}
    Per ogni \(a \in \Omega\) e \(\overline{D_\rho}{(a)} \subseteq \Omega \)
    \[
        \frac{|f^{{(n)}}{(a)}|}{n!} \le {\rho^{-n}} \max_{\partial D_{\rho}{(a)}
        }|f|
    \]
\end{proposition}
\begin{proof}
    Da~\eqref{eq:cauchy-derivata} si ottiene che
    \[
        \frac{|f^{{(n)}}{(a)}|}{n!} \le  \frac{1}{ 2\pi } \frac{1}{\rho^{n+1}}
        {\left( \max_{\partial D_{\rho} } |f| \right)} \cdot \underbrace{\text{lungh}
        \partial D_{\rho}{(a)}}_{2\pi \rho} 
    \]
\end{proof}
\begin{theorem}[Liouville]
    Se \(f \in \mathcal{H}{(\mathbb{C})}\) è limitata, allora \(f\) è costante 
\end{theorem}
\begin{proof}
    Fissiamo \(a \in \mathbb{C}\) e \(\rho > 0\) arbitrario. Consideriamo lo
    sviluppo di Taylor di centro \(z = a\) 
    \[
        f{(z)} = \sum_{n=0}^{\infty} c_{n} {(z-a)}^{n} \quad c_{n} =
        \frac{f^{{(n)}}{(a)}}{n!}
    \]
    è valido per ogni \(z \in \mathbb{C}\).

    Sappiamo per la proposizione precedente che
    \[
        |c_{n}| \le \frac{1}{\rho^{n}} \max_{\partial D_{\rho}{(a)}} |f| \le
        \rho^{-n} \max_{\mathbb{C}} |f| \to 0 \text{ per } \rho \to \infty
    \]
    da cui \(c_{n} = 0\) per \( n \ge 1\) da cui \(f{(z)}  = c_{0}\) è costante.
\end{proof}
\begin{corollary}[Teorema Fondamentale dell'Algebra]
    Sia 
    \[
        p_{n}{(z)} = a_{n} z^{n} + a_{n-1} z^{n-1} +~\dots + a_{0}
    \]
    con \(a_{n} \neq\) 0. Allora \(p_{n}\) ha almeno uno zero
    
\end{corollary}
\begin{proof}
    Per assurdo sia \(p_{n}{(z)} \neq 0\) per ogni \(z\). Allora sia
    \[
        f{(z)} = \frac{1}{p_{n}{(z)}}
    \]
    Si ha che 
    \[
        p_{n}{(z)} = a_{n} z^{n} {\left( 1 + \frac{a_{n-1}}{a_{n} z} +
        \frac{a_{n-2}}{a_{n} z^2} +~\dots+ \frac{a_{0}}{a_{n} z^{n}} \right)}
    \]
    il cui valore assoluto va a \(+\infty\) per \(|z| \to \infty\). Allora \(f\)
    è limitata perché \(\displaystyle \lim_{|z| \to \infty} \frac{1}{p_{n}{(z)}}
    = 0\).
    Ma allora per il teorema di Liouville \(f\) è costante, assurdo.
\end{proof}

\subsection{Sviluppo di \textbf{Laurent}}
Con
\begin{equation}\label{eq:serie-bilatera}
    \sum_{m=-\infty}^{\infty} c_m {(z-a)}^{m}
\end{equation}
intendiamo
\begin{equation}\label{eq:serie-bilatera-2}
    \sum_{m=-1}^{-\infty} c_m {(z-a)}^{m} + \sum_{m=0}^{\infty} c_m {(z-a)}^{m}
\end{equation}
cioè diremo che la serie~\eqref{eq:serie-bilatera} converge (uniformemente,
assolutamente, ecc) se tali sono le serie~\eqref{eq:serie-bilatera-2}.

Siano \(0 \le r_{1} < r_{2} < +\infty\) e \(a \in \mathbb{C}\). Consideriamo
\[
    \Omega = \{z \in \mathbb{C}: r_{1} < |z-a| < r_{2}\} 
\]
Nel caso di \(r_{1} = 0, r_{2}= r\) allora si indica anche \(D^{*}_r{(a)} =
D_r{(a)} \sminus \{a\} \) 
\begin{theorem}
    Sia \(f \in \mathcal{H}{(\Omega)}\). Allora esiste unica una successione
    \(c_{n}\) tale che 
    \[
        f{(z)} = \sum_{m=-\infty}^{\infty} c_{m} {(z-a)}^{m} \quad z \in \Omega
    \]
    Tale serie converge assolutamente in modo uniforme sui compatti di
    \(\Omega\).
    Inoltre si ha che
    \begin{equation}\label{eq:coeff-laurent}
        c_m = \frac{1}{2\pi i} \int_{\partial D_\rho} \frac{f{(\zeta)}}{{(\zeta
        - a)}^{m+1}} d \zeta
    \end{equation}
    per un qualunque \(\rho \in (r_{1}, r_{2})\) 
\end{theorem}
\begin{proof}
\begin{figure}[ht]
    \centering
    \incfig[.4]{disegno_laurent}
\end{figure}
    Siano \(r_{1} < \rho_{1} < \rho_{2} < r_{2}\) come in figura. Rappresentiamo \(f\) in forma
    integrale in \(\{z : \rho_{1} < |z-a| < \rho_{2}\} \). Sia 
    \[
        g{(\zeta)} = \begin{cases}
            \frac{f{(\zeta)} - f{(z)}}{\zeta - z} & \zeta \neq z \\
            f'{(z)} & \zeta = z
        \end{cases}
    \]
    Allora \(g\) è olomorfa in \(\Omega\), infatti se \(\zeta \neq z\), poiché
    \(\displaystyle f{(\zeta)} = \sum_{n=0}^{\infty} a_{n} {(\zeta - z)}^{n} \),
    con \(a_{0} = f{(z)}\),
    \[
        \frac{f{(\zeta)} - f{(z)}}{\zeta -z} = \frac{1}{\zeta-
        z}\sum_{n=1}^{\infty} a_{n} {(\zeta -z)}^{n} = \sum_{n=1}^{\infty}
        a_{n}{(\zeta - z)}^{n-1} 
    \]
    che è una funzione olomorfa anche in un intorno di \(z\) e vale
    \(a_{1} = f'{(z)}\) in \(\zeta = z\).

    Ne consegue che per il teorema di Cauchy
    \[
        \partial D_{\rho_{1}} \sim \partial D_{\rho_{2}} \implies \int_{\partial
        D_{\rho_{1}}{(a)}} g = \int_{\partial D_{\rho_{2}}{(a)}} g
    \]
    ma allora
    \[
        \int_{\partial D_{\rho_{1}}{(a)}} \frac{f{(\zeta)}}{\zeta- z} d\zeta -
        f{(z)} \int_{\partial D_{\rho_{1}}{(a)}} \frac{d\zeta}{\zeta - z} = 
        \int_{\partial D_{\rho_{2}}{(a)}} \frac{f{(z)}}{\zeta - z} d\zeta -
        f{(z)} \int_{\partial D_{\rho_{2}}{(a)}} \frac{d\zeta}{\zeta - z} = 0
    \]
ma \(\displaystyle \int_{\partial D_{\rho_{1}}} \frac{1}{\zeta -z} d\zeta = 0\)
poiché la funzione \(\zeta \mapsto \frac{1}{\zeta -z}\) è olomorfa in un
disco contenente \(\partial D_{\rho_{1}} \) e non contenente \(z\) e in tale
disco \(D_{\rho_{1}} \sim 0\). Allora 
\[
- f{(z)} \int_{\partial D_{\rho_{2}}} \frac{d \zeta}{\zeta - z} d \zeta = -f{(z)} 2 \pi i = \int_{\partial D_{\rho_{1}}{(a)}} \frac{f{(\zeta)}}{\zeta
    - z} d\zeta - \int_{\partial D_{\rho_{2}}{(a)}} \frac{f{(\zeta)}}{\zeta - z}
    d \zeta
\]
e quindi ora per il secondo passo usiamo la rappresentazione
\[
    f{(z)} = \frac{1}{2\pi i} {\left( \int_{\partial D_{\rho_{2}} {(a)}}
    \frac{f{(\zeta)}}{\zeta -z} d \zeta - \int_{D_{\rho_{1}} {(a)}}
\frac{f{(\zeta)}}{\zeta - z} d \zeta \right)} 
\]
Sia ora \(\zeta \in \partial D_{\rho_{1}} {(a)}\) e allora
\[
    \frac{1}{\zeta -z} = \frac{1}{\zeta -a - {(z- a)}} = - \frac{1}{{(z-a)} -
    {(\zeta -a)}} = -\frac{1}{{(z-a)}{\left( 1 - \frac{\zeta-a}{z -a} \right)} }
\]
e poiché \(\left| \frac{\zeta -a}{z-a} \right|= \frac{\rho_{1}}{z-a}  < 1 \)
abbiamo che la precedente
\[
    \frac{1}{ \zeta -z } = -\frac{1}{z-a} \sum_{n=0}^{\infty} {\left(
    \frac{\zeta -a}{z- a} \right)} ^{n} 
\]
converge uniformemente per \(\zeta \in \partial D_{\rho_{1}} \) e quindi
\[
    - \int_{\partial D_{\rho_{1}}{(a)}} \frac{f{(\zeta)}}{\zeta - z} d \zeta =
    \sum_{n=0}^{\infty} {\left( \int_{\partial D_{\rho_{1}} }
    f{(\zeta)}{(\zeta-a)}^{n}d \zeta \right)}  \frac{1}{{(z-a)}^{n+1}}
\]
da cui
\[
    -\frac{1}{2\pi i} \int_{\partial D_{\rho_{1}} } \frac{f{(\zeta)}}{\zeta -
    z}d \zeta = \sum_{n=0}^{\infty} {\left( \frac{1}{2\pi i}\int_{\partial
    D_{\rho_{1}} } \frac{f{(\zeta)}}{{(\zeta -a)}^{-n}}d \zeta \right)}
    {(z-a)}^{-{(n-1)}} = \cdots
\]
se ora \( m := -{(n+1)}\) si ha che
\[
    \dots = \sum_{m=-1}^{\infty} \underbrace{\left(\frac{1}{2 \pi i } \int_{\partial
        D_{\rho_{1}}{\left( a \right)} } \frac{f{(\zeta)}}{{(\zeta -a)}^{m+1}}d
\zeta \right)}_{c_m}{(z-a)}^{m}
\]
che è esattamente la forma promessa dal teorema per le potenze negative.

Consideriamo ora invece \(\zeta \in \partial D_{\rho_{2}} {(a)}\) e allora
\[
    \frac{1}{\zeta -z} = \frac{1}{\zeta - a - {(z-a)}} = \frac{1}{{(\zeta
    -a)}{\left( 1 - \frac{z -a}{\zeta -a} \right)} }
\]
e come prima poiché \(\left| \frac{z -a}{\zeta -a} \right|\le \frac{|z-a|}{\rho_{2}} <
1 \)  la precedente 
\[
    \frac{1}{\zeta -z} = \frac{1}{\zeta -a} \sum_{n=0}^{\infty} {\left( \frac{z
    -a}{\zeta -a} \right)} ^{n} 
\]
converge uniformemente per \(\zeta \in \partial D_{\rho_{2}} \) e quindi
\[
    \int_{\partial D_{\rho_{2}}{(a)}} \frac{f{(\zeta)}}{\zeta - z} d \zeta =
    \sum_{n=0}^{\infty} {\left( \int_{\partial D_{\rho_{2}} }
            \frac{f{(z)}}{{(\zeta -a)}^{n+1}}d \zeta \right)} {(z -a)}^{n}
\]
da cui
\[
    \frac{1}{2\pi i} \int_{\partial D_{\rho_{2}} } \frac{f{(\zeta)}}{\zeta - z}
    d \zeta = \sum_{n=0}^{\infty} \underbrace{\left( \frac{1}{2 \pi i} \int_{\partial
    D_{\rho_{2}} } \frac{f{(\zeta)}}{{(\zeta - a)}^{n+1}}d \zeta \right)}_{c_{n}}  {(z-
a)}^{n}
\]
Infine poiché \(\zeta \mapsto \frac{f{(z)}}{{(\zeta-a)}^{n+1}}\) è olomorfa in
\(\Omega\) l'espressione dei coefficienti coincide con
\[
    c_{m} = \frac{1}{2\pi i} \int_{\partial D_\rho {(a)}}
    \frac{f{(\zeta)}}{{(\zeta -a)}^{m+1}} d \zeta
\]
per \(\rho\) arbitrario con \(r_{1} < \rho < r_{2}\) e allora
\[
    f{(z)} = \sum_{m=-\infty}^{\infty} c_{m} {(z-a)}^{m}
\]
\end{proof}
\begin{proof}[Convergenza assoluta uniforme]
    La validità della convergenza dimostrata assicura automaticamente la
    convergenza assoluta uniforme sui compatti di \(\Omega\). Infatti:
    Per ipotesi
    \[
        \sum_{m=0}^{\infty} c_{m} {(z-a)}^{m}
    \]
    converge per \(|z-a| \in (r_{1},r_{2})\). Ne segue che il raggio di
    convergenza è almeno \(r_{2}\); pertanto la serie converge assolutamente in
    modo uniforme nei compatti di \(D_{r_{2}} {(a)}\) e quindi in particolare
    sui compatti di \(\Omega\). 

    Si ha inoltre che
    \[
        \sum_{m=-1}^{-\infty} c_{m} {(z-a)}^{m} = \sum_{n=-1}^{\infty} c_{-n}
        {(z-a)}^{-n} = \sum_{n=1}^{\infty} c_{-n} \zeta^{n}
    \]
    con \(\zeta - \frac{1}{z-a}\). L'ultima serie è una serie di potenze e
    converge per \(|\zeta| \in (\frac{1}{r_{2}}, \frac{1}{r_{1}})\) con la
    convenzione \(\frac{1}{0} = \infty\) e \(\frac{1}{\infty} = 0\) e quindi il
    raggio di convergenza è \(R \ge \frac{1}{r_{1}}\) per cui converge in modo
    assoluto uniforme sui dischi \(\overline{D_{\rho} }{(0)}\) con \(\rho <
    \frac{1}{r_{1}}\) cioè sugli insiemi
\[
    \{z \in \mathbb{C} : |\zeta| \le \rho\} 
\]
    e quindi la serie \(\sum_{m=-1}^{-\infty} c_{m} {(z-a)}^{m} \) converge in
    modo assoluto uniforme sugli insiemi
\[
    \{z \in \mathbb{C} : \frac{1}{|z-a|} \le \rho\} = \{z \in \mathbb{C} : |z-a|
    \ge \frac{1}{\rho}\}
\]
    con \(\frac{1}{\rho}\) un qualunque valore maggiore di \(r_{1}\). Quindi la
    convergenza è assoluta uniforme sugli insiemi
\[
    \{z \in \mathbb{C}: |z-a| \ge \gamma\} \quad \gamma > r_{1}
\]
    in particolare si ha convergenza assoluta uniforme sui compatti di
    \(\Omega\).
\end{proof}
\begin{proof}[Unicità dei \(c_{m}\) ]
    Mostriamo che se \({(c_{m})}_{m \in \mathbb{Z}}  \) sono tali che 
\[
    f{(z)} = \sum_{m=-\infty}^{\infty} c_{m} {(z-a)}^{m}, \quad z \in \Omega 
\]
    allora, necessariamente sono dati dalla formula dell'enunciato. Fissiamo
    \(\overline{m} \in \mathbb{Z}\) e calcoliamo
\[
    \int_{\partial D_{\rho} {(a)}} \frac{f{(z)}}{{(z-a)}^{\overline{m}+1}} dz
\overset{\text{conv. unif.}}{=} \sum_{m=-\infty}^{\infty} c_{m} \int_{\partial
D_{\rho} {(a)}} \frac{{(z-a)}^{m}}{{(z-a)}^{\overline{m}+1}} dz 
\]
    l'argomento dell'integrale è una potenza. Ha primitiva se \(m -\overline{m}
    - 1 \neq -1\) e quindi tutti gli integrali della serie sono nulli tranne per
    \(m - \overline{m} - 1 = -1\) ossia \(m = \overline{m}\). Allora
\[
    \int_{\partial D_{\rho} {(a)}} \frac{f{(z)}}{{(z-a)}^{\overline{m}-1}}dz =
    c_{\overline{m}} \int_{\partial D_{\rho} {(a)}} \frac{1}{z-a} dz = 2\pi i
    c_{\overline{m}} 
\]
    e quindi
\[
    c_{\overline{m}} = \frac{1}{2\pi i} \int_{\partial D_{\rho} {(a)}}
    \frac{f{(z)}}{{(z-a)}^{\overline{m}+1}} dz
\]
\end{proof}

\begin{example}
    Sia \(\displaystyle f{(z)} = \frac{1}{z{(z-1)}}\). Allora \(z\neq 0\) e
    \(z\neq 1\). Consideriamo i seguenti \(\Omega_{1}\) e \(\Omega_{2}\) 
    \begin{align*}
        \Omega_{1} &= \{z \in \mathbb{C} : 0 < |z| < 1\}  \\
        \Omega_{2} &= \{z \in \mathbb{C} : |z| > 1\}
    \end{align*}
    Consideriamo prima \(\Omega_{1}\). Allora
\[
    -\frac{1}{z} \frac{1}{1-z} = -\frac{1}{z} {(1 + z + z^2 + z^3 + \dots)} =
    -\frac{1}{z} - 1 - z - z^2 - \dots
\]
che è lo sviluppo di Laurent di \(f\) in \(\Omega_{1}\)

    In \(\Omega_{2}\) si ha che
\[
    \frac{1}{z{(z-a)}} = \frac{1}{z^2 {\left(1 - \frac{1}{z}\right)}} =
    \frac{1}{z^2} \sum_{n=0}^{\infty} \frac{1}{z^{n}} = \frac{1}{z^2} +
    \frac{1}{z^3} + \frac{1}{z^{4}} + \dots
\]
    Che è lo sviluppo di Laurent di \(f\)  in \(\Omega_{2}\) 
\end{example}

\begin{definition}{Sviluppo di Laurent relativo a un punto}
    Sia \(f \in \mathcal{H}{(D_{r}^{*} {(a)})}\) con \(D_{r}^{*} {(a)} =
    D_r{(a)} \sminus \{a\} \). Lo sviluppo di cui al teorema precedente è detto
    \textbf{sviluppo in serie di Laurent} relativo al punto \(z=a\) (non dipende
    da \(r\))
\end{definition}

\begin{eser}
    Sia \(\displaystyle f{(z)} = \frac{3}{iz^2 - z + 2i}\). Si calcoli:
\begin{itemize}[label = --]
    \item La serie di Laurent centrata in entrambi i punti in cui \(f\) non è
        definita.
    \item La serie di Taylor centrata in \(z=0\) 
\end{itemize}
\end{eser}

\begin{eser}
    Sia \(f \in \mathcal{H}{(D_{r} {(a)})}\) e \(N \in \mathbb{N}\). Allora
    esiste unico \(P_N{(z-a)}\) polinomio di grado al più \(N\) tale che 
\[
    \lim_{z \to a} \frac{f{(z)} - P_N{(z-a)}}{{(z-a)}^{N}} =0
\]
    Infatti il polinomio di Taylor di grado \(N\) soddisfa questa condizione
\begin{proof}
    \(f\) è olomorfa, quindi
\[
    f{(z)} = \sum_{n=0}^{\infty} c_{n} {(z-a)}^{n}, \quad c_{n} =
    \frac{f^{{(n)}}{(a)}}{n!}
\]
    e quindi
\[
    f{(z)} = P_N{(z-a)} + R_N{(z-a)} = P_N{(z-a)} + \sum_{n=N+1}^{\infty}
    c_{n}{(z-a)}^{n}
\]
\[
    R_N{(z-a)} = {(z-a)}^{N+1}\sum_{n=N+1}^{\infty} c_{n}{(z-a)}^{n-N-1} 
\]
    per cui la serie dà luogo ad una funzione olomorfa \(g\); quindi
\[
    \frac{f{(z)} - P_N{(z-a)}}{{(z-a)}^{N}} = {(z-a)}g{(z)} \to 0 \quad \text{
    per } z \to a
\] 
    e \(P_N\) è l'unico con tale proprietà. Infatti se \(Q_N\) avesse la stessa
    proprietà, allora
\[
    \frac{P_N{(z-a)} - Q_N{(z-a)}}{{(z-a)}^{N}} = \frac{P_N - f + f -
    Q_N}{{(z-a)}^{N}}
\]
    Sia ad esempio \(N = 2\), allora questo significa
    \begin{align*}
        P_N{(z-a)} &= a_{0} + a_{1}{(z-a)} + a_{2}{(z-a)}^2 \\
        Q_N{(z-a)} &= b_{0} + b_{1}{(z-a)} + b_{2}{(z-a)}^2
    \end{align*}
    Se ora \(\frac{P_N - Q_N}{{(z-a)}^{N}} \to 0\) allora \(P_N{(0)} =
    Q_N{(0)}\) da cui \(a_{0} = b_{0}\). Si può proseguire mostrando che allora
    \(a_{1} = b_{1}\), \(a_{2} = b_{2}\) e così via.
\end{proof}
\end{eser}
\begin{eser}
    Calcolare il Polinomio di Taylor di grado 4 di
\[
    f{(z)} = e^{z} \sin{(z)}
\]
    relativamente a \(z=0\) 
\end{eser}

Sia \(f \in \mathcal{H}{(D^{*}_{r} {(a)})}\) e \(f{(z)} =
\sum_{m=-\infty}^{\infty} c_{m} {(z-a)}^{m} \). Sia inoltre \(\zeta =
\frac{1}{z-a}\) e quindi
\[
    \sum_{m=-1}^{-\infty} c_{m} {(z-a)}^{m} = \sum_{n=1}^{\infty} c_{-n}
    \zeta^{n}
\]
Sappiamo che (vedasi dimostrazione teorema sopra) tale serie ha raggio di
convergenza \(R \ge  \frac{1}{r_{1}}\) e definisce pertanto una funzione
\(\varphi \in \mathcal{H}{(\mathbb{C})}\). Dunque
\[
    \sum_{m=-1}^{-\infty} c_{m} {(z-a)}^{m} = \varphi {\left( \frac{1}{z-a}
    \right)} \text{ è olomorfa in \(\mathbb{C} - \{a\}\)} 
\]
Che motiva la seguente definizione
\begin{definition}{Parte principale}
    La funzione olomorfa in \(\mathbb{C} \sminus  \{a\} \) 
\[
    \sum_{m=-1}^{-\infty} c_{m} {(z-a)}^{m} 
\]
    è detta parte principale dello sviluppo di Laurent di \(f\) relativo a \(z=a\) 
\end{definition}
Sappiamo che tale serie converge assolutamente in modo uniforme sui compatti di
\(\mathbb{C} \sminus \{a\} \). 

\begin{proposition}
    La parte principale è l'unica funzione \(g\) tale che
\begin{itemize}
    \item \(g \in \mathcal{H}{(\mathbb{C} \sminus \{a\} )}\) 
    \item \(g{(z)} \to 0\) per \(|z| \to \infty\) 
    \item \(f-g\) è estendibile in modo olomorfo in un intorno di \(a\) 
\end{itemize}
\end{proposition}
\begin{proof}[pezzettino di dim]
    Sappiamo che \(g{(z)} = \varphi {\left( \frac{1}{z-a} \right)} \) è olomorfa
    in \(\mathbb{C} \sminus \{a\} \). Inoltre \(\lim_{|z|\to \infty} g{(z)} =
    \varphi {(0)} = 0\). Infine \((f-g){(z)} = c_{0}+c_{1}{(z-a)} +
    c_{2}{(z-a)}^2 + \dots\) e il secondo membro definisce una funzione olomorfa
    in un intorno di \(a\).
\end{proof}
\begin{proof}[Unicità]
    % TODO
\end{proof}
\begin{definition}{Residuo}
    Sia \(f \in \mathcal{H}{(D^{*}_r{(a)})}\) e sia \(f{(z)} =
    \sum_{m=-\infty}^{\infty} c_{m} {(z-a)}^{m}\) lo sviluppo di Laurent. Il
    valore \(c_{-1} \) è detto \textbf{residuo} di \(f\)  in \(z=a\) 
\end{definition}

\begin{definition}{Funzioni meromorfe}
    Sia \(\Omega \subseteq  \mathbb{C}\) aperto ed \(E \subseteq  \Omega\) un
    insieme chiuso e discreto (tutti i punti di \(E\) sono isolati).

    Una funzione \(f\) si dice \textbf{meromorfa} in E se \(f \in
    \mathcal{H}{(\Omega \sminus  E)}\) e per ogni \(a \in E\) esiste \(r > 0\) e
    due funzioni \(h, g \in \mathcal{H}{(D_r{(a)})}\) con \(h \not\equiv 0\)
    tali che 
    \[
        h \cdot f = g \quad \text{ in } D_r{(a)}
    \]
\end{definition}
\begin{remark}
    Vorrei dire che \(f\) è rapporto di due funzioni olomorfe, ma non volendomi
    preoccupare della definizione dò invece dale definizione.
\end{remark}
Sia \(f\) meromorfa in \(\Omega\) come da definizione. Sia \(a \in E\); poiché
\(h \not\equiv 0\) 
\[
    h{(z)} = c_N{(z-a)}^{N} + c_{N+1}{(z-a)}^{N+1} + \dots
\]
con \(c_N \neq 0\) e allora da \(hf = g\) si ha
\[
    {(z-a)}^{N}\underbrace{(c_N + c_{N+1}{(z-a)} + \dots)}_{:= \psi {(z)}}
    f{(z)} = g{(z)}
\]
con \(\psi {(a)} = c_N\). In un intorno \(U\) di \(a\) si ha \(\psi \neq 0\) e
quindi
\begin{equation}\label{eq:meromorfa}
    f{(z)} = \frac{g{(z)}}{{(z-a)}^{N}\psi {(z)}} \quad z \in U, z \neq a
\end{equation}
Possiamo quindi dire che la condizione che \(f\) sia meromorfa in \(\Omega\) è
che nell'intorno di ogni \(a \in E\) la funzione \(f\) è quoziente di due
funzioni olomorfe, con denominatore nullo al più in \(a\).
\begin{remark}
    Ovviamente \(f \in \mathcal{H}{(\Omega)}\) allora \(f\) è meromorfa con \(h
    \equiv 1\) 
\end{remark}

\begin{proposition}\label{prp:meromorfa-troncato-sinistra}
    Sia \(\Omega \subseteq  \mathbb{C}\) aperto ed \(E \subseteq \Omega \)
    chiuso e discreto. Sia \(f \in \mathcal{H}{(\Omega - E)}\). Allora \(f\) è
    meromorfa in \(\Omega\) se e solo se per ogni \(a \in E\) lo sviluppo di
    Laurent di \(f\) in\(a\) è ``troncato a sinistra''.
\end{proposition}
\begin{proof}
In \(\eqref{eq:meromorfa}\) la funzione \(g / \psi\) è olomorfa in \(U\) e
quindi la possiamo sviluppare in serie di Taylor di centro \(z=a\) ottenendo
\[
    \frac{g{(z)}}{\psi {(z)}} = \sum_{n=0}^{\infty} a_{n} {(z-a)}^{n} \implies
    f{(z)} = \frac{a_{0}}{{(z-a)}^{N}} + \frac{a_{1}}{{(z-a)}^{N-1}} +
    \frac{a_{2}}{{(z-a)}^{N-2}} + \dots
\]
in un opportuno intorno bucato di \(a\). Per l'unicità dello sviluppo di Laurent
di \(f\) in \(a\), il precedente è lo sviluppo di Laurent di \(f\) in \(a\).
Evidentemente è ``troncato a sinistra''.

Viceversa se \(f\) ha uno sviluppo come sopra allora
\[
    f{(z)} = \underbrace{\frac{1}{{(z-a)}^{N}}}_{:= \frac{1}{\tilde{h}{(z)}}}
    \underbrace{\left( a_{0} + a_{1}{(z-a)} + a_{2}{(z-a)}^2 +
    \dots\right)}_{:=\tilde{g}{(z)}} 
\]
con \(\tilde{h}, \tilde{g}\) olomorfe. Allora \(f\) è meromorfa.
\end{proof}
\begin{theorem}[Estensione di Riemann]\label{thm:estensione-riemann}
    Sia \(f \in \mathcal{H}{(D^{*}_r{(a)})}\) tale che 
\[
    \lim_{z \to a} {(z-a)}f{(z)} = 0
\]
    Allora \(f\) è estendibile in modo olomorfo a tutto \(D_r{(a)}\) 
\end{theorem}
\begin{corollary}
    Dal teorema si ricava che se \(f \in \mathcal{H}{(D^{*}_r{(a)})}\) allora
    \(f\) è estendibile in modo olomorfo a \(D_r{(a)}\) se e solo se \(f\)
    è limitata in un intorno di \(a\) 
\end{corollary}
\begin{proof}\( \)
\begin{itemize}
    \item[\(\implies \)] ovvio perché l'estensione è continua in \(z=a\) 
    \item[\(\impliedby \)] Se \(f\) è limitata in un intorno di \(a\) allora
        \({(z-a)}f{(z)} \to 0\) per \(z-a\) e quindi per il teorema \(f\) è
        estendibile
\end{itemize}
\end{proof}
\begin{proof}[Dimostrazione del Teorema~\ref{thm:estensione-riemann}]
    Sia
\[
    f{(z)} = \sum_{m=-\infty}^{\infty} c_{m} {(z-a)}^{m} 
\]  Dimostriamo che \(c_{m} = 0\) per \(m < 0\). Ricordando che \(c_{m}\) è dato
da~\eqref{eq:coeff-laurent} si ha, prendendo \(\rho\) aribtrario, con \(\rho \in
(0, z)\) 
\begin{equation}\label{help:est_riemann}
    |c_{m}| \le \frac{1}{2\pi} \max_{z \in \partial D_\rho {(a)}}
    \frac{|f{(z)}|}{|{(z-a)}|^{m+1}} \cdot 2\pi \rho =
    \frac{1}{\rho^{m}}\max_{\partial D_\rho{(a)}} |f|
\end{equation}
Per ipotesi fissato \(\varepsilon > 0\) esiste un \(\rho_{\varepsilon} > 0\)
tale che 
\[
    \forall \eta < \rho_\varepsilon, \quad \forall z \in D_\eta^*{(a)} \quad
    |{(z-a)}f{(z)}| < \varepsilon
\]
in particolare per ogni \(\rho < \rho_\varepsilon\) e per ogni \(z \in
\partial D^{*}_\rho {(a)}\) 
\[
    |{(z-a)}f{(z)}| < \varepsilon \implies \rho|f{(z)}| < \varepsilon \implies
    |f{(z)}| < \frac{\varepsilon}{\rho}
\]
Allora, fissato \(\varepsilon>0\) e posto \(\rho_\varepsilon\) come sopra, si ha
\(\forall \rho < \rho_\varepsilon\), da \eqref{help:est_riemann} otteniamo
\[
    |c_m| \le \frac{\varepsilon}{\rho^{m}} \frac{1}{\rho}
\]
e quindi per \(m < 0\) si ha
\begin{align*}
    m &= -1 \implies |c_{-1} | \le \varepsilon \text{ quindi per arbitrarietà di
    \(\varepsilon\) } c_{-1} = 0 \\
        m &< -1 \implies |c_{m}| \le \varepsilon \rho^{-{(m+1)}} \to 0 \text{
        per } \rho \to 0 \implies c_{m} = 0
\end{align*}
\end{proof}
\begin{definition}{Singolarità eliminabile}
    Se \(f \in \mathcal{H}{(D^{*}_r{(a)})}\) è estendibile in modo olomorfo a
    tutto \(D_r{(a)}\), si dice che \(z=a\) è una singolarità
    \textbf{eliminabile}
\end{definition}
\begin{proposition}
    Sia \(\Omega \subseteq \mathbb{C} \) aperto e \(E \subseteq \Omega \) chiuso
    e discreto. Sia \(f \in \mathcal{H}{(\Omega - E)}\). Allora la funzione
    \(f\) è meromorfa se e solo se per ogni \(a \in E\) vale una delle seguenti
    proprietà:
\begin{enumerate}[label = \alph*)]
    \item \(f\) è limitata in un intorno di \(a\) \emph{(singolarità
        eliminabile)}
    \item \(\lim_{z \to a} |f{(z)}| = +\infty\) \emph{(polo)}
\end{enumerate}
\end{proposition}
\begin{proof}
    Sia \(a \in E\). Se \(f\) è meromorfa, per la
    proposizione~\ref{prp:meromorfa-troncato-sinistra} in un intorno di \(a\)
    sia ha \[
    f{(z)} = c_{m_{0}} {(z-a)}^{m_{0}} + c_{m_{0}+1}{(z-a)}^{m_{0}+1} + \dots
    \quad m_{0} \in \mathbb{Z},\,\, c_{m_{0}} \neq 0
\]
    ossia lo sviluppo è troncato a sinistra. 

    Se \(m_{0} \ge 0\) allora lo sviluppo dà una funzione olomorfa in un intorno
    di \(a\), e quindi è verificato il caso \((a)\), altrimenti \(m_{0} =: -N
    <0\) e quindi
\begin{equation}\label{eq:meromorfa2}
    f{(z)} = \frac{c_{-N} }{{(z-a)}^{N}} + \frac{c_{-N + 1} }{{(z-a)}^{N-1}} +
    \dots = \frac{1}{{(z-a)}^{N}}\underbrace{\left( c_{-N} + c_{-N+1} {(z-a)} +
    \dots \right)}_{:= g{(z)}}  
\end{equation}
    dove \(g \in \mathcal{H}{(U)}\), con \(U\) intorno di \(a\), e quindi
    necessariamente \(\lim_{z \to a} |f{(z)}| = +\infty\) che è la \((b)\).
    
    Viceversa, se \({(a)}\) \(f\) è limitata in un intorno di \(a\) allora per il
    teorema~\ref{thm:estensione-riemann} è estendibile in modo olomorfo anche in
    \(a\) ed è quindi olomorfa (dunque anche meromorfa) in un intorno di \(a\).
    Se invece \({(b)}\) \(\lim_{z \to a} |f{(z)}| = +\infty\) allora \(\frac{1}{f}\) è
    definito in un intorno \(U\) di \(a\) ed è limitato in tale intorno. Allora per
    il teorema~\ref{thm:estensione-riemann} esiste \(\tilde{\varphi } \in
    \mathcal{H}{(U)}\) con \(\tilde{\varphi} = \frac{1}{f}\) in \(U \sminus
    \{a\} \) e dunque \(\tilde{\varphi} f = 1\) in \(U \sminus \{a\} \) ossia
    \(f\) è meromorfa.

\end{proof}
\begin{definition}{Polo}\label{def:polo}
    Sia \(f \in \mathcal{H}{(D^*_r{(a)})}\) Si dice che \(z = a\) è un
    \textbf{polo} se 
\[
    \lim_{z \to a} |f{(z)}| = + \infty
\]
\end{definition}
In tal caso \(f{(z)}\) è del tipo~\eqref{eq:meromorfa2}. Se \(c_{-N} \neq 0\) si
dice che \(z=a\) è un polo di \textbf{ordine} \(N\)

\begin{definition}{Singolarità essenziale}\label{def:sing-essenziale}
    Se \(f \in \mathcal{H}{(D^{*}_r{(a)})}\) ha sviluppo di Laurent con infiniti
    termini \(c_m\) con \(m < 0\) si dice che \(z=a\) è una singolarità
    \textbf{essenziale}
\end{definition}
\begin{eser}
    Mostrare che \(f{(z)} = e^{\frac{1}{z}}\) ha una singolarità essenziale in
    \(z=0\) 
\end{eser}
\begin{theorem}[Casorata-Weierstrass]\label{thm:casorata-weierstrass}
    Sia \(f \in \mathcal{H}{(D^*_r{(a)})}\) con \(z=a\) singolarità essenziale.
    Allora l'immagine di \(f\) è densa in \(\mathbb{C}\) 
\end{theorem}
\begin{proof}
    Per assurdo supponiamo che esista un \(w \in \mathbb{C}\) e \(\rho > 0\)
    tali che
\[
    D_\rho{(w)} \cap f{(D^{*}_r{(a)})} = \varnothing
\]
    Allora \(\displaystyle \frac{1}{f-w}\) è ben definita in tutto \(D^{*}_r{(a)}\) ed è
    limitata, dunque per il teorema~\ref{thm:estensione-riemann} di estensione
    di Riemann, esiste \(\varphi  \in \mathcal{H}{(D_r{(a)})}\) con
    \(\displaystyle \varphi = \frac{1}{f - w}\) se \(z\neq a\). Ne segue che
    \(f-w\) è meromorfa che è assurdo, perché \(f\) ha una singolarità
    essenziale
\end{proof}
In realtà in particolare esiste un risultato ancora più sorprendente
\begin{theorem}[Grande Teorema di Picard]\label{thm:lol-1eccezione}
    Sia \(f \in \mathcal{H}{(D^{*}_r{(a)})}\) con \(a\) singolarità essenziale.
    Allora \(f\) assume ogni valore di \(\mathbb{C}\), con al più un'eccezione,
    un numero infinito di volte.
\end{theorem}
Da questo teorema ne possiamo dedurre la versione ``piccola'', ossia
\begin{theorem}[Piccolo Toerema di Picard]
    Una funzione olomorfa su \(\mathbb{C}\) che non sia un polinomio assume
    tutti i valori complessi con al più un eccezione un numero infinito di
    volte.
\end{theorem}
\begin{proof}
    Se \(f\) è olomorfa su tutto \(\mathbb{C}\) allora \(f{(z)} = c_{0} + c_{1}
    z + c_{2}z^2 + \dots\). Sia ora 
\[
    \tilde{f}{(\zeta)} = f{\left(\frac{1}{\zeta}\right)} = c_{0}+ \frac{c_{1}}{\zeta} +
    \frac{c_{2}}{\zeta^2} + \dots
\]
    allora \(\zeta=0\) è una singolarità essenziale perché \(f\) non è un
    polinomio. Si applica dunque il teorema precedente.
\end{proof}

\paragraph{\(\mathbb{C}\) esteso.} Indichiamo con \(\hat{\mathbb{C}}\)
l'estensione \(\mathbb{C} \cup \{\infty\} \) in cui gli intorni aperti di
\(\infty\) sono i complementari dei compatti di \(\mathbb{C}\) (\(\infty\) è
visto come una sorta di ``punto ad infinito''). È la compattificazione 1-punto.
Geometricamente \(\hat{\mathbb{C}}\) può essere visto come una sfera (detta
\textbf{sfera di Riemann}) mediante la \emph{proiezione stereografica}.
\begin{figure}[ht]
    \centering
    \incfig[.5]{sfera-di-riemann}
    \caption{Sfera di Riemann}
    \label{fig:sfera-di-riemann}
\end{figure}

\subsection{Residui}
\begin{proposition}\label{prop:residuo-1frac}
    Sia \(f \in \mathcal{H}{(D^{*}_r{(a)})}\) della forma
    \[
      f{(z)} = \frac{g{(z)}}{h{(z)}} \quad {(z \neq a)}
    \]
    con \(g, h \in \mathcal{H}{(D_r{(a)})}\), \(g{(a)} \neq 0\), \(h{(a)} =
    0\) e \(h'{(a)} \neq 0\).
    Allora \(z = a\) è un polo del primo ordine per \(f\) e 
    \[
        \mathrm{Res} (f, a) = \frac{g{(a)}}{h'{(a)}}
    \]
\end{proposition}
\begin{proof}
    Lo sviluppo di \(h\) in \(a\) è 
    \begin{align*}
        h{(z)} &= h'{(a)}{(z-a)} + a_{2}{(z-a)}^2+~\dots = \\
               &= {(z-a)}\underbrace{(h'{(a)} + a_{2}{(z-a)} + \dots)}_{=: \varphi {(z)}} 
    \end{align*}
    allora \(\displaystyle f{(z)} = \frac{g{(z)}}{{(z-a)}\varphi {(z)}} =
    \frac{\psi{(z)}}{z-a}\) con \(\psi\) olomorfa in un intorno di \(z=a\) e
    \(\psi{(a)} = g{(a)} / h'{(a)}\). Se ora
    \begin{align*}
        \varphi {(z)} &= \varphi {(a)} + \varphi '{(a)}{(z-a)} + \dots \implies
        \\
        \implies f{(z)} &= \frac{\varphi {(a)}}{z-a} + \varphi '{(a)} + \dots
    \end{align*}
    e quindi \(z=a\) è un polo del primo ordine per \(f\) con residuo \(\psi{(a)}\) 
\end{proof}
\begin{eser}
    Sia \(\displaystyle f{(z)} = \frac{z}{z^2 + 1}\) per \(z \in \mathbb{C}
    \sminus \{\pm i\}\). Calcolare il residuo in \(z=i\)

    Operando come nella proposizione~\ref{prop:residuo-1frac} si ha \(g{(z)} =
    z\), \(h{(z)} = z^2 + 1\) e effettivamente abbiamo \(g{(i)} = i \neq 0\),
    \(h{(i)} = i^2 + 1 = 0\) e \(h'{(i)} = 2i \neq 0\) e quindi il residuo è 
    \[
        \mathrm{Res} (f, i) = \frac{i}{2i} = \frac{1}{2}
    \]
\end{eser}
Il risultato della proposizione~\ref{prop:residuo-1frac} si può generalizzare a
poli di ordine superiore
\begin{proposition}\label{prop:residuo-kfrac}
    Sia \(f \in \mathcal{H}{(D^{*}_r{(a)})}\). Allora \(z = a\) è polo di ordine
    \(k\) se e solo se
    \[
      \lim_{z \to a} {(z-a)}^{k}f{(z)} = l \neq 0
    \] e in tal caso il residuo è
    \[
        \mathrm{Res} (f, a) = \frac{\phi^{{(k-1)}}{(a)}}{(k-1)!} \quad \varphi
        {(z)} := {(z-a)}^{k}f{(z)}
    \]
\end{proposition}
\begin{proof}
    Ripercorriamo la dimostrazione del caso \(k = 1\): se \(z = a\) è un
    polo di ordine \(k\) allora
    \[
        f{(z)} = \frac{c_{-k} }{{(z-a)}^{k}} + \frac{c_{-k+1} }{{(z-a)}^{k-1}} +
        \dots + \frac{c_{-1} }{z-a} + c_{0} + c_{1}{(z-a)} + \dots
    \]
    con \(c_{-k} \neq 0 \) e quindi \(\displaystyle {(z-a)}^{k}f{(z)} \to c_{-k}
    \neq 0\). In tal caso allora
    \[
        \phi{(z)} := {(z-a)}^{k}f{(z)} = c_{-k} + c_{-k+1}{(z-a)} +~\dots +
        c_{-1} {(z-a)}^{k-1} + \dots
    \]
    e quindi il residuo \(\mathrm{Res} (f, a) \) è il coefficiente dello
    sviluppo di Taylor di \(\phi\) in \(a\), ossi
    \[
        \mathrm{Res} (f, a) = c_{-1} = \frac{\phi^{{(k-1)}}{(a)}}{(k-1)!}
    \]
    Viceversa se \(\displaystyle \lim_{z \to a} {(z-a)}^{k}f{(z)} = l \neq 0\)
    allora \({(z-a)}\phi{(z)} \to 0 \) per \(z\to a\) allora è estensibile, con
    \(\phi {(a)} \neq 0\) e quindi
    \[
        f{(z)} = \frac{\phi{(z)}}{{(z-a)}^{k}} \quad \text{ e quindi } z=a
        \text{ è un polo di ordine } k
    \]
\end{proof}
\begin{eser}
    Sia \(\displaystyle f{(z)} = \frac{1}{z^3 - z^{5}}\) per \(z \in \mathbb{C}
    \sminus \{0, 1, -1\}\). Calcolare il residuo in \(z=0\)
\end{eser}

\begin{eser}
    Calcolare i residui di \(f\) nei poli, nel caso delle funzioni
    \[
      f{(z)} = \frac{e^{\frac{1}{z}} - e^{z}}{z^2-1} \quad , \quad
      \frac{1}{\sin^3 z}
    \]
\end{eser}
\subsection{Indice di avvolgimento}
\paragraph{Logatitmo} non è una funzione di per se su \(\mathbb{C}\), infatti la
funzione esponenziale \(z |> e^{z}\) è suriettiva ma non iniettiva. Infatti dato
\(w \in \mathbb{C}\sminus \{0\} \) cerchiamo un \(z \in \mathbb{C}\) tale che
\(e^{z} = w\). Se \(w = re^{i\theta}\) e \(z = x + i y\) allora
\[
      e^{x + iy} = r e ^{i\theta} \implies \begin{cases}
          e^{x} = r \\
          e^{iy} = e^{i\theta}
      \end{cases}
      \implies 
      \begin{cases}
          x = \log z\\
          y = \theta + 2k \pi
      \end{cases}
\]
quindi abbiamo \(\mathrm{exp}^{-1}{(r e^{i\theta})} = \{\log r + i{(\theta +
2k\pi)} : k \in \mathbb{Z}\}\). Evidentemente quindi risulta difficile definire
una funzione logaritmo. Possiamo dunque limitarne il dominio
\begin{definition}{Logaritmo principale}
    Diciamo \textbf{logaritmo principale} la funzione \(\log : \mathbb{C}
    \sminus (-\infty, 0] \to \mathbb{C}\) definita da
    \[
        z = r e ^{i\theta} \mapsto \log r + i\theta \quad \theta \in (-\pi,
        \pi)
    \]
\end{definition}
Il motivo per cui non si definisce sul semiasse negativo è per evitare la
discontinuità. Vorremmo infatti che tale funzione fosse olomorfa. Lo è.
Mostriamo infatti che \(\log z\) è una primitiva di \(\frac{1}{z}\) in
\(\mathbb{C} - (-\infty, 0]\). \(\frac{1}{z}\) è olomorfa su \(\Omega =
\mathbb{C} \sminus (-\infty, 0]\) che è semplicemente connesso, dunque ha
primitiva, data da:
\[
  F{(z)} = \int _{\gamma_{z} } \frac{1}{\zeta} \,d \zeta
\]
dove \(\gamma_z\) è una curva da \(z_{0}\) a \(z\). In particolare, la definiamo
come in figura~\ref{fig:log-primitiva} e allora
\[
  F{(z)} = \int _{\gamma_z} \frac{1}{\zeta} \,d \zeta = \int_{1}^{r}
  \frac{1}{\rho} \,d \rho + r\int_{0}^{\theta} e^{i\theta} \,d \theta = \log z +
  c % TODO: adj
\]
\begin{figure}[ht]
    \centering
    \incfig[.5]{log-primitiva}
    \caption{Curva \(\gamma\) usata per mostrare che \(\log z\) è primitiva di
    \(\frac{1}{z}\), in questa figura \(z = re^{i\theta}\)} 
    \label{fig:log-primitiva}
\end{figure}
Quindi la funzione \(f{(z)} = \log{(1 + z)}\) è definita in \(\mathbb{C} \sminus
(-\infty, 1]\}\) e quindi in particolare è definita in 0. Consideriamo dunque lo
sviluppo di \(f\) in \(z = 0\). Per determinare i coefficienti abbiamo due modi:
facendo come al solito e calcolando tutte le derivate \(f^{{(n)}}{(0)}\) oppure
oppure sfruttando la primitiva trovata sopra. Infatti
\[
  f'{(z)} = \frac{1}{1 + z} \overset{|z| < 1}{=} 1 - z + z^2 - z^3 + \dots
\]
e sappiamo però che, se \(f{(z)} = c_{0} + c_{1}{(z)} + c_{2}z^2 +\dots\) allora
la serie derivata \(f'{(z)} = c_{1} + 2c_{2}z +3c_{3}z^2 +\dots\) e per confronto
abbiamo quindi (poiché \(c_{0} = f{(0)} = \log{(1)} = 0\))
\[
    f{(z)} = \log{(1 + z)} = z - \frac{z^2}{2} + \frac{z^3}{3} - \dots
\]
\begin{remark}
    Notare che abbiamo usato l'accortezza di confrontare con la serie derivata,
    invece che ``integrare''. Infatti in \(\mathbb{C}\) dovremmo specificare
    una curva ecc è un po' strano dire ``integrando la serie''. Tuttavia così ha
    senso.
\end{remark}
Risulta, in particolare
\[
  \lim_{z \to 0} \frac{\log{(1 + z)}}{z} = 1
\]
Quanto abbiamo detto sul logaritmo principale diventa problematico se abbiamo un
\(\Omega \not\subseteq \mathbb{C}^2 \sminus (-\infty, 0]\). Si consideri
l'esempio della figura~\ref{fig:omega-brutto}
\begin{figure}[ht]
    \centering
    \incfig[.3]{omega-brutto}
    \caption{Omega brutto}
    \label{fig:omega-brutto}
\end{figure} % TODO: chiarire
e anche nel caso di \(\Omega\) brutto siamo riusciti a trovare una funzione
\(\log\) che è inversa di \(e^{z}\) in \(\Omega\). Questo è più generale:
\begin{theorem}
    Sia \(\Omega \subseteq \mathbb{C} \) semplicemente connesso. Sia \(f \in
    \mathcal{H}{(\Omega)}\) mai nulla. Allora esiste \(g \in
    \mathcal{H}{(\Omega)}\) tale che
    \[
      e^{g} = f
    \]
    Tale \(g\) è univocamente individuata a meno di una costante additiva della
    forma \(2k\pi i\) 
\end{theorem}
Idea (illuminazione) (intuito): \(g = \log f\) allora \(g' = f' / f\)
\begin{proof}
    \(f' / f \in \mathcal{H}{(\Omega)}\), \(\Omega\) semplicemente connesso, e
    allora ammette primitiva \(\phi\), con \(\phi' = f' / f\). Consideriamo
    \[
        \frac{d}{dz}{(f e ^{-\phi})} = f' e^{-\phi} - f e^{-\phi} \phi' =0
    \]
    quindi \(f e^{-\phi} =: e^{c}\) è costante su \(\Omega\) e mai nulla. Ma
    allora \(f = e^{\phi + c}\) e dunque \(\phi + c\) è la funzione \(g\) di cui
    nell'enunciato.

    Per l'unicità, se \(g_{1}\) e \(g_{2}\) sono tali che
    \[
      e^{g_{1}} = f = e^{g_{2}}
    \] allora \(e^{g_{1} - g_{2}} = 1\) e dunque \(g_{1}{(z)} - g_{2}{(z)} \in
    2\pi i \mathbb{Z}\) ma poiché \(g_{1} - g_{2}\) è continua e \(\Omega\) è
    connesso, deve essere costante, esiste dunque un \(k \in \mathbb{Z}\) tale
    che
    \[
      \forall  z \in \Omega, \quad g_{1}{(z)} - g_{2}{(z)} = 2k\pi i 
    \]
\end{proof}
\begin{definition}{Ramo del logaritmo}
    Ogni \(g\) di cui al teorema precedente è detta \textbf{determinazione} o
    \textbf{ramo} di \(\log f\) in \(\Omega\) 
\end{definition}
    Siano \(\Omega\) e \(f\) come sopra; sia \(\gamma\) una curva in \(\Omega\)
    con punto iniziale \(z_{0}\) e terminale \(z_1\). Vogliamo calcolare
    l'integrale 
    \[
      \int_{\gamma} \frac{f'{(z)}}{f{(z)}} \,dz \overset{\star}{=} g{(z_{1})} -
      g{(z_{0})}
    \]
    dove in \(\star\) abbiamo preso \(g\) una determinazione di \(\log f\) in
    \(\Omega\). Allora ne segue
    \[
      \exp{\left( \int _{\gamma} \frac{f'{(z)}}{f{(z)}} \,dz \right)} =
      e^{g{(z_{1})} - g{(z_{0})}} = \frac{e^{g{(z_{1})}}}{e^{g{(z_{0})}}} =
      \frac{f{(z_{1})}}{f{(z_{0})}}
    \]
Tale risultato continua a sussistere anche se \(\Omega\) \textbf{non} è
semplicemente connesso, ossia vale il teorema
\begin{theorem}
    Sia \(\Omega \subseteq \mathbb{C} \) aperto e \(f \in
    \mathcal{H}{(\Omega)}\) mai nulla. Sia \(\gamma\) una curva in
    \(\mathbb{C}\) con punto iniziale \(z_{0}\) e punto terminale \(z_{1}\).
    Allora
    \[
      \exp {\left( \int_{\gamma} \frac{f'{(z)}}{f{(z)}} \,dz \right)} =
      \frac{f{(z_{1})}}{f{(z_{0})}}
    \]
\end{theorem}

\begin{corollary}
    Siano \(\Omega, f\) come sopra, sia \(\gamma\) una curva chiusa in
    \(\Omega\). Allora
    \[
      \int _{g} \frac{f'{(z)}}{f{(z)}} \,dz \in 2\pi i \mathbb{Z}
    \]
\end{corollary}
\begin{proof}
    Se \(\gamma\) è chiusa nel teorema si ha 
    \[
        \exp{\left( \int_{\gamma} \frac{f'{(z)}}{f{(z)}} \,dz \right)} = 1
    \]
\end{proof}
\textbf{Caso particolare:} Sia \(a \in \mathbb{C}\) e sia \(\gamma\) una curva
chiusa che non passa per \(a\). Sia \(f{(z)} = z-a\), con \(z \in \mathbb{C}
\sminus \{a\} =: \Omega\) (notare \(\Omega\) non semplicemente connesso). Allora
dal corollario
\[
    \int _{\gamma} \frac{1}{z-a} \,dz = 2\pi i k \quad k \in \mathbb{Z}
\]
e quindi 
\[
    \frac{1}{2\pi i} \int_{\gamma} \frac{1}{z-a} \,dz = k \in \mathbb{Z}
\]
\begin{definition}{Indice di avvolgimento}
    Siano \(\gamma\) e \(a\) come sopra. Il valore (intero) 
    \[
      n{(\gamma; a)} = \frac{1}{2\pi i} \int _{\gamma} \frac{1}{z-a} \,dz
    \] è detto \textbf{indice di avvolgimento} di \(\gamma\) rispetto al punto
    \(a\) 
\end{definition}
Valgono alcune proprietà:
\begin{itemize}[label = --]
    \item \(n{(\gamma; \cdot )}\) è costante sulle componenti connesse di
        \(\mathbb{C} - \gamma\), infatti la funzione \(a \to n{(\gamma; a)}\)  è
        continua ed ha valori in \(\mathbb{Z}\). 
    \item \(n{(\gamma; \cdot )}\) è nullo su \(\Omega_\infty\), infatti sia
        \(D\) un disco contenente \(\gamma\) e sia \(a \not\in \overline{D}\).
        Allora
        \[
          \int _{\gamma} \frac{1}{z-a} \,dz = 0 \text{ perché } z \mapsto
          \frac{1}{z-a} \in \mathcal{H}{(D)}
        \]
\end{itemize}

\begin{figure}[ht]
    \centering
    \incfig[.4]{indice-di-avvolgimento}
    \caption{Indice di avvolgimento}
    \label{fig:indice-di-avvolgimento}
\end{figure}

% TODO : MANCA UN SACCO DI ROBA

\begin{theorem}
    Sia \(g\) olomorfa ad eccezione di un numero finito di punti, su un aperto
    contenente 
    \[
      H = \{z \in \mathbb{C} : \Im(z)\ge  0\} 
    \]
    Nessuna delle singolarità sia sull'asse reale. Sia 
    \[
      f{(z)} = e^{i\omega z}g{(z)}
    \]
    con \(\omega \in \mathbb{R}\). Supponiamo inoltre che
    \[
      \lim_{|z| \to \infty} g{(z)} = 0 ; \quad \omega > 0
    \]

    Allora \(f\) è integrabile su \(\mathbb{R}\) e 
    \[
      \int _{-\infty} ^{+\infty}e^{i\omega x}g{(x)} \,dx = 2\pi i \sum_{\Im(a) >
      0} \mathrm{Res} (f, a)
    \]
\end{theorem}
\begin{proof}
    Sappiamo che
    \[
        \int_{-\infty}^{\infty} = \lim_{R, S \to +\infty} \int_{-S}^{R}
    \]
    Fissiamo \(R, S > 0\). Fissato poi \(M > 0\) consideriamo la curva \(\gamma
    = \gamma_{1} + \gamma_{2} + \gamma_{3} + \gamma_{4}\) in
    figura~\ref{fig:curva-della-dim}
\begin{figure}[ht]
    \centering
    \incfig[.5]{curva-della-dim}
    \caption{Costruzione della curva \(\gamma\) }\label{fig:curva-della-dim}
\end{figure}

    Fissato \(\varepsilon>0\) sia \(R_\varepsilon\) tale che 
    \[
      \forall R, S, M \ge R_\varepsilon, \quad |g{(z)}| \le \varepsilon
    \]
    e le singolarità di \(g\) siano all'interno del rettangolo.
    Allora 
    \[
        \int _{\gamma} e^{i\omega z}g{(z)} \,dz = 2\pi i \sum_{\Im(a) > 0}
        \mathrm{Res} (f, a)
    \]
    Il primo membro è uguale a 
    \[
        \int_{-S}^{R} e^{i\omega z}g{(x)} \,dx + \int _{\gamma_{2}} + \int
        _{\gamma_{3}} + \int _{\gamma_{4} }
    \]
    Consideriamo 
    \[
      \int _{\gamma_{2}} e^{i\omega z}g{(z)} \,dz \overset{z = R + it}{=}
      \int_{0}^{M}e^{i\omega{(R+it)}}g{(R+it)}  \,d t =: I_{2}
    \]
    \[
      |I_{2}| \le \varepsilon \int_{0}^{M} \left| e^{i\omega R} \right|
      e^{-\omega t} \,dt = \varepsilon \left[ -\frac{1}{\omega}e^{-\omega t}
      \right]_0^{M} = \frac{\varepsilon}{\omega} {\left( 1 - e^{-\omega M}
      \right)} \le \frac{\varepsilon}{\omega}
    \]
    qualunque sia \(M> 0\). Analogamente abbiamo anche che \(\left| \int
    _{\gamma_{4}} \right| \le \frac{\varepsilon}{\omega} \) 

    Per \(\gamma_{3}\) abbiamo invece
    \begin{align*}
      \int _{\gamma_{3}} e^{i\omega z}g{(z)} \,dz \overset{z=t+Mi}{=} \int _{R}
      ^{-S}e^{i\omega(t+Mi)}g{(t+Mi)}\,dt =: I_{3} \\
      |I_3| \le \varepsilon \int_{-S}^{R} e^{i\omega M} \,dt = \varepsilon
      e^{-\omega M}{(R+S)} 
    \end{align*}
    Fissati \(R\) e \(S\) si prenda quindi \(M\) tale che
    \[
      \left| \int _{\gamma_{3}} \right| \le \varepsilon
    \]
    Allora possiamo concludere perché
    \[
      \forall \varepsilon > 0,\quad \exists R_\varepsilon : \forall R, S \ge
      R_\varepsilon \quad \left| \int_{-S}^{R} e^{i\omega x}g{(x)} \,dx - 2\pi i
      \sum_{\Im(a) > 0} \mathrm{Res}{(f, a)} \right| \le 2\cdot
      \frac{\varepsilon}{w} + \varepsilon
    \]
\end{proof}
\begin{eser}
    Si calcoli l'integrale
    \[
      \frac{1}{\pi} \int_{-\infty}^{\infty} e^{i\omega x} \frac{\gamma}{x^2 +
      \gamma^2} \,dx \quad \gamma, \omega > 0 
    \]
    e poi per \(\omega < 0\) 
\end{eser}

\begin{eser}
    Si calcoli l'integrale
    \[
      I = \int_{-\infty}^{\infty} \frac{x^3\sin x}{{(x^2 + 1)}^2} \,dx 
    \]
    Si osserva che
    \[
        I = \Im {(J)}, \quad J = \int_{-\infty}^{\infty} e^{ix}
      \frac{x^3}{{(x^2+1)}^2} \,dx 
    \]
    \[
    g{(z)} = \frac{z^3}{{(z^2+1)}^2}\quad z_{0} = i, \quad z_{1} = -i
    \]
    Allora \(z_{0}\) e \(z_{1}\) sono poli del secondo ordine poiché
    \[
        \varphi {(z)} = {(z-i)}^2 f{(z)} = e^{iz} \frac{z^3}{{(z+i)}^2} \to
        e^{-1}\frac{-i}{-4}\neq 0
    \]
    e quindi il residuo è
    \[
      \mathrm{Res}{(f,i)} = \varphi '{(i)} = \frac{1}{4e} \implies J = 2\pi i
      \cdot \frac{1}{4e} \implies I = \Im(J) = \frac{\pi}{2e}
    \]
\end{eser}

\paragraph{Singolarità sull'asse reale} 
Per gli integrali impropri richiamiamo che se \(\xi \in {(a, b)}\), con \(a, b
\in \mathbb{R}\) e \(f : [a, b] \sminus \{\xi\} \to \mathbb{R}\) è continua
allora si dice che l'integrale \(\int_{a}^{b} f \,dx \) è convergente se esiste
finito
\[
    \lim_{\varepsilon, \sigma \to 0} \int _{[a, b] \sminus {(\xi - \varepsilon,
    \xi + \sigma)}} f{(x)} \,dx
\]
(Analogamente per per gli integrali impropri verso \(\pm \infty\) con gli intorni
di \(\pm \infty\) e con \(f : \mathbb{R} \sminus {(-\infty, \infty)} \to
\mathbb{R}\))

\begin{definition}{Valor principale}
    Si dice che \(f\) è \textbf{integrabile in valor principale} su \([a,b]\) se
    esiste finito
    \[
        \lim_{\varepsilon \to 0} \int _{[a,b] \sminus {(\xi - \varepsilon,
        \xi+\varepsilon)}} f{(x)} \,dx =: p.v. \int_{a}^{b} f{(x)} \,dx
    \]
\end{definition}
\begin{example}
    Per \(f{(x)} = \frac{1}{x}\), con \(x \in [-1, 1] \sminus \{0\} \) allora
    \[
      p.v. \int_{-1}^{1} \frac{1}{x} \,d x = \int_{-1}^{-\varepsilon}
      \frac{1}{x} \,dx + \int_{\varepsilon}^{1} \frac{1}{x} \,dx = 0 
    \]
\end{example}
Scriveremo anche
\[
  p.v. \int_{-\infty}^{\infty} f{(x)} \,dx 
\] se \(f\) è integrabile in un intorno di \(\pm \infty\) e presenta singolarità
al finito considerate in valor principale.

\begin{theorem}\label{thm:valor-principale}
    Sia \(f\) olomorfa, ad eccezione di un numero finito di singolarità, in un
    intorno di \(H\). Le singolarità sull'asse reale siano poli semplici.
    Supponiamo che valga una o l'altra delle seguenti condizioni (vedi teoremi
    precedenti)
\begin{itemize}[label = --]
    \item Esista \(p > 1\) tale che \[\limsup_{\substack{|z| \to +\infty \\ z \in
      M}}|z|^{p} |f{(z)}| < +\infty \] 
    \item Esistano \(\omega > 0\) e \(g\) (con le stesse singolarità di \(f\) )
        tali che
        \[
          \lim_{\substack{|z| \to \infty \\ z \in H}} g{(z)} = 0, \quad f{(z)} = e^{i
          \omega z}g{(z)}
        \]
\end{itemize}
Allora \(f\) è integrabile (in valor principale) su \(R\) e 
\[
    p.v. \int_{-\infty}^{\infty} f{(x)} \,dx = 2\pi i \sum_{\Im {(a)} > 0}
    \mathrm{Res}{(f, a)} + \pi i \sum_{\Im {(a)} = 0} \mathrm{Res}{(f, a)}
\]
\end{theorem}
\begin{proof}
    La dimostrazione ripercorre ciascuna delle due precedenti, ``isolando'' le
    singolarità sull'asse reale. Siano ad esempio valide le prime ipotesi.
    Allora Sia \(R > \max_{z \text{sing.}}  |z|\) e consideriamo la curva
    \(\gamma = \eta_{1} + \gamma_\varepsilon + \eta_{2} + \gamma_R\). Allora
    come nella dimostrazione precedente abbiamo
    \[
      \left| \int _{\gamma_R} f{(z)} \,dz \right| \le \frac{M}{R^p}\pi R \to 0
      \text{,  per } R \to \infty
    \]
\begin{figure}[ht]
    \centering
    \incfig[.6]{curva-dim-teo3}
    \caption{curva dim teo3}
    \label{fig:curva-dim-teo3}
\end{figure}
    Per quanto riguarda \(\gamma_\varepsilon\) abbiamo
    \[
      -\gamma_\varepsilon {(\theta)} = a + \varepsilon e ^{i\theta}, \quad
      \theta \in [0, \pi]
    \]
    Consideriamo lo sviluppo di Laurent di \(f\) relativo a \(a\):
    \[
      f{(z)} = \frac{c_{-1}}{z-a} + h{(z)} \quad \text{(polo semplice)}
    \]
    dove \(h\) è olomorfa in un intorno di \(a\). Allora
    \begin{align*}
        \int_{\gamma_\varepsilon}f{(z)}  \,dz &=
      \int_{\gamma_\varepsilon}\frac{c_{-1}}{z-a}  \,d z +
      \underbrace{\int_{\gamma_\varepsilon} h{(z)} \,d z}_{\to  0 \text{ per }
      \varepsilon \to 0 \text{ (\(h\) limitato in un intorno di \(a\))}} \\
      &= -\int_{0}^{\pi} \frac{c_{-1} }{\varepsilon e^{i\theta}}\varepsilon i
      e^{i\theta} \,d \theta = -i \pi \mathrm{Res} {(f,a)}
    \end{align*}
    
    Abbiamo dunque
    \[
      \int _{\gamma} f{(z)} \,dz = \int_{-R}^{a-\varepsilon} f{(x)} \,dx +
      \int_{a+\varepsilon}^{R} f{(x)}  \,dx + \int_{\gamma_\varepsilon} f{(z)}
      \,dz + \underbrace{\int_{\gamma_R}  \,dz}_{\to 0 } 
    \]
    Infine possiamo concludere
    \begin{align*}
        \left| \int_{[-R, R] \sminus {(a-\varepsilon, a+\varepsilon)}} f{(x)}
        \,dx  - 2\pi i \sum_{\Im {(\xi)}> 0} \mathrm{Res}{(f, \xi)} - \pi i
        \mathrm{Res} {(f, a)}\right|\le \\ \le  \left| \int _{\gamma_R}\right| +
        \left| \int _{\gamma_\varepsilon} f + \pi i \mathrm{Res}{(f, a)}\right|
        \to 0
    \end{align*}
\end{proof}

\begin{eser}
    Calcolare 
    \[
        \int_{-\infty}^{\infty} \frac{1-e^{iz}}{z^2} \,dz 
        \quad \text{   e   }\quad
        \int_{-\infty}^{\infty} \frac{1-\cos x}{x^2} \,dx \tag*{[$\pi$
        entrambi]}
    \]
\end{eser}

\subsection{Altri risultati}

\begin{theorem}[Prolungamento analitico]\label{thm:prolungamento-analitico}
    Sia \(\Omega \subseteq \mathbb{C} \) un aperto connesso e \(f\) olomorfa su
    \(\Omega\). Se esiste \(U \mathbb{C} \Omega\) aperto tale che \(f \equiv 0\)
    su \(U\) allora \(f \equiv 0\) su \(\Omega\).
\end{theorem}
\begin{proof}
    Consideriamo \(E_{n} = \{f^{{(n)}} = 0\} \) con \(n = 0, 1, \dots\). Allora
    \(E = \bigcap_{n} E_{n}\). \(E\) è chiuso in quanto intersezione di chiusi.
    Vogliamo mostrare che è anche aperto. Sia \(a \in E \subseteq \Omega \),
    sviluppiamo \(f\)  intorno a \(a\) e abbiamo
    \[
      f{(z)} = \sum_{n=0}^{\infty} c_{n} {(z-a)}^{n}, \quad z \in D_r{(a)}, quad
      c_{n} = \frac{f^{{(n)}}{(a)}}{n!}
    \]
    per \(r\) opportuno, con \(D_r{(a)} \subseteq \Omega \). Quindi poiché \(a
    \in E\) risulta \(c_{n}=0\) per ogni \(n\) quindi \(f \equiv 0\) in
    \(D_r{(a)}\). Allora \(D_r{(a)} \subseteq E\) e quindi \(E\) è aperto.

    Infine poiché \(\Omega\) è connesso e \(E \neq \varnothing\) ne consegue
    necessariamente che \(E = \Omega\) 
\end{proof}
\begin{corollary}\label{cor:zero-isolato}
    Sia \(0 \not\equiv f \in \mathcal{H}{(\Omega)} con {(\Omega)} connesso. Allora l'insieme
    degli zeri di {(f)} è un insieme chiuso in {(\Omega)} e discreto\) 
\end{corollary}
\begin{proof}
    Sia \(f{(a)} = 0\): mostriamo che \(z = a\) è uno zero isolato. Sia
    \(D_r{(a)}\) tale che
    \[
      f{(z)} = \sum_{n=0}^{\infty} c_{n} {(z-a)}^{n}, \quad z \in D_r{(a)} 
    \]
    Se \(c_{n} = 0\) per ogni \(n\) allora \(f\equiv 0\) su \(D_r{(a)}\): per il
    teorema~\ref{thm:prolungamento-analitico} quindi \(f \equiv 0\) su \(\Omega\), che è escluso.

    Sia quindi \(N = \min \{ n : c_{n} \neq 0\} \) e si ha che
    \[
        f{(z)} = c_N {(z-a)}^{N} + c_{N+1} {(z-a)}^{N+1} +~\dots = {(z-a)}^{N}
        g{(z)}
    \]
    con \(g{(a)} = c_N \neq 0\). Quindi in un opportuno intorno di \(a\), \(f\)
    si annulla solamente in \(a\).
\end{proof}
\begin{corollary}\label{cor:estensione-analitica}
    Siano \(f, g \in \mathcal{H}{(\Omega)}\) con \(\Omega\) aperto connesso. Se
    \( \{ f = g\} \) ha un punto di accumulazione in \(\Omega\) allora \(f
    \equiv g\) in \(\Omega\) 
\end{corollary}
\begin{proof}
    Si applichi il corollario~\ref{cor:zero-isolato} a \(f-g\)
\end{proof}
\begin{example}
    Consideriamo \(f = z \mapsto e^{z} : \mathbb{C}\to \mathbb{C}\). Allora
    \(f\) è l'unica estensione olomorfa di \(x \mapsto e^{x} : \mathbb{R} \to
    \mathbb{R}\). Analogamente per le funzioni trigonometriche.

    Così pure si possono estendere a tutto \(C\) alcune formule. Infatti se, ad
    esempio \(\cos^2 x + \sin^2 x = 1\) su tutto \(\mathbb{R} \subseteq \mathbb{C} \), che ha punti di
    accumulazione, allora l'uguaglianza vale su tutto \(\mathbb{C}\)
\end{example}

\begin{theorem}\label{thm:olomorfa-aperta}
    Ogni funzione olomorfa non costante su un aperto connesso è un'applicazione aperta
\end{theorem}

Da cui segue il seguente teorema
\begin{theorem}[del massimo modulo]
Sia \(\Omega \subseteq \mathbb{C} \) aperto e connesso e \(f \in
\mathcal{H}{(\Omega)}\). Se \(f\) non è costante allora \(|f|\) non può avere
punti di massimo in \(\Omega\) 
\end{theorem}

\begin{proof}
    Sia \(z \in \Omega\), vogliamo mostrare che non può essere un punto di
    massimo. Sia \(w = f{(z)}\). Sia \(D_r{(z)} \subseteq \Omega \). Allora
    \(f{(D_r{(z)})}\) è un aperto contenente \(f{(z)} = w\).
    Esiste \(w' \in f{(D_r{(z)})}\) (quindi \(w' = f{(z')}\)) tale che \(|w'| >
    |w|\) cioè \(|f{(z')}| > |f{(z)}|\). Allora \(z\) non può essere punto di
    massimo per \(|f|\) 
\end{proof}

\begin{corollary}
    Sia \(\Omega\) aperto connesso e \textbf{limitato}. Sia \(f :
    \overline{\Omega} \to \mathbb{C}\) una funzione continue con \(f \in
    \mathcal{H}{(\Omega)}\). 

    Allora
    \begin{equation}\label{eq:massimo-modulo}
      \sup_{\Omega} |f| = \max_{\overline{\Omega}} |f| = \max_{\partial \Omega}
      |f|
    \end{equation}
\end{corollary}
\begin{remark}
    Come ricordare il tutto: è chiaro, vogliamo dire che il massimo ``non c'è
    perché è sul bordo''. Per farlo dobbiamo definire la funzione su
    \(\overline{\Omega}\), aggiungendo come ipotesi la continuità e per avere il
    massimo dobbiamo richiedere la limitatezza di \(\Omega\). Infine poiché il
    massimo non può essere in \(\Omega\) deve essere sul bordo. 
\end{remark}
\begin{proof}
    Se \(f\) è costante valgono le uguaglianze.

    Altrimenti: i punti di massimo di \(|f|\) su \(\overline{\Omega}\)  (che
    esistono per Weierstrass) devono essere su \(\partial \Omega\), altrimenti
    sarebbero massimo per \(|f|\) in \(\Omega\). Quindi
    \[
        \max_{\overline{\Omega}} |f| = \max_{\partial \Omega} |f|
    \]
    Del resto se \(z_M \in \partial \Omega\) è di massimo per \(|f|\) su
    \(\overline{\Omega}\) e \(z_k \to z_M\) con \(z_k \in \Omega\), allora
    \[
        \sup_{\Omega} |f| \ge |f{(z_k)}| \to |f{(z_M)}| = \max_{\partial \Omega}
        |f|
    \]
\end{proof}
\begin{theorem}
    Sia \(f \in \mathcal{H}{(\Omega)}\). Allora
\begin{enumerate}[label = \roman*)]
    \item Se \(f\) è iniettiva allora \(f'\) non si annulla mai. Quindi posto
        \(G = f{(\Omega)}\) si ha che \(G\) è aperto, \(f : \Omega \to G\) è
        biettiva. Allora \(f^{-1} : G \to \Omega\) è olomorfa e 
        \[
            {(f^{-1})}'{(w)} = \frac{1}{{f'{(z)}}} \quad \text{ con } w = f{(z)}
        \]
    \item Se \(a \in \Omega\) è tale che \(f'{(a)} \neq 0\) allora \(f\) è
        localmente iniettiva, quindi localmente invertibile.
\end{enumerate}
\end{theorem}

Sia \(f \in \mathcal{H}{(\Omega)}\) e prendiamo un punto \(a \in \Omega\).
Consideriamo la curva \(\gamma : (-1, 1) \to \Omega\) con \(\gamma{(0)} = a\) e
consideriamo la curva \(\tilde{\gamma} = f \circ \gamma\). Sia \(v =
\gamma'{(0)}\) allora
\[
  \gamma'{(0)} = f'{(\gamma{(0)})} \gamma'{(0)} = f'{(a)} v
\]
dove l'ultimo dipende solo da \(f, a\) (e \(v\)) ma non dalla curva.

Sia \(0 \neq f'{(a)} = r e ^{i\theta}\). Allora \(\gamma'{(0)}\) si ottiene da \(v\)
per rotazione dell'angolo \(\theta\) (e dilatazione di fattore \(r\)). Quindi se
\(f'{(a)} \neq 0\) e \(\gamma, \eta\) sono curve che si tagliano in \(a\)
secondo l'angolo \(\alpha\), anche le immagini si tagliano in \(f{(a)}\) con
secondo \(\alpha\). Tale proprietà geometrica è chiamata \textbf{conformità}.

\begin{definition}{Applicazione conforme}
    Un'applicazione \(f : \Omega \to G \) con \(\Omega, G \subseteq \mathbb{C}
    \) aperti che sia olomorfa e biettiva (quindi con anche \(f^{-1}\)
    olomorfa) è detta applicazione conforme (biolomorfismo)
\end{definition}

\begin{example}
    La funzione \(f = z \mapsto e^{z} : \{x + iy : |y| < \pi\} \to G =
    \mathbb{C} \sminus (-\infty, 0]\) è conforme. In particolare ha inversa il
    logaritmo principale
    \[
        \log z = \log \rho + i \theta \quad \text{ con } z = \rho e^{i\theta}
    \]

    % se avessi il mouse farei un disegno ma è facile: griglia va in raggiera
\end{example}
\begin{example}
    \(f{(z)} = \frac{i - z}{i + z}\) con \(z \neq -i\). Allora \(f : \mathbb{C}
    \sminus \{-i\} \to \mathbb{C}\sminus \{-1\} \). Mostriamo che è biettiva.
    Sia \(\zeta \neq -1\) e cerchiamo \(z\) tale che \(f{(z)} = \zeta\)
    \[
      \frac{i - z}{i + z} = \zeta \iff i - z = i \zeta + z \zeta \iff z = i
      \frac{1 - \zeta}{1 + \zeta} = f^{-1}{(\zeta)}
    \]
    Sia \(H ^{+} = \{ z \in \mathbb{C} : \Im {(z)} > 0\} \) e \(D =
    D_{1}{(0)}\). Allora mostriamo che \(f : H^{+} \to D\) è biettiva. Questo
    segue da 
    \[
      {|f{(z)}|} = \frac{|i-z|}{|i+z|} = \frac{d{(z, i)}}{d{(z, -i)}}
    \]
    e dunque \(|f{(z)}| < 1 \iff z \in H^{+}\) 
\end{example}

\begin{lemmao}[Schwartz]\label{lem:schwartz}
    Sia \(f : D \to D\) con \(f{(0)} = 0\). Allora 
\begin{enumerate}[label = \alph*.]
    \item \(|f{(z)}| \le |z|\) per ogni \(z \in D\); inoltre se esiste \(z_{0}
        \neq 0\) con \(|f{(z_{0})}| = |z_{0}|\). Allora \(f\) è una rotazione
    \item \(|f'{(0)}| \le 1\) e se \(f'{(0)}| = 1\) allora \(f\) è una rotazione
\end{enumerate}
\end{lemmao}
\begin{proof}
\begin{enumerate}[label = \alph*.]
    \item Sia \(g{(z)} = \frac{f{(z)}}{z}\) con \(z \neq 0\). Allora per il
        teorema di estensione di Riemann~\ref{thm:estensione-riemann} \(g\) è
        olomorfa anche in \(z=0\) con \(g{(0)} = \lim_{z \to 0} \frac{f{(z)}}{z}
        = f'{(0)}\). Dunque \(g \in \mathcal{H}{(D)}\).

        Sia \(r \in {(0, 1)}\) e \(z \in \partial D_r{(0)}\) allora
        \[
          |g{(z)}| = \frac{|f{(z)}|}{|z|} \le \frac{1}{r}
      \]
      e per il principio del massimo modulo~\ref{thm:massimo-modulo} si ha che
      \(|g| \le \frac{1}{r}\) su \(D_r{(0)}\). Per \(r\to 1\) si ha \(|g{(z)}|
      \le 1\) su \(D\) cioè \(|f{(z)}| \le |z|\) su \(D\).

      Se esiste \(z_{0} \neq 0\) con \(|f{(z_0)}| = |z_{0}|\) cioè
      \(|g{(z_{0})}| = 1\) allora per il principio del massimo modulo si ha che
      \(|g| = c \) e dunque \(g{(z)} = e^{i\theta}\) per cui \(f{(z)} =
      e^{i\theta} z\) cioè \(f\) è una rotazione
    \item \(|f'{(0)}| = |g{(0)}| \le  1\). Se \(|f'{(0)}| = 1\) allora
        \(|g{(0)}| = 1\), come prima.
\end{enumerate}
\end{proof}

\begin{theorem}
    Gli automorfismi di \(D\) sono tutte e sole le applicazioni della forma 
    \[
        f{(z)} = e^{i\theta} \psi_w{(z)} \quad \text{ con } \psi_w{(z)} =
        \frac{z-w}{\overline{w}z - 1}
    \]
\end{theorem}
\begin{proof}
Sia \(f : D \to D\) biettiva e olomorfa. Definiamo \(\psi_w : D\to D\) biettiva
e olomorfa come una funzione tale che \(\varphi_w {(0)} = w = f^{-1}({0})\) una
possibile funzione è \(\psi_w{(z)} = \frac{z-w}{\overline{w}z - 1}\) (si
verifichi). Allora abbiamo che \(h = f \circ \psi_w\) fissa l'origina.
Applichiamo~\ref{lem:schwartz} a \(h\) e otteniamo che \(h\) è una rotazione.
Ne consegue la tesi
\end{proof}
Le precedenti sono casi particolari delle \textbf{trasformazioni di Möbius} che
sono della forma
\[
  f{(z)} = \frac{az + b}{cz + d}
\]
che hanno proprietà interessanti

\begin{theorem}[Mappa conforme di Riemann]
    Sia \(\Omega \subseteq \mathbb{C} \) un aperto semplicemente connesso e
    diverso da \(\mathbb{C}\). Fissato \(z_{0} \in\Omega\) esiste \(f : \Omega
    \to D\) \textbf{conforme} tali che \(f{(z_{0})} = 0\) e \(f'{(z_{0})} \in \mathbb{R}\)
\end{theorem}

\newpage
\section{Richiamo delle forme differenziali}

Sia \(\Omega \subseteq \mathbb{R}^{2} \) aperto. Una forma differenziale su \(\Omega\) è
un'espressione formale della forma 
\[
    \omega(x, y) = A(x,y)dx + B(x,y) dy
\]
con \(A, B \in C^{0}(\Omega)\). Più precisamente \(\omega\) è una funzione
continua \(\omega: \Omega \to (\mathbb{R}^{2})'\). Se \(\gamma\) è una curva
\(C^{1}\) a tratti in \(\Omega\), \(\gamma : [a,b] \to \Omega\), allora
\[
    \int_{\gamma}\omega \overset{\text{def}}{=} \int_{a}^{b} A{(x(t), y(t))}
    x'{(t)} + B{(x(t), y(t))} y'{(t)} dt \quad \gamma{(t)} = (x(t), y(t))
\]
\begin{definition}{Forma esatta}
    La forma differenziale \(\omega\) si dice \textbf{esatta} se esiste \(F \in
    C^{1}(\Omega)\) tale che \(\omega = dF\), e \(F\) è detta \emph{primitiva}
    di \(\omega\) 
\end{definition}

Se \(\omega\) è \(C^{1}\) (cioè \(A, B \in C^{\Omega}\)) allora, se è esatta,
ossia \(\frac{\partial F}{\partial x} = A\) e \(\frac{\partial F}{\partial y} =
B\) risulta
\begin{equation}\label{eq:forma-chiusa}
    \frac{\partial A}{\partial y} = \frac{\partial B}{\partial x}
\end{equation}

\begin{definition}{Forma chiusa}
    Se \(\omega\) è una forma differenziale \(C^{1}\) e
    soddisfa~\eqref{eq:forma-chiusa} allora si dice \textbf{chiusa} 
\end{definition}
\begin{theorem}\label{thm:1_forme}
    Sia \(\Omega\) connesso. Allora 
    \[
        \omega \text{ esatta } \iff \int_{\gamma} \omega = 0 \text{ per ogni
        \(\gamma\) in \(\Omega\) chiusa }
    \]
\end{theorem}
\begin{proof}[idea di dimostrazione]\( \)
\begin{itemize}
    \item[\(\implies \)] semplice
    \item[\(\impliedby \)] Fissiamo \((x_{0},y_{0}) \in \Omega\). Definiamo ora 
        \[
            F(x, y) = \int_{\gamma(x,y)} \omega
        \]
        dove \(\gamma_{x,y} \) è una qualunque curva in \(\Omega\) che unisce
        \((x_{0},y_{0})\) a \((x,y)\). La definizione è ben posta perché se
        \(\gamma_{(x,y)} e \tilde{\gamma}_{(x,y)}\) sono due tali curve allora
        \[
            0 = \int_{\gamma_{(x,y)} - \tilde{\gamma}_{(x,y)}} \omega =
            \int_{\gamma_{(x,y)}} \omega - \int_{\tilde{\gamma}_{(x,y)}} \omega
        \]

        La dimostrazione procede dimostrando che \(dF = \omega\)
\end{itemize}
\end{proof}

\begin{theorem}\label{thm:2_forme}
    Sia \(\Omega\) connesso e \(\omega\) \textbf{chiusa}. Allora se
    \(\gamma_{0}\) e \(\gamma_{1}\) sono curve chiuse \(C^{1}\) a tratti omotope
    in \(\Omega\) allora
    \[
        \int_{\gamma_{0}} \omega = \int_{\gamma_{1}} \omega
    \]
\end{theorem}
\begin{remark}
    Il teorema vale anche per curve non necessariamente chiuse purché siano omotope 
    mediante un'omotopia che fissa gli estremi.
\end{remark}

\begin{corollary}
    Sia \(\Omega\) semplicemente connesso e \(\omega\) chiusa. Allora \(\omega\)
    è esatta.
\end{corollary}
\begin{proof}
    Se \(\Omega\) è semplicemente connesso ogni curva chiusa è omotopa a
    costante \(0\) e quindi \(\int_{\gamma} \omega = 0\), ossia \(\omega\) è
    esatta 
\end{proof}

\end{document}
