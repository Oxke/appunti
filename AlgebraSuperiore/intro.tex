In topologia: sia \(X\) uno spazio topologico, allora omologia / coomologia (a
coeff in \(A\)) 
\[
  H_{n}{(X, A)} / H^{n}{(X, A)}
\]
Ottenuti a partire da \textbf{complessi} di catene o cocatene, cioè successioni
\[
    C_{n}{(X, A)} \overset{d_{n}}{\to } C_{n-1} {(X, A)} \overset{d_{n-1} }{\to } \dots
\]
e \(d_{n-1} \cdot d_{n} = 0\) (cioè \(\mathrm {Im} {(d_{n})} \subseteq \mathrm{Ker}{(d_{n-1} )} \))  e \(H_{n} {(X, A)} := \frac{\mathrm{Ker}{(d_{n-1} )}}{\mathrm{Im}{(d_{n})}}\) 
In astratto introdurremo le categorie \emph{abeliane} (che generalzzano le
strutture dei moduli su un anello) e studieremo i ``funtori derivati'' di
funtori (additivi) tra categorie abeliane.

I funtori derivati misureranno la mancanza di \textbf{esattezza} (ossia mandare
successioni esatte in successioni esatte) del funtore di
partenza.

\paragraph{Libri utili}
\begin{itemize}[label = --]
    \item Per la parte di algebra omologica Hilton-Stammbach, Osborne e
Weibel.
    \item Dispense sui moduli (su KIRO) utili
    \item Aluffi, \emph{Algebra Chapter 0}
\end{itemize}
Il corso è di 60 ore, non perché sia più pesante ma perché dovrebbero esserci
ore di esercitazioni (non sarà necessariamente vero ma Canonaco cercherà di
andare un po' nel dettaglio, fornire esempi e controesempi per quanto possibile)
