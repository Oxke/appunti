\section{Richiami sui Moduli}
\begin{definition}{\(A\)-modulo}
    Un \(A\)-modulo (di default sinistro) \(M\) è un gruppo abeliano \({(M, +)}\) con una
    moltiplicazione per scalare definita da 
    \begin{align*}
        \cdot : A\times M &\longrightarrow M \\
        {(a,x)}&\longmapsto ax \in M
    \end{align*}
    e tale che, \(\forall a,b \in A\) e \(\forall x,y \in M\):
\begin{enumerate}[label = \arabic*)]
    \item \(a{(x+y)} = ax + ay\) 
    \item \({(a+b)}x = ax + bx\) 
    \item \({(ab)}x = a{(bx)}\) 
    \item \(1x = x\) 
\end{enumerate}
\end{definition}
\begin{remark}{}
    Se \(\mathbb{K}\) è un campo, allora un \(\mathbb{K}\)-modulo è uno spazio
    vettoriale.
\end{remark}
\begin{remark}{}
    Se \({(M, +)}\) è un gruppo abeliano, data \(f : A \times  M \to M\) posso
    \\ definire \(\alpha : A \to M^{M}\) come \(\alpha{(a)} = (x \mapsto ax)\),
    e quindi le proprietà precedenti si traducono in 
\begin{enumerate}[label = \arabic*.]
    \item \(\alpha{(a)}{(x+y)} = \alpha{(a)}{(x)} + \alpha{(a)}{(y)}\) e dunque
        \(\alpha{(a)}\) è omomorfismo di gruppi, dunque \(\alpha{(A)} \subseteq \mathrm{End}{(M)}\) 
    \item \(\alpha{(a+b)} = \alpha{(a)}+\alpha{(b)}\) dunque \(\alpha : A \to \mathrm{End}{(M)}\) è omomorfismo di gruppi
    \item \(\alpha{(a)}\circ \alpha{(b)} = \alpha{(ab)}\) 
    \item \(\alpha{(1)} = \mathrm{id}_M\) 
\end{enumerate}
Dalla 2,3,4 \(\alpha : A \to \mathrm{End}{(M)}\) è omomorfismo di anelli.
\end{remark}


\begin{theorem}{Secondo teorema di isomorfismo}
    Sia \(M\) un modulo, con \(M', M'' \subseteq M \) sottomoduli. Allora 
    \[
      M' / (M' \cap  M'') \cong {(M' + M'')} / M''
    \]
\end{theorem}
\begin{proof}{}
    Si prenda \(f : M' \to (M' + M'') / M''\) composizione dell'inclusione di \(M'\) in \(M' + M''\) e della proiezione a quoziente, dunque è un omomorfismo.

    Allora \(\mathrm{Ker} f = \{x \in M' : x + M'' = M''\} = M' \cap  M'' \).

    Preso \(y \in {(M' + M'')}/M''\), \(y = x' + x'' + M''
     = x' + M'' = f{(x')}\) dunque \(f\) è suriettiva. Dal primo
     teorema di isomorfismo segue la tesi.
\end{proof}


\begin{theorem}{Terzo teorema di isomorfismo}
    Dati \(M'' \subseteq M' \subseteq M\)  sottomoduli e modulo, allora
    \[
      {({M}/{M'})} / {(M'/ M'')} \cong M / M'
    \]
\end{theorem}
\begin{proof}{}
    Sia \(f\) la composizione delle due proiezioni a quoziente, dunque è
    suriettiva. Allora
    \[
      x \in \mathrm{Ker}f \iff \pi{(x)} \in \mathrm{Ker}\pi' = M'/M''
    \]
    dunque \(\mathrm{Ker}f  = M'\) da cui la tesi per il primo teorema di
    isomorfismo.
\end{proof}

\begin{proposition}{} \( \) 
\begin{enumerate}[label = \arabic*.]
    \item Sia \(A\) un anello, allora un \(A\)-modulo \(M\) è ciclico se e solo se \(\exists I \subseteq A \) ideale sinistro tale che \(M \cong A / I\) 
    \item \(M\) è semplice se e solo se \(\exists  I \subseteq A \) ideale
        sinistro massimale tale che \(M \cong A/I\) 
\end{enumerate}
\end{proposition}
\begin{proof}{}
\begin{enumerate}[label = \arabic*.]
    \item \({(\impliedby)}\) \(A / I\) è ciclico (generato da \(\overline{1}\)). Viceversa per \({(\implies )}\) so che \(M = Ax\) per un qualche \(x \in M\). Considerata \(
        f : _AA \to M\) data da \(a \mapsto ax\), \(\mathrm{Ker}f\) è
        sottomodulo di \(A\), ovvero ideale sinistro. Concludo per il primo
        teorema di isomorfismo.
    \item Se \(M\) è semplice allora \(\forall 0 \neq x \in M\), \(M = Ax\),
        dunque \(M\) è ciclico e per il punto 1. esiste \(I\) ideale sinistro
        tale che \(M \cong A/I\). La proposizione si riduce a dire che \(A/I\) è
        semplice se e solo se \(I\) è massimale. Sappiamo che i sottomoduli di
        \(A/I\) sono tutti e soli della forma \(J /I\) con \(I \subseteq J \subseteq A  \) ideale sinistro.
        Allora \(A / I \neq 0 \iff I \neq A\) e gli unici sottomoduli di \(A / I \) sono \(I / I\) e \(A / I\), ossia gli unici ideali sinistri \(J\) tali che \(I \subseteq J \subseteq A  \) sono \(I\) e \(A\).
\end{enumerate}
\end{proof}
\begin{remark}{}
    Con il lemma di Zorn si dimostra che \(A \neq 0 \implies \) esiste un ideale
    sinistro massimale (e dunque esiste un sottomodulo semplice)
\end{remark}

\subsection{Prodotti}
\begin{definition}{Prodotto}
Supponiamo di avere \(M_{\lambda} \) \(A\)-moduli, per \(\lambda \in \Lambda\). Allora 
\[
  M := \prod_{\lambda \in \Lambda} M_{\lambda} \text{ è un \(A\)-modulo detto \textbf{prodotto} degli \(M_{\lambda}\)}
\]
con \({(x + y)}_\lambda := x_\lambda + y_\lambda\)  e \({(ax)}_\lambda =
ax_\lambda\) per ogni \(\lambda \in \Lambda\) e \(x, y \in M\).

\(\forall \mu \in \Lambda\) esiste \(p_\mu : M \to M_\mu\), \({(x_{\lambda})}_{\lambda \in \Lambda} \mapsto x_\mu  \) che è \(A\)-lineare e suriettivo.
\end{definition}

\begin{proposition}[Proprietà universale del prodotto] \( \) 
    \\ Dati \(f_\mu : N \to M_\mu \) \(A\)-lineari \(\forall \mu \in \Lambda\),
    allora esiste unico \(f : N\to M\) \(A\)-lineare tale che \(f_\mu = p_\mu \circ f\) 
\end{proposition}
\[\begin{tikzcd}
	N \\
	{M_\mu} & \prod_{\lambda \in \Lambda} M_{\lambda} 
	\arrow["{f_\mu}"', from=1-1, to=2-1]
	\arrow["{\exists ! f}", dashed, from=1-1, to=2-2]
	\arrow["{p_\mu}", from=2-2, to=2-1]
\end{tikzcd}\]

\begin{eser}{}
    Dimostrare la proprietà universale del prodotto
\end{eser}


\begin{definition}{Somma diretta}
    La \textbf{somma diretta} (o coprodotto) degli \(M_{\lambda} \) è 
    \[
        M' := \bigoplus_{\lambda \in \Lambda} M_\lambda = \{{(x_\lambda)}_{\lambda \in \Lambda} \in M : x_\lambda >0 \text{ per
  finiti \(\lambda\)} \subseteq M \}
    \]
    è sottomodulo.
\end{definition}

\(\forall \mu \in \Lambda\) esiste
\begin{align*}
    i_\mu: M_\mu &\longrightarrow M' \\
    x &\longmapsto i_\mu(x) = {(x_{\lambda} )}_{\lambda \in \Lambda}, \quad
    x_\lambda := \begin{cases}{}
        x & \lambda = \mu \\
        0 & \lambda \neq \mu
    \end{cases}
\end{align*}
che è \(A\)-lineare e iniettivo.

\begin{proposition}[Proprietà universale somma diretta]
\[\begin{tikzcd}
	N \\
	{M_\mu} & \bigoplus_{\lambda \in \Lambda} M_\lambda
	\arrow["{f_\mu}", from=2-1, to=1-1]
	\arrow["{i_\mu}"', from=2-1, to=2-2]
	\arrow["{\exists! f}"', dashed, from=2-2, to=1-1]
\end{tikzcd}\]
\end{proposition}
\begin{remark}{}
    Se \(\#\Lambda < +\infty\) allora 
    \[
      \bigoplus_{\lambda \in \Lambda} M_\lambda = \prod_{\lambda \in \Lambda} M_{\lambda} 
    \]
\end{remark}

\begin{note}[zione]
    Se \(M_\lambda = M\) per ogni \(\lambda \in \Lambda\), si denota
    \[
      \prod_{\lambda \in \Lambda} M =: M^{\Lambda} \quad \text{e} \quad
      \bigoplus_{\lambda \in \Lambda} M =: M^{{(\Lambda)}}
    \]
\end{note}

    Dati \(M_\lambda \subseteq M \) sottomoduli, con \(\lambda \in \Lambda
    \), sia \[f : \oplus_{\lambda \in \Lambda} M_\lambda \to M\] l'omomorfismo indotto
    dalle inclusioni \(M_\lambda \overset{i_{\lambda} }{\hookrightarrow }  M \), allora
    \[
      \mathrm{im}f =: \sum_{\lambda \in \Lambda} M_\lambda \subseteq M \text{ è sottomodulo } 
    \]
    Inoltre \(f\) è iniettiva se e solo se \(M_{\mu} \cap \sum_{\lambda \in \Lambda \sminus \{\mu\} } = 0  \) per ogni \(\mu \in \Lambda\) e in tal caso \(f\) induce un isomorfismo tra \(\bigoplus_{\lambda \in \Lambda} M_\lambda \) e \(\sum_{\lambda \in \Lambda} M_\lambda\) e si può scrivere \(\bigoplus_{\lambda \in \Lambda} M_\lambda\) per indicare il sottomodulo di \(M\) 


\begin{definition}{Linearmente indipendente, base, modulo libero}
    Sia \(U \subseteq M \) un insieme, con \(M\) \(A\)-modulo. Si dice che \(U\) è \(A\)-linearmente indipendente se dati \(x_{1},\dots,x_{n} \subseteq U \) distinti 
    \[
      a_{1}, \dots, a_{n} \in A \text{ t.c. } \sum_{i=1}^{n} a_{i} x_{i} = 0 \implies a_{1} = \dots = a_{n} = 0 
    \]
    \(U\) è detta \textbf{base} di \(M\) se è linearmente indipendente e genera
    \(M\), ossia \(M = AU\). Si dice che \(M\) è \textbf{libero} se ammette una
    base
\end{definition}

\begin{example}{}
    Per ogni \(\Lambda\), \(A^{{(\Lambda)}}\) è libero con base \(\{e_\lambda : \lambda \in  \Lambda\} \) dove, per ogni \(\lambda \in \Lambda\),
    \[
        (e_\lambda)_i = \begin{cases}{}
            1 & \lambda = i \\
            0 & \lambda \neq i
        \end{cases}
    \]
\end{example}

\begin{proposition}{}
    Siano \(L, M\) \(A\)-moduli, con \(L\) libero con base \(\{l_\lambda : \lambda \in \Lambda\} \) tale che \(l_\lambda \neq l_\mu\) se \(\lambda \neq \mu\), allora
    \[
      \forall \lambda \in  \Lambda \,\,\,\, \exists ! f :
  L\to M \,\, A\text{-lineare t.c. }f{(l_\lambda)} = x_\lambda
    \]
\end{proposition}
\begin{corollary}{}
    Un \(A\)-modulo è libero se e solo se è isomorfo a \(A^{{(\Lambda)}}\) per
    qualche \(\Lambda\) 
\end{corollary}
\begin{proof}\( \)
\begin{itemize}
    \item[\(\implies \)] \(M\) libero con base \(\{x_\lambda : \lambda \in \Lambda\}\) con \(x_\lambda\neq x_\mu\) se \(\lambda \neq \mu\). Allora per la proposizione
        \[
          \exists ! f : A^{\Lambda} \to M \text{ \(A\)-lineare t.c. } f{(e_\lambda)} = x_\lambda 
        \]
        per ogni \(\lambda \in \Lambda\).
        Allora \(\mathrm{im}f = \Span{x_{\lambda} : \lambda \in \Lambda}_A = M\) e \(f\) è iniettivo perché gli \(x_\lambda\) sono linearmente indipendenti.
    \item[\(\impliedby \)] ovvio
\end{itemize}
\end{proof}

\begin{corollary}{}
    Ogni \(A\)-modulo è insomorfo a un quoziente di un modulo libero (\(A^{{(\Lambda)}}\) per un qualche \(\Lambda\) ).

    Inoltre un \(A\)-modulo è finitamente generato se e solo se è isomorfo a un quoziente di \(A^{n}\), \(n \in \mathbb{N}\) 
\end{corollary}
\begin{proof}{}
    Sia \(\{x_{\lambda} \}_{\lambda \in \Lambda}  \) un insieme di generatori di
    un modulo \(M\). Per la proposizione \(\exists ! f: A^{{(\Lambda)}}\to M\)
    \(A\)-lineare tale che f\(l_\lambda = x_\lambda\) per ogni \(\lambda \in
    \Lambda\). Allora \(\mathrm{Im}f = M\) e dunque per il primo teorema di
    isomorfismo \(M\neq A^{{(\Lambda)}} / \mathrm{ker}f\).

    Per la seconda parte se \(M\) è finitamente generato posso scegliere \(\Lambda\) finito e viceversa \(M\neq A^{n} / N\) è finitamente generato perché \(A^{n}\) lo è e \(\pi : A^{n}\to A^{n} / N\) è un omomorfismo suriettivo.

\end{proof}

\begin{proposition}{}
    \(A\) è con divisione se e solo se ogni suo \(A\)-modulo è libero
\end{proposition}
\begin{proof}\( \)
\begin{itemize}
    \item[\(\implies \)] (complementi di algebra)
    \item[\(\impliedby \)] Sia \(M\) un \(A\)-modulo semplice. Per ipotesi è
        libero, allora \(M \cong A^{{(\Lambda)}}\) per un qualche \(\Lambda\).
        Ma se \(\# \Lambda > 1\) allora \(A^{{(\Lambda)}}\) non è semplice (\(A \subseteq A^{{(\Lambda)}} \) è un sottomodulo non banale). Inoltre \(\Lambda\neq \varnothing\) (\(A^{{(\varnothing)}} = \{0\} \) non è semplice).

        Ne consegue che \(M\cong A\) e dunque \(A\) è con divisione 
\end{itemize}
\end{proof}
\begin{example}{}
    Con \(A = \mathbb{Z}\), \(\mathbb{Z} / p \mathbb{Z}\) non è libero
\end{example}

Si può dimostrare che se \(A\) è con divisione, allora tutte le basi di un \(A\)-modulo (libero) \(M\) hanno la stessa cardinalità, che viene detta rango e indicata con \(\mathrm{rk}_A M\).

In generale non tutte le basi di un \(A\)-modulo libero hanno la stessa
cardinalità, esistono infatti anelli \(A\) non banali tali che \(\prescript{A}{A} \cong _AA
%TODO pedice giusto
^{n}\) per \emph{ogni} \(n \in \mathbb{N}\).

\begin{example}{}
    Sia \(A = \mathrm{End}_{\mathbb{K}} {(V)}\) con \(\mathbb{K}\) campo e \(\dim_{\mathbb{K}} {(V)} = +\infty\) 
\end{example}


Si dimostra che se \(A \to  B\) è omomorfismo di anelli e il rango dei \(B\)-modulli
liberi è ben definito allora anche il rango degli \(A\)-moduli liberi è ben
definito. Di conseguenza se \(A\neq 0 \) è commutativo allora il rango degli \(A\)-moduli liberi è ben definito (\(\exists I \subseteq A \) ideale massimale e \(\pi : A \to A/I\) omomorfismo con \(A / I\) campo)

\subsection{restrizione degli scalari}
Siano \(A, B\) anelli, con \(f : A \to B\) omomorfismo di anelli. Allora se \(M\) è un \(B\)-modulo allora \(M\) è anche un \(A\)-modulo con \(ax := f{(a)}x\). Si dice allora che \(
_A M\) è ottenuto da \(_B M\) per \textbf{restrizione degli scalari} attraverso
\(f\).

Inoltre se \(M' \subseteq M \) è \(B\)-sottomodulo allora è anche un
\(A\)-sottomodulo e se \(g : M \to N\) è \(B\)-lineare allora \(g\) è anche \(A\)-lineare.


Prima della prossima definizione ricordiamo che il \textbf{centro} di un anello
è sottoanello, con il centro l'insieme degli elementi che commutano con tutti
gli altri elementi e indicato con \(Z{(A)}\),
\[
  Z{(A)} := \{z \in A : za = az \,\,\forall a \in A\} 
\]
\begin{definition}{}
    Sia \(A\) commutativo. Allora una \(A\)-algebra è un omomorfismo di anelli
    \(f:A\to B\) tale che \(\mathrm{im}f \subseteq Z{(B)}\) 
\end{definition}
Se \(f\) è evidente si dice che \(B\) è una \(A\)-algebra
\begin{example}{}
    \(M_n{(A)}\) è una \(A\)-algebra con \(a \mapsto \begin{pmatrix}
        
        a & 0 \\
        0 & a
    \end{pmatrix}\) 
\end{example}

\begin{example}{}
    Se \(A = \mathbb{Z}\) per ogni \(B\) anello l'unico omomorfismo di anelli \(\mathbb{Z}\to B\) è una \(\mathbb{Z}\)-algebra. Infatti l'omomorfismo unico \(\mathbb{Z} \to Z{(B)}\) deve essere lo stesso di \(\mathbb{Z} \to B\) 
\end{example}

\begin{definition}{Morfismo di \(A\)-algebre}
    Siano \(f : A\to B\), \(g : A\to C\) \(A\)-algebre. Un (omo/iso/\dots)morfismo di
    \(A\)-algebre da \(f\) a \(g\) è \(h : B\to C\) (omo/iso/\dots)morfismo di anelli tale
    che \(h \circ f = g\) 
\end{definition}
\[\begin{tikzcd}
	A \\
    B & C
    \arrow["{f}"', from=1-1, to=2-1]
	\arrow["{g}", from=1-1, to=2-2]
    \arrow["{h}", from=2-1, to=2-2]
\end{tikzcd}\]

\begin{example}{}
    Ogni omomorfismo di anelli è omomorfismo di \(\mathbb{Z}\)-algebre. 
\end{example}
\begin{example}{}
    Sia \(f : A\to B\) una \(A\)-algebra. Allora \(\forall I \subseteq B \) ideale \(B / I\) è \(A\)-algebra con \(\pi \circ f\) 
\end{example}

% TODO : algebre libere

\begin{remark}[\textbf{motivazione della definizione}]
    Se \(f : A\to B\) \(A\)-algebra, allora \(B\) è un anello e \(A\)-modulo
    (per restrizione degli scalari) tale che 
    \[
      a{(b b')} = {(ab)}b' = b{(ab')}
    \]
\end{remark}

% TODO UN BEL PO' di ROBA


\begin{lemma}{}\label{lem:com}
    Sia \(0 \to M' \overset{i}{\to } M \overset{p}{\to } M''\) una successione
    esatta di \(A\)-moduli. Siano \(f': A^{m} \to M'\) e \(f'' : A^{n}\to M''\)
    omomorfismi. Allora esiste un diagramma commutativo con righe esatte
\[\begin{tikzcd}
	0 & A^{m} & A^{m+n} & A^{n} & 0 \\
	0 & M' & M & M'' & 0
	\arrow[from=1-1, to=1-2]
	\arrow[from=1-2, to=1-3]
	\arrow["{f'}", from=1-2, to=2-2]
	\arrow[from=1-3, to=1-4]
	\arrow["f", from=1-3, to=2-3]
	\arrow[from=1-4, to=1-5]
	\arrow["{f''}", from=1-4, to=2-4]
	\arrow[from=2-1, to=2-2]
	\arrow["i"', from=2-2, to=2-3]
	\arrow["p"', from=2-3, to=2-4]
	\arrow[from=2-4, to=2-5]
\end{tikzcd}\]
\end{lemma}

\begin{proof}{}
    % TODO manca
\end{proof}

\begin{proposition}{}
\label{prop:varie}
    Sia \(0 \to M' \overset{i}{\to } M \overset{p}{\to } M''\) esatta di
    \(A\)-moduli. Allora
\begin{enumerate}[label = \arabic*.]
    \item \(M f.g. \implies M'' f.g.\) 
    \item \(M', M'' f.g. \implies M f.g.\) 
    \item \(M', M'' f.p. \implies M f.p.\) 
    \item \(M f.g., M'' f.p. \implies M' f.g.\) 
    \item \(M' f.g., M f.p. \implies M'' f.g.\) 
\end{enumerate}
\end{proposition}
\begin{proof}{} \( \) 
\begin{enumerate}[label = \arabic*.]
    \item già visto
    \item In esercizio la dimostrazione diretta. In alternativa possiamo
        applicare il lemma~\ref{lem:com}. Infatti esistono \(f': A^{m} \to M'\) e \(f'' : A^{n} \to M''\) omomorfismi suriettivi e per il lemma~\ref{lem:com} il diagramma
\[\begin{tikzcd}
	0 & A^{m} & A^{m+n} & A^{n} & 0 \\
	0 & M' & M & M'' & 0
	\arrow[from=1-1, to=1-2]
	\arrow[from=1-2, to=1-3]
	\arrow["{f'}", from=1-2, to=2-2]
	\arrow[from=1-3, to=1-4]
	\arrow["f", from=1-3, to=2-3]
	\arrow[from=1-4, to=1-5]
	\arrow["{f''}", from=1-4, to=2-4]
	\arrow[from=2-1, to=2-2]
	\arrow["i"', from=2-2, to=2-3]
	\arrow["p"', from=2-3, to=2-4]
	\arrow[from=2-4, to=2-5]
\end{tikzcd}\]
    commuta. Applicando ora il lemma del serpente otteniamo la successione
    esatta 
    \[
      \mathrm{coKer}f' = 0 \to \mathrm{coKer}f \to  0= \mathrm{coKer} f'' \implies \mathrm{coKer} f = 0 \implies M f.g.
    \]
    \item Si può fare una dimostrazione simile a quella precedente e applicando
        il lemma del serpente troviamo la successione esatta
    \[
    0 \to \mathrm{Ker} f' \to \mathrm{Ker} f \to \mathrm{Ker}f'' \to 0 = \mathrm{coKer} f'
    \]
    per il punto 1. \(M\) è finitamente generato e dunque \(M\) è finitamente
    presentato.
    \item \(M'' f.p. \implies \exists \) successione esatta \(A^{m} \overset{g}{\to } A^{n} \overset{h}{\to }M'' \to 0\). Esiste dunque il diagramma commutativo
\[\begin{tikzcd}
	 & A^{m} & A^{n} & M'' & 0 \\
	0 & M' & M & M'' & 0
	\arrow["g", from=1-2, to=1-3]
	\arrow["{f'}", from=1-2, to=2-2]
	\arrow["h", from=1-3, to=1-4]
	\arrow["f", from=1-3, to=2-3]
	\arrow[from=1-4, to=1-5]
	\arrow["{\mathrm{id}}", from=1-4, to=2-4]
	\arrow[from=2-1, to=2-2]
	\arrow["i"', from=2-2, to=2-3]
	\arrow["p"', from=2-3, to=2-4]
	\arrow[from=2-4, to=2-5]
\end{tikzcd}\]
    infatti voglio \(f\) tale che \(p \circ f = h\) e posso come prima (esercizio).
    % TODO controllare che esista f' e che funzioni tutto
    allora per il lemma del serpente trovo che
    \[
      0 \to \mathrm{coKer}f' \to \mathrm{coKer}f \to 0 = \mathrm{coKer_{
              \mathrm{id}_{M''}
      } }
    \]
    è una successione esatta, e dunque \(\mathrm{coKer}f' \cong \mathrm{coKer}f = M / \mathrm{Im} f\) per cui
    \[
    0 \to \mathrm{Im}f' \to M' \to \mathrm{coKer}f' \to 0
    \]
    è esatta. Concludiamo che \(\mathrm{Im} f' \cong A^{m} / \mathrm{Ker} f'\) e
    dunque è \(f.g.\), da cui anche \(M'\) è finitamente generato per il punto
    1.
    \item \(M\) è finitamente generato, dunque \(M''\) è finitamente generato
        per il punto 0. Come prima \(\exists\,\, A^{m} \to A^{n}, A^{n} \to M''\)
        omomorfismi suriettivi e trovo il diagramma del lemma~\ref{lem:com} e
        applicando il lemma del serpente ottengo la successione esatta
        \[
          \mathrm{Ker} f \to \mathrm{Ker}f'' \to 0 = \mathrm{coKer}f'
        \]
        Sia ora \(f: A^{m+n}\to M\) suriettiva, allora 
        \[
          0 \to \mathrm{Ker} f \to A^{m+n} \to M \to 0
        \]
        è esatta, \(A^{m+n}\) è finitamente generata, \(M\) è finitamente
        presentato, dunque \(\mathrm{Ker}f\) è f.g., quindi per il punto 3. \(\mathrm{Ker}f''\) è f.g. e per il punto 0. \(M\) è f.p.
\end{enumerate}
\end{proof}

\begin{eser}{}
    Dimostrare il punto 1. della dimostrazione precedente direttamente.
\end{eser}
\begin{corollary}{}
    Sia \(M = \bigoplus_{i=1}^{n} M_i \). Allora \(M\) è f.g. / f.p. se e solo
    se \(M_{i} \) è f.g. / f.p. per ogni \(i\).
\end{corollary}
\begin{proof}{}
    La successione \(0 \to \bigoplus_{i=1} ^{n-1}M_i \to M \to M_{n} \to 0\) è
    esatta, dunque
    \begin{itemize}
        \item[\(\implies \)] unsando induzione su \(n\) e i punti 1. e 2. della
            proposizione precedente
        \item[\(\impliedby \)] \(M_n\) è f.g. per il punto 0., ed è f.p. per il
            punto 4.
    \end{itemize}
\end{proof}
\begin{remark}{}
    per il punto 3., se \(M\) è f.p. allora ogni \(A^{n} \overset{p}{\to } M \to 0\) esatta si estende a 
    \[
      A^{m} \to A^{n} \overset{p}{\to } M \to 0
    \]
\end{remark}

\begin{remark}{}
    Sia \(A\) non noetheriano. Allora \(\exists M\) \(A\)-modulo f.g. non
    noetheriamo, ad esempio \(M = A\), ossia \(\exists  M' \subseteq M \)
    sottomodulo non f.g. e in tal caso anche quando \(M\) è finitamente
    presentato, ad esempio nel caso \(M=A\), \(M/M'\) non è f.p. perché
    contraddirrebbe il punto 3.
\end{remark}

Uno può chiedersi dato un modulo se i suoi sottomoduli finitamente generati
siano anche moduli finitamente presentati. Quessto non è in generale vero ma
motiva la seguente definizione

\begin{definition}{Modulo coerente}
    Uno modulo \(M\) è detto \textbf{coerente} se è finitamente generato e tutti
    i suoi sottomoduli finitamente generati sono finitamente presentati.
\end{definition}
\begin{remark}{}
    Chiaramente essendo \(M \subseteq M \) un sottomodulo, se è coerente è anche
    f.p.
\end{remark}
\begin{definition}{Anello coerente}
    Un anello \(A\) è \textbf{coerente} (a sinistra) se \(_AA\) è \(A\)-modulo
    coerente (ossia tutti gli ideali sinistri f.g. di \(A\) sono f.p.)
\end{definition}
\begin{remark}{}
    Se \(A\) è noetheriano e \(M\) è un \(A\)-modulo, allora
    \[
      M \text{ coerente } \iff M \text{ f.p. } \iff M \text{ f.g. } \iff M
      \text{ noetheriano }
    \]
    in particolare \(A\) è coerente. 
\end{remark}
\begin{proof}{}
    Sappiamo già che \(M\) noetheriano se e solo se \(M\) f.g. Resta da
    dimostrare dunque che \(M\) noetheriano se e solo se \(M\) è coerente. So
    che \(M' \subseteq M \) f.g. è noetheriano (perché \(M\) lo è). Allora esiste la successione esatta 
    \[
      0 \to \mathrm{Ker} p \to A^{n} \overset{p}{\to } M' \to 0
    \]
    Ora poiché \(A\) è noetheriano, anche \(A^{n}\) lo è, e dunque \(\mathrm{Ker} p \) è noetheriano, dunque \(\mathrm{Ker}p\) è f.g. e infine \(M'\) è f.p.
\end{proof}
\begin{remark}{}
    Sia \(A\) coerente non noetheriano, allora \(_AA\) è coerente non
    noetheriano
\end{remark}
\begin{example}{}
    Sia \(A\) non noetheriano, \(I \subseteq A \) ideale sinistro non f.g.,
    allora \(A/I\) è f.g. non f.p. e \(A / I\) può anche essere notheriano.

    Un esempio è \(A = \mathbb{K}[X_n | n \in \mathbb{N}]\), \(I = {(X_n | n \in \mathbb{N})}\), \(A / I = \mathbb{K}\) 

\end{example}

\begin{remark}{}
    Sia \(f : M \to N\) \(A\)-lineare, con \(M, N\) finitamente generati. Allora
    \(\mathrm{Im} f \cong M / \mathrm{Ker} f\) e \(\mathrm{coKer} f \cong N / \mathrm{Im} f \)
    sono finitamente generati (punto 0. della proposizione) ma non
    necessariamente anche \(\mathrm{Ker} f\) se \(A\) è non noetheriano.
    % TODO inserire esempio
\end{remark}


\begin{proposition}{}
    Sia \(0 \to M' \overset{i}{\to } m \overset{p}{\to } M'' \to 0\) esatta di
    \(A\)-moduli.
\begin{enumerate}[label = \arabic*.]
    \item \(M'\) f.g. e \(M\) coerente, allora \(M''\) è coerente
    \item \(M', M''\) coerenti, allora \(M\) è coerente
    \item \(M\) è coerente, \(M''\) è f.p., allora \(M'\) è coerente
\end{enumerate}
in particolare \(M', M, M''\) sono coerenti se due di essi lo sono.
\end{proposition}
\begin{proof}{}
\begin{enumerate}[label = \arabic*.]
    \item \(M''\) è f.g. per il punto 0. della proposizione~\ref{prop:varie}.
        \(N'' \subseteq M'' \)  è f.g., allora c'è una successione esatta
        \[
          0 \to M' \to N := p^{-1}{(N'')} \to N'' \to 0
        \]
        Allora \(N\) è f.g. per 1. di~\ref{prop:varie} e dunque \(N\) è f.p.
        perché \(M\) è coerente, da cui \(N''\) è f.p. per 4. di \ref{prop:varie}
    \item \(M\) è f.g. per 1. di~\ref{prop:varie}, se \(N \subseteq M \)
        sottomodulo finitamente generato, allora esiste la successione esatta
        \[
          0 \to N' := i^{-1}{(N)} \to N \to N'' := p{(N)} \to 0
        \]
        Allora \(N''\) è f.g. per 0. di~\ref{prop:varie} da cui \(N''\) è f.p.
        per la coerenza di \(M\), dunque \(N'\) è f.g. per 3. di~\ref{prop:varie}. Segue dalla oerenza di \(M'\) che \(N'\) è f.p. e dunque \(N\) lo è per 2. di~\ref{prop:varie}
    \item M' è f.g. per 3. di~\ref{prop:varie} dunque \(M'\) è coerente perché
        \(M' \cong i{(M')} \subseteq M \) sottomodulo è f.g. e \(M\) è coerente.
\end{enumerate}
\end{proof}
\begin{eser}{}
    mostrare che
    \[
      \bigoplus_{i=1} ^{n} M_i \text{ coerente } \iff M_{i} \text{ coerente } \forall i
    \]
\end{eser}

\begin{corollary}{}
    Sia \(f: M \to N\) \(A\)-lineare, \(M, N\) coerenti, allora \\ \(\mathrm{Ker}f, \mathrm{Im}f, \mathrm{coKer}f\) sono coerenti.
\end{corollary}
\begin{proof}{}
    \(\mathrm{Im}f \cong M / \mathrm{Ker} f\) è f.g. per 0. di~\ref{prop:varie}
\end{proof}
\begin{corollary}{}
    Se \(A\) è coerente e \(M\) è un \(A\)-modulo f.p., allora \(M\) è coerente.
\end{corollary}
\begin{proof}{}
    Basta osservare che per definizione esiste una successione esatta 
    \[
      A^{m} \overset{f}{\to } A^{n} \to M \to 0
    \]
    e in particolare dunque \(M \cong \mathrm{coKer}f\) e poiché \(A^{m}\) e \(A^{n}\) sono coerenti, lo è pure \(M\) 
\end{proof}
\begin{example}{}
    Sia \(A\) commutativo tale che \(A[X_{1}, \dots, X_{n}]\) sia coerente \(\forall n \in \mathbb{N}\) (ad esempio \(A\) noetheriano, per il teorema della base di Hilbert)
    Allora \(A[X_{\lambda} | \lambda \in \Lambda]\) è coerente \(\forall \Lambda\), anche se non è noetheriano per \(\# \Lambda = +\infty\) e \(A \neq 0\).
    \begin{proof}[Idea della dimostrazione]
        Sia \(I \subseteq B \) ideale f.g., ossia \(I = {(f_{1}, \dots, f_{n})}\). Allora \(\exists \Lambda_{0} \subseteq \Lambda \) finito tale che \(f_{1} \in B_{0} := A[X_\lambda | \lambda \in \Lambda_{0}]\) 
    \end{proof}
\end{example}
\begin{example}[Anello non coerente]
    Presi \(A\) e \(B\) come prima, ma supponiamo che \(A = \mathbb{K}\) campo.
    Prendiamo dunque \(J := {(X_{\lambda} | \lambda \in \Lambda)}\), con \(\#
    \Lambda = +\infty\). Allora preso
    \[
      C = B / J^2
    \]
    non è coerente. Preso infatti ad esempio 
    \[
      I = C \overline{X_{\lambda} } \text{  con }\lambda \in \Lambda
    \]
    è f.g. ma non f.p. perché c'è la successione esatta 
    \[
      0 \to J / J^2 \to C \to I \to 0
    \]
    e \(J / J^2\) è \(C\)-modulo annullato da \(J / J^2\) e come \(C / {(J / J^2)} \cong B / J \cong \mathbb{K}\)-modulo ha dimensione \(\infty\)  con base \(\{\overline{x_{\lambda} }| \lambda \in \Lambda \} \) 
\end{example}
