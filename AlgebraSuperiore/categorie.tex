\chapter{Categorie}

\begin{definition}{Categoria}
    Una \textbf{categoria} \(\mathcal{C}\) è data da una classe di oggetti \(\mathrm{Ob}{(\mathcal{C})}\) e \(\forall X, Y \in \mathrm{Ob}{(\mathcal{C})}\) da un insieme di morfismi da \(X \) a \(Y\) indicato con
    \(\mathrm{Hom}{(X,Y)} = \mathrm{Hom}_\mathcal{C} {(X, Y)} = \mathcal{C}{(X, Y)}\)  e da una
   azione  composizione di morfismi, cioè \(\forall X, Y, Z \in \mathrm{Ob}{(\mathcal{C})}\)
    (anche scritto \(X, Y, Z \in \mathcal{C}\) ) un'operazione 
    \begin{align*}
        \mathcal{C}{(X, Y)} \times  \mathcal{C}{(Y, Z)} &\to  \mathcal{C}{(X, Z)}
        {(f, g)} &\mapsto g \circ f
    \end{align*}

    tale che 
\begin{enumerate}[label = \arabic*.]
    \item[0.] \(\mathcal{C}{(X,Y)} \cap  \mathcal{C}{(X', Y')} \neq \varnothing \implies X = X' \text{ e } Y = Y'\) 
    \item[1.] \(\circ\) è associativa, cioè \(\forall X, Y, Z, W \in \mathcal{C}\) e \(\forall f \in \mathcal{C}{(X,Y)}\) e \(\forall g \in \mathcal{C}{(Y, Z)}\) e \(\forall h \in \mathcal{C}{(Z, W)}\)  allora 
        \[
          h \circ {(g \circ f)} = {(h \circ g)} \circ f
        \]
    \item[2.] \(\forall  X \in  \mathcal{C} \) esiste \(1_X = \mathrm{id}_X \in \mathcal{C}{(X, X)}\)
        che è eleemento neutro di \(X\) cioè \(\forall Y \in \mathcal{C}\) e \(\forall f
        \in \mathcal{C}{(X, Y)}\), 
        \[
          f \circ 1_X = f \quad, \quad 1_Y \circ f = f
        \]
\end{enumerate}
\end{definition}
\begin{example}{}
    La categoria degli insiemi \(\mathtt{Set} \) che ha come oggetti tutti gli
    insiemi e \(\forall X, Y \in \mathtt{Set} \) i morfismi \(\mathtt{Set}{(X, Y)} = \{f : X \to Y\}  \) le funzioni e \(\circ\) la composizione di funzioni
\end{example}
\begin{remark}{}
    Se ho \(\mathcal{C}\) tale che valgano solo 1. e 2. e non necessariamente 0. posso
    ottenere la categoria \(\mathcal{C}'\) che soddisfa anche 0. ponendo \(\mathrm{Ob}{(\mathcal{C}')} := \mathrm{Ob}{(\mathcal{C})}\) e 
    \[
      \mathcal{C}'{(X, Y)} := \{X\} \times \mathcal{C}{(X,Y)} \times \{Y\} 
    \]
\end{remark}
\begin{example}{}
    Le categorie concrete, in cui gli oggetti sono insiemi con qualche struttura
    e i morfismi sono funzioni tra insiemi che preservano la struttura (con \(\circ\) sempre la composizione di funzioni). In particolare:
\begin{itemize}
    \item La categoria \(\mathtt{Grp}\) dei gruppi, dove gli oggetti sono i
        gruppi e i morfismi gli omomorfismi di gruppi
    \item La categoria \(\mathtt{Rng}\) degli anelli
    \item Dato un anello \(A\), la categoria \(\mathtt{A-Mod}\) / \(\mathtt{Mod-A}\) degli \(A\)-moduli sinistri / destri
    \item Dato un anello commutativo \(A\), la categoria \(\mathtt{A-Alg}\)
        delle \(A\)-algebre
    \item La categoria \(\mathtt{Top}\) degli spazi topologici (con funzioni
        continue come morfismi)
\end{itemize}
\end{example}
\begin{note}{}
    Dato \(f \in \mathcal{C}{(X, Y)}\) si può indicare con \(f : X \to Y\) ``come fosse
    una funzione''
\end{note}

\begin{example}{}
    Le categorie discrete, cioè tali che gli unici morfismi sono \(1_X\) per
    ogni \(X \in \mathcal{C}\).
\end{example}

\begin{example}{}
    \(\mathcal{C}\) tale che \(\forall X, Y \in \mathcal{C}\), \(\# \mathcal{C} {(X, Y)} \le 1\), ottengo
    una relazione \(\preccurlyeq \) su \(\mathrm{Ob}{(\mathcal{C})}\) in cui
    \[
      X \preccurlyeq Y \iff \mathcal{C}{(X, Y)} \neq \varnothing
    \]
    e \(\preccurlyeq\)  è riflessivo (perché \(\exists  1_X \in \mathcal{C}{(X, X)} \forall X \in \mathcal{C}\)) e transitivo, perché \(\exists \circ\). Ne consegue che \(\preccurlyeq\) è un \emph{preordine}

    Viceversa, data una relazione di preordine \(\preccurlyeq\) su un insieme (o
    una classe) \(S\), ottengo una categoria \(\mathcal{C}\) con \(\mathrm{Ob}{(\mathcal{C})} := S\) e \(\forall X, Y \in S\), 
    \[
      \mathcal{C}{(X, Y)} := \begin{cases}{}
          \{f_{X,Y} \} & \text{ se } X \preccurlyeq Y \\
          \varnothing & \text{ altrimenti}
      \end{cases}
    \]
    con l'unica composizione possibile
\end{example}

\begin{example}[Categoria Vuota]
    Prendendo \(\mathrm{Ob}{(\mathcal{C})} = \varnothing\)
\end{example}
\begin{remark}{}
    \(\forall X \in \mathcal{C}\) con \(\mathcal{C}\) una categoria, \(\mathrm{End}_\mathcal{C}{(X)} := \mathcal{C}{(X, X)}\) è un monoide con \(\circ\), ne consegue il prossimo esempio
\end{remark}
\begin{example}[Monoide]
    Una categoria con un solo oggetto è un monoide. Viceversa ogni monoide può
    essere visto come categoria di un solo oggetto.
\end{example}

\begin{example}[Diagrammi]
    Possiamo definire categorie date da diagrammi, in cui si rappresentano i
    morfismi (non l'identità). Ad esempio:
\[\begin{tikzcd}
	\bullet & \bullet & \bullet & \bullet & \bullet & \bullet & \bullet
	\arrow[from=1-1, to=1-2]
	\arrow[shift left, from=1-3, to=1-4]
	\arrow[shift right, from=1-3, to=1-4]
	\arrow[from=1-5, to=1-6]
	\arrow[from=1-7, to=1-6]
\end{tikzcd}\]
    sono tre categorie diverse, rispettivamente con 2, 2, e 3 oggetti
\end{example}

\begin{definition}{Categoria opposta}
    La \textbf{categoria opposta} di \(\mathcal{C}\) è denotata \(\mathcal{C}^{op}\) ed è definita
    da
    \[
        \mathrm{Ob}{(\mathcal{C}^{op})} := \mathrm{Ob}{(\mathcal{C})} \quad \mathcal{C}^{op}{(X, Y)} := \mathcal{C}{(Y, X)}
    \]
    con composizione in \(\circ^{op}\) data da \(f \circ^{op} g := g \circ f\) 
\end{definition}
\begin{remark}{}
    \[
      {(\mathcal{C}^{op})}^{op} = \mathcal{C}
    \]
\end{remark}
\begin{example}[Categoria Prodotto]
    Siano \(\mathcal{C}_{\lambda} \) per \(\lambda \in \Lambda\) delle categorie. Allora
    la categoria prodotto
    \[
      \mathcal{C} := \prod_{\lambda \in \Lambda} \mathcal{C}_{\lambda} 
    \]
    è definita da 
    \begin{align*}
      \mathrm{Ob}{(\mathcal{C})} &:= \prod_{\lambda \in \Lambda} \mathrm{Ob}{(\mathcal{C}_\lambda)} \\
      \mathcal{C}{({(X_{\lambda} )}_{\lambda \in \Lambda}, {(Y_{\lambda} )}_{\lambda \in \Lambda} )} &:= \prod_{\lambda \in \Lambda} \mathcal{C}_{\lambda} {(X_{\lambda} , Y_\lambda)} \\
      {(g_{\lambda} )}_{\lambda \in \Lambda} \circ {(f_{\lambda} )}_{\lambda \in \Lambda} &:= {(g_{\lambda} \circ f_{\lambda} )}_{\lambda \in \Lambda} 
    \end{align*}
\end{example}

\begin{example}[Cateogoria Coprodotto]
    La categoria coprodotto
    \[
      \mathcal{C} := \coprod_{\lambda \in \Lambda} \mathcal{C}_{\lambda} 
    \]
    è definita con \(\mathrm{Ob}{(\mathcal{C})} := \coprod_{\lambda \in \Lambda} \mathrm{Ob}{(\mathcal{C}_{\lambda} )}\) l'unione disgiunta. 
    \[
      \forall X, Y \in \mathcal{C} \quad \mathcal{C}{(X, Y)} := \begin{cases}{}
          \mathcal{C}_{\lambda} {(X, Y)} & \text{ se } X, Y \in \mathcal{C}_{\lambda} \text{ per
          qualche } \lambda \in \Lambda \\
              \varnothing & \text{ altrimenti}
      \end{cases}
    \]
    con \(\circ\) ovvia.
\end{example}

\begin{definition}{Sottocategoria}
    Sia \(\mathcal{C}\) una categoria. Allora una sottocategoria \(\mathcal{C}'\) di \(\mathcal{C}\) è data
    da una sottoclasse \(\mathrm{Ob}{(\mathcal{C}')} \subseteq \mathrm{Ob}{(\mathcal{C})} \) e \(\forall X, Y \in \mathcal{C}'\) da un sottoinsieme \(\mathcal{C}'{(X, Y)} \subseteq \mathcal{C}{(X, Y)} \) tale che \(\circ\) si restringe a \(\mathcal{C}'\) e \(1_X \in \mathcal{C}'{(X, X)}\) per ogni \(X \in \mathcal{C}'\).

    In particolare \(\mathcal{C}'\) è una categoria.
\end{definition}

\begin{example}{}
    Se \(\mathcal{C}\) è un monoide (cateogoria di un oggetto), allora le sottocategorie
    non vuote di \(\mathcal{C}\) sono i sottomonoidi.
\end{example}

\begin{definition}{Sottocategoria Piena}
    Una sottocategoria \(\mathcal{C}'\) di \(\mathcal{C}\) si dice \textbf{piena} se \(\mathcal{C}'{(X, Y)} =
    \mathcal{C}{(X, Y)}\) per ogni \(X, Y \in \mathcal{C}'\) 
\end{definition}

\begin{remark}{}
    Una sottocategoria piena di \(\mathcal{C}\) equivale a dare una sottoclasse di \(\mathrm{Ob}{(\mathcal{C})}\) 
\end{remark}

\begin{example}[Gruppi Abeliani]
    \(\mathtt{Ab} \subseteq \mathtt{Grp} \) sottocategoria piena dei gruppi
    abeliani. Similmente anche \(\mathtt{\mathcal{C}Rng} \subseteq \mathtt{Rng} \)
    sottocategoria piena degli anelli commutativi.
\end{example}

Oltre alle sotto-strutture sono anche importanti i quozienti, e anche qui
possiamo dare una definizione astratta

\begin{definition}{Congruenza}
    Una congruenza \(\sim \) su una categoria \(\mathcal{\mathcal{C}}\) è data da una relazione di
    equivalenza \(\sim \) su \(\mathcal{\mathcal{C}}{(X, Y)}\) \(\forall X, Y \in \mathcal{\mathcal{C}}\) tale che
    \[
      \forall X, Y, Z \in \mathcal{\mathcal{C}}, \, \forall f, f' \in \mathcal{\mathcal{C}}{(X, Y)} \, \forall g,
      g' \in \mathcal{\mathcal{C}}{(Y, Z)} \quad f \sim f', g \sim g' \implies g \circ f \sim g'
      \circ f'
    \]
    equivalentemente \(g \sim g' \implies g \circ f \sim g' \circ f\) e \(h \circ g \sim h \circ g'\) 
\end{definition}

\begin{definition}{Quoziente}
    Sia \(\sim \) una congruenza su \(\mathcal{\mathcal{C}}\), allora possiamo definire la categoria
    quoziente \(\mathcal{\mathcal{C}} /_\sim \) definita da
    \[
      \mathrm{Ob}{(\mathcal{\mathcal{C}} /_\sim )} = \mathrm{Ob}{(\mathcal{\mathcal{C}})} \quad {(\mathcal{\mathcal{C}} /_\sim )}{(X, Y)} :=
      \mathcal{\mathcal{C}}{(X, Y)} /_\sim \quad \forall X, Y \in \mathcal{\mathcal{C}}
    \]
    e \(\circ\) è indotta da quella di \(\mathcal{\mathcal{C}}\), ossia
    \[
      \overline{g} \circ \overline{f} := \overline{g \circ f}
    \]
\end{definition}

\begin{example}[Omotopia]
    Sia \(\mathcal{\mathcal{C}} = \mathtt{Top}\) e \(\sim_h \) l'omotopia, ossia \(f, g : X \to Y\)
    sono omotope se \(\exists  H : X \times [0, 1] \to Y\) continue tali che 
    \[
      f{(x)} = H{(x, 0)}, \quad g{(x)} = H{(x, 1)} \quad \forall x \in X
    \]
    e si ottiene \(\mathtt{Toph} := \mathtt{Top} / \sim_h \) 
\end{example}

\begin{example}[Gruppo quoziente]
    Sia \(G\) un gruppo (visto come monoide, ossia categoria di un oggetto) e
    sia \(H \vartriangleleft G\) e \(\sim \) su \(G\) data da \(a \sim b \iff aH
    = bH\). Allora \(G/N\) è la categoria quoziente \(G / \sim \).
    Viceversa ogni \(\sim \) congruenza su \(G\) si può scrivere in tal
    modo prendendo \(H = \{a \in G : a \sim 1\} \vartriangleleft G \) (esercizio).
\end{example}

\begin{definition}{morfismo invertibile}
    Sia \(f : X\to Y\) un morfismo in una categoria \(\mathcal{C}\). Allora esso è
    invertibile a sinistra (destra) se \(\exists  f' : Y \to X\) tale che
    \(f' \circ f = 1_X\) (\(f \circ f' = 1_Y\)).
\end{definition}
\begin{remark}{}
    \(f\) è invertibile a sinistra (destra) in \(\mathcal{C}\), allora \(f\) è invertibile
    a destra (sinistra) in \(\mathcal{C}^{op}\) 
\end{remark}

\begin{definition}{Isomorfismo}
    \(f : X \to Y\) è un \textbf{isomorfismo} se \(\exists f' : Y \to X\) tale che \(f'
    \circ f = 1_X\) e \(f \circ f' = 1_Y\) 
\end{definition}
\begin{remark}{}
    \(f\) è isomorfismo se e solo se \(f\) è invertibile a destra e a sinistra. 
\end{remark}
\begin{proof}\( \)
\begin{itemize}
    \item[\(\implies \)] ovvio
    \item[\(\impliedby \)] \(\exists f', f''\) tale che \(f' \circ f = 1_X\)  e
        \(f \circ f'' = 1_Y\), allora
        \[
          f' \circ {( f \circ f'')} = f' = f'' = {(f' \circ f)} \circ f''
        \]
        e dunque \(f\) è invertibile.
\end{itemize}
    In particolare dunque la \(f'\) della definizione di isomorfismo è unica e
    viene denotata \(f^{-1}\) 
\end{proof}

\begin{definition}{}
    Siano \(X, Y \in \mathcal{C}\). Allora \(X\) e \(Y\) sono isomorfe (\(X \cong Y\)) se
    esiste un \(f: X \to Y\) isomorfismo.
\end{definition}

\begin{remark}{}
    \(1_X\) è isomorfismo e \(1_X^{-1} = 1_X\). Se \(f\) isomorfismo
    allora \(f^{-1}\) isomorfismo e \({(f^{-1})}^{-1} = f\). Se \(f, g\)
    isomorfismi componibili, allora \(g \circ f\) è isomorfismo e \({(g \circ f)}^{-1} = f^{-1} \circ g^{-1}\) 

    Ne segue che \(\cong\) è una relazione di equivalenza su \(\mathrm{Ob}{(\mathcal{C})}\) 
\end{remark}

\begin{definition}{}
    Un morfismo \(f : X \to Y\) in \(\mathcal{C}\) è detto \textbf{monomorfismo} se \(\forall Z \in \mathcal{C}\) la funzione 
    \begin{align*}
        f_*: \mathcal{C}{(Z, X)} &\longrightarrow \mathcal{C}{(Z, Y)} \\
        g &\longmapsto f_*(g) = f \circ g
    \end{align*}
    è iniettiva
\end{definition}
\begin{definition}{Epimorfismo}
    \(f\) è un \textbf{epimorfismo} in \(\mathcal{C}\) se è monomorfismo in \(\mathcal{C}^{op}\),
    ossia \(\forall Z \in \mathcal{C}\) la funzione
    \begin{align*}
        f^*: \mathcal{C}{(Y, Z)} &\longrightarrow \mathcal{C}{(X, Z)} \\
        g &\longmapsto f^*(g) = g \circ f
    \end{align*}
    è iniettiva.
\end{definition}

\begin{proposition}{}
    \(f\) è invertibile a sinistra (destra), allora \(f\) è monomorfismo
    (epimorfismo)
\end{proposition}
\begin{proof}{}
    Basta dimostrare che se \(f\) è invertibile a sinistra, allora è mono.

    Sappiamo che \(\exists f' : Y\to X\) tale che \(f' \circ f = 1_X\). Dobbiamo
    dimostrare che \(f_*\) è iniettiva. Siano \(g, h \in \mathcal{C}{(Z, X)}\) tali che \(f_*{(g)} = f_*{(h)}\). Allora \(f \circ g = f \circ g\), da cui \(f' \circ f \circ g = f' \circ f \circ h\) e dunque \(g = h\) 
\end{proof}

\begin{proposition}{}
    Sia \(\mathcal{C}\) concreta. Allora
    \[
      f \text{ invertibile a \emph{sinistra}/\emph{destra}} \implies f \text{ \emph{iniettiva}/\emph{suriettiva}} \implies f \text{ \emph{mono}/\emph{epi}}
    \]
\end{proposition}
\begin{proof}{}
    Non possiamo usare il trick della categoria opposta, perché non è detto che
    \(\mathcal{C}^{op}\) sia ancora concreta.

    Sia \(f'\) tale che \(f' \circ f = 1_X\) (\(f \circ f' = 1_Y\)), allora
    chiaramente \(f\) iniettiva (suriettiva) perché le composizioni \(1_X\) e \(1_Y\) sono biunivoche.

    Se \(f\) è iniettiva, allora siano \(g_{1}, g_{2}: Z \to X\). Dunque \(\forall x \in X\) 
    \[
      f{(g_{1}{(x)})} = f{(g_{2}{(x)})} \overset{f \text{ inj.}}{\implies } g_{1}{(x)} = g_{2}{(x)}
    \]
    ossia \(f_*\) è iniettiva, ossia \(f\) è monomorfismo.

    se \(f\) è suriettiva, allora siano \(
    g_{1}, g_{2} : Y \to Z\). Sappiamo che \(\forall y \in Y\) esiste \(x_y \in X\) tale che \(f{(x_y)} = y\). Allora abbiamo che, assumendo che \(g_{1}\circ f = g_{2}\circ f\) 
    \[
        g_{1}{(y)} = g_{1}{(f{(x_y)})} = g_{2}{(f{(x_y)})} = g_{2}{(y)} \quad \forall y \in Y
    \]
    ossia \(f^{*}\) è iniettiva e dunque \(f\) è epimorfismo
\end{proof}

In generale non vale nessuna delle \(\impliedby\).
\begin{example}{}
    In \(\mathtt{Set}\) se \(f : X \to Y\) è suriettiva, allora \(f\) è invertibile a
    sinistra. Infatti basta scegliere (AOC) \(f'{(y)} \in f^{-1}\{y\} \) per
    ogni \(y \in Y\). Inoltre se \(X \neq \varnothing\) e \(f : X \to Y\) è
    iniettiva, allora \(f\) è invertibile a sinistra.
\end{example}

\begin{eser}{}
    In \(\mathtt{A-Mod}\), mostrare che \(f : M\to N\) iniettiva è invertibile a
    sinistra se e solo se \(\mathrm{Im}{(f)} \subseteq N \) è addendo diretto.

    Mostrare che \(f : M \to N\) suriettiva è invertibile a destra se e solo se
    \(\mathrm{Ker}{(f)} \subseteq M \) è addendo diretto

    Concludere che valgono sempre entrambe le implicazioni se e solo se \(A\) è
    semisemplice.
\end{eser}

\begin{example}{}
    In \(\mathtt{Set}\), se \(f\) è mono (epi), allora \(f\) è iniettiva
    (suriettiva).

    Infatti, poniamo per assurdo \(f : X \to Y\) non iniettiva, dunque siano \(x, y \in X\) tali che \(f{(x)} = f{(y)}\). Allora preso \(Z = \{z\} \) e \(g, h : Z\to X\) tali che \(g
    {(z)} = x\) e \(h{(z)} = y\) abbiamo che \(f \circ g = f \circ h\) da cui \(g = h\) e dunque \(x = y\) 

    Supponiamo \(f\) non suriettiva, mostrare \emph{per esercizio} \(\exists g, h : Y \to Z\) tali che
    \(g \neq h\) ma \(g \circ f = h \circ f\) 
\end{example}

\begin{example}{}
    In \(\mathtt{A-Mod}\) \(f : M \to N\) è mono (epi), allora \(f\) è iniettiva (suriettiva).

    Infatti \(i : \mathrm{Ker}f \to M\) inclusione tale che \(f \circ i = 0\) e
    anche \(0 : \mathrm{Ker}f \to M\) è tale che \(f \circ 0 = 0\). Concludiamo
    che \(i = 0\) e dunque \(\mathrm{Ker}f = 0\).

    Similmente \(\pi : N \to \mathrm{coker}f\) è tale che \(\pi \circ f = 0\) e
    se \(f\) è epi allora \(0 = \pi \) e dunque \(\mathrm{coker} f= 0\) e dunque
    \(f\) è suriettiva.
\end{example}

\begin{example}{}
    In \(\mathtt{Grp}\) \(f\) mono (epi), allora \(f\) iniettiva (suriettiva)

    Per mono \(\implies \) iniettiva si può usare la stessa dei moduli, mentre
    per l'altra è un po' più complicato, ma si dimostra che è vero lo stesso
\end{example}

\begin{example}{}
    In \(\mathtt{Rng}\) \(f : A\to B\) mono, allora \(f\) iniettiva. % TODO eser

    Tuttavia \(f\) epi \textbf{non implica} f suriettiva. Ad esempio preso \(i :
    \mathbb{Z} \hookrightarrow \mathbb{Q}\) è epi, infatti \(\forall A\) anello
    esiste al più un omomorfismo \(\mathbb{Q} \to A\) (\(f : \mathbb{Q}\to A\)
    sia omomorfismo, allora \(f|_{\mathbb{Z}} \) è l'unico omomorfismo e \(f{(\frac{a}{b})} = f{(a)}f{(b)}^{-1}\)). Chiaramente però non è suriettiva.
\end{example}

\begin{definition}{Funtore}
    Un funtore \(F : \mathcal{C} \to D\) tra 2 categorie è dato da una funzione
    \(F : \mathrm{Ob}{(\mathcal{C})} \to \mathrm{Ob}{(D)}\) e \(\forall X, X' \in \mathcal{C}\) una
    funzione \(F = F_{X, X'} : \mathcal{C}{(X, X')} \to D{(F{(X)}, F{(X')})}\) tale che
    \[
      F{(g\circ f)} = F{(g)}\circ F{(f)} 
    \]
    (se \(f\) e \(g\) sono componibili in \(\mathcal{C}\)) e \(F{(1_X)} = 1_{F{(X)}} \)
    per ogni \(X \in \mathcal{C}\) 
    
\end{definition}
\begin{proposition}{}
    Sia \(F\) un funtore e \(f\) invertibile a sinistra (destra). Allora \(F{(f)}\) è invertibile a sinistra (destra)
\end{proposition}
\begin{proof}{}
    \(\exists f'\) tale che \(f' \circ f = 1_X\), allora \(F{(f')} \circ F{(f)} = F{(f'\circ f)} = F{(1_X)} = 1_{F{(X)}} \).
\end{proof}
\begin{remark}{}
    Segue che \(f\) iso, allora \(F{(f)}\) iso e \(F{(f)}^{-1} = F{(f^{-1})}\) 
\end{remark}

\begin{example}{}
    Sia \(\mathcal{C}' \subseteq \mathcal{C} \) sottocategoria. Allora \(\mathcal{C}' \to \mathcal{C}\), \(X \mapsto X\) e \(f \mapsto f\) è un funtore
\end{example}

\begin{example}{}
    Se \(\sim \) è una congruenza, allora \(\mathcal{C} \to \mathcal{C} /\sim\) è un funtore, con
    \(X \mapsto X\) e \(f \mapsto \overline{f}\) 
\end{example}

\begin{example}[Funtore dimenticante]
    \(\mathcal{C} \to \mathtt{Set}\) con \(\mathcal{C}\) categoria discreta e \(X \mapsto X\), \(f
    \mapsto f\) è un funtore, che ``dimentica'' la struttura aggiunta.

    Analogamente anche \(\mathtt{Rng} \to \mathtt{Ab}\), con \({(A, +, \cdot )} \to {(A, +)}\) è un funtore dimenticante.
\end{example}
\begin{remark}{}
    Notare che il secondo funtore dimenticante non preserva gli epimorfismi.
    Sarebbe infatti \(i : \mathbb{Z} \to \mathbb{Q}\) l'inclusione è
    un'epimorfismo in \(\mathtt{Rng}\) ma non in \(\mathtt{Ab}\) 
\end{remark}


\begin{example}{}
    Sia \(A \to B\) un omomorfismo di anelli. Allora la restrizione degli
    scalare è un funtore \(\mathtt{B-Mod} \to \mathtt{A-Mod}\) 
\end{example}


\begin{example}{}
    Funtore tra 2 categorie discrete \(\mathcal{C}\) e \(D\) è una funzione \(\mathrm{Ob}{(\mathcal{C})} \to \mathrm{Ob}{(D)}\) 
\end{example}

\begin{example}{}
    Un funtore tra 2 preordini \(\mathcal{C}\) e \(D\) è una funzione \(\mathrm{Ob}{(\mathcal{C})} \to \mathrm{Ob}{(D)}\)  che preserva la relazione di preordine.
\end{example}

\begin{example}{}
    Un funtore tra 2 monoidi è un omomorfismo di monoidi.

    Più in generale dato \(G\)  un monoide e una categoria \(\mathcal{C}\) , un funtore \(G \to \mathcal{C}\) è dato da \(X \in \mathcal{C}\) e da un omomorfismo di monoidi \(G \to \mathrm{End}_\mathcal{C}{(X)}\) 

    Se \(G\) è un gruppo un funtore \(G \to \mathcal{C}\) è dato da \(X \in \mathcal{C}\) e un
    omomorfismo di gruppi \(G \to \mathrm{Aut}_\mathcal{C}{(X)}\). Ad esempio se \(\mathcal{C} = \mathtt{Set}\) il funtore dà un'azione di un gruppo su un insieme.
    Similmente se \(\mathcal{C} = \mathbb{K}\)-spazi vettoriali ho una rappresentazione di
    \(G\).
\end{example}

\begin{example}[Funtore costante]
    Date \(\mathcal{C}, D\) categorie preso \(Y \subseteq D \) si può considerare il
    funtore costante di valore \(Y\), \(\mathcal{C} \to D\), \(X \mapsto Y\) e \(f \mapsto 1_Y\) 
\end{example}

\begin{example}{}
    Presa \(\mathtt{Top}_*\) la categoria degli spazi topologici puntati, il
    gruppo fondamentale 
    \[
        \pi_{1} : \mathtt{Top}_* \to \mathtt{Grp}
    \]
    è un funtore
\end{example}

\begin{example}{}
    \(\forall n \in \mathbb{N}\) ci sono funtori di omologia (singolare)
    \[
      H_{n} : \mathtt{Top} \to \mathtt{Ab}
    \]
\end{example}
\begin{theorem}{Omomorfismo}
    Sia \(\sim \) una congruenza su \(\mathcal{C}\) e \(F : \mathcal{C} \to D\) un funtore tale che
    se \(f \sim f'\) in \(\mathcal{C}\) allora \(F{(f)} = F{(f')}\).
    Allora esiste un unico funtore \(\overline{F} : \mathcal{C} /_\sim \to D\) tale
    che \(\overline{F}{(\overline{f})} = F{(f)}\) per ogni \(f\) morfismo di \(\mathcal{C}\) 
\end{theorem}
\begin{example}{}
    Negli esempi precedenti se \(f\) e \(f'\) sono omotope, allora \(\pi_{1}{(f)} = \pi_{1}{(f')}\) e \(H_{n}{(f)} = H_{n}{(f')}\), dunque inducono funtori
    \[
      \pi_{1} : \mathtt{Toph}_* \to \mathtt{Grp} \quad H_{n} : \mathtt{Toph} \to \mathtt{Ab}
    \]
\end{example}

\begin{note}{}
    I funtori che abbiamo definito si dicono anche funtori covarianti
\end{note}
\begin{definition}{funtore controvariante}
    Un funtore \textbf{controvariante} \(\mathcal{C} \to D\) è un funtore (covariante) \(\mathcal{C}^{op} \to D\).
\end{definition}
\begin{example}{}
    \(\forall n \in \mathbb{N}\) i funtori di coomologia (singolare) sono funtori
    controvarianti \(H^{n} : \mathtt{Top(h)}^{op} \to \mathtt{Ab}\) 
\end{example}
\begin{example}{}
    Sia \(\mathcal{C}\) una categoria, \(X \in \mathcal{C}\)
    \begin{align*}
        \mathcal{C}{(X, \--)} : \mathcal{C} &\to \mathtt{Set} \\
        Y \mapsto \mathcal{C}{(X, Y)} \quad {(f : Y \to Y')} &\mapsto {(f_{*} : \mathcal{C}{(X, Y)} \to \mathcal{C}{(X, Y')})} \\
            & \quad \quad \quad \quad\quad\quad\quad  g \mapsto f \circ g
    \end{align*}
    è un funtore perché \({(f' \circ f)}_* = f'_* \circ f_*\) 

    Analogamente 
    \begin{align*}
        \mathcal{C}{(\--, Y)} : \mathcal{C}^{op} &\to \mathtt{Set} \\
        X \mapsto \mathcal{C}{(X, Y)} \quad {(f : X \to X')} &\mapsto {(f^{*} : \mathcal{C}{(X', Y)} \to \mathcal{C}{(X, Y)})} \\
            & \quad \quad \quad \quad\quad\quad\quad  g \mapsto g \circ f
    \end{align*}
    è un funtore controvariante perché \({(f' \circ f )}^{*} = f^{*} \circ f'^{*}\) 
\end{example}
\begin{remark}{}
    C'è anche un funtore 
    \begin{align*}
        \mathcal{C}{(\--, =)} : \mathcal{C}^{op} \times \mathcal{C} &\to \mathtt{Set} \\
        {(X, Y)} &\mapsto \mathcal{C}{(X, Y)} \\
        {(f : X \to X', g : Y \to Y')} &\mapsto {(f^* : \mathcal{C}{(X', Y)} \to  \mathcal{C}{(X,Y)}, g_* : \mathcal{C}{(X, Y)} \to \mathcal{C}{(X, Y')})}
    \end{align*}
\end{remark}
\begin{example}{}
    Per ogni gruppo \(G\), preso il sottogruppo dei commutatori \([G, G]\),
    allora per ogni \(f : G \to H\) omomorfismo di gruppi, \(f{([G, G])} \subseteq [H, H] \) 
    quindi si ottiene un funtore
    \begin{align*}{}
        \mathtt{Grp} &\to \mathtt{Grp} \\
        G &\mapsto [G, G] \\
        {(f : G \to H)} &\to {(f |_{[G, G]} : [G, G] \to [H, H] )}
    \end{align*}
    e anche 
    \begin{align*}
        \mathrm{Abel} : \mathtt{Grp} &\to \mathtt{Ab} \\
        G &\mapsto \frac{G}{[G, G]} \text{ (abelianizzato di \(G\) ) } \\
        {(f : G \to H)} &\mapsto {\left(\overline{f} : \frac{G}{[G,G]} \to \frac{H}{[H,H]}\right)}
    \end{align*}
\[\begin{tikzcd}
	{G} & {H} \\
    {\frac{G}{[G,G]}} & {\frac{H}{[H,H]}}
	\arrow["f", from=1-1, to=1-2]
	\arrow["{p}", from=1-1, to=2-1]
	\arrow["{q}", from=1-2, to=2-2]
	\arrow["\overline{f}"', from=2-1, to=2-2]
\end{tikzcd}\]
\end{example}

\begin{eser}{}
    Indicando con \(Z{(X)}\) il centro di \(X\),

\begin{enumerate}[label = \alph*.]
    \item Mostrare che non esiste un funtore \(F : \mathtt{Rng} \to \mathtt{Rng}\) tale
    che \(\forall A \in \mathtt{Rng}\) \(F{(A)} = Z{(A)}\).
    \item Mostrare che non esiste un funtore \(F : \mathtt{Grp} \to \mathtt{Ab}\) tale
    che \(\forall G \in \mathtt{Grp}\) \(F{(G)} = Z{(G)}\).
\end{enumerate}
    \tcblower
    Supponiamo l'esistenza di \(F\).
\begin{enumerate}[label = \alph*.]
    \item Se prendo \(i : \mathbb{\mathcal{C}} \hookrightarrow \mathbb{H}\),
        allora \(F{(\mathbb{\mathcal{C}})} = \mathbb{\mathcal{C}} \) e \(F{(\mathbb{H})} = \mathbb{R}\). A tal punto però \(
        F{(i)} : \mathbb{\mathcal{C}} \to \mathbb{R}\) che non esiste perché altrimenti \[-1 = F{(i)}{(-1)} = F{(i)}{(i^2)} = F{(i)}{(i)}^2\]
    \item Consideriamo
        \[
          \{(1), (12)\} \overset{i}{\hookrightarrow } S_{3} \overset{\varepsilon}{\to } \{\pm 1\} 
        \]
        Allora \(\varepsilon \circ i = \mathrm{Id}_{\mathcal{C}_{2}} \). Allora avremmo
        \[
          0_{\mathrm{End}{(\mathcal{C}_{2})}}  = F{(\varepsilon)} \circ F{(i)} = F{(\varepsilon \circ i)} = F{(\mathrm{id}_{\mathcal{C}_{2}})} = \mathrm{id}_{\mathcal{C}_{2}}
        \]
\end{enumerate}

\end{eser}

    L'identità
    \[
      \mathrm{id}_\mathcal{C} : \mathcal{C} \to \mathcal{C} \quad X \mapsto X \quad f \mapsto f
    \]
    è un funtore
    Si possono comporre i funtori. Dati ad esempio 
    \[
      \mathcal{C} \overset{F}{\to } D \overset{G}{\to } E
    \]
    funtori, possiamo definire \(G \circ F : \mathcal{C} \to E\) come \(X \mapsto G{(F{(X)})}\)  e \(f \mapsto G{(F{(f)})}\) è un funtore.

    La composizione è associativa  e \(F \circ \mathrm{id}_\mathcal{C} = F = \mathrm{id}_\mathcal{C} \circ F\) 

In tal modo otteniamo una categoria \(\mathtt{\mathcal{C}at}\) delle categorie (piccole\footnote{si potrebbe anche fare di tutte le categorie, ma per motivi
insiemistici/logici dovremmo introdurre gli universi di Grothendieck e fare le
cose per bene. Al fine di evitare questo inutile sforzo, ci limitiamo a
considerare le categorie piccole.})

\begin{definition}{}
    Un funtore \(F : \mathcal{C} \to D\) è un isomorfismo se lo è in \(\mathtt{\mathcal{C}at}\),
    cioè se \(\exists G : D \to \mathcal{C}\)  funtore tale che \(G \circ F = \mathrm{id}_\mathcal{C} = F \circ G\) 
\end{definition}

\begin{definition}{iniettivo e suriettivo}
    Un funtore \(F : \mathcal{C} \to D\) è \emph{iniettivo}/\emph{suriettivo} se \(F : \mathrm{Ob}{(\mathcal{C})} \to \mathrm{Ob}{(D)}\) è \emph{iniettivo}/\emph{suriettivo}.

    Nel caso in cui \(F\) sia sia iniettivo che suriettivo, è \textbf{biunivoco}.
\end{definition}

\begin{definition}{Fedele e pieno}
    \(F\) è detto \textbf{fedele} (\textbf{pieno}) se \(\forall X, Y \in \mathcal{C}\), \(F : \mathcal{C}{(X, Y)} \to D{(F{(X)}, F{(Y)})}\) è iniettivo (suriettivo).

    Nel caso in cui \(F\) sia sia fedele che pieno, si dice che è
    \textbf{pienamente fedele}
\end{definition}
\begin{eser}{}
    \(F\) funtore è isomorfismo se e solo se \(F\) è pienamente fedele e
    biunivoco.
\end{eser}


\begin{example}{}
    Se \(\mathcal{C}' \subseteq \mathcal{C} \) è una sottocategoria, allora il funtore di inclusione
    \(i : \mathcal{C}' \to \mathcal{C}\) è iniettivo e fedele ed è pieno se e solo se \(\mathcal{C}' \subseteq \mathcal{C}\) è piena.

    Ad esempio se \(\sim \) è una congruenza in \(\mathcal{C}\) , allora il funtore
    quoziente \(\mathcal{C} \to \mathcal{C} / \sim \) è biunivoco e pieno.
\end{example}

\begin{example}{}
    Un omomorfismo di monoidi (categorie di un oggetto) è iniettivo (suriettivo)
    se e solo se come funtore è fedele (pieno). In ogni caso è biunivoco.
\end{example}

\begin{example}{}
    I funtori dimenticanti \(\mathtt{\mathbb{Z}-Mod} \to \mathtt{Ab}\) e \(\mathtt{\mathbb{Z}-Alg} \to \mathtt{Rng}\) sono isomorfismi.
\end{example}

\begin{example}{}
    Anche \(\mathtt{Mod-A}\cong \mathtt{A^{op}-Mod}\) ed esiste un isomorfismo
    (anche se non sono categorie piccole).
\end{example}

\begin{definition}{}
    Un funtore \(F: \mathcal{C} \to D\) è \textbf{essenzialmente
    \emph{iniettivo}/\emph{suriettivo}} se la funzione ridotta 
    \[
        \mathrm{Ob}{(\mathcal{C})}/_{\cong} \to \mathrm{Ob}{(D)}/_{\cong}
    \]
    è \emph{iniettivo}/\emph{suriettivo}
\end{definition}

\begin{remark}{}
    Se \(F\) è suriettivo allora \(F\) è essenzialmente suriettivo. L'altra
    implicazione non vale. Ad esempio
\[\begin{tikzcd}
	{(\bullet)} & {(\bullet} & {\bullet)}
	\arrow[from=1-1, to=1-2]
	\arrow[shift left, from=1-2, to=1-3]
	\arrow[shift left, from=1-3, to=1-2]
\end{tikzcd}\]
    Nessuna delle implicazioni tra iniettiva e essenzialmente iniettiva è vera.
    Basti considerare
\[\begin{tikzcd}
	{(\bullet)} & {(\bullet} & {\bullet)}
	\arrow[from=1-2, to=1-1]
	\arrow[shift left, from=1-2, to=1-3]
	\arrow[shift left, from=1-3, to=1-2]
\end{tikzcd}\]
    per essenzialmente iniettiva \(\nRightarrow\) iniettiva
    e
\[\begin{tikzcd}
	{(\bullet} & {\bullet)} & {(\bullet} & {\bullet)}
	\arrow[from=1-2, to=1-3]
	\arrow[shift left, from=1-3, to=1-4]
	\arrow[shift left, from=1-4, to=1-3]
\end{tikzcd}\]
    per iniettiva \(\nRightarrow\) essenzialmente iniettiva.
\end{remark}

\begin{proposition}{}
    Sia \(F: \mathcal{C} \to D\) un funtore pienamente fedele. Allora \(F\) è
    essenzialmente iniettivo
\end{proposition}
\begin{proof}{}
    Siano \(X, Y \in \mathcal{C}\) tali che \(F{(X)} \cong F{(Y)}\) in \(D\). Devo
    dimostrare che \(X \cong Y\) in \(\mathcal{C}\).

    Sappiamo che esiste \(g : F{(X)} \to F{(Y)}\) isomorfismo in \(D\). Poiché
    \(F\) è pieno esiste \(f \in \mathcal{C}{(X, Y)}\) tale che \(F{(f)} = g\).
    Analogamente \(\exists f' \in \mathcal{C}{(Y, X)}\) tale che \(F{(f')} = g^{-1}\).
    \[
    F{(f' \circ f)} = F{(f')} \circ F{(f)} = g^{-1} \circ g = 1_{F{(X)}} = F{(1_X)}
    \]
    Se \(F\) è fedele, allora \(f' \circ f = 1_X\)  e analogamente \(f \circ f' = 1_Y\) da cui \(f\) è isomorfismo e duque \(X \cong Y\) 
\end{proof}

\begin{definition}{Trasformazione naturale}
    Siano \(F, F' : \mathcal{C} \to D\) funtori.

    Una \textbf{trasformazione naturale} \(\alpha : F \to F'\) (si può anche
    scrivere \(\alpha : F \implies F'\)) è il dato di un morfismo
    \[
      \alpha_X : F{(X)} \to F'{(X)} \text{ in \(D\) } \forall X \in \mathcal{C}
    \]
    tale che \(\forall f : X \to Y\) morfismo di \(\mathcal{C}\) il diagramma
\[\begin{tikzcd}
	{F(X)} & {F(Y)} \\
	{F'(X)} & {F'(Y)}
	\arrow["{F(f)}", from=1-1, to=1-2]
	\arrow["{\alpha_X}", from=1-1, to=2-1]
	\arrow["{\alpha_Y}", from=1-2, to=2-2]
	\arrow["{F'(f)}"', from=2-1, to=2-2]
\end{tikzcd}\]
    commuta in \(D\), cioè \(\alpha_Y \circ F{(f)} = F'{(f)} \circ \alpha_X\) 
\end{definition}

\begin{example}{}
    Consideriamo i due funtori \(\mathrm{Abel} : \mathtt{Grp} \to \mathtt{Grp}\) e \(\mathrm{id} : \mathtt{Grp} \to \mathtt{Grp}\). 
    C'è una trasformazione naturale \(\alpha : \mathrm{id} \to \mathrm{Abel}\)
    definita per ogni \(G \in \mathtt{Grp}\) da 
    \begin{align*}
        \alpha_G: G &\longrightarrow \frac{G}{[G, G]} \\
        a &\longmapsto \alpha_G(a) = a[G, G]
    \end{align*}
    è naturale perché \(\forall f : G \to H\) in \(\mathtt{Grp}\) il diagramma
\[\begin{tikzcd}
	{G} & {H} \\
    {\frac{G}{[G,G]}} & {\frac{H}{[H,H]}}
	\arrow["f", from=1-1, to=1-2]
	\arrow["{\alpha_G}", from=1-1, to=2-1]
	\arrow["{\alpha_H}", from=1-2, to=2-2]
	\arrow["\overline{f}"', from=2-1, to=2-2]
\end{tikzcd}\]
\end{example}

\begin{example}{}
    Supponendo di avere \(F, F' : G \to \mathtt{Set}\) funtori (\(G\) gruppo
    visto come categoria con un oggetto), cioè \(G\)-insiemi (azioni di \(G\) su
    insiemi).
    Allora una trasformazione naturale \(\alpha : F \to F'\) è un morfismo di \(G\)-insiemi cioè una
    funzione \(\alpha : F{(G)} \to F'{(G)}\) tale che \(\alpha{(gx)} = g \alpha{(x)}\) per ogni \(g \in G\) e per ogni \(x \in F{(G)}\).
\end{example}
\begin{remark}{}
    \(\forall F : \mathcal{C}\to D\), \(\mathrm{id}_F : F \to F\) data da \({(\mathrm{id}_F)}_X = \mathrm{id}_{F{(X)}} \) per ogni \(X \in \mathcal{C}\) è una trasformazione naturale.
\end{remark}

\begin{eser}{}
    Dati \(F, F', F'' : \mathcal{C} \to D\) funtori, \(\alpha : F \to F'\) e \(\beta : F'
    \to F''\) trasformazioni naturali, allora la composizione \(\beta \circ \alpha : F \to F''\) è definita da
    \[
      \beta_X \circ \alpha_X =: {(\beta \circ \alpha)}_X : F{(X)} \to F''{(X)}
    \]
    Mostrare che \(\alpha \circ \beta\) è una trasformazione naturale
\end{eser}

La composizione dell'esercizio precedente è anche detta \emph{composizione
verticale}
di trasformazioni naturali, per via di questo disegno esplicativo:
\[\begin{tikzcd}
	\mathcal{C} & \mathcal{D}
	\arrow[""{name=0, anchor=center, inner sep=0}, "F", curve={height=-20pt}, from=1-1, to=1-2]
	\arrow[""{name=1, anchor=center, inner sep=0}, "{F''}"', curve={height=20pt}, from=1-1, to=1-2]
	\arrow[""{name=2, anchor=center, inner sep=0}, "F"{description}, from=1-1, to=1-2]
	\arrow["\alpha", between={0.2}{0.8}, Rightarrow, from=0, to=2]
	\arrow["\beta", between={0.2}{0.8}, Rightarrow, from=2, to=1]
\end{tikzcd}\]

Considerando funtori e trasformazioni naturali, si ottiene (assumiamo sempre \(\mathcal{C}\) piccola) la categoria \(\mathtt{Fun}{(\mathcal{C}, \mathcal{D})}\) (anche denotata \(\mathcal{D}^{\mathcal{C}}\) ) con oggetti i funtori \(\mathcal{C} \to \mathcal{D}\), morfismi le trasformazioni naturali e composizione la composizione verticale.

\begin{definition}{}
    Data una categoria \(\mathcal{C}\), la categoria dei morfismi di \(\mathcal{C}\) è
    \[
      \mathtt{Mor}{(\mathcal{C})} := \mathtt{Fun}{(\cdot \to \cdot , \mathcal{C})}
    \]
    che ha come oggetti esattamente \(\{f : X \to Y \text{ morfismo di }\mathcal{C}\} \) e trasformazioni naturali date da \({(X \overset{f}{\to } Y )} \to  {(X' \overset{f'}{\to } Y' )}\) è data da \({(g : X \to  X' , h : Y \to  Y')}\) tale che 
\[\begin{tikzcd}
	{X} & {Y} \\
    {X'} & {Y'}
	\arrow["f", from=1-1, to=1-2]
	\arrow["g", from=1-1, to=2-1]
	\arrow["h", from=1-2, to=2-2]
	\arrow["f'", from=2-1, to=2-2]
\end{tikzcd}\]
\end{definition}

\begin{definition}{}
    Date \(F, G : \mathcal{C} \to D\) funtori, \(\alpha : F \to G\) trasformazione
    naturale, allora \(\alpha\) è \emph{isomorfismo} (\emph{naturale} o di funtori) se è isomorfismo in \(\mathtt{Fun}{(\mathcal{C}, D)}\) cioè se 
    \(\exists \beta : G \to F\) trasformazione naturale tale che \(\beta \circ
    \alpha = \mathrm{id}_F\), \(\alpha \circ \beta = \mathrm{id}_G\).

    In tal caso \(F\) e \(G\) si dicono \emph{isomorfi} (denotato \(F \cong
    G\)).
\end{definition}
\begin{remark}{}
    \(\cong\) di funtori è una relazione di equivalenza 
\end{remark}
\begin{example}{}

Il primo gruppo di omologia si può vedere come l'abelianizzato del gruppo
fondamentale. In linguaggio categorico abbiamo 
\begin{align*}
  \mathtt{Top_*} \overset{\pi_{1}}{\to } &\mathtt{Grp} \overset{\mathrm{Abel}}{\to } \mathtt{Ab} \\ 
  &\text{ e } \\
  \mathrm{Top_*} \to &\mathrm{Top} \overset{H_{1}}{\to } \mathtt{Ab} \\
  {(X, x_{0})} \mapsto &\text{ comp. c.p.a. che contiene \(x_{0}\)}
\end{align*}
sono funtori isomorfi
\end{example}

\begin{remark}{}
    \(F \cong F'\) allora \(F\) e \(F'\) inducono la stessa funzione \(\mathrm{Ob}{(\mathcal{C})} /_{\cong} \to \mathrm{Ob}{(D)}/_{\cong}\) quindi \(F\) è essenzialmente 
   \emph{iniettiva} / \emph{suriettiva} se e solo se \(F'\) lo è.
\end{remark}

\begin{eser}{}
    Mostrare che non necessariamente la precedente osservazione vale per le
    proprietà di iniettività e suriettività.
\end{eser}

\begin{proposition}{}
    Se \(F \cong F'\) allora \(F\) è \emph{fedele}/\emph{pieno} se e solo se \(F'\) è \emph{fedele}/\emph{pieno}.
\end{proposition}
\begin{proof}{}
    Sia \(\alpha : F \to F'\) l'isomorfismo. Sia allora \(\overline{\alpha} :
    \mathcal{D}{(F{(X)}, F{(Y)})} \to \mathcal{D}{(F'{(X)}, F'{(Y)})}\) definita
    da \(\overline{\alpha}{(g)} := \alpha_Y \circ g \circ \alpha_X^{-1}\).
    Per ogni \(X, Y \in \mathcal{C}\), il diagramma
\[\begin{tikzcd}
	& {\mathcal{D}(F(X),F(Y))} \\
	{\mathcal{C}(X,Y)} \\
	& {\mathcal{D}(F'(X),F'(Y))} \\
	\arrow["{\overline{\alpha}}", from=1-2, to=3-2]
	\arrow["F", from=2-1, to=1-2]
	\arrow["{F'}", from=2-1, to=3-2]
\end{tikzcd}\]
commuta. Infatti preso \(f \in \mathcal{C}{(X, Y)}\), per la trasformazione
naturale \(\alpha_Y \circ F{(f)} = F'{(f)} \circ \alpha_X\) si ha che 
\[
  {(\overline{\alpha} \circ F)}{(f)} = \alpha_Y \circ F{(f)} \circ \alpha_X^{-1}
  = F'{(f)}
\]
Inoltre \(\overline{\alpha}\) è una biezione, infatti ha inversa \(\overline{\alpha}^{-1}{(h)} = \alpha_Y^{-1} \circ h \circ \alpha_X\): \[\overline{\alpha} {( \overline{\alpha}^{-1} \circ \overline{\alpha})}{(g)} = \overline{\alpha}^{-1}
{(\alpha_Y \circ g \circ \alpha_X^{-1})} = \alpha_Y^{-1} \circ \alpha_Y \circ g
\circ \alpha_X^{-1} \circ \alpha_X = g
\]
e similmente l'altra composizione. Allora chiaramente se \(F\) è \emph{fedele}/\emph{pieno}, allora \(F'  = \overline{\alpha} \circ F \) è \emph{iniettivo}/\emph{suriettivo}.

\end{proof}

\begin{proposition}{}
\(\alpha, \beta\) trasformazioni naturali inducono una trasformazione naturale
\(\beta * \alpha : G \circ F \to G' \circ F'\) 
\[\begin{tikzcd}
	{G{(F{(X)})}} & {G{(F'{(X)})}} \\
    {G'{(F{(X)})}} & {G'{(F'{(X)})}}
	\arrow["G{(\alpha_X)}", from=1-1, to=1-2]
	\arrow["\beta_{F{(X)}}  ", from=1-1, to=2-1]
	\arrow["\beta_{F'{(X)}} ", from=1-2, to=2-2]
	\arrow["G'{(\alpha_X)}", from=2-1, to=2-2]
\end{tikzcd}\]
dunque \((\beta * \alpha)_X := \beta_{F'{(X)}} \circ G{(\alpha_X)} = G'{(\alpha_x)} \circ \beta_{F{(X)}} \).
\end{proposition}
\begin{proof}[Dimostrazione che è una trasformazione naturale]
Vogliamo mostrare che \(b * a\) è naturale, cioè \(\forall f : X \to Y\) in \(\mathcal{C}\) il diagramma
\[\begin{tikzcd}
	{G{(F{(X)})}} & {G{(F{(X)})}} \\
    {G'{(F'{(X)})}} & {G'{(F'{(X)})}}
	\arrow["G{(F{(f)})}", from=1-1, to=1-2]
	\arrow["{(\beta * \alpha)}_X", from=1-1, to=2-1]
	\arrow["{(\beta * \alpha)}_Y", from=1-2, to=2-2]
	\arrow["G'{(F'{(f)})}", from=2-1, to=2-2]
\end{tikzcd}\]
commuta. Ma questo è vero perché
\begin{align*}
    G'{(F'{(f)})} \circ {(\beta * \alpha)}_X &= G'{(F'{(f)})} \circ G'{(\alpha_x)}
  \circ \beta_{F{(X)}} = G'{(F'{(f)} \circ \alpha_X)} \circ \beta_{F{(X)}} = \\
    &\overset{\alpha \text{ nat.}}{=} G'{(\alpha_Y \circ F{(f)})} \circ
    \beta_{F{(X)}} = G'{(\alpha_Y)} \circ G'{(F{(f)} \circ \beta_{F{(X)}})} = \\
    &\overset{\beta \text{ nat.}}{=} G'{(\alpha_Y)} \circ \beta_{F{(Y)}}
    \circ G{(F{(f)})} = {(\beta * \alpha)}_Y \circ G{(F{(f)})}
\end{align*}
\end{proof}

Ovviamente è chiaro che si potrebbe definire allora la categoria delle
trasformazioni naturali eccetera e andare avanti all'infinito. Per
assiomatizzare queste cose in realtà bisognerebbe esplicitare che abbiamo
definito le ``2-frecce'' e che quindi siamo in una \emph{2-categoria}

\begin{note}[zione]
    Se \(\beta = \mathrm{id}_G \) invece di \(\mathrm{id_G} * \alpha\) si scrive
    \(G \circ \alpha\) (dunque con \({(G \circ \alpha)}_X = G{(\alpha_X)}\)). Se
    \(\alpha = \mathrm{id}_F\) invece di \(\beta * \mathrm{id}_F\) si scrive \(\beta \circ F\) (con \({(\beta \circ F)}_X = \beta_{F{(X)}} \)). In generale 
    \[
      \beta * \alpha = {(\beta \circ F')} \circ {(G \circ \alpha)} = {(G' \circ
      \alpha)} \circ {(\beta \circ F)}
    \]
\end{note}
\begin{remark}{}
    Se \(\alpha, \beta\) sono isomorfismi, allora \(\beta * \alpha\) è
    isomorfismo. Questo significa che se 
    \[
      F \cong F', \quad G \cong G' \implies G \circ F \cong G' \circ F'
    \]
    cioè l'isomorfismo di funtori è una congruenza su \(
    \mathtt{\mathcal{C}at}\) e quindi si ottiene la categoria \(\mathtt{\mathcal{C}at} /_{\cong}\)
\end{remark}
\begin{definition}{Equivalenza}
    Un funtore \(F: \mathcal{C} \to D\) è un'equivalenza se \(\exists G : D \to \mathcal{C}\)
    funtore tale che \(G \circ F \cong \mathrm{id}_G\) e \(F \circ G \cong \mathrm{id}_D\).

    Un tale \(G\) si dice un \emph{quasi-inverso} di \(F\).
\end{definition}
\begin{remark}{}
    \(F\) è un'equivalenza se e solo se \(\overline{F} \text{ in } \mathtt{\mathcal{C}at}/_{\cong} \) è un isomorfismo.
\end{remark}

Segue che se \(F \cong F'\), allora \(F\) è un'equivalenza se e solo se \(F'\) è
un'equivalenza e un quasi-inverso di \(F\) è unico a meno di isomorfismo e
l'equivalente di categorie è una relazione di equivalenza su \(\mathtt{\mathcal{C}at}\) 

\begin{definition}{Scheletro}
    Una sottocategoria piena \(\mathcal{C}' \subseteq \mathcal{C} \) è detta \emph{scheletro} se \(\forall X \in \mathcal{C}\), \(\exists ! X' \in \mathcal{C}'\) tale che \(X \cong X'\) 
\end{definition}

\begin{lemma}{}\label{lem:ext-scheletro}
    Sia \(F: \mathcal{C}\to D\) un funtore, e si supponga che \(\forall X \in \mathcal{C}\), \(\alpha_X : F{(X)} \to F'{(X)}\) sia un isomorfismo in \(D\). Allora \(F' : \mathrm{Ob}{(\mathcal{C})} \to \mathrm{Ob}{(D)}\) Si estende in modo unico a un funtore \(F' : \mathcal{C} \to D\) tale che \(\alpha : F \to F'\) è isomorfismo.
\end{lemma}

\begin{theorem}{Finalmente un teorema}
    Un funtore \(F : \mathcal{C} \to D\) è un'\emph{equivalenza} se e solo se \(F\) è \emph{pienamente fedele} e \emph{essenzialmente suriettivo}
\end{theorem}
\begin{remark}{}
    Non è necessario aggiungere l'ipotesi che \(F\) sia essenzialmente iniettivo
    perché come mostrato prima pienamente fedele implica essenzialmente
    iniettivo (ma non essenzialmente suriettivo).
\end{remark}
\begin{example}{}
    Supponiamo che \(\mathcal{C}' \subseteq \mathcal{C} \) sia una sottocategoria piena. Allora il
    funtore di inclusione \(\mathcal{C}' \hookrightarrow \mathcal{C}\) è pienamente fedele ed è
    essenzialmente suriettivo (quindi è un'equivalenza) se e solo se \(\forall X \in \mathcal{C}\) esiste \(X' \in \mathcal{C}'\) tale che \(X \cong X'\).
\end{example}

\begin{proof}\( \)
\begin{itemize}
    \item[\(\implies \)] Sia \(G : D \to \mathcal{C}\) un quasi-inverso di \(F\). Allora
        \(F \circ G \cong \mathrm{id}_G\) che è essenzialmente suriettivo, e
        dunque \(F\) è essenzialmente suriettivo. D'altra parte lo stesso \(F
        \circ G\) è fedele, e dunque \(G\) è fedele.
        
        Ora, per ogni \(X, Y \in \mathcal{C}\),
        \[
          \mathcal{C}{(X, Y)} \overset{F_{X,Y} }{\to } D{(F{(X)}, F{(Y)})} \overset{G_{F{(X)}, F{(Y)}}  \text{ inj}}{\to} \mathcal{C}{(G{(F{(X)})}, G{(F{(Y)})})}
        \]
        poiché la composizione è biunivoca e \(G\) è fedele, allora entrambi
        devono essere biunivoci, ossia in particolare \(F\) è pienamente fedele.
    \item[\(\impliedby \)] Consideriamo prima il caso di un'inclusione \(\mathcal{C}' \subseteq \mathcal{C} \) sottocategoria piena tale che \(\forall X \in \mathcal{C}\) esista \(X' \in \mathcal{C}\) tale che \(X \cong X'\). 
        Sia \(I : \mathcal{C}' \to \mathcal{C}\) il funtore di inclusione (pienamente fedele e
        essenzialmente suriettivo).

        Allora \(\forall X \in \mathcal{C}\) scelto (AoC) un
        isomorfismo \(\alpha_X : X \to \tilde{P}{(X)} \in \mathcal{C}'\) e se \(X \in 
        \mathcal{C}'\) in particolare prendiamo \(\alpha_X = 1_X\). Applico ora il
        lemma~\ref{lem:ext-scheletro} con \(F = \mathrm{id}_\mathcal{C}\) e dunque \(\exists !\) estensione di \(\tilde{P}\) a un funtore \(\tilde{P} : \mathcal{C} \to \mathcal{C}\) tale che \(\alpha : \mathrm{id}_\mathcal{C} \to \tilde{P}\) è isomorfismo. Allora \(\exists ! P : \mathcal{C} \to \mathcal{C}'\) funtore tale che \(\tilde{P} = I \circ P\) e \(P\) è un quasi-inverso di \(I\) poiché \(I \circ P = \tilde{P} \cong \mathrm{id}_\mathcal{C}\) e \(P \circ I = \mathrm{id}_{\mathcal{C}'}\).

        In generale, dato \(F : \mathcal{C} \to D\) pienamente fedele. Siano allora \(
        I : \mathcal{C}' \to \mathcal{C}\) e \(J : D' \to D\) due scheletri. Per il caso qui fatto
        \(I, J\) sono equivalenze e siano \(P : \mathcal{C}\to \mathcal{C}'\) quasi-inverso di \(I\) e \(Q : D \to D'\) quasi-inverso di \(J\).
\[\begin{tikzcd}
	\mathcal{C} & D \\
	{\mathcal{C}'} & {D'}
	\arrow["F"{description}, from=1-1, to=1-2]
	\arrow["P"', shift right, from=1-1, to=2-1]
	\arrow["Q", shift left, from=1-2, to=2-2]
	\arrow["I"', shift right, from=2-1, to=1-1]
	\arrow["{F'}"{description}, from=2-1, to=2-2]
	\arrow["J", shift left, from=2-2, to=1-2]
\end{tikzcd}\]
    Sia \(F' := Q \circ F \circ I : \mathcal{C}' \to D'\) come nel diagramma. Allora \(
    I, F, Q\) sono pienamente fedeli e essenzialmente suriettivi (\(I\) per
    definizione, \(F\) per ipotesi e \(Q\) perché è un'equivalenza e vale il
    punto \({(\implies )}\)) dunque \(
    F'\) è pienamente fedele e essenzialmente suriettivo.

    \(F'\) è essenzialmente biunivoco, \(\mathcal{C}'\) e \(D'\) sono scheletri, dunque \(F'\) è biunivoco, quindi isomorfismo e quindi equivalenza.
    \[
        F = \mathrm{id}_D \circ F \circ \mathrm{id}_c \cong J \circ Q \circ F
        \circ I \circ P = J \circ F' \circ P
    \]
    equivalenza perché lo sono \(J, F'\) e \(P\) 
\end{itemize}
\end{proof}

\begin{example}{}
    Sia \(\sim \) una relazione d'equivalenza su un insieme \(X\) che vedo come
    categoria \(\mathcal{C}\) con \(\mathrm{Ob}{(\mathcal{C})} = X\)  e \(\mathcal{C}{(x,y)} \neq \varnothing \iff x \sim y\).

    Il funtore \(\mathcal{C} \to X/_\sim\) (categoria discreta) definito da \(x \mapsto
    \overline{x}\) è un'equivalenza poiché pienamente fedele e essenzialmente
    suriettiva.
\end{example}

\begin{eser}{}
    Mostrare che ogni categoria equivalente a una categoria discreta è una relazione di
    equivalenza, ossia una categoria dove \(\forall X, Y \in \mathcal{C}, \,\,
    \mathcal{C}{(X,Y)} \neq O \iff x \sim y \) per una qualche \(\sim \) relazione di
    equivalenza.
\end{eser}

\section{Categorie preadditive}
\begin{definition}{Categoria preadditiva}
    Una categoria \emph{preadditiva} è una categoria \(\mathcal{A}\) con una
    struttura di gruppo abeliano (notazione: additivo) su \(\mathcal{A}{(X, Y)}\) per ogni \(X, Y \in \mathcal{A}\) ed è tale che la composizione di morfismi sia \(\mathbb{Z}\)-bilineare, ossia
    \[
      g \circ {(f + f')} = g \circ f + g \circ f' \quad \text{ e } \quad {(g +
      g')} \circ f = g \circ f + g' \circ f 
    \]
    per ogni \(
    X, Y, Z \in \mathcal{A}\), \(f, f' \in \mathcal{A}{(X, Y)}\) e \(g, g' \in \mathcal{A}{(Y, Z)}\).
\end{definition}
\begin{remark}{}
    Si dice anche che \(\mathcal{A}\) è una \(\mathtt{Ab}\)-Categoria. Si può
    studiare quando si può generare una categoria simile partendo da altre
    categorie invece di \(\mathtt{Ab}\) ma non è argomento di questo corso.

    Si può anche dire che \(\mathcal{A}\) è \(\mathbb{Z}\)-lineare. Più in
    generale \(\forall\)\footnote{normalmente in mezzo alla frase così avrei scritto esplicitamente ``per ogni'' ma trovavo divertente la quantità di \(\mathcal{A}\) e di \(A\) nella frase quindi ho valutato simpatico aggiungere anche un \(\forall\)} anello commutativo \(A\) una categoria \(A\)-lineare è
    una categoria \(\mathcal{A}\) con una struttura di \(A\)-modulo su \(\mathcal{A}{(X,Y)}\) tale che la composizione sia \(A\)-bilineare.
\end{remark}
\begin{proposition}{}
    Se \(A\) è non commutativo, allora \(\forall a, b \in A\) e \(\forall f :
    X\to Y\) morfismo di \(\mathcal{A}\),
    \[
      {(ab)}f = {(ba)}f
    \]
\end{proposition}
\begin{proof}{}
    \[
      {(ab)}f = a{(bf)} = a{({(bf)}\circ 1_X)} = {(bf)} \circ {(a1_x)} = f \circ
      {(b{(a 1_X)})} = f \circ {(ba)}1_X = {(ba)}f
    \]
\end{proof}

\begin{example}{}
    Sia \(A\) un anello, allora \(A\mathtt{-Mod}\) è preadditiva. Infatti per
    ogni \(M, N \in {(A\mathtt{-Mod})}\), \(A\mathtt{-Mod}{(M, N)} = \mathrm{Hom}_A{(M, N)}\) è un gruppo abeliano e \(\circ\) è \(\mathbb{Z}\)-bilineare. Se \(A\) è commutativo, allora \(A\mathtt{-Mod}\) è anche \(A\)-lineare.
    Più in generale se \(B\) è una \(A\)-algebra allora \(B\mathtt{-Mod}\) è \(A\)-lineare.
\end{example}

\begin{remark}{}
    Sia \(X \in \mathcal{A}\) categoria \(A\)-lineare (quindi \(A\)
    commutativo). Allora \(\mathrm{End}_\mathcal{A}{(X)}\) è una \(A\)-algebra.
    Infatti \({(\mathrm{End}_{\mathcal{A}} {(X)}, \circ)}\) è un monoide e \(\mathrm{End}_\mathcal{A}\) è \(A\)-modulo e \(\circ\) è \(A\)-lineare.

    Quindi le categorie \(A\)-lineari con un solo oggetto sono \(A\)-algebre. In
    particolare le categorie preadditive con un solo oggetto sono anelli.
\end{remark}

\begin{remark}{}
Sia \(\mathcal{A}\) preadditiva, allora \(\mathcal{A}^{op}\) è preadditiva con
la stessa struttura di gruppo abeliano su \(\mathcal{A}^{op}{(X, Y)} = \mathcal{A}{(Y, X)}\) per ogni \(X, Y \in \mathcal{A}\).
\end{remark}

\begin{remark}{}
Se \(\mathcal{A}\) è preadditiva, allora \(\mathcal{A}' \subseteq \mathcal{A}\) sottocategoria tale che \(\mathcal{A}'{(X, Y)} < \mathcal{A}{(X,Y)}\) per ogni \(X, Y \in \mathcal{A}'\), allora \(\mathcal{A}'\) è preadditiva. In particolare la condizione è sempre verificata per le categorie piene.
\end{remark}

    Sia \(\mathcal{A}\) preadditiva, \(\sim \) una congruenza tale che \(\forall X, Y \in \mathcal{A}\), \(\forall f, f', g \in \mathcal{A}{(X, Y)}\) allora \(f \sim f' \implies f + g \sim f' + g\). In tale ipotesi \(\mathcal{A}/_\sim \) è preadditiva con \(\overline{f} + \overline{g} = \overline{f+g}\).

    Data una tale congruenza, sia \(\forall X, Y \in \mathcal{A}\) 
    \[
      \mathfrak{I}{(X,Y)} = \{f \in \mathcal{A}{(X, Y)} : f \sim 0\} 
    \]
    e indichiamo con \(\mathfrak{I} \subseteq \mathcal{A}\) la collezione di
    tutti gli \(\mathfrak{I}{(X, Y)}\). Allora vale la proprietà di ideale, cioè
    dati \(f, g\) morfismi di \(\mathcal{A}\) componibili, 
    \[
      f \in  \mathfrak{I} \text{ o } g \in \mathfrak{I} \implies g \circ f \in \mathfrak{I}
    \]
    Se per esempio \(f \in \mathfrak{I}\) ossia \(f \sim 0\), allora \(g \circ f
    \sim g \circ 0 = 0 \) e dunque \(g \circ f \in \mathfrak{I}\).

    Arriviamo dunque alla seguente definizione

\begin{definition}{}
    Definiamo un ideale \(\mathfrak{I}\) in una categoria preadditiva \(\mathcal{A}\) come \(\mathfrak{I}{(X,Y)} < \mathcal{A}{(X,Y)}\) per ogni \(X, Y \in \mathcal{A}\) tale che
    \[
      f \in \mathfrak{I} \text{ o } g \in \mathfrak{I} \implies g \circ f \in
      \mathfrak{I}
    \]
\end{definition}
Viceversa, dato \(\mathfrak{I} \subseteq \mathcal{A} \) ideale, si ottiene una
congruenza \(\sim \) su \(\mathcal{A}\) definito da 
\[
  f \sim f' \iff f' - f \in \mathfrak{I}{(X, Y)} \quad \forall X, Y \in \mathcal{A} \quad \forall f, f' \in  \mathcal{A}{(X, Y)}
\]
ed è tale che \(f \sim f' \implies f + g \sim f'+ g\).

In tali ipotesi si può anche denotare \(\mathcal{A} / \mathfrak{I}\) invece di
\(\mathcal{A}/_\sim\).

Una categoria \(\mathcal{\mathcal{C}}\) può non avere nessuna struttura di categoria
    preadditiva (ad esempio se \(\exists X, Y \in \mathcal{\mathcal{C}}\)) tale che \(\mathcal{\mathcal{C}}{(X, Y)} = \varnothing\) o averne più di una.

\begin{example}{G}
    Possiamo pensare ad anelli \(A\) e \(B\) tali che \({(A, \cdot )} \cong {(B, \cdot )}\) e \({(A, +)} \not\cong {(B, +)}\).

    Ad esempio possiamo prendere \(A = \mathbb{Z} / 4 \mathbb{Z}\) e \(B = \mathbb{Z} /_{2 \mathbb{Z}} [X] / {(X^2)}\). Allora evidentemente 
    \[
      {(A, +)} \cong \mathcal{C}_4 \not\cong \mathcal{C}_{2} \times \mathcal{C}_{2} \cong B
    \]
    ma gli elementi diversi da \(0\) e \(1\) di \(A\) sono \(\overline{2}\) e \(\overline{3}\) e sono tali che \(\overline{2}^2 = \overline{0}\), \(\overline{3}^2 = \overline{1}\) e \(\overline{2} \cdot \overline{3} = \overline{2}\). Similmente in \(B\) abbiamo che
    \(
      \overline{X}^2 = \overline{0}
    \), \(\overline{1+X}^2 = \overline{1}\) e \(\overline{X} \cdot \overline{1+X} = \overline{X}\)
\end{example}

\begin{definition}{}
    Un funtore \(F : \mathcal{A} \to \mathcal{B}\) tra categorie preadditive è
    additivo se 
    \[
      F_{X,Y}  : \mathcal{A}{(X, Y)} \to \mathcal{B}{(F{(X)}, F{(Y)})}
    \]
    è omomorfismo di gruppi \(\forall X, Y \in \mathcal{A}\).

    Più in generale \(F: \mathcal{A} \to \mathcal{B}\) tra categorie
    \(A\)-lineari è detto \(A\)-lineare se \(F_{X, Y} \) è \(A\)-lineare \(\forall X, Y \in \mathcal{A}\).
\end{definition}

\begin{example}{}
    Sia \(\mathcal{A}' \subseteq \mathcal{A} \) sottocategoria tale che \(\mathcal{A}'{(X, Y)} < \mathcal{A}{(X,Y)}\) per ogni \(X,Y \in \mathcal{A}'\). Allora l'inclusione \(\mathcal{A}' \to \mathcal{A}\) è addditivo.
\end{example}

\begin{example}{}
    Se \(\mathcal{A}\) è preadditiva e \(\mathfrak{I} \subseteq \mathcal{A} \)
    ideale, allora il funtore \(f : \mathcal{A} \to \mathcal{A}/\mathfrak{I}\) 
    definito da \(X \mapsto X\) e \(f \mapsto \overline{f}\) è additivo.
\end{example}

\begin{eser}{}
    Sia \(F : \mathcal{A} \to \mathcal{B}\) additivo tale che ``\(\mathfrak{I} = \ker F\)'' cioè \(F{(f)} = 0 \), \(\forall f \in \mathfrak{I}\), allora mostrare che esiste un unico \(\overline{F} : \mathcal{A}/\mathfrak{I} \to \mathcal{B}\) funtore additivo tale che \(F = \overline{F} \circ P\) 
\end{eser}

\begin{example}{}
    Siano \(A, B\) anelli (categorie preadditive con un solo oggetto), allora un
    funtore additivo \(A\to B\) è un omomorfismo di anelli.

    Più in generale per ogni anello \(A\) e per ogni \(\mathcal{A}\) categoria
    preadditiva un funtore additivo \( A \to \mathcal{A}\) è dato da un oggetto
    \(X \in \mathcal{A}\) e un omomorfismo di anelli \(A \to \mathrm{End}_\mathcal{A}{(X)}\).
    Quindi un \(A\)-modulo è un funtore additivo \(A \to \mathtt{Ab}\) 
\end{example}

\begin{example}{}
    Sia \(A\to B\) un omomorfismo di anelli. Allora il funtore di restrizione
    degli scalari \(B\mathtt{-Mod} \to A\mathtt{-Mod}\) è additivo.
\end{example}

\begin{example}{}
    Se \(\mathcal{A}\) preadditiva, allora \(\forall X \in \mathcal{A}\) ci sono
    funtori additivi
    \[
      \mathcal{A}{(X, \--)} : \mathcal{A} \to \mathtt{Ab} \quad , \quad \mathcal{A}{(\-- , X)} : \mathcal{A}^{op} \to \mathtt{Ab}
    \]
    e in generale se \(
    \mathcal{A}\) è \(A\)-lineare, allora i due funtori hanno codominio \(A\mathtt{-Mod}\) 
    e sono \(A\)-lineari
\end{example}

Notare che se \(\mathcal{A} \overset{F}{\to } \mathcal{B} \overset{G}{\to } \mathcal{\mathcal{C}}\) 
sono funtori additivi, allora \(G \circ F\) è additivo.
Inoltre \(\mathrm{id}_\mathcal{A}\) è additivo. 
Dunque si ottiene una categoria contenente le \textbf{categorie preadditive}
(piccole) e morfismi i funtori additivi.

    Sia \(\mathcal{\mathcal{C}}\) una categoria (piccola) e \(\mathcal{A}\) una categoria
    preadditiva. Allora \(\mathrm{Fun}{(\mathcal{\mathcal{C}}, \mathcal{A})}\) è
    preadditiva (in modo naturale) con la seguente struttura

    \(\forall F, G \in \mathtt{Fun}{(\mathcal{\mathcal{C}}, \mathcal{A})}\) e \(\forall \alpha, \beta : F \to G\) trasformazioni
    naturali allora \(\alpha + \beta : F \to G\) trasformazione naturale
    definita \(\forall X \in \mathcal{\mathcal{C}}\) da
    \[
      {(\alpha + \beta)}_X := \alpha_X + \beta_X : F{(X)} \to G{(X)} \text{ in }
      \mathcal{A}
    \]
    è naturale perché \(
    \forall f : X \to \mathcal{C}\), 
\[\begin{tikzcd}
	{F(X)} & {F(Y)} \\
	{G{(X)}} & {G{(Y)}}
	\arrow["{F(f)}", from=1-1, to=1-2]
	\arrow["{\alpha_X + \beta_X}", from=1-1, to=2-1]
	\arrow["{\alpha_Y + \beta_Y}", from=1-2, to=2-2]
	\arrow["{G{(f)}}"', from=2-1, to=2-2]
\end{tikzcd}\]
commuta

Se anche \(\mathcal{\mathcal{C}}\) è preadditiva sia 
\[
  \mathtt{AddFun}{(\mathcal{\mathcal{C}}, \mathcal{A})}
\]
la sottocategoria piena di \(\mathtt{Fun}{(\mathcal{\mathcal{C}}, \mathcal{A})}\) con 
oggetti i funtori additivi. Allora tale categoria è preadditiva.

\begin{example}{}
    Se \(A\) è anello, allora gli \(A\)-moduli sono gli oggetti di \(\mathtt{AddFun}{(A, \mathtt{Ab})}\) (già visto)
    e in effetti \(A\mathtt{-Mod} \cong \mathtt{AddFun}{(A, \mathtt{Ab})}\),
    poiché dati \(M, N : A \to \mathtt{Ab}\) funtori (cioè \(A \to \mathrm{End}{(M)}\) e \(A \to \mathrm{End}{(N)}\) omomorfismi di anelli) la trasformazione naturale \(\alpha : M \to N\) è data da
    \(\alpha : M \to N\) in \(\mathtt{Ab}\) tale che \(\forall a \in A\) 
\[\begin{tikzcd}
	{M} & {N} \\
	{N} & {N}
	\arrow["{a}", from=1-1, to=1-2]
	\arrow["{\alpha}", from=1-1, to=2-1]
	\arrow["{\alpha}", from=1-2, to=2-2]
	\arrow["{a}"', from=2-1, to=2-2]
\end{tikzcd}\]
commuta, cioè \(a \alpha {(x)} = \alpha {(a x)}\) per ogni \(x \in M\), ossia \(
\alpha \) è omomorfismo di \(A\)-moduli.
\end{example}
\begin{remark}{}
    \(\forall \mathcal{A}\) preadditiva (piccola) si può definire la categoria
    (preadditiva) \(\mathcal{A-}\mathtt{Mod} := \mathtt{AddFun}{(\mathcal{A}, \mathtt{Ab})}\).
\end{remark}
\begin{proposition}{}
    Siano \(\mathcal{A}, \mathcal{B}\) preadditive, \(F, G : \mathcal{A} \to
    \mathcal{B}\) funtori tali che \(F \cong G\) e \(F\) additivo. Allora \(G\)
    è additivo
\end{proposition}
\begin{proof}{}
    Sia \(\alpha : F \to G\) isomorfismo. Allora \(\forall  f : X \to Y\) in \(\mathcal{A}\), 
    \(G{(f)} = \alpha_Y \circ F{(f)} \circ \alpha_X^{-1}\).
    Inoltre \(\forall f, f' : X \to Y\),
    \begin{align*}G{(f + f')} &= \alpha_Y \circ F{(f +
    f')} \circ \alpha_X^{-1} = \alpha_Y \circ {(F{(f)} + F{(f')})} \circ \alpha_X^{-1} = \\
    &= \alpha_Y \circ F{(f)} \circ \alpha_X^{-1} + \alpha_Y \circ F{(f')} \circ
    \alpha_X^{-1} = G{(f)} + G{(f')}
\end{align*}
\end{proof}

\begin{remark}{}
    Sia \(F : \mathcal{A} \to \mathcal{B}\) un funtore pienamente fedele con \(\mathcal{B}\) preadditiva.
    Allora esiste un'unica struttura preadditiva su \(\mathcal{A}\) tale che \(F\) 
    sia additivo
\end{remark}
\begin{proof}{}
    \(\forall X, Y \in \mathcal{A}\) voglio che \(F_{X,Y}  : \mathcal{A}{(X, Y)} \to \mathcal{B}{(F{(X)}, F{(Y)})}\) sia isomorfismo di gruppi (già è biettiva), che è vero se e solo se \(\forall f, f' \in \mathcal{A}{(X,Y)}\) 
    \[
      f + f' = F^{-1}{(F{(f)} + F{(f')})} 
    \]
    Infatti è ovvio che se \(+\) è così definita, allora \(F\) sia omomorfismo
    di gruppi. Viceversa, se \(F\) è omomorfismo di gruppi, allora anche \(F^{-1}\) è omomorfismo di gruppi, 
    e dunque \(F^{-1}{(F{(f)} + F{(f')})} = F^{-1}{(F{(f)})} + F^{-1}{(F{(f')})} = f + f'\).

\end{proof}

\subsection{Prodotti, coprodotti, proprietà universali}

\begin{definition}{Prodotto}
Sia \(\mathcal{\mathcal{C}}\) una categoria, siano \(
X_\lambda \in \mathcal{\mathcal{C}}\) con \(\lambda \in \Lambda\) insieme. Un prodotto
degli \(
X_\lambda\) è dato da \(X \in \mathcal{\mathcal{C}}\) con morfismi morfismi \(p_\lambda
\in \mathcal{C}{(X, X_\lambda)}\) per ogni \(\lambda \in \Lambda\) tale che
vale la seguente proprietà universale:
\[
  \forall Y \in \mathcal{\mathcal{C}}, \,\, \forall \lambda \in \Lambda, \,\, \forall f_{\lambda} \in \mathcal{\mathcal{C}}{(Y, X_\lambda)} \quad \exists ! f \in \mathcal{\mathcal{C}}{(X, Y)} : f_\lambda = p_\lambda \circ f
\]
o equivalentemente i seguenti diagramma commutano :
\[\begin{tikzcd}
	Y \\
	{X_\lambda} & X
	\arrow["{f_\lambda}"', from=1-1, to=2-1]
	\arrow["{\exists ! f}", dashed, from=1-1, to=2-2]
	\arrow["{p_\lambda}", from=2-2, to=2-1]
\end{tikzcd} \text{ al variare di \(\lambda \in \Lambda\) }\]
\end{definition}

\begin{proposition}{}
    Siano \({(X, \{p_\lambda\}_{\lambda \in \Lambda} )}\) e \({(X', \{p'_{\lambda} \}_{\lambda \in \Lambda} )}\).
    due prodotti in \(\mathcal{C}\) degli \(X_{\lambda} \). Allora \(\exists ! f \in \mathcal{\mathcal{C}}{(X', X)}\) 
    tale che \(p'_\lambda = p_\lambda \circ f\) per ogni \(\lambda\) e \(f\) è
    isomorfismo. 

    Viceversa se \({(X, \{p_\lambda\}_{\lambda \in \Lambda} )}\) è un prodotto
    degli \(X_\lambda\) e \(f : X' \to X\) è isomorfismo, anche \({(X', \{p_\lambda \circ f\}_{\lambda \in \Lambda} )}\) è un prodotto degli \(X_\lambda\) 
\end{proposition}
\begin{remark}{}
    Si dice che il prodotto è \emph{unico a meno di unico isomorfismo}
\end{remark}
\begin{proof}
    \emph{(Prima parte)}
    Esiste unico \(f\) per la proprietà universale, analogamente \(\exists ! f'
    \in \mathcal{\mathcal{C}}{(X, X')}\) tale che \(p_\lambda = p'_\lambda \circ f\), \(\forall \lambda \in \Lambda\). Dunque \(p_\lambda = p'_\lambda \circ f' = p_\lambda \circ f \circ f' = p_\lambda
     \circ 1_X\). Ne consegue che \(1_X = f \circ f'\) per la proprietà
     universale e analogamente \(1_Y = f'\circ f\).

     \emph{(Seconda parte)} Dati \(Y \in \mathcal{\mathcal{C}}\) e \(g_\lambda : Y \to
     X'_\lambda\) devo dimostrare che \(\exists ! g : Y \to X\) tale che \(
     g_\lambda = f_\lambda \circ f \circ g\) per ogni \(\lambda \in \Lambda\).
     Ora per la proposizione universale di \(X\) \(\exists ! g : Y \to X\) tale
     che \(g_\lambda = p_\lambda \circ g\). Voglio \(g = f \circ g'\) cioè \(g'
     = f^{-1} \circ g\) 
\end{proof}

\begin{note}[zione]
    L'oggetto prodotto \(X\) si indica con
    \[
      X =: \prod_{\lambda \in \Lambda} X_\lambda
    \]
\end{note}

\begin{example}{}
    In \(\mathtt{Set}\) il prodotto di insiemi \(X_\lambda\) per \(\lambda \in \Lambda\) 
    è dato dall'usuale prodotto cartesiano con le proiezioni.

    In categorie concrete come \(\mathtt{Grp}, \mathtt{Rng}, A\mathtt{-Mod}\) un
    prodotto si ottiene dal prodotto in \(\mathtt{Set}\) mettendo la struttura
    disgiuntiva componente per componente.
\end{example}

\begin{example}{}
    In \(\mathtt{FinSet}\) (la sottocategoria piena di \(\mathtt{Set}\)) con
    oggetti insiemi finiti, non esiste \(\prod_{\lambda \in \Lambda} X_\lambda\) se \(\# \Lambda = \infty\) e \(\# X_\lambda > 1\) per ogni \(\lambda \in \Lambda\) 

    Infatti se per assurdo supponiamo il prodotto essere \(X\) per la proprietà
    universale \(\forall Y \in \mathtt{FinSet}\), \( \infty > 
    \# \mathtt{Set}{(Y, X)} = \prod_{\lambda \in \Lambda} \# \mathtt{Set}(Y,
    X_\lambda) = \infty\) 
\end{example}

\begin{remark}{}
    Se \(\# \Lambda = 1\) allora un prodotto di \(X_1\) è \(p_1 : X \to X_{1}\)
    in \(\mathcal{\mathcal{C}}\) tale che \(\forall Y \in \mathcal{\mathcal{C}}\) e \(\forall f_{1} : Y \to X_{1}\) 
    \(\exists ! f : Y \to X\) tale che \(f_{1} = p_{1} \circ f\) 

    Questo è vero se \(p_{1}\) è isomorfismo (\(f = p_{1}^{-1} \circ f_{1}\)).
    D'altra parte se \(p_{1}\) non è isomorfismo non fattorizza unicamente ogni
    altro morfismo. Quindi un prodotto di \(
    X \in \mathcal{\mathcal{C}}\) è qualunque isomorfismo \(X' \to X\) (in particolare \(1_X\)).
\end{remark}

\begin{definition}{Oggetto terminale}
    Un oggetto terminale di una categoria \(\mathcal{\mathcal{C}}\) è un prodotto vuoto in
    \(\mathcal{\mathcal{C}}\), cioè \(X \in \mathcal{\mathcal{C}}\) tale che \(\forall Y \in \mathcal{\mathcal{C}}\) esiste un unico \(Y \to X\) morfismo di \(\mathcal{C}\), cioè \(\#\mathcal{\mathcal{C}}{(Y, X)} = 1\) 
\end{definition}

\begin{example}{}
    In \(\mathtt{Set}\) \(X\) è terminale se e solo se \(\# X = 1\).
    Analogamente in \(\mathtt{Grp}, \mathtt{Rng}, A\mathtt{-Mod}\) è ogni gruppo
    anello o \(A\mathtt{-Mod}\) banale.
\end{example}

\begin{example}{}
    Se \(G\) è un monoide non banale, allora \(G\) (come categoria con un solo
    oggetto) non ha oggetto terminale.
\end{example}

\begin{proposition}{}
    Una categoria \(\mathcal{\mathcal{C}}\) ha tutti i prodotti finiti se e solo se ha
    oggetto terminale e i prodotti di coppie di oggetti.
\end{proposition}
\begin{proof}{}
    Dimostro solo per induzione l'implicazione non ovvia.

    Il passo base è dato dalla presenza dell'oggetto terminale.
    Per induzione supponiamo che esista
    il prodotto di \(n-1\) oggetti \(X'=\prod_{i = 1}^{n-1} X_i\) con \(p'_i :
    X' \to X_{i}\) per ogni \(i\). Sia ora un elemento \(X_n\) e per ipotesi
    esiste \(X = X' \times X_{n}\) con \(p_{n} : X \to X_{n}, p' : X \to X'\).
    Allora \(X\) è prodotto di tutti gli \(\{X_i\}_{i=1}^{n}\) con \(p_{i} :=
    p'_i \circ p'\) per ogni \(i < n\). 
\end{proof}

\begin{definition}{Coprodotto}
    Un coprodotto di \(X_\lambda\) \((\lambda \in \Lambda)\) in una categoria \(\mathcal{\mathcal{C}}\) è un prodotto degli \(X_\lambda\) in \(\mathcal{\mathcal{C}}^{op}\), cioè è dato da \(
    X \in \mathcal{\mathcal{C}}\) e da morfismi \(i_\lambda \in \mathcal{\mathcal{C}}{(X_\lambda, X)}\) tali che vale la proprietà universale (duale di quella del prodotto)

\[
  \forall Y \in \mathcal{C}, \,\, \forall \lambda \in \Lambda,\,\, \forall f_\lambda \in \mathcal{\mathcal{C}}{(X_\lambda, Y)}, \quad \exists ! \,f \in \mathcal{\mathcal{C}}{(X, Y)} : f_\lambda = f \circ i_\lambda
\]
e viene denotato 
\[
  X =: \coprod_{\lambda \in \Lambda} X_\lambda
\]
Diagrammaticamente, il diagramma
\[\begin{tikzcd}
	Y \\
	{X_\lambda} & \coprod_{\lambda \in \Lambda} X_\lambda
	\arrow["{f_\lambda}", from=2-1, to=1-1]
	\arrow["{i_\lambda}"', from=2-1, to=2-2]
	\arrow["{\exists! f}"', dashed, from=2-2, to=1-1]
\end{tikzcd}\]
commuta
\end{definition}

Nelle categorie preadditive si può parlare di \textbf{somma diretta} invece di
coprodotto e usare \(\bigoplus\) invece di \(\coprod\).

\begin{example}{}
    In \(\mathtt{Set}\) il coprodotto è l'unione disgiunta. In \(A\mathtt{-Mod}\) è la somma diretta usuale. In \(\mathtt{Grp}\) i coprodotti sono i \textbf{prodotti liberi}.
\end{example}

\begin{definition}{Oggetto iniziale}
    Un \emph{oggetto iniziale} di \(\mathcal{\mathcal{C}}\) è un coprodotto vuoto in \(\mathcal{C}\), ossia \(X \in \mathcal{\mathcal{C}}\) tale che
    \(\forall Y \in \mathcal{\mathcal{C}}\), \(\# \mathcal{\mathcal{C}}{(X, Y)} = 1\) 
\end{definition}

Può succedere che uno stesso oggetto sia terminale che iniziale. In tal caso per entrambe le definizioni esiste un solo morfismo da uno all'altro. Se tale morfismo è un isomorfismo
allora l'oggetto è sia iniziale che terminale.

\begin{definition}{Oggetto nullo}
    Un oggetto sia iniziale che terminale si dice \emph{nullo}
\end{definition}

\begin{example}{}
    In \(\mathtt{Set}\), \(\varnothing\) è iniziale (non nullo).
    In \(\mathtt{Grp}/A\mathtt{-Mod}\) ogni \emph{gruppo}/\emph{modulo} banale è
    nullo.
    In \(\mathtt{Rng}\), \(\mathbb{Z}\) è iniziale (non nullo)
\end{example}

    Se \(X \in \mathcal{\mathcal{C}}\) è nullo allora \(\forall Y, Z \in \mathcal{\mathcal{C}}\)
    esiste il morfismo \(0 \in \mathcal{\mathcal{C}}{(Y, Z)}\) dato dalla composizione
    \(Y \overset{\exists !}{\to } X \overset{\exists !}{\to } Z\). In tal caso
    abbiamo che effettivamente \(f \circ 0 = 0\) e \(0 \circ g = 0\)  per ogni
    \(f, g\) componibili con \(0\).
\begin{example}{}
    In \(\mathcal{A}\) preadditiva (in cui esiste un oggetto nullo) il morfismo
    0 di cui sopra coincide con 0 dello struttura preadditiva.
\end{example}

\begin{definition}{Preservazione del prodotto}
    Un funtore \(F : \mathcal{\mathcal{C}} \to \mathcal{D}\) \emph{preserva un prodotto}
    \({(X, \{p_\lambda\} _{\lambda \in \Lambda} )}\) di \(\{X_\lambda\}_{\lambda \in \Lambda} \subseteq \mathcal{\mathcal{C}} \) se \({(F{(X)}, \{F{(p_\lambda)}\}_{\lambda \in \Lambda}  )}\) è un prodotto degli \(F{(X_{\lambda} )}\) in \(\mathcal{D}\).

    Diremo inoltre che \(F\) \emph{preserva i prodotti (o prodotti finiti)} se preserva
    tutti i prodotti (o prodotti finiti) che esistono in \(\mathcal{\mathcal{C}}\) 
\end{definition}
\begin{remark}{}
    Se \(F\) preserva un prodotto degli \(X_\lambda\), allora li preserva tutti.
\end{remark}

\begin{definition}{Preservazione del coprodotto}
    \(F: \mathcal{\mathcal{C}} \to \mathcal{D}\) preserva un coprodotto di \(\mathcal{\mathcal{C}}\) se \(F^{op}\) preserva il
    corrispondente prodotto di \(\mathcal{C}^{op}\) 
\end{definition}

\begin{example}{}
    I funtori dimenticanti \(\mathtt{Grp}/\mathtt{Rng}/A\mathtt{-Mod} \to \mathtt{Set}\) preservano i prodotti ma non i coprodotti.
\end{example}

Ora vedremo in particolare cosa succede nelle categorie preadditive, in cui
alcune valgono alcune simpatiche proprietà non ovvie.

\begin{definition}{Biprodotto}
    Sia \(\mathcal{A}\) una categoria preadditiva. Un \emph{biprodotto} di \(X_{1}, \dots, X_{n} \in \mathcal{A}\) 
    è dato da \(X \in \mathcal{A}\) e morfismi (in \(\mathcal{A}\)) \(p_{j}: X\to X_j\) e \(i_{j}:
    X_j \to X\), \(\forall j=1.. n\).
    tali che
    \begin{align*}
        p_{j}\circ i_{j} = 1_{X_{j}} \quad ; \quad p_k \circ i_{j} &= 0 \quad \forall j,k=1 .. n \,\, \text{ con }j \neq k \\
        \text{ e }\sum_{j=1}^{n} i_{j} \circ p_j &= 1_X 
    \end{align*}
\end{definition}
\begin{remark}{}
    \({(X, i_{1}, \dots, i_n, p_{1}, \dots, p_n )}\) è un biprodotto di \(X_{1}, \dots, X_{n}\) in \(\mathcal{A}\) 
    se e solo se \({(X, p_{1}, \dots, p_n, i_1, \dots, i_n)}\) è un biprodotto
    in \(\mathcal{A}^{op}\) 
\end{remark}
Se \(n = 0\) la condizione diventa \(1_X = 0\), dunque l'anello degli
endomorfismi \(\mathrm{End}_{\mathcal{A}} {(X)}\) è banale.

Se \(n = 1\) \(i_{1}\) e \(p_{1}\) sono isomorfismi e l'uno l'inverso
dell'altro.

\begin{remark}{}
Basta verificare \(p_k \circ i_{j} = 0\) per \(j,k=1. .n-1\) e \(j\neq k\)
(dunque per \(n=2\)) non è necessario verificare quella parte della definizione
\end{remark}
\begin{proof}{}
    Sia \(k < n\). Allora supponendo che \(p_k \circ i_j = 0\) se \(k \neq j < n\),
    \begin{align*}
        p_n \circ i_k &= p_n \circ 1_X \circ i_k = p_n \circ {\left( \sum_{j=1}^{n} i_j \circ p_j  \right)} \circ i_k = \sum_{j=1}^{n} p_n \circ i_j \circ p_j \circ i_k = \\
        &= p_n \circ i_n \circ p_n \circ i_k + \sum_{j=1}^{n-1} p_n \circ i_j
        \circ p_j \circ i_k = \\
        &= 1_{X_{n}} \circ p_n \circ i_k + p_{n} \circ
        i_k \circ 1_{X_k} = p_{n} \circ i_k + p_{n} \circ i_k
    \end{align*}
    \(\implies p_{n} \circ i_k = 0\) 
\end{proof}

\begin{proposition}{}
    Sia \(\mathcal{A}\) una categoria preadditiva e siano \(X_{1}, \dots, X_{n} \in \mathcal{A}\).
    Allora \(X \in \mathcal{A}\) è biprodotto di \(X_{1}, \dots, X_{n}\) \emph{se e solo se} \(X\) è un prodotto di \(X_{1}, \dots, X_{n}\) \emph{se e solo se }\(X\) è
    un coprodotto di \(X_{1}, \dots, X_{n}\).

    Più precisamente \((X, p_{1}, \dots, p_{n})\) è un prodotto di \(X_{1}, \dots, X_{n}\) se e solo se esistono (unici) \(i_{1},\dots, i_n\) tali che \({(X, \{i_j\}_{j=1. .n}, \{p_{j}\}_{j = 1. .n}  )}\) è un biprodotto di \(X_{1}, \dots, X_{n}\).
    Analogamente e dualmente per il coprodotto.
\end{proposition}
\begin{proof}\( \)
\begin{itemize}
    \item[\(\implies \)] Per la proprietà universale del prodotto \(\exists ! i_j : X_j \to X\) per ogni \(j=1 . .n\) tali che
        \[
          p_k \circ i_j = (f_{j} = ) \begin{cases}{}
              1_{X_j} & \text{ se } j = k \\
              0 & \text{ se } j\neq k
          \end{cases}\quad
\begin{tikzcd}
	X_j  \\
	{X_k} & X
	\arrow["{\delta_{kj} 1_{X_{j}} }"', from=1-1, to=2-1]
	\arrow["{\exists ! i_{j}}", dashed, from=1-1, to=2-2]
	\arrow["{p_k}", from=2-2, to=2-1]
\end{tikzcd}\]
        Resta da dimostrare che 
        \[
          \sum_{j=1}^{n} i_{j} \circ p_j = 1_X \iff \forall k, \,\, p_k \circ
          \sum_{j=1}^{n} i_{j} \circ p_{j} = p_k \text{ (proprietà universale)}
        \]
        e effettivamente 
        \[
          p_k \circ \sum_{j=1}^{n} i_{j} \circ p_{j} = \sum_{j=1}^{n} p_k \circ
          i_{j} \circ p_{j} = \underbrace{p_k \circ i_k}_{1_{X_k} }  \circ p_k = p_k
        \]
    \item[\(\impliedby \)] Dati \(f_{j} : Y \to X_j \) devo dimostrare che \(\exists  ! \, f : Y \to X\) tale che \(f_j = p_j \circ f\) per ogni \(j = 1..n\). Allora posso definire
        \[
          f := \sum_{k=1}^{n} i_k \circ f_k : Y \to  X \text{ e allora } p_j
          \circ f = \sum_{k=1}^{n} p_j \circ i_k \circ f_k = f_j 
        \]
        essa è unica poiché se \(f' : Y\to X\) è tale che \(\forall j\), \(f_j = p_j \circ f'\),
        allora \[f' = 1_X \circ f' = \sum_{j=1}^{n} i_j \circ p_j \circ f' = \sum_{j=1}^{n} i_j \circ f_j = f\] 
\end{itemize}
\end{proof}
\begin{remark}[\(n =0\)]
    In una categoria preadditiva un oggetto \\ è terminale \(\iff\)  è iniziale \(\iff\) è nullo \(\iff\) \(1_X = 0\) 
\end{remark}

\begin{remark}{}
    Sia \(F : \mathcal{A} \to \mathcal{B}\) un funtore additivo e sia \(X, \{i_{j}\}_{j = 1. .n}, \{p_{j}\} _{j = 1. .n}  \) un biprodotto di \(X_{1}, \dots, X_{n} \in \mathcal{A}\). Allora 
    \({(F{(X)}, \{F{(i_{j})}\}_{j=1. .n}, \{F{(p_{j})}\} _{j=1. .n}  )}\) è un
    biprodtto di \(F{(X_{1})}, \dots, F{(X_{n})}\) in \(\mathcal{B}\) 
\end{remark}
\begin{corollary}{}
    Un funtore \(F: \mathcal{A} \to \mathcal{B}\) additivo preserva i prodotti e
    i coprodotti finiti.
\end{corollary}

\begin{note}[zione matriciale nelle categorie preadditive]
    Sia \(\mathcal{A}\) una categoria preadditiva. Siano \(X_{1}, \dots, X_{n}\) e \(Y_{1},\dots, Y_{n}\) in \(\mathcal{A}\) tali che \(\exists {(X, i_j^{X}, p_j^{X})}\) biprodotto di \(X_{1}, \dots, X_{n}\) 
    e \(Y, i_j^{Y}, p_j^{Y}\) biprodotto di \(Y_{1}, \dots, Y_{n}\).
    Dati \(f_{j,k}  : X_k\to Y_j \) morfismi di \(\mathcal{A}\) indico con la
    matrice \(m \times  n\) \({(f_{j,k} )}\) il morfismo \(f: X\to Y\)
    definito da
    \[
      f = \sum_{j=1}^{m} \sum_{k=1}^{n} i_j^{Y} \circ f_{j, k} \circ p_k^{X}
    \]
\end{note}

\begin{definition}{Categoria additiva}
    Una \emph{categoria additiva} è una categoria preadditiva in cui esistono
    tutti i biprodotti.
\end{definition}

\begin{example}{}
    \(A\mathtt{-Mod}\) è additiva per ogni anello \(A\).
\end{example}
\begin{example}{}
    \(A\) anello come categoria preadditiva con un solo oggetto è additiva se e solo se
    \(A=0\) 
\end{example}

\begin{remark}{}
    Se \(\mathcal{A}\) è additiva, allora anche \(\mathcal{A}^{op}\) lo è.
\end{remark}

\begin{example}{}
    Le sottocategorie piene di \(A\mathtt{-Mod}\) con oggetti i moduli
    \emph{f.g.} / \emph{f.p.} / \emph{coerenti} / \emph{noetheriani} /
    \emph{artiniani} / \emph{liberi} sono additive. Non vale per esempio per
    ciclici.
\end{example}

\begin{proposition}{}
    Sia \(F: \mathcal{A} \to \mathcal{B}\) un funtore additivo essenzialmente
    suriettivo. Allora se \(\mathcal{A}\) è additiva, anche \(\mathcal{B}\) è
    additiva. In particolare \(\mathcal{A}/\mathfrak{I}\) è additivo per ogni \(\mathfrak{I} \subseteq \mathcal{A} \) 
    ideale.
\end{proposition}

\begin{proof}{}
    \(\forall Y_{1}, \dots, Y_{n}\), se \(F\) è ess. suriettivo, allora \(\exists X_{j} \in \mathcal{A}\) 
    tale che \(F{(X_{j})} \cong Y_j\). A meno di cambiare \(F\) con \(F' \cong F\) 
    posso supporre \(F {(X_{j})} = Y_{j}\).

    Allora se \(\exists X\) biprodotto di \(X_{1}, \dots, X_{n}\) in \(\mathcal{A}\), \(F{(X)}\) è 
    biprodotto di \(Y_{1}, \dots, Y_{n}\) 
\end{proof}

\begin{remark}{}
    Sia \(\mathcal{C}\) una categoria (piccola) e \(\mathcal{A}\) additiva.
    Allora \(\mathtt{Fun}{(\mathcal{C}, \mathcal{A})}\) è additiva.
\end{remark}
\begin{proof}{}
    Dati \(F_{1}, \dots, F_{n} \in \mathtt{Fun}{(\mathcal{C}, \mathcal{A})}\)
    devo dimostrare che esiste il loro biprodotto.
    Per ogni \(X \in \mathcal{C}\) esiste \(F{(X)}\) biprodotto di
    \(F_{1}{(X)}, \dots, F_{n}{(X)}\) in \(\mathcal{A}\).

    \(\forall f : X \to Y\) morfismo di \(C\) definisco \(F{(f)}\) come il
    morfismo definito con la notazione matriciale da
    \[
      F{(X)} = \bigoplus_{i = 1}^{n}F_{i}{(X)} \overset{
        \begin{pmatrix}
            F_{1}{(f)} &  & 0 \\
             & \dots & \\
             0 &  & F_n{(X)}
        \end{pmatrix}
      }{\longrightarrow} F{(Y)} = \bigoplus_{i=1} ^{n} F_{i}{(Y)}
    \]
    Allora \(F: \mathcal{C} \to \mathcal{A}\) è un funtore e ci sono
    trasformazioni naturali \(i_j : F_{j} \to F\) e \(p_{j} : F \to F_{j}\) le
    cui componenti sono
    \[
      {(i_{j})}_X := i^{X}_j : F_{j}{(X)} \to F{(X)} \quad ; \quad {(p_j)}_X :=
      p_{j}^{X} : F{(X)} \to F_{j}{(X)}
    \]
    Poiché \(\mathcal{C}\) è preadditiva, \(\mathtt{AddFun}{(\mathcal{C},\mathcal{A})}\) è additiva.
    Poiché \(F_{1}, \dots, F_{n} \in \mathtt{AddFun}{(\mathcal{C}, \mathcal{A})}\), allora \(F = \bigoplus_{i=1} ^{n}\) è additivo, infatti
    \begin{align*}
        F{(f + g)} &= \begin{pmatrix}
          F_{1}{(f+g)} &  & 0 \\
           & \dots &  \\
           0 &  & F_n{(f + g)}
           \end{pmatrix} = \\ &=\begin{pmatrix}
          F_{1}{(f)}+F_{1}{(g)} &  & 0 \\
           & \dots &  \\
      0 &  & F_n{(f)} + F_n{(g)}
      \end{pmatrix} = F{(f)} + F{(g)}
    \end{align*}
\end{proof}

\begin{example}{}
    Se \(\mathcal{A}\) è preadditiva (piccola), allora \(\mathcal{A}\mathtt{-Mod} := \mathtt{AddFun}{(\mathcal{A}, \mathtt{Ab})}\) è additiva.
\end{example}

\begin{proposition}{}\label{prop:F_add}
    Sia \(F : \mathcal{A} \to \mathcal{B}\) un funtore con \(\mathcal{A}\)
    additiva, \(\mathcal{B}\) preadditiva, \(F\) che preserva i prodotti (o
    coprodotti) finiti. Allora \(F\) è additivo
\end{proposition}
\begin{proof}[Dimostrazione (idea)]
    Sia \(0 \in \mathcal{A}\) oggetto nullo, allora \(F{(0)}\) è termianle,
    dunque \(F{(0)}\) è nullo (perché \(B\) è preadditiva), ossia \(F{(0)} = 0\).

    Allora segue che \(F\) preserva i biprodotti. Infatti se \((X, \{i_j\}_{j = 1. .n}, \{p_{j}\}_{j= 1. .n})\)
    è un biprodotto di \(X_{1}, \dots, X_{n}\) in \(\mathcal{A}\), allora \({(X, p_{1}, \dots, p_{n})}\) è un prodotto di \(X_{1}, \dots, X_{n}\) in \(\mathcal{A}\)
    e dunque \(F{(X)}, \{F{(p_{j})}\}_{j = 1. .n} \) è un prodotto di 
    \(F{(X_{1})}, \dots, F{(X_{n})}\) in \(\mathcal{B}\). Allora \(\exists !
    i'_j : F{(X_{j})} \to F{(X)}\) tali che \({(F{(X)}, \{i_{j}'\}_{j = 1. .n}, \{F{(p_{j})}\}_{j = 1. .n})}\)
    è un biprodotto di \(F{(X_{1})}, \dots, F{(X_{n})}\). Ma allora deve essere
    \(i_{j}' = F{(i_{j})}\) poiché 
    \begin{align*}
        F{(p_k \circ i_j)} &= F{(p_k)} \circ F{(i_{j})} = F{(p_k)} \circ i'_j = \begin{cases}{}
          1 & \text{ se } j = k \\
          0 & \text{ se } j\neq k
      \end{cases} \\
            F{(0 : Y \to Z)} &= 0 : F{(Y)} \to F{(Z)}
    \end{align*}
    E allora se \(f, g : X \to Y\) in \(\mathcal{A}\), allora \(f + g\) è la
    composizione
    \[
      X \overset{\begin{pmatrix}
          1 \\
          1
      \end{pmatrix}}{\longrightarrow} X \oplus X \overset{\begin{pmatrix}
          f & 0 \\
          0 & g
      \end{pmatrix}}{\longrightarrow} Y \oplus Y \overset{\begin{pmatrix}
          1 & 1
      \end{pmatrix}}{\longrightarrow} Y
    \]
    Allora \(F {(f + g)}\) è la composizione
    \[
      F{(X)} \overset{\begin{pmatrix}
          1 \\
          1
      \end{pmatrix}}{\longrightarrow} F{(X)} \oplus F{(X)} \overset{\begin{pmatrix}
          F{(f)} & 0 \\
          0 & F{(g)}
      \end{pmatrix}}{\longrightarrow} F{(Y)} \oplus F{(Y)} \overset{\begin{pmatrix}
          1 & 1
      \end{pmatrix}}{\longrightarrow} F{(Y)}
    \]
    che è \(F{(f)} + F{(g)}\) 
\end{proof}

\begin{corollary}{}
    Se \(\mathcal{A}\) è additiva, allora \(\mathcal{A}\) ha un'unica struttura
    preadditiva.
\end{corollary}
\begin{proof}{}
    Considero \(\mathcal{B} := \mathcal{A}\) come categoria con una qualunque
    struttura preadditiva e \(F = \mathrm{id}\). Allora per la
    proposizione~\ref{prop:F_add}, \(F\) è additivo. Poiché \(F\) è pienamente
    fedele, esiste un'unica struttura preadditiva su \(\mathcal{A}\) tale che \(F=\mathrm{id}\) è additivo, dunque la struttura di \(\mathcal{B}\) coincide con quella di \(\mathcal{A}\) 
\end{proof}
\begin{remark}{}
    Sia \(\mathcal{A}\) un anello commutativo, \(\mathcal{A}\) una categoria
    \(A\)-lineare e additiva. Allora \emph{non necessariamente} la struttura
    \(A\)-lineare è unica. Data infatti una struttura \(A\)-lineare posso
    cambiarla su \(\mathcal{A}{(X, Y)}\), \(\forall X, Y \in \mathcal{A}\). Se è
    data da omomorfismi di anelli \(\mathcal{A} \overset{\varphi_{X,Y} }{\to } \mathrm{End}{(\mathcal{A}{(X, Y)})}\)
    considero \(A \overset{\alpha}{\to } A \overset{\varphi_{X, Y} }{\to } \mathrm{End}{(\mathcal{A}{(X, Y)})}\) con \(\alpha\) omomorfismo di anelli non banale. % NOTE: non banale l'ho aggiunto io
\end{remark}
\begin{corollary}{}
    Se \(\mathcal{A}\) e \(\mathcal{B}\) sono categorie equivalenti e \(\mathcal{A}\) è additiva, allora \(B\) è additiva.
\end{corollary}
\begin{proof}
    So già che \(\mathcal{B}\) è preadditiva. Allora \(\exists F: \mathcal{A} \to \mathcal{B}\) equivalenza, additivo per la proposizione~\ref{prop:F_add}.
    \(F\) è essenzialmente suriettivo, dunque \(\mathcal{B}\) è additivo.
\end{proof}

\section{Limiti e colimiti}

L'obiettivo nostro è riuscire a generalizzare il concetto di \(\ker\) e \(\mathrm{coker}\)
che era definito sui moduli. Vedremo che il giusto contesto sarà quello delle
categorie \emph{abeliane}. (probabilmente) A scopo di definire tale struttura,
introduciamo i seguenti nuovi concetti

\begin{definition}{}
    Sia \(\mathcal{C}\) una categoria, \(I : \mathcal{L} \to \mathcal{C}\) un
    funtore (con \(\mathcal{L}\) piccola).

    Un \emph{limite} di \(I\) è dato da \(X \in \mathcal{C}\) con una
    trasformazione naturale \(\alpha : K_X \to I\) dove \(K_X : \mathcal{L} \to \mathcal{C}\) è il funtore 
    costante di valore \(X\), ossia
\(\begin{tikzcd}
	\mathcal{L} & \mathcal{C}
	\arrow[""{name=0, anchor=center, inner sep=0}, "K_X", curve={height=-10pt}, from=1-1, to=1-2]
	\arrow[""{name=1, anchor=center, inner sep=0}, "I"', curve={height=10pt}, from=1-1, to=1-2]
	\arrow["\alpha", between={0.1}{0.9}, Rightarrow, from=0, to=1]
\end{tikzcd}\).

    In altri termini, \(\forall L \in \mathcal{L}\), \(\alpha_L ; X \to I{(L)}\) è tale che \(\forall l : L \to L'\) in \(\mathcal{L}\), il seguente diagramma commuta:
\[\begin{tikzcd}
	X \\
	{I(L)} & {I(L')}
	\arrow["{\alpha_L}"', from=1-1, to=2-1]
	\arrow["{\alpha_{L'}}", from=1-1, to=2-2]
	\arrow["{I(l)}"', from=2-1, to=2-2]
\end{tikzcd}\]

    Inoltre, vale la seguente proprietà universale:
    \[
      \forall Y \in \mathcal{C}, \,\, \forall \beta : K_Y \to I \text{
      trasformazione naturale, } \, \exists  ! \, f \in \mathcal{C}{(Y, X)} \text{ t.c. } \beta = \alpha \circ K_f
    \]
    dove \(K_f : K_Y \to K_X\) è la trasformazione naturale definita da \({(K_f)}_L := f \) per ogni \(L \in \mathcal{L}\). In altre parole \(\beta_L = \alpha_L \circ f\) per ogni \(L \in \mathcal{L}\), ossia il seguente diagramma commuta:
\[\begin{tikzcd}
	X \\
	{I(L)} & {I(L')} \\
	Y
	\arrow["{\alpha_L}"', from=1-1, to=2-1]
	\arrow["{\alpha_{L'}}", from=1-1, to=2-2]
	\arrow["{I(l)}"', from=2-1, to=2-2]
    \arrow["{\exists ! f}"', curve={height=-35pt}, dashed, from=3-1, to=1-1]
	\arrow["{\beta_L}", from=3-1, to=2-1]
	\arrow["{\beta_{L'}}"', from=3-1, to=2-2]
\end{tikzcd}\]
\end{definition}

\begin{example}[Prodotto]
    Se \(\mathcal{L} = \Lambda\) insieme (categoria discreta), allora \(I\) è
    dato dagli \(I{(\lambda)} =: X_\lambda \in \mathcal{C}\) (\(\lambda \in \Lambda\)), e un
    limite di \(I\) è un prodotto degli \(I{(\lambda)}\) (\(\lambda \in \Lambda\)) perché un limite 
    è dato da \(X \in \mathcal{C}\) con morfismi \(\alpha_\lambda : X \to X_\lambda\) (qualunque, perché \(l\) può essere solo \(1_\lambda\)) tali che \(\forall Y \in \mathcal{C}\) e \(\forall \beta_\lambda : Y \to X_\lambda\),
    \[
      \exists ! \, f \in \mathcal{C}{(Y, X)} : \beta_\lambda  = \alpha_\lambda
      \circ f \quad \forall \lambda \in \Lambda
      \quad 
\begin{tikzcd}
	X \\
	{X_\lambda} & {X_\lambda} \\
	Y
	\arrow["{\alpha_\lambda}"', from=1-1, to=2-1]
    \arrow["{\alpha_\lambda}", from=1-1, to=2-2]
	\arrow["{1_\lambda}"', from=2-1, to=2-2]
    \arrow["{\exists ! f}"', curve={height=-35pt}, dashed, from=3-1, to=1-1]
	\arrow["{\beta_\lambda}", from=3-1, to=2-1]
    \arrow["{\beta_\lambda}"', from=3-1, to=2-2]
\end{tikzcd}\]
\end{example}

\begin{example}{}
    Sia \(\mathcal{L} = {(\bullet \rightrightarrows \bullet )}\). Allora \(I\) è
    dato da \( I{(\mathcal{L})} = Y \underset{g}{\overset{f}{\rightrightarrows}} Z\) in \(\mathcal{C}\). Un limite di \(I\) è dato da \(X \in \mathcal{C}\) 
    con morfismi \(\alpha_{1} : X\to Y\) e \(\alpha_{2}: X \to Z\) tali che
    \(\alpha_{2} = f \circ \alpha_{1} = g \circ \alpha_{1}\) o equivalentemente
    \(h: X \to Y\) tale che \(f \circ h = g \circ h\).

    La proprietà universale diventa:
    \[
      \forall X' \in \mathcal{C}, \, \forall h' : X' \to Y : f \circ h' = g \circ\,\, \exists ! \,k :
  X' \to X : h' = h \circ k\]
\[\text{ossia il diagramma }\begin{tikzcd}
	X & Y & Z \\
	X'
	\arrow["h", from=1-1, to=1-2]
	\arrow["g"', shift right, from=1-2, to=1-3]
	\arrow["f", shift left, from=1-2, to=1-3]
	\arrow["{\exists ! \,k}", dashed, from=2-1, to=1-1]
	\arrow["{h'}"', from=2-1, to=1-2]
\end{tikzcd} \text{ commuta}\]
    tale \(h\) si dice \emph{equalizzatore} di \(f\) e \(g\).

    In \(\mathtt{Set}\) si può prendere \(X := \{y \in Y : f{(y)} = g{(y)}\} \overset{i}{\hookrightarrow } Y\).
    Analogamente in \(\mathtt{Grp}\), \(\mathtt{Rng}\) e \(A\mathtt{-Mod}\) e
    sarà sottogruppo, sottoanello e sottomodulo come diagramma abbiamo
\(\begin{tikzcd}
	X & Y & Z \\
	X'
	\arrow["i", from=1-1, to=1-2]
	\arrow["g"', shift right, from=1-2, to=1-3]
	\arrow["f", shift left, from=1-2, to=1-3]
	\arrow["{\exists ! \,k}", dashed, from=2-1, to=1-1]
	\arrow["{h'}"', from=2-1, to=1-2]
\end{tikzcd}\)
\end{example}

\begin{example}[Prodotto fibrato o \emph{pullback}]
    Sia \(\mathcal{L} = {(\bullet \to  \bullet \leftarrow \bullet)}\). Allora \(I\) è dato
    da\(\begin{tikzcd}[baseline=-4.5ex]
	& {Y_2} \\
	{Y_1} & Z
	\arrow["{g_2}"', from=1-2, to=2-2]
	\arrow["{g_1}", from=2-1, to=2-2]
\end{tikzcd}\) in \(\mathcal{C}\). Un limite di \(I\) è \(X \in \mathcal{C}\)
    con morfismi \(f_{1}: X \to Y_{1}\) e \(f_{2}: X \to Y_{2}\) (e \(h: X \to Z\)) tale che
    \[
      {(h = )} g_{1} \circ f_{1} = g_{2} \circ f_{2}
    \]
    e la proprietà universale diventa: dati \(f_{i}' : X' \to Y_{i}\) tali che
    \(g_{1} \circ f_{1}' = g_{2} \circ f_{2}'\), allora \(\exists ! f : X' \to X\) tale che
    \(f_{1}' = f_{1}\circ f\). Equivalentemente si ha che il seguente diagramma
    commuta:
    \[\begin{tikzcd}
	{X'} \\
	& X & {Y_2} \\
	& {Y_1} & Z
	\arrow["{\exists!\, f}"{description}, dashed, from=1-1, to=2-2]
	\arrow["{f'_2}"', curve={height=-12pt}, from=1-1, to=2-3]
	\arrow["{f'_1}"', curve={height=12pt}, from=1-1, to=3-2]
	\arrow["{f_2}"', from=2-2, to=2-3]
	\arrow["{f_1}", from=2-2, to=3-2]
	\arrow["{g_2}"', from=2-3, to=3-3]
	\arrow["{g_1}", from=3-2, to=3-3]
    \end{tikzcd}\]
    In tal caso \({(X, f_{1}, f_{2})}\) si dice \emph{prodotto fibrato} di \(g_{1}, g_{2}\) (o \emph{pullback})
\end{example}

\begin{example}[Prodotto fibrato su \(\mathtt{Set}\)]
    \(\mathtt{Set}\) (o altre categorie concrete) si può prendere \(X := \{y_{1}, y_{2} \in Y_{1} \times  Y_{2} : g_{1}{(y_{1})} = g_{2}{(y_{2})}\}\) 
\end{example}

\begin{example}{}
    Se prendiamo \(\mathcal{L} = \varnothing\) e \(I\) il funtore vuoto, allora
    la definizione di limite diventa (notare l'assenza di \(\alpha\) e \(\beta\) che sono funzioni vuote)
    \[
      X \in \mathcal{C} : \forall Y \in \mathcal{C} \exists ! f \in
      \mathcal{C}{(Y, X)}
    \]
    ossia \(X\) è oggetto terminale
\end{example}


\begin{definition}{Colimite}
    Un \emph{colimite} di \(I : \mathcal{L} \to \mathcal{C}\) è un limite di \(I^{op} : \mathcal{L}^{op} \to \mathcal{C}^{op}\), cioè è dato da \(X \in \mathcal{C}\) 
    con una trasformazione naturale \(\alpha : I \to K_X\) tale che \(\forall Y
    \in \mathcal{C}\), \(\forall \beta: I \to K_Y\) trasformazione naturale,
    \(\exists ! f \in \mathcal{C}{(X, Y)}\) tale che \(\beta = K_f \circ
    \alpha\) (cioè \(\beta_L = f \circ \alpha_L\) per ogni \(L \in \mathcal{L}\)
    ).

    Equivalentemente, \(L, L' \in \mathcal{L}, Y \in \mathcal{C}, \beta_L : I(L) \to Y\) e \(b_{L'} : I(L') \to  Y\), esiste unico \(f : X \to Y\) tale che il seguente diagramma commuta:
\[\begin{tikzcd}
	X \\
	{I(L)} & {I(L')} \\
	Y
	\arrow["{\exists!f}", curve={height=35pt}, dashed, from=1-1, to=3-1]
	\arrow["{\alpha_L}", from=2-1, to=1-1]
	\arrow["{I(l)}"', from=2-1, to=2-2]
	\arrow["{\beta_L}"', from=2-1, to=3-1]
	\arrow["{\alpha_{L'}}"', from=2-2, to=1-1]
	\arrow["{\beta_{L'}}", from=2-2, to=3-1]
\end{tikzcd}\]
\end{definition}
\begin{example}{}
    Se \(\mathcal{L} = {(\bullet \rightrightarrows \bullet)}\), allora un
    colimite di \(I\) (dato da \(Y \underset{g}{\overset{f}{\rightrightarrows}} Z\)) è dato da \(h : Z \to X\) tale che \(h \circ f = h \circ g\) tale che \(\forall h' : Z \to X'\) tale che \(h' \circ f = h' \circ g\), \(\exists  !\, k : X \to X'\) tale che \(h'= k \circ h\). Equivalentemente il diagramma
\begin{tikzcd}
	Y & Z & X \\
	&& {X'}
	\arrow["f", shift left, from=1-1, to=1-2]
	\arrow["g"', shift right, from=1-1, to=1-2]
	\arrow["h", from=1-2, to=1-3]
	\arrow["{h'}"', from=1-2, to=2-3]
	\arrow["{\exists ! \, k}", dashed, from=1-3, to=2-3]
\end{tikzcd}
commuta e \(h\) si dice \emph{coequalizzatore} di \(f\) e \(g\).
\end{example}
\begin{eser}{Coprodotto fibrato o \emph{pushout}}
    Definire il coprodotto fibrato, che è il duale del prodotto fibrato.
    \tcblower
    Sia \(\mathcal{L} = {(\bullet \leftarrow \bullet \to \bullet)}\) (duale di
    quella del prodotto fibrato). Allora una colimite di \(I\) (la cui immagine in \(\mathcal{C}\)  è \(\begin{tikzcd}[baseline=-4.5ex]
	& {Y_2} \\
	{Y_1} & Z
	\arrow["{g_2}"', from=2-2, to=1-2]
	\arrow["{g_1}", from=2-2, to=2-1])
\end{tikzcd}\) è dato da \(X \in \mathcal{C}\), \(f_{1} : Y_{1}\to X, f_{2} : Y_{2}\to X(, h : Z \to X)\) morfismi tali che \(f_{1} \circ
g_{1} = f_{2} \circ g_{2} (= h)\). Inoltre la proprietà universale dice che \(\forall X' \in \mathcal{C}\), \(f_{1}' \in \mathcal{C}{(Y_{1}, X)}\) e \(f_{2}' \in \mathcal{C}{(Y_{2}, X)}\) tali che \(f_{1}' \circ g_{1} = f_{2}' \circ g_{2}\), allora \(\exists ! \, f : X \to Y\) tale che \(f \circ f_i = f_i'\) per \(i \in \{1, 2\}\).

Equivalentemente, il seguente diagramma commuta:
    \[\begin{tikzcd}
	{X'} \\
	& X & {Y_2} \\
	& {Y_1} & Z
	\arrow["{\exists!\, f}"{description}, dashed, from=2-2, to=1-1]
	\arrow["{f'_2}"', curve={height=12pt}, from=2-3, to=1-1]
	\arrow["{f'_1}", curve={height=-12pt}, from=3-2, to=1-1]
	\arrow["{f_2}"', from=2-3, to=2-2]
	\arrow["{f_1}", from=3-2, to=2-2]
	\arrow["{g_2}"', from=3-3, to=2-3]
	\arrow["{g_1}", from=3-3, to=3-2]
    \end{tikzcd}\]
    In tal caso \({(X, f_{1}, f_{2})}\) si dice \emph{prodotto cofibrato} di \(g_{1}, g_{2}\) (o \emph{pushout})
\end{eser}

\begin{example}{}
    In \(\mathtt{Set}\) un coequalizzatore di \(Y \underset{g}{\overset{f}{\rightrightarrows}} Z\) è dato dalla
    proiezione al quoziente \(Z \to Z/_\sim \) dove \(\sim \) è la più piccola
    relazione di equivalenza su \(Z\) che contiene \(\{{(f{(y)}, g{(y)})} | y
    \in Y\} \subseteq Z \times  Z  \)
\end{example}
\begin{note}[zione]
    In alcuni casi invece di \emph{limiti}/\emph{colimiti} si può parlare di \emph{limiti inversi}/\emph{limiti diretti}.
\end{note}

\begin{example}{}
    Similmente a prima, prendendo \(\mathcal{L} = \varnothing\) e \(I\) il
    funtore vuoto, la definizione diventa
    \[
      X \in \mathcal{C} : \forall Y \in \mathcal{C}\quad \exists ! f \in \mathcal{C}{(X, Y)}
    \]
    ossia \(X\) è un oggetto iniziale.
\end{example}

\begin{proposition}[Unicità del limite]\label{prop:unic_limite}
    Se \({(X, \alpha)}\) limite esiste, allora è unico a meno di isomorfismo. In
    altre parole dati \({(X, \alpha)}\) e \({(Y, \beta)}\) limiti di \(I\),
    allora \(\exists ! \, f \in C{(X', X)}\) tale che \(\beta = \alpha \circ
    K_f\) e \(f \) è isomorfismo.

    Viceversa, se \({(X, \alpha)}\) è un limite di \(I\) e \(f : Y \to X\) è
    isomorfismo, allora anche \({(X', \alpha \circ K_f)}\) è un limite di \(I\) 
\end{proposition}
\begin{eser}
    Dimostrare la proposizione~\ref{prop:unic_limite}
\end{eser}

\begin{note}[zione]
    Un limite di \(I\) si indica con \(\lim I \) o \(\displaystyle\lim_{\longleftarrow} I \)
    e analogamente un colimite di \(I\) si indica con \(\mathrm{colim} \, I\) o
    \(\displaystyle\lim_{\longrightarrow} I\).

    Un (co)equalizzatore d \(f, g ; Y \to Z\) si indica con \(\mathrm{(co)eq} {(f, g)}\)


    Un prodotto fibrato di \(Y_{1} \overset{g_{1}}{\to } Z
    \overset{g_{2}}{\leftarrow} Y_{2}\) si indica con \(Y_{1} \times_Z Y_{2}\).
    Similmente per i coprodotti fibrati si usa \(Y_{1} \coprod_Z Y_{2}\) 
\end{note}

\begin{proposition}{}
    Sia \(\gamma : I \to I'\) una trasformazione naturale tra funtori \(\mathcal{L} \to \mathcal{C}\) 
    e sia \({(\lim I )}\) un limite di \(I\). Allora
\begin{enumerate}[label = \arabic*.]
    \item Se \({(\lim I ' , \alpha')}\) è un limite di \(I'\) allora \(\exists ! \lim \gamma \in \mathcal{C}{(\lim I, \lim I')}\) tale che \(\alpha' \circ K_{\lim \gamma } = \gamma \circ \alpha \),
        inoltre se \(\gamma\) è isomorfismo, allora anche \(\lim \gamma\) è
        isomorfismo.
\[\begin{tikzcd}
	{K_{ \lim I}} & {K_{ \lim I'} } \\
	{I} & {I'}
	\arrow["{K_{ \lim \gamma} }", from=1-1, to=1-2]
	\arrow["{\alpha}", from=1-1, to=2-1]
	\arrow["{\alpha'}", from=1-2, to=2-2]
	\arrow["{\gamma}"', from=2-1, to=2-2]
\end{tikzcd}\]
\item Se \(\gamma\) è isomorfismo, allora \({( \lim I, \gamma \circ \alpha)}\) è
    un limite di \(I'\) 
\end{enumerate}
\end{proposition}
\begin{proof}{}
\begin{enumerate}[label = \arabic*.]
    \item Segue direttamente dal fatto che \({( \lim I', \alpha')}\) è un limite
        di \(I'\) e \(\gamma \circ \alpha\) è una trasformazione naturale
    \item Sia \(Y \in \mathcal{C}\) e \(\beta : K_Y \to I'\) trasformazione naturale. Allora \(\gamma^{-1} \circ \beta : K_Y \to I\) è trasformazione naturale e dunque \(\exists !\, f \in C{(Y,  \lim I)} \) tale che \(\gamma^{-1} \circ \beta = \alpha \circ K_f\), da cui \(\beta = \gamma \circ \alpha \circ K_f\) 
\end{enumerate}
\end{proof}

\begin{theorem}{Condizione sufficiente per l'esistenza dei limiti}\label{th:suff_coeq_prod}
    Sia \(\mathcal{L}\) una categoria piccola. Allora esistono tutti i (co)limiti dei funtori \(\mathcal{L} \to \mathcal{C}\) se in \(\mathcal{C}\) esistono tutti i (co)equalizzatori e tutti i (co)prodotti indicizzati dagli oggetti e dai morfismi di \(\mathcal{L}\).

    Più precisamente un (co)limite di \(I : \mathcal{L} \to \mathcal{C}\) è dato
    da \({(X, \alpha)}\) con
    \begin{align*}
    X &= \mathrm{eq}{\left( \prod_{L \in \mathcal{L}} I{(L)} \underset{h}{\overset{g}{\rightrightarrows}} \prod_{{(l : L \to L')} \in \mathrm{Mor}{(\mathcal{L})}} I{(L')}) \right)} \\
    \text{dove } p_l \circ g &:= p_{L'}, \quad p_l \circ h := I{(l)} \circ p_L \quad \forall {(l : L \to L')} \in \mathrm{Mor}{(\mathcal{L})}
\end{align*}

Indicando con \(i : X \to \prod_{L \in \mathcal{L}} I{(L)}\) il morfismo
naturale, 
\[
  \forall L \in \mathcal{L} \quad \alpha_L := p_L \circ i : X \to I{(L)}
\]
\end{theorem}

\begin{proof}[idea di dimostrazione]
    \(\alpha\) è naturale perché \(g \circ i = h \circ i\). Più in generale,
    dato \(j : Y \to \prod_{L \in \mathcal{L}} I{(L)}\) in \(\mathcal{C}\), sia
    \(\beta_L := p_L \circ j : Y \to I{(L)}\). Allora \(\beta : K_Y \to I\) è
    una trasformazione naturale se e solo se \(g \circ j = h \circ j\).
    Osservando questo, una qualunque trasformazione naturale \(\beta : K_Y \to I\) è ottenuta da \(j : Y \to \prod_{L \in \mathcal{L}} I {(L)}\) con \(g \circ j = h \circ j\) ponendo \(\beta_L = p_L \circ j\).

    Per definizione di equalizzatore, \(\exists ! \, f \in \mathcal{C}{(Y, X)}\) tale che \(j = i \circ f\) che è equivalente a \(\beta_L = \alpha_L \circ f\) e \(\beta = \alpha \circ K_f\).

    \(\beta\) è naturale se e solo se \(\forall l : L \to L'\) morfismo di
    \(\mathcal{L}\), \(I{(l)} \circ \beta_L = \beta_{L'}\) cioè \(I{(l)} \circ p_L \circ j = p_{L'}  \circ j \iff 
    p_l \circ h \circ j = p_l \circ g \circ j \overset{\text{ propr. univ.
    prodotto}}{\iff} h \circ j = g \circ j \)
\end{proof}

\begin{definition}{}
    Sia \(\mathcal{C}\) una categoria.

    \(\mathcal{C}\) \emph{ha i (co)limiti di forma \(\mathcal{L}\)} se esistono i
    (co)limiti di tutti i funtori \(\mathcal{L} \to \mathcal{C}\).

    \(\mathcal{C}\) è \emph{(co)completa} se ha i \emph{(co)limiti} di forma \(\mathcal{L}\) per ogni categoria piccola \(\mathcal{L}\).

    \(\mathcal{C}\) è \emph{finitamente (co)completa} se ho i (co)limiti di
    forma \(\mathcal{L}\) per ogni categoria finita \(\mathcal{L}\) (ossia se \(\mathrm{Mor} \mathcal{L}\) è finito)
\end{definition}

\begin{corollary}{}
    \(\mathcal{C}\) categoria è (finitamente) (co)completa se e solo se ha (co)equalizzatori e
    (co)prodotti (finiti).
\end{corollary}

\begin{example}{}
    \(\mathtt{Set}\) è completa e cocompleta (anche \(\mathtt{Grp}\), \(\mathtt{Rng}\) e \(A\mathtt{-Mod}\))
\end{example}

\begin{definition}{}
    Sia \(F : \mathcal{C} \to \mathcal{D}\) funtore. \(F\) preserva un
    (co)limite \({(X, \alpha)}\) di \(I : \mathcal{L} \to \mathcal{C}\) se \({(F{(X)}, F \circ \alpha\footnote{Composizione orizzontale: \(\alpha: K_X \to I \implies F \circ \alpha : F \circ K_X = K_{F{(X)}} \to F \circ I \)})}\) è un (co)limite di \(F \circ I : \mathcal{L} \to \mathcal{D}\)
\end{definition}
\begin{remark}{}
    Se \(F\) preserva un limite di \(I\), allora \(F\) preserva tutti i
    limiti di \(I\).
\end{remark}

\(F\) preserva i (co)limiti (finiti) se preserva i (co)limiti di forma \(\mathcal{L}\) per ogni \(\mathcal{L}\) categoria piccola (finita)

\begin{corollary}{}
    Sia \(\mathcal{C}\) una categoria (finitamente) (co)completa.
    Allora \(F\) preserva i (co)limiti (finiti) se e solo se \(F\) preserva i
    (co)prodotti (finiti) e i (co)equalizzatori.
\end{corollary}

\begin{example}{}
    I funtori dimenticanti \(\mathtt{Grp} , \mathtt{Rng} , A\mathtt{-Mod} \to \mathtt{Set}\) preservano i limiti ma non i colimiti.
\end{example}

\begin{example}{}
    \(\forall X \in \mathcal{C}\) i funtori \(\mathcal{C}{(X, -)} : \mathcal{C} \to \mathtt{Set}\) e \(\mathcal{C}{(-, X)} : C^{op} \to \mathtt{Set}\) preservano i limiti, ma in generale non i colimiti.

    Analogamente se \(\mathcal{A}\) è una categoria preadditiva, allora \(\forall X \in \mathcal{A}\) i funtori \\ \(\mathcal{A}{(X, \--)} : \mathcal{A} \to \mathtt{Ab}\) e \(\mathcal{A}{(-, X)}
    : \mathcal{A}^{op} \to \mathtt{Ab}\) preservano i limiti, ma in generale
    non i colimiti.

\begin{proof}[controesempio colimiti]
    Se \(\mathcal{C} = \mathtt{Set}\) e \(\# X \ge 2\), allora \(\mathcal{C}{(X, -)}\) non preserva i coprodotti. % TODO: eser
\end{proof}

\begin{proof}[\(\mathcal{C}{(X, -)}\) preserva i limiti]
    Infatti sia \(I : \mathcal{L} \to \mathcal{C}\) funtore con limite \({(Y, \alpha)}\). Allora abbiamo che \(\forall L \in \mathcal{L}\) \(\alpha_L \in \mathcal{C}{(Y, I{(L)})}\) tale che \(\forall l : L \to L'\) morfismo di \(\mathcal{L}\), \(I{(l)} \circ \alpha_L = \alpha_L'\). Ma allora
    \[
      {(\alpha_L)}_{L \in \mathcal{L}} \in \mathrm{eq}{\left( \prod_{L \in \mathcal{L}} C{(Y, I{(L)})} \underset{h}{\overset{g}{\rightrightarrows}} \prod_{{(l : L \to L')} \in \mathrm{Mor}{(\mathcal{L})}} C{(Y, I{(L')})}) \right)}
    \]
    % TODO : aggiustare e finire
\end{proof}
\end{example}

\begin{remark}{}
    Sia \(F, F' : \mathcal{C} \to \mathcal{D}\) con \(F \cong F'\). Se \(F\)
    preserva un limite \({(X, \alpha)}\) di \(I : \mathcal{L} \to
    \mathcal{C}\), allora anche \(F'\) lo preserva. 

    Infatti \({(F{(X)}, F \circ \alpha)}\) è un limite di \(F \circ I\) e \(\gamma : F \to F'\) è
    isomorfismo, dunque \(\gamma \circ I : F \circ I \to  F' \circ I\) è
    isomorfismo, dunque \({(F{(X)}, {(\gamma \circ I)} \circ {(F \circ \alpha)})}\) è un limite di \(F' \circ I\).
    Segue dunque che \({(F'{(X)}, {(\gamma \circ I)} \circ {(F \circ \alpha)} \circ K_{\gamma_X^{-1}} = F' \circ \alpha)}\) dove l'ultima uguaglianza è perché \(\alpha\) è naturale.
\end{remark}

\begin{remark}{}
    Vedremo che ogni equivalenza di categorie preserva sia i limiti che i
    colimiti.

    Sgue che se \(\mathcal{C}\) e \(\mathcal{D}\) sono equivalenti e \(\mathcal{C}\) ha i limiti di forma \(\mathcal{L}\), lo stesso vale per \(\mathcal{D}\). Infatti sia \(I : \mathcal{L} \to \mathcal{D}\) funtore e sia \(F : \mathcal{C} \to \mathcal{D}\) equivalenza con quasi-inverso \(G\). 
    Allora \(G \circ I : \mathcal{L} \to \mathcal{C}\) ha limite \({(X, \alpha)}\) e dunque anche \(F \circ G \circ I \cong I\) ha limite 
\end{remark}

Siano \(\mathcal{C}\) e \(\mathcal{D}\) categorie (con \(\mathcal{C}\) piccola)
e sia \(X \in \mathcal{C}\). Allora c'è un funtore di valutazione in \(X\) 
\begin{align*}
    \mathrm{ev}_X: \mathtt{Fun}{(\mathcal{C}, \mathcal{D})} &\longrightarrow
    \mathcal{D} \\
    F &\longmapsto \mathrm{ev}_X(F) = F{(X)} \\
    {(\alpha : F \to G)} &\longmapsto {(\alpha_X : F{(X)} \to G{(X)})} \\
\end{align*}
E \(\forall f : X \to Y\), \(\mathrm{ev}\) induce una trasformazione naturale
\(\mathrm{ev}_f : \mathrm{ev}_X \to \mathrm{ev}_Y\) definita da 
\({(\mathrm{ev_f})}_X : \mathrm{ev}_X{(F)} = F{(X)} \overset{F{(f)}}{\to } \mathrm{ev}_Y{(F)} = F{(Y)}\) 

\begin{proposition}{}
    Se in \(\mathcal{D}\) esistono i limiti di forma \(\mathcal{L}\), allora
    esistono anche in \(\mathtt{Fun}{(\mathcal{C}, \mathcal{D})}\) e si
    calcolano ``puntualmente'', cioè, dato \(I: \mathcal{L} \to \mathtt{Fun}{(\mathcal{C}, \mathcal{D})}\)
    si ha che 
    \[
      {( \lim I)}{(X)} =  \lim {(\mathrm{ev}_X \circ I)} % NOTE: canonaco qui
                                            % aveva usato := invece di =
    \]
    e \(\forall f : X \to Y\) morfismo di \(\mathcal{C}\) 
    \[
      {( \lim I)}{(f)} =  \lim {(\mathrm{ev}_f \circ I)}
    \]
    con trasformazioni naturali \(\alpha{(X)} : K_{ \lim {(\mathrm{ev}_X \circ I)}} \to \mathrm{ev}_X \circ I\) e \(\forall L \in \mathcal{L}\) \(\alpha{(X)}_L :  \lim {(\mathrm{ev}_X \circ I)} \to {(\mathrm{ev}_X \circ I)}{(L)} = I{(L)}{(X)}\).

    La trasformazione naturale \(\alpha: K_{ \lim I} \to I\) è definita \(\forall L \in \mathcal{L}\) da 
    \[
      a_L :  \lim I \to I{(L)} \text{ morfismo in \(\mathtt{Fun}{(\mathcal{C}, \mathcal{D})}\) }
    \]
    cioè una trasformazione naturale definita \(\forall X \in \mathcal{C}\) da
    \[
        \underbrace{{(\alpha_L)}_X}_{\alpha{(X)}_L} :  \underbrace{{( \lim I)}{(X)}}_{ \lim {(\mathrm{ev}_X \circ I)}} \to I{(L)}{(X)}
    \]
\end{proposition}

\subsection{Limiti in categorie preadditivive}

\begin{definition}{(co)Nucleo}
    Sia \(\mathcal{A}\) una categoria preadditiva. Un (co)nucleo di \(f: X \to Y\) morfismo di \(\mathcal{A}\) è un (co)equalizzatore di \(X \underset{0}{\overset{f}{\rightrightarrows}} Y\)
\end{definition}
\begin{remark}{}
    Il (co)equalizzzatore di \(X \underset{g}{\overset{f}{\rightrightarrows}} Y \) è (co)nucleo di \(f-g\) 
\end{remark}
Un nucleo di \(f : X \to Y\) è dunque \( i : K \to X\) tale che \(f \circ i = 0\) 
e \(\forall i', K' \to X\) tale che \(f \circ i' = 0\), allora 
\(\exists ! \, g : K' \to K\) tale che \(i' = i \circ g\). Equivalentemente il
seguente diagramma commuta:

\[\begin{tikzcd}
	K & {i'} & Y \\
	& {K'}
	\arrow["i", from=1-1, to=1-2]
	\arrow["0", curve={height=-18pt}, from=1-1, to=1-3]
	\arrow["f", from=1-2, to=1-3]
	\arrow["{\exists! \,g}", dashed, from=2-2, to=1-1]
	\arrow[from=2-2, to=1-2]
	\arrow["0"', from=2-2, to=1-3]
\end{tikzcd}\]

Un conucleo di \(f\) è \(p : Y \to C\) tale che \(p \circ f = 0\) e \(\forall p' : Y\to C'\) tale che
\(p' \circ f = 0\) allora \(\exists ! \,h : C \to C' \) tale che \(p ' = h \circ p\).
Equivalentemente il seguente diagramma commuta:

\[\begin{tikzcd}
	X & Y & C \\
	& {K'}
	\arrow["f", from=1-1, to=1-2]
	\arrow["0", curve={height=-18pt}, from=1-1, to=1-3]
	\arrow["0"', from=1-1, to=2-2]
	\arrow["p", from=1-2, to=1-3]
	\arrow["{p'}"', from=1-2, to=2-2]
	\arrow["{\exists!\,h}", dashed, from=1-3, to=2-2]
\end{tikzcd}\]

\begin{example}{}
    In \(A\mathtt{-Mod}\) un nucleo di \(f: M \to N\) è dato dall'inclusione \(\mathrm{Ker}f \hookrightarrow M\).

    Un conucleo di \(f\) è dato invece dalla proiezione a quoziente 
    \[
      \pi : N \to \mathrm{coker}f := \frac{N}{\mathrm{Im}f}
    \]
    effettivamente 
\[\begin{tikzcd}
	M & N & \frac{N}{\mathrm{Im}f} \\
	& {P}
	\arrow["f", from=1-1, to=1-2]
	\arrow["0", curve={height=-18pt}, from=1-1, to=1-3]
	\arrow["0"', from=1-1, to=2-2]
	\arrow["\pi", from=1-2, to=1-3]
	\arrow["{g}"', from=1-2, to=2-2]
	\arrow["{\exists!\,g'}", dashed, from=1-3, to=2-2]
\end{tikzcd}\]
    rappresenta il teorema di omomorfismo per moduli.
\end{example}
\begin{remark}{}
    In quanto limiti, nuclei e conuclei sono unici a meno di unico isomorfismo
\end{remark}

\begin{definition}{Categoria preabeliana}
    Una categoria \emph{preabeliana} e una categoria additiva in cui esistono
    nuclei e conuclei.
\end{definition}
\begin{remark}{}
    Una categoria è preabeliana se e solo se è preadditiva e finitamente
    completa e cocompleta.
\end{remark}
\begin{remark}{}
    Se \(\mathcal{A}\) è preabeliana, allora \(\mathcal{A}^{op}\) è preabeliana.
    \(\mathcal{A}' \subseteq \mathcal{A}\) sottocategoria piena è preabeliana se
    è chiusa per prodotti e coprodotti finiti, per nuclei e per conuclei (cioè
    \(f: X \to Y\) in \(\mathcal{A}'\) \(\implies \) \(\exists i: K \to X\)
    nucleo di \(f\) in \(\mathcal{A}\) con \(K \in \mathcal{A}'\)).

    Ad esempio, se \(\mathcal{A} = A\mathtt{-Mod}\) e \(\mathcal{A}'\) con
    oggetti i moduli noeheriani/artiniani/coerenti, allora \(\mathcal{A}'\) è
    preabeliana.
\end{remark}


\begin{example}{}
    Sia \(\mathcal{A}\) una sottocategoria piena di \(A\mathtt{-Mod}\) con
    oggetti i moduli \emph{finitamente generati}/\emph{finitamente presentati}
    (è sempre additiva).

    Allora \(\mathcal{A}\) è preabeliana se e solo se \(A\) è
    \emph{noetheriano}/\emph{coerente}.

    Infatti \(M\) \(A\)-modulo è \emph{f.g.}/\emph{f.p.} se e solo se \(M\) è \emph{noeth.}/\emph{coerente}.
    Viceversa \(\mathcal{A}\) è in ogni caso chiusa per conuclei in \(A\mathtt{-Mod}\), ma in generale non per nuclei. Sia \(A\) non \emph{noeth.}/\emph{coerente}. Allora in entrambi i casi \(\exists f : M \to N\) in \(\mathcal{A}\) tale che \(\mathrm{Ker}f\) non è f.g.
    Infatti nel primo caso si può prendere \(f = \pi : A \to A/\mathfrak{I}\) con \(\mathfrak{I}\)
    ideale sinistro non f.g., mentre nel secondo caso \(\exists  \mathfrak{I} \subseteq A \) ideale sinistro f.f.g. non f.p. se prendo \(f : A^{n} \to A\) con \(\mathrm{im}f = \mathfrak{I}\).

    Per assurdo sia ora \(g : K \to M\) un nucleo di \(f\) in \(\mathcal{A}\)
    (dunque \(f \circ g = 0\)). In ogni caso \(K\) è f.g. e dunque \(\mathrm{im}g\) è f.g. 
    \(\implies  x \in \mathrm{Ker}f \sminus \mathrm{im}g\).

    Sia \(h : A \to M\), \(a \mapsto ax\) in \(\mathcal{A}\) tale che \(f \circ h = 0\), ma \(\not
    \exists  h' : A \to K\) tale che \(h = g \circ h'\), e il che contraddice la
    proprietà universale.
\end{example}
\begin{example}{}
    Sia \(A\) un dominio d'integrità, \(\mathcal{A} \subseteq A\mathtt{-Mod} \)
    sottocategoria piena con oggetti i moduli senza torsione, ossia tali che
    \[
      T{(M)} := \{x \in M : \mathrm{Ann}{(x)} \neq 0\} = 0 \quad (T{(M)} \subseteq M \text{ sottomodulo})
    \]
    Allora \(\mathcal{A}\) è chiusa per nuclei (\(M \in \mathcal{A}, M' \subseteq M \) sottomodulo, allora \(M' \in \mathcal{A}\)),
    ma non è chiusa per conuclei se \(A\) non è un campo (esiste infatti \(0 \neq I \neq A\)) ideale
    \(I, A \in \mathcal{A}\) ma \(\mathrm{coker}(I \hookrightarrow A) = A/I
    \not\in A\).

    \(\mathcal{A}\) è comunque preabeliana, infatti \(\forall f : M \to N\) in
    \(\mathcal{A}\), un conucleo di \(f\) in \(\mathcal{A}\) è 
    \[
      N \overset{\pi}{\to } \mathrm{coker}f \overset{\pi'}{\to } C := \frac{\mathrm{coker}f}{T{(\mathrm{coker}f)}}
    \]
    Allora \(\pi' \circ \pi \circ f = 0\) evidentemente. Inoltre vale la
    proprietà universale, infatti
\[\begin{tikzcd}
	M & N & {\mathrm{coker}f} & C \\
	& {P\in\mathcal{A}}
	\arrow["f", from=1-1, to=1-2]
	\arrow["0"', from=1-1, to=2-2]
	\arrow["\pi", from=1-2, to=1-3]
	\arrow["g"', from=1-2, to=2-2]
	\arrow["{\pi'}", from=1-3, to=1-4]
	\arrow["{\exists!\,g'}"', dashed, from=1-3, to=2-2]
	\arrow["{\exists!\,g''}", curve={height=-12pt}, dotted, from=1-4, to=2-2]
\end{tikzcd}\]
    dove la freccia tratteggiata è dovuta alla proprietà universale del conucleo
    in \(A\mathtt{-Mod}\) mentre quella puntinata (ossia quella voluta in \(\mathcal{A}\)) è dovuta al fatto, più generale, che dato \(g' : L \to P\) in \(A\mathtt{-Mod}\) con \(P \in \mathcal{A}\) e \(\pi' : L \to L/T{(L)}\) 
    \[
      \exists ! \,g'' : \frac{L}{T{(L)}} \to P \text{ t.c. } g' = g'' \circ \pi'
    \]
    poiché \(g'{(T{(L)})} \subseteq T{(P)} = 0\) e dunque si può applicare il
    teorema di omomorfismo.
\end{example}

\begin{note}{zione}
    In una categoria preabeliana, \(\forall f : X \to Y\), indico con 
    \begin{align*}
      &\mathrm{ker}f : \mathrm{Ker} f \to X \text{ un nucleo di \(f\)} \\
      &\mathrm{coker}f : Y \to \mathrm{Coker}f \text{ un conucleo di \(f\)}
    \end{align*}
\end{note}

\begin{proposition}{}
    Sia \(C\) una categoria piccola, \(\mathcal{A}\) una categoria preabeliana,
    allora \(\mathtt{Fun}{(\mathcal{C}, \mathcal{A})}\) è preabeliana. Se \(\mathcal{C}\) è preadditiva, allora \(\mathtt{AddFun}{(\mathcal{C}, \mathcal{A})}\) (che è additiva) è chiusa per conuclei e nuclei in \(\mathtt{Fun}{(\mathcal{C}, \mathcal{A})}\) e quindi è preabeliana.
\end{proposition}
\begin{proof}{}
    Sia \(\alpha : F \to G\) un morfismo in \(\mathtt{AddFun}{(\mathcal{C}, \mathcal{A})}\). Devo dimostrare che \(\mathrm{Ker}\alpha\), \(\mathrm{Coker}\alpha\) (in \(\mathtt{Fun}{(\mathcal{C}, \mathcal{A})}\)) sono additivi. Effettivamente
    \[
      \forall X \in \mathcal{C} \quad \mathrm{Ker}(\alpha) {(X)} = \mathrm{Ker}{(\alpha_X)}
    \]
    e \(\forall f : X \to Y\) morfismo di \(C'\), \(\mathrm{Ker}{(\alpha)}{(f)}\) è l'unico morfismo 
    \(f'\) tale che il seguente diagramma commuta:
\[\begin{tikzcd}
	{\mathrm{Ker}(\alpha_X)} & {F(X)} & {G(X)} \\
	{\mathrm{Ker}(\alpha_X)} & {F(Y)} & {G(Y)}
	\arrow["{\mathrm{ker}(\alpha_X)}", from=1-1, to=1-2]
	\arrow["{f'}", dashed, from=1-1, to=2-1]
	\arrow["{\alpha_X}", from=1-2, to=1-3]
	\arrow["{F(f)}", from=1-2, to=2-2]
	\arrow["{G(f)}", from=1-3, to=2-3]
	\arrow["{\mathrm{ker}(\alpha_Y)}"', from=2-1, to=2-2]
	\arrow["{\alpha_Y}"', from=2-2, to=2-3]
\end{tikzcd}\]
    Analogamente dato \(g: X \to Y\), allora \(\mathrm{Ker}{(\alpha)}{(g)} = g'\) tale che l'analogo diagramma commuta. Dunque abbiamo
    \[
      F{(f + g)} = F{(f)} + F{(g)} \,\,,\,\, G{(f + g)} = G{(f)} + G{(g)} \implies
      f' + g' = {(f + g)}'
    \]
\end{proof}

\begin{proposition}{}
    Sia \(\mathcal{C}\) una categoria. Allora ogni
    \emph{equalizzatore}/\emph{coequalizzatore} in \(\mathcal{C}\) è un
    \emph{mono}/\emph{epi}morfismo in \(\mathcal{C}\).
\end{proposition}
\begin{proof}{}
    Sia \(f : X \to Y\) un equalizzatore di \(Y \underset{h}{\overset{g}{\rightrightarrows}} Z\). Dati \(l, l' : W \to X\) tali che \(f \circ l = f \circ l'\). Devo dimostrare che \(l = l'\). 
    Effettivamente
    \[
      g \circ {(f \circ l)} = h \circ {(f \circ l)} \implies \exists ! \, \overline{l} : W \to X \text{ t.c. } f \circ l = f \circ \overline{l} \implies l = \overline{l} = l'
    \]
\[\begin{tikzcd}
	X & Y & Z \\
	W
	\arrow["f", from=1-1, to=1-2]
	\arrow["g"', shift right, from=1-2, to=1-3]
	\arrow["h", shift left, from=1-2, to=1-3]
	\arrow["{\exists!\,\overline{l}}", dashed, from=2-1, to=1-1]
	\arrow["{f\circ l}"', from=2-1, to=1-2]
\end{tikzcd}\]
    in particolare in una categoria preadditiva i \textbf{nuclei} sono \textbf{mono} e i \textbf{conuclei} sono \textbf{epi}.
\end{proof}
\begin{remark}{}
    In \(\mathcal{A}\) categoria preadditiva, \(f : X \to Y\) è mono \(\iff\)
    dato \(g : X' \to X\) t.c. \(f \circ g = 0\), allora \(g = 0\). Analogamente
    per gli epimorfismi.
\end{remark}

\begin{definition}{Categoria abeliana}
    Una categoria \emph{abeliana} è una categoria preabeliana in cui ogni
    monomorfismo è un nucleo e ogni epimorfismo è un conucleo
\end{definition}
\begin{example}{}
    \(A\mathtt{-Modl}\) è abeliana \(\forall A\) anello.

    Infatti se \(f : M \to N\) in \(A\mathtt{-Mod}\) è \emph{mono}/\emph{epi} se
    e solo se \(f\) è \emph{iniettivo}/\emph{suriettivo}.

    Se \(f\) è iniettivo, allora \(\mathrm{im}f \cong M\) e l'inclusione \(\mathrm{im}f \overset{i}{\hookrightarrow } N\) è un nucleo di \(\mathrm{coker}f : N \to \mathrm{Coker}f\). 
    Dunque anche \(f\) è un nucleo di \(\mathrm{coker}f\) poiché \(f = i \circ f'\) con \(f' : M \to \mathrm{im}f\) isomorfismo.

    Se \(f\) è suriettivo, per il primo teorema di isomorfismo \(N \cong M / \mathrm{Ker}f\) e sia 
    \(\overline{f}\) l'isomorfismo. Allora
\[\begin{tikzcd}
	{\mathrm{Ker}f} & M & N \\
	&& {\mathrm{Coker}(j)}
	\arrow["{j:=\mathrm{ker}f}", from=1-1, to=1-2]
	\arrow["f", from=1-2, to=1-3]
	\arrow["{\mathrm{coker}j}"', from=1-2, to=2-3]
	\arrow["{\exists!\,\overline{f}}"', dashed, from=2-3, to=1-3]
\end{tikzcd}\]
    commuta e dunque \(f\) è un conucleo di \(j\).
\end{example}

\begin{remark}{}
    Se \(\mathcal{A}\) è abeliana, anche \(\mathcal{A}^{op}\) è abeliana.
\end{remark}

\begin{proposition}{}
    Sia \(\mathcal{A}\) una categoria preadditiva con oggetto nullo \(0\).
    Allora un morfismo \(f : X \to Y\) di \(\mathcal{A}\) è
    \emph{mono}/\emph{epi}morfismo \(\iff\) \emph{\(\mathrm{Ker}f =
    0\)}/\emph{\(\mathrm{Coker}f = 0\)}
\end{proposition}
\begin{proof}\( \)
\begin{itemize}
    \item[\(\implies \)] Devo dimostrare che \(0 \overset{0}{\hookrightarrow }X \) è un nucleo di \(f\).
        Se \(f \circ 0 = 0\), dato \(g : Z \to X\) tale che \(f \circ g = 0\),
        allora \(g = 0\) (perché \(f\) è mono) e dunque fattorizza in modo unico
        attraverso \(0\).
    \item[\(\impliedby \)] Se \(\mathrm{Ker}f = 0\) dato \(g : Z \to X\) tale
        che \(f \circ g = 0\), allora per definizione di nucleo, \(g\) fattorizza in modo unico attraverso \(0 \to X\), dunque \(g = 0\) 
\end{itemize}
\end{proof}

\begin{definition}{Immagine e coimmagine}
    Un'\emph{immagine} di \(f : X \to Y\) in una categoria preabeliana è un
    nucleo di un conucleo di \(f\). Una \emph{coimmagine} di \(f\) è un conucleo
    di un nucleo di \(f\). Diagrammaticamente abbiamo
\[\begin{tikzcd}
	\mathrm{Ker}f & X & Y & \mathrm{Coker}f \\
	& \mathrm{Coim}f & \mathrm{Im}f
	\arrow["{\mathrm{ker}f}", from=1-1, to=1-2]
	\arrow["f", from=1-2, to=1-3]
	\arrow["{\mathrm{coker}f}", from=1-3, to=1-4]
\arrow["{\mathrm{coker}{(\mathrm{ker}f)} = \mathrm{coim}f}"', from=1-2, to=2-2]
\arrow["{\mathrm{im}f = \mathrm{ker}{(\mathrm{coker}f)}}"', from=2-3, to=1-3]
\end{tikzcd}\]
\end{definition}
\begin{note}[zione]
    Se \(f\) è nucleo di \(g\) si scrive \(f = \mathrm{ker}g\) 
\end{note}
\begin{proposition}{}
    Sia \(f\) un (co)nucleo in una categoria preabeliana. Allora \[f = (co)im(f)\]
\end{proposition}

\begin{proof}{}
    Sia \(f :X \implies Y\) un nucleo di \(g : Y \to Z\). Voglio dimostrare che
    il seguente diagramma commuta:
    
\[\begin{tikzcd}
	X & Y & Z \\
	& {Y'} & {\mathrm{Coker}f}
	\arrow["f", from=1-1, to=1-2]
	\arrow["g", from=1-2, to=1-3]
	\arrow["{\mathrm{coker}f}"{description}, from=1-2, to=2-3]
	\arrow["{\exists!\,h'}", dashed, from=2-2, to=1-1]
	\arrow["h", from=2-2, to=1-2]
	\arrow["0"', from=2-2, to=2-3]
	\arrow["{\exists!\,g'}"', dashed, from=2-3, to=1-3]
\end{tikzcd}\]
    Ossia \(f\) è nucleo di \(\mathrm{coker}f\). È sempre vero che \(\mathrm{coker}f \circ f = 0\). Dato \(h : Y' \to Y\) tale che \(\mathrm{coker}f \circ h = 0\), devo dire che \(\exists !\, h' : Y'\to X\) tale che \(h = f \circ h'\).

    % TODO : finire
\end{proof}

\begin{corollary}{}
    Sia \(f\) \emph{mono}/\emph{epi} in una categoria abeliana. Allora \(f = \text{(co)}\mathrm{im}{(f)}\).
\end{corollary}


\begin{proposition}{}
    Sia \(\mathcal{A}\) una categoria abeliana. Sia \(\mathcal{A}' \subseteq
    \mathcal{A}\) una sottocategoria piena chiusa per (co)prodotti finiti,
    nuclei e conuclei. Allora \(\mathcal{A}'\) è abeliana.
\end{proposition}
\begin{proof}{}
    \(\mathcal{A}'\) è già preabeliana.

    \(f : X \to Y\) mono di \(\mathcal{A}'\) \(\iff\) \(\mathrm{Ker}f = 0\) in
    \(\mathcal{A}'\) e quindi in \(\mathcal{A}\), ossia \(f\) è mono in \(\mathcal{A}\), ma allora \(f = \mathrm{im}f \) in \(\mathcal{A}\) e quindi in \(\mathcal{A}'\) 
\end{proof}

\begin{example}{}
    In particolare le sottocategoria di \(A\mathtt{-Mod}\) con oggetti i moduli
    \emph{noeth.}/\emph{artin.}/\emph{coerenti} sono abeliane
\end{example}

\begin{example}{}
    Sia \(\mathcal{C}\) piccola, \(\mathcal{A}\) categoria abeliana. Allora \(\mathtt{Fun}{(\mathcal{C}, \mathcal{A})}\) è abeliana.
    \begin{proof}{}
        Sappiamo già che \(\mathtt{Fun}{(\mathcal{C},\mathcal{A})}\) è
        preabeliana.

        ora \(\alpha : F \to G\) è mono in \(\mathtt{Fun}{(\mathcal{C}, \mathcal{A})}\) \(\iff\) \(\mathrm{Ker}\alpha = 0 \iff \mathrm{Ker}{(\alpha_X)} = 0\) per ogni \(X \in \mathcal{C}\), cioè \(\alpha_X\) mono in \(\mathcal{A}\) abeliana e dunque \(\alpha_X = \mathrm{im}{(\alpha_X)} \implies \alpha = \mathrm{im}\alpha\).
    \end{proof}

    Se \(\mathcal{C}\) è preadditiva, allora \(\mathtt{AddFun}{(\mathcal{C},\mathcal{A})}\) è chiusa per 
    nuclei e conuclei in \(\mathtt{Fun}{(\mathcal{C}, \mathcal{A})}\) e dunque
    \(\mathtt{AddFun}{(\mathcal{C},\mathcal{A})}\) è abeliana.
    In particolare \(\mathcal{A}\mathtt{-Mod} = \mathtt{AddFun}{(\mathcal{A}, \mathtt{Ab})}\) è abeliana \(\forall \mathcal{A}\) preadditiva piccola.
\end{example}

\begin{proposition}{}
    Sia \(f: X \to Y\) monomorfismo e epimorfismo in una categoria abeliana.
    Allora \(f \) è isomorfismo.
\end{proposition}
\begin{proof}{}
    \(f\) è mono, dunque \(f = \mathrm{im}f = \mathrm{ker}{(\mathrm{coker}f)}\).
    Poiché \(f\) è epi, \(\mathrm{Coker}f = 0\), dunque un'immagine di \(f\) è
    \(1_Y\), e quindi \(f\) è isomorfismo.
\end{proof}

\begin{example}{}
    Sia \(\mathcal{A}\) sottocategoria piena di \(A\mathtt{-Mod}\) (\(A\)
    dominio d'integrità non campo) dei moduli senza torsione. Allora \(\mathcal{A}\) non è abeliana.

    \begin{proof}{}
        Sia \(0 \neq I \neq A\) ideale. Allora l'inclusione \(i : I \to A\) è
        mono (\(\mathrm{Ker}i = 0\) ), epi (\(\mathrm{Coker}i = {(A/I)}/{(T{(A /I)})} = 0\) ) ma non è isomorfismo.
    \end{proof}
\end{example}

\begin{proposition}{}
    Se \(\mathcal{A}\) e \(\mathcal{B}\) sono due categorie equivalenti, allora
    \(\mathcal{A}\) è abeliana se e solo se \(\mathcal{B}\) è abeliana.
\end{proposition}
\begin{proof}{}
    So già che \(\mathcal{B}\) è preabeliana. Per dualità basta dimostrare che
    se \(f : X \to Y\) è monomorfismo di \(\mathcal{B}\), allora è nucleo.

    Sia \(F : \mathcal{B} \to \mathcal{A}\) un'equivalenza (quindi preserva i
    (co)nuclei). Allora se \(f\) è mono \(\mathrm{Ker}f = 0\), dunque \(\mathrm{Ker}{(F{(f)})} = F{(0)} = 0\) è dunque \(F{(f)}\) è mono.

    Inoltre \(F{(f)} = \mathrm{im}{(F{(f)})} = \mathrm{Ker}{(\mathrm{coker}{(F{(f)})})}\). Ma \(\mathrm{im}{(F{(f)})} = F{(\mathrm{im}{(f)})}\). A meno di isomorfismo, \(F{(f)} = F{(\mathrm{im}{(f)})}\) e dunque \(f = \mathrm{im}{(f)}\) poiché \(F\) è pienamente fedele.
\end{proof}

\begin{note}[zione]
    I mono si possono indicare con \(\rightarrowtail\) e gli epi con \(\) 
\end{note}

\begin{theorem}{}
    Sia \(\mathcal{A}\) una categoria preabeliana. Allora
\begin{enumerate}[label = \arabic*.]
    \item \(\forall f : X \to Y\) di \(\mathcal{A}\), \(\exists !\, s{(f)} : \mathrm{Coim}f \to \mathrm{Im}f\) tale che \(f = \mathrm{im}(f) \circ s{(f)} \circ \mathrm{coim}{(f)}\) 
    \item \(\mathcal{A}\) è abeliana se e solo se \(s{(f)}\) è isomorfismo per
        ogni morfismo \(f\) di \(\mathcal{A}\) 
\end{enumerate}
\end{theorem}
\begin{proof}{}
    \[\begin{tikzcd}
	\mathrm{Ker}f & X & Y & \mathrm{Coker}f \\
	& \mathrm{Coim}f & {\mathrm{Im}f}
	\arrow["\mathrm{ker}f", tail, from=1-1, to=1-2]
	\arrow["f", from=1-2, to=1-3]
	\arrow["{\mathrm{coim}f}"', two heads, from=1-2, to=2-2]
	\arrow["\mathrm{coker}f", two heads, from=1-3, to=1-4]
	\arrow["{f'}", dashed, from=2-2, to=1-3]
	\arrow["{s(f)}"', dashed, from=2-2, to=2-3]
	\arrow["{\mathrm{im}f}"', tail, from=2-3, to=1-3]
    \end{tikzcd}\]

\begin{enumerate}[label = \arabic*.]
    \item \(f \circ \mathrm{ker}f = 0\) e \(\mathrm{coim}f = \mathrm{coker}{(\mathrm{ker}f)}\). Dunque 
        esiste unico \(f' : \mathrm{Coim}f \to Y\) tale che \(f = f' \circ \mathrm{coim}f\).

        \[
          0 = \mathrm{coker}f \circ f = \mathrm{coker}f \circ f' \circ
            \mathrm{coim}f \implies \mathrm{coker}f \circ f' = 0
        \]
        poiché \(\mathrm{coim}\) è epi.

        Inoltre 
        \[
        \mathrm{im}f = \mathrm{ker}{(\mathrm{coker}f)} \implies \exists ! \, s{(f)} : \mathrm{Coim}f \to \mathrm{Im}f : f' = \mathrm{im}f \circ s{(f)}
        \]
        dove \(s{(f)}\) è unico perché \(\mathrm{im}f\) è mono e \(\mathrm{coim}f\) è epi.
    \item 
    \begin{itemize}
        \item[\(\implies \)] Per dualità basta dimostrare che \(f\) mono \(\implies \) \(f\) nucleo
            \[
              \mathrm{coim}f = \mathrm{coker}{(\mathrm{ker}f)} =
              \mathrm{coker}{(0 \to X)} \implies \mathrm{coim}{(f)} \text{ iso}
              \implies f = \mathrm{im}f \circ iso
            \]
            cioè \(f\,``="\, \mathrm{im}f\) è un nucleo
        \item[\(\implidby \)] \(s{(f)}\) è isomorfismo se e solo se \(
            s{(f)}\) è mono e epi. Per dualità basta dimostrare che \(s{(f)}\) è mono, che è vero se \(f'\) è mono. Dunque dimostreremo che \(f'\) è mono.

            Sia \(g : Z \to \mathrm{Coim}f\) tale che \(f' \circ g = 0\), devo
            dimostrare che \(g=0\).

            \[
\begin{tikzcd}
	{\mathrm{Ker}f} & X & Y & {\mathrm{Coker}f} \\
	W & {\mathrm{Coim}f} & {\mathrm{Im}f} \\
	& Z & {\mathrm{Coker}h} & {\mathrm{Coker}g}
	\arrow["{\mathrm{ker}f}", tail, from=1-1, to=1-2]
	\arrow["f", from=1-2, to=1-3]
	\arrow["{{\mathrm{coim}f}}"', two heads, from=1-2, to=2-2]
	\arrow["{\mathrm{coker}f}", two heads, from=1-3, to=1-4]
	\arrow["{h'}", dashed, from=2-1, to=1-1]
	\arrow["h", from=2-1, to=2-2]
	\arrow["{{f'}}", dashed, from=2-2, to=1-3]
	\arrow["{{s(f)}}", dashed, from=2-2, to=2-3]
	\arrow["{\mathrm{coker}g}"{description}, shift left, from=2-2, to=3-4]
	\arrow["{{\mathrm{im}f}}", tail, from=2-3, to=1-3]
	\arrow["g", from=3-2, to=2-2]
	\arrow["q"{description}, dashed, from=3-3, to=2-2]
	\arrow["{=}"{description}, draw=none, from=3-3, to=3-4]
	\arrow["{f''}"', curve={height=6pt}, dashed, from=3-4, to=1-3]
\end{tikzcd}\]
    ma \(f' \circ g = 0 \), quindi \(\exists !\, f'' : \mathrm{Coker}g \to Y\)
    tale che \(f'= f'' \circ \mathrm{coker}g\).

    Inoltre \(\mathrm{coker}g \circ \mathrm{coim}f : X \to \mathrm{Coker}g\)
    epi e \(\mathcal{A}\) è abeliana, dunque
    \[
      \exists h : W \to X : \mathrm{coker}g \circ \mathrm{coim}f = \mathrm{coker}h
    \]
    \(f \circ h = f' \circ \mathrm{coim}f \circ h = f'' \circ \mathrm{coker}g \circ \mathrm{coim}f \circ h = f'' \circ \mathrm{coker}h \circ h = 0\) dunque \(\exists !\, h' : W \to \mathrm{Ker}f\) tale che \(h = \mathrm{Ker}f \circ h'\). 
    \end{itemize}
    %TODO finire
\end{enumerate}
\end{proof}

\begin{corollary}{}
    Sia \(f: X \to Y\) morfismo in una categoria abeliana. Allora \(f\) si
    fattorizza come \(X \overset{e}{\to } I \overset{m}{\to } Y\) con \(e\) epi
    e \(m\) mono. Inoltre tale fattorizzazione è essenzialmente unica, cioè se
    \(X \overset{e'}{\twoheadrightarrow } I' \overset{}{\rightarrowtail} Y\) è un'altra
    fattorizzazione simile, allora \(\exists ! \, t : I \to I'\) (iso)morfismo
    tale che \(e' = t \circ e\) e \(m = m' \circ t\) 
\end{corollary}
\begin{proof}[dimostrazione unicità]
\[\begin{tikzcd}
	X & I & Y \\
	Z & {I'}
	\arrow["e", two heads, from=1-1, to=1-2]
	\arrow["{e'}", two heads, from=1-1, to=2-2]
	\arrow["m", tail, from=1-2, to=1-3]
	\arrow["g", from=2-1, to=1-1]
	\arrow["t"', dashed, from=2-2, to=1-2]
	\arrow["{m'}"', tail, from=2-2, to=1-3]
\end{tikzcd}\]
    Poiché \(e\) è epi, allora \(e = \mathrm{coker}g \) con \(g : Z \to X\).
    Allora \(e \circ g = 0 \implies m \circ e \circ g = m' \circ e' \circ g = 0 \implies e' \circ g = 0\) perché \(m'\) è mono. Poiché \(e = \mathrm{coker}g\), 
    \[
      \exists ! \, t : I \to I' : e' = t \circ e
    \]
    Allora \(m' \circ t \circ e = m' \circ e' = m \circ e\). Inoltre \(e\) è
    epi, dunque \(m' \circ t = m\).

    Verificare come esercizio che \(t\) è iso.
\end{proof}

\begin{definition}{Sottoggetto}
    Un \emph{sottoggetto} di \(X\) è una classe di equivalenza di monomorfismi
    \(Y \overset{i}{\rightarrowtail } X\), dove \({(Y \overset{i}{\rightarrowtail } X)} \sim {(Y' \overset{i'}{\rightarrowtail } X)}\) se \(\exists t : Y \to Y'\) isomorfismo tale che \(i = i' \circ t\), ossia \(\begin{tikzcd}
	Y & X \\
	{Y'}
	\arrow["i", from=1-1, to=1-2]
	\arrow["t"', from=1-1, to=2-1]
	\arrow["{i'}"', from=2-1, to=1-2]
\end{tikzcd}\) 
\end{definition}
\begin{remark}{}
    Sia \(\mathcal{C}_X\) la sottocategoria di \(\mathrm{Mor}{(\mathcal{C})}\) con oggetti i mono
    con codominio \(X\). Sia ora
    \[
      \mathcal{C}'_X{({(Y \overset{i}{\to }X)}, {(Y' \overset{i}{\to }X)})} := \{t \in \mathcal{C}{(Y, Y')} : i' \circ t = i\} 
    \]
    Inoltre \(t\) è tale che \(i' \circ t = i\) è mono, dunque \(t\) è mono.
    Allora \(i'\) è mono, quindi \(t\) è unico, quindi \(\mathcal{C}'_X\) è un
    preordine.

    Una relazione di preordine induce una relazione d'ordine sulle classi di
    isomorfismo degli oggetti. Quindi ottengo una relazione d'ordine sui
    sottoggetti di \(X\) (che indico con \(\subseteq \)), cioè se \(Y, Y'\)
    sottoggetti di \(X\), si scrive \(Y \subseteq Y' \) se \(\exists t \) tale
    che. % TODO non ho capito.
    Dualmente si ottiene la nozione di quoziente di \(
    X\) considerando gli epi \(X \twoheadrightarrow  Y\).
\end{remark}
\begin{example}{}
    Sia \(f : X \to 
    Y\) un monomorfismo in una categoria abeliana. Allora \(f = \mathrm{im}{(f)}\) come sottoggetti di \(X\).
\end{example}
\begin{definition}{Complesso}
    \(X \overset{f}{\to } Y \overset{g}{\to } Z\) in una categoria preadditiva è
    un \emph{complesso} se \(g \circ f = 0\) 
\end{definition}
\begin{remark}{}
    In \(\mathcal{A}\) abeliana,
    \[\begin{tikzcd}
	X & Y & Z \\
	{\mathrm{Im}f} && {\mathrm{Ker}g}
	\arrow["f", from=1-1, to=1-2]
	\arrow["{\mathrm{coim}f}"', two heads, from=1-1, to=2-1]
	\arrow["g", from=1-2, to=1-3]
	\arrow["{\mathrm{im}f}"'{pos=0.7}, from=2-1, to=1-2]
	\arrow["t"', dashed, from=2-1, to=2-3]
	\arrow["{\mathrm{ker}g}"'{pos=0.3}, tail, from=2-3, to=1-2]
\end{tikzcd}\]
    \[
      g \circ f = 0 \iff \exists (!)\, t : \mathrm{Im} f \to \mathrm{Ker}g \text{ t.c. } \mathrm{im}f = \mathrm{ker}g \circ t
    \]

    Infatti \(\impliedby\) è vero perché
    \[
        g \circ f = g \circ \mathrm{im}f \circ \mathrm{coim}f = \underbrace{g \circ \mathrm{ker}g}_0 \circ t \circ \underbrace{\mathrm{coim}f}_0
    \]
    e viceversa
    \[
      0 = g \circ f = g \circ \mathrm{im}f \circ \mathrm{coim}f \,, \mathrm{coim}f \text{ epi} \implies g \circ \mathrm{im}f = 0
    \]
    da cui \(\exists ! \, t : \mathrm{Im}f \to \mathrm{Ker}g\) tale che
    \(\mathrm{im}f = \mathrm{ker}g \circ t\).
\end{remark}

\begin{definition}{}
    \(X \overset{f}{\to } Y \overset{g}{\to } Z\) complesso in una categoria
    abeliana è \emph{esatta} se \(t : \mathrm{Im}f \to \mathrm{Ker}g \) tale che
    \(\mathrm{im}f = \mathrm{ker}g \circ t\) è isomorfismo (ossia \(\mathrm{Im}f = \mathrm{Ker}g\) come sottoggetti di \(Y\))
\end{definition}

\begin{remark}{}
\[\begin{tikzcd}
	X & Y & Z \\
	{\mathrm{Coker}f} && {\mathrm{Im}g}
	\arrow["f", from=1-1, to=1-2]
	\arrow["g", from=1-2, to=1-3]
	\arrow["{\mathrm{coker}f}"'{pos=0.9}, from=1-2, to=2-1]
	\arrow["{\mathrm{coim}g}"'{pos=0.3}, from=1-2, to=2-3]
	\arrow["t"', dashed, from=2-1, to=2-3]
	\arrow["{\mathrm{im}g}"', tail, from=2-3, to=1-3]
\end{tikzcd}\]
   \[
     g \circ f \mathrm{iso} \iff \exists (!) t : \mathrm{coim}g \circ t = \mathrm{coker}f
   \] 
   e la successione è esatta se e solo se \(
   t\) è isomorfismo.
\end{remark}

\begin{proposition}{}
\begin{enumerate}[label = \arabic*.]
    \item \(0 \to X \overset{f}{\to } Y\) è esatta \(\iff\) \(\mathrm{Ker}f = 0\) \(\iff f \) mono.
    \item \(X \overset{f}{->} Y \to 0\) è esatta \(\iff \mathrm{Coker}f = 0 \iff f\) epi
    \item \(0 \to X \overset{f}{\to } Y \overset{g}{\to } Z\) è esatta (in \(X\) e \(Y\)) \(\iff f = \mathrm{ker}g\)
    \item[3'] \(X \overset{f}{\to } Y \overset{g}{\to } Z \to 0\) è esatta \(\iff g = \mathrm{coker}f\)
    \item \(\forall f : X \to Y\) la successione \(0 \to \mathrm{Ker}f \overset{\mathrm{ker}f}{\to } X \overset{f}{\to} Y \overset{\mathrm{coker}f}{\to } \mathrm{Coker}f \to 0\) è esatta. In particolare se \(f\) è mono, allora \(0 \to X \overset{f}{\to } Y \overset{\mathrm{coker}f}{\to } \mathrm{Coker}f \to 0\) è esatta e \(f\) epi, dunque \(0 \to \mathrm{Ker}f \overset{\mathrm{ker}f}{\to } X \to Y \to 0\) è esatta
\end{enumerate}
\end{proposition}

\begin{proof}{}
    A quanto pare l'unica non ovvia è la 3.     
    \begin{itemize}
        \item[\(\implies \)] Se la successione è esatta in \(X\) allora \(f\) è
            mono, dunque \(f = \mathrm{im}f\). Se la successione è esatta in \(Y\) allora \(\mathrm{im}f = \mathrm{ker}g \implies f = \mathrm{ker}g\) 
        \item[\(\impliedby \)] Se \(f = \mathrm{ker}g \implies f\) mono, dunque successione
    esatta in \(X\). Ma quindi \((\mathrm{ker}f) = f = \mathrm{im}f\) e dunque esatta in \(Y\) 
    \end{itemize} 
\end{proof}

\begin{definition}{Successione esatta corta}
    Le successioni esatte della forma \( 0 \to X \to Y \to Z \to 0\) si dicono
    esatte corte.
\end{definition}
\begin{proposition}{}
    Una successione esatta corta \(0 \to X \overset{f}{\to } Y \overset{g}{\to } Z \to 0\) in una
    categoria abeliana si spezza se vale una delle seguenti condizioni
    equivalenti:
\begin{enumerate}[label = \arabic*.]
    \item \(f\) è invertibile a sinistra 
    \item \(g\) è invertibile a destra
    \item \(\exists r : Y \to X\) e \(s : Z \to Y\) tale che \({(Y, f, s, r, g)}\) è un biprodotto di \(X \) e \(Z\) 
\end{enumerate}
\end{proposition}
\begin{proof}[Buona definizione]
    Per dualità basta dimostrare che \(1. \iff 3.\). Inoltre \(
    3. \implies 1.\) è ovvia, dunque dimostriamo solo \(1. \implies 3.\).

    Sia \(h:= 1_Y - f \circ r : Y \to Y\). Allora \(h \circ f = f - f \circ
    \underbrace{r \circ f}_{1_X} = 0\). Allora
    \[g = \mathrm{coker}f \implies \exists !\, s : Z \to Y : h = s \circ g
    \implies 1_Y = s \circ g + f \circ r \]
    Dunque resta da dimostrare che \(g \circ s = 1_Z\). Infatti \(g \circ s \circ g = g \circ h = g - g \circ f \circ r = g = 1_Z \circ g\) e \(g\) è epimorfismo, dunque \(g \circ s = 1_Z\).
\end{proof}

\begin{definition}{Categoria semisemplice}
    Una categoria abeliana \(\mathcal{A}\) è detta \emph{semisemplice} se tutte le
    successioni esatte corte di \(\mathcal{A}\) si spezzano 
\end{definition}

\begin{example}{}
    \(A\mathtt{-Mod}\) è semisemplice se e solo se \(A\) è semisemplice.
\end{example}


