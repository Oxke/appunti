\chapter{Categorie}

\begin{definition}{Categoria}
    Una \textbf{categoria} \(C\) è data da una classe di oggetti \(\mathrm{Ob}{(C)}\) e \(\forall X, Y \in \mathrm{Ob}{(C)}\) da un insieme di morfismi da \(X \) a \(Y\) indicato con
    \(\mathrm{Hom}{(X,Y)} = \mathrm{Hom}_C {(X, Y)} = C{(X, Y)}\)  e da una
   azione  composizione di morfismi, cioè \(\forall X, Y, Z \in \mathrm{Ob}{(C)}\)
    (anche scritto \(X, Y, Z \in C\) ) un'operazione 
    \begin{align*}
        C{(X, Y)} \times  C{(Y, Z)} &\to  C{(X, Z)}
        {(f, g)} &\mapsto g \circ f
    \end{align*}

    tale che 
\begin{enumerate}[label = \arabic*.]
    \item[0.] \(C{(X,Y)} \cap  C{(X', Y')} \neq \varnothing \implies X = X' \text{ e } Y = Y'\) 
    \item[1.] \(\circ\) è associativa, cioè \(\forall X, Y, Z, W \in C\) e \(\forall f \in C{(X,Y)}\) e \(\forall g \in C{(Y, Z)}\) e \(\forall h \in C{(Z, W)}\)  allora 
        \[
          h \circ {(g \circ f)} = {(h \circ g)} \circ f
        \]
    \item[2.] \(\forall  X \in  C \) esiste \(1_X = \mathrm{id}_X \in C{(X, X)}\)
        che è eleemento neutro di \(X\) cioè \(\forall Y \in C\) e \(\forall f
        \in C{(X, Y)}\), 
        \[
          f \circ 1_X = f \quad, \quad 1_Y \circ f = f
        \]
\end{enumerate}
\end{definition}
\begin{example}{}
    La categoria degli insiemi \(\mathtt{Set} \) che ha come oggetti tutti gli
    insiemi e \(\forall X, Y \in \mathtt{Set} \) i morfismi \(\mathtt{Set}{(X, Y)} = \{f : X \to Y\}  \) le funzioni e \(\circ\) la composizione di funzioni
\end{example}
\begin{remark}{}
    Se ho \(C\) tale che valgano solo 1. e 2. e non necessariamente 0. posso
    ottenere la categoria \(C'\) che soddisfa anche 0. ponendo \(\mathrm{Ob}{(C')} := \mathrm{Ob}{(C)}\) e 
    \[
      C'{(X, Y)} := \{X\} \times C{(X,Y)} \times \{Y\} 
    \]
\end{remark}
\begin{example}{}
    Le categorie concrete, in cui gli oggetti sono insiemi con qualche struttura
    e i morfismi sono funzioni tra insiemi che preservano la struttura (con \(\circ\) sempre la composizione di funzioni). In particolare:
\begin{itemize}
    \item La categoria \(\mathtt{Grp}\) dei gruppi, dove gli oggetti sono i
        gruppi e i morfismi gli omomorfismi di gruppi
    \item La categoria \(\mathtt{Rng}\) degli anelli
    \item Dato un anello \(A\), la categoria \(\mathtt{A-Mod}\) / \(\mathtt{Mod-A}\) degli \(A\)-moduli sinistri / destri
    \item Dato un anello commutativo \(A\), la categoria \(\mathtt{A-Alg}\)
        delle \(A\)-algebre
    \item La categoria \(\mathtt{Top}\) degli spazi topologici (con funzioni
        continue come morfismi)
\end{itemize}
\end{example}
\begin{note}{}
    Dato \(f \in C{(X, Y)}\) si può indicare con \(f : X \to Y\) ``come fosse
    una funzione''
\end{note}

\begin{example}{}
    Le categorie discrete, cioè tali che gli unici morfismi sono \(1_X\) per
    ogni \(X \in C\).
\end{example}

\begin{example}{}
    \(C\) tale che \(\forall X, Y \in C\), \(\# C {(X, Y)} = 1\), ottengo
    una relazione \(\preccurlyeq \) su \(\mathrm{Ob}{(C)}\) in cui
    \[
      X \preccurlyeq Y \iff C{(X, Y)} \neq \varnothing
    \]
    e \(\preccurlyeq\)  è riflessivo (perché \(\exists  1_X \in C{(X, X)} \forall X \in C\)) e transitivo, perché \(\exists \circ\). Ne consegue che \(\preccurlyeq\) è un \emph{preordine}

    Viceversa, data una relazione di preordine \(\preccurlyeq\) su un insieme (o
    una classe) \(S\), ottengo una categoria \(C\) con \(\mathrm{Ob}{(C)} := S\) e \(\forall X, Y \in S\), 
    \[
      C{(X, Y)} := \begin{cases}{}
          \{f_{X,Y} \} & \text{ se } X \preccurlyeq Y \\
          \varnothing & \text{ altrimenti}
      \end{cases}
    \]
    con l'unica composizione possibile
\end{example}

\begin{example}[Categoria Vuota]
    Prendendo \(\mathrm{Ob}{(C)} = \varnothing\)
\end{example}
\begin{remark}{}
    \(\forall X \in C\) con \(C\) una categoria, \(\mathrm{End}_C{(X)} := C{(X, X)}\) è un monoide con \(\circ\), ne consegue il prossimo esempio
\end{remark}
\begin{example}[Monoide]
    Una categoria con un solo oggetto è un monoide. Viceversa ogni monoide può
    essere visto come categoria di un solo oggetto.
\end{example}

\begin{example}[Diagrammi]
    Possiamo definire categorie date da diagrammi, in cui si rappresentano i
    morfismi (non l'identità). Ad esempio:
\[\begin{tikzcd}
	\bullet & \bullet & \bullet & \bullet & \bullet & \bullet & \bullet
	\arrow[from=1-1, to=1-2]
	\arrow[shift left, from=1-3, to=1-4]
	\arrow[shift right, from=1-3, to=1-4]
	\arrow[from=1-5, to=1-6]
	\arrow[from=1-7, to=1-6]
\end{tikzcd}\]
    sono tre categorie diverse, rispettivamente con 2, 2, e 3 oggetti
\end{example}

\begin{definition}{Categoria opposta}
    La \textbf{categoria opposta} di \(C\) è denotata \(C^{op}\) ed è definita
    da
    \[
        \mathrm{Ob}{(C^{op})} := \mathrm{Ob}{(C)} \quad C^{op}{(X, Y)} := C{(Y, X)}
    \]
    con composizione in \(\circ^{op}\) data da \(f \circ^{op} g := g \circ f\) 
\end{definition}
\begin{remark}{}
    \[
      {(C^{op})}^{op} = C
    \]
\end{remark}
\begin{example}[Categoria Prodotto]
    Siano \(C_{\lambda} \) per \(\lambda \in \Lambda\) delle categorie. Allora
    la categoria prodotto
    \[
      C := \prod_{\lambda \in \Lambda} C_{\lambda} 
    \]
    è definita da 
    \begin{align*}
      \mathrm{Ob}{(C)} &:= \prod_{\lambda \in \Lambda} \mathrm{Ob}{(C_\lambda)} \\
      C{({(X_{\lambda} )}_{\lambda \in \Lambda}, {(Y_{\lambda} )}_{\lambda \in \Lambda} )} &:= \prod_{\lambda \in \Lambda} C_{\lambda} {(X_{\lambda} , Y_\lambda)} \\
      {(g_{\lambda} )}_{\lambda \in \Lambda} \circ {(f_{\lambda} )}_{\lambda \in \Lambda} &:= {(g_{\lambda} \circ f_{\lambda} )}_{\lambda \in \Lambda} 
    \end{align*}
\end{example}

\begin{example}[Cateogoria Coprodotto]
    La categoria coprodotto
    \[
      C := \coprod_{\lambda \in \Lambda} C_{\lambda} 
    \]
    è definita con \(\mathrm{Ob}{(C)} := \coprod_{\lambda \in \Lambda} \mathrm{Ob}{(C_{\lambda} )}\) l'unione disgiunta. 
    \[
      \forall X, Y \in C \quad C{(X, Y)} := \begin{cases}{}
          C_{\lambda} {(X, Y)} & \text{ se } X, Y \in C_{\lambda} \text{ per
          qualche } \lambda \in \Lambda \\
              \varnothing & \text{ altrimenti}
      \end{cases}
    \]
    con \(\circ\) ovvia.
\end{example}

\begin{definition}{Sottocategoria}
    Sia \(C\) una categoria. Allora una sottocategoria \(C'\) di \(C\) è data
    da una sottoclasse \(\mathrm{Ob}{(C')} \subseteq \mathrm{Ob}{(C)} \) e \(\forall X, Y \in C'\) da un sottoinsieme \(C'{(X, Y)} \subseteq C{(X, Y)} \) tale che \(\circ\) si restringe a \(C'\) e \(1_X \in C'{(X, X)}\) per ogni \(X \in C'\).

    In particolare \(C'\) è una categoria.
\end{definition}

\begin{example}{}
    Se \(C\) è un monoide (cateogoria di un oggetto), allora le sottocategorie
    non vuote di \(C\) sono i sottomonoidi.
\end{example}

\begin{definition}{Sottocategoria Piena}
    Una sottocategoria \(C'\) di \(C\) si dice \textbf{piena} se \(C'{(X, Y)} =
    C{(X, Y)}\) per ogni \(X, Y \in C'\) 
\end{definition}

\begin{remark}{}
    Una sottocategoria piena di \(C\) equivale a dare una sottoclasse di \(\mathrm{Ob}{(C)}\) 
\end{remark}

\begin{example}[Gruppi Abeliani]
    \(\mathtt{Ab} \subseteq \mathtt{Grp} \) sottocategoria piena dei gruppi
    abeliani. Similmente anche \(\mathtt{CRng} \subseteq \mathtt{Rng} \)
    sottocategoria piena degli anelli commutativi.
\end{example}

Oltre alle sotto-strutture sono anche importanti i quozienti, e anche qui
possiamo dare una definizione astratta

\begin{definition}{Congruenza}
    Una congruenza \(\sim \) su una categoria \(C\) è data da una relazione di
    equivalenza \(\sim \) su \(C{(X, Y)}\) \(\forall X, Y \in C\) tale che
    \[
      \forall X, Y, Z \in C, \, \forall f, f' \in C{(X, Y)} \, \forall g,
      g' \in C{(Y, Z)} \quad f \sim f', g \sim g' \implies g \circ f \sim g'
      \circ f'
    \]
    equivalentemente \(g \sim g' \implies g \circ f \sim g' \circ f\) e \(h \circ g \sim h \circ g'\) 
\end{definition}

\begin{definition}{Quoziente}
    Sia \(\sim \) una congruenza su \(C\), allora possiamo definire la categoria
    quoziente \(C / \sim \) definita da
    \[
      \mathrm{Ob}{(C / \sim )} = \mathrm{Ob}{(C)} \quad {(C / \sim )}{(X, Y)} :=
      C{(X, Y)} / \sim \quad \forall X, Y \in C
    \]
    e \(\circ\) è indotta da quella di \(C\), ossia
    \[
      \overline{g} \circ \overline{f} := \overline{g \circ f}
    \]
\end{definition}

\begin{example}[Omotopia]
    Sia \(C = \mathtt{Top}\) e \(\sim_h \) l'omotopia, ossia \(f, g : X \to Y\)
    sono omotope se \(\exists  H : X \times [0, 1] \to Y\) continue tali che 
    \[
      f{(x)} = H{(x, 0)}, \quad g{(x)} = H{(x, 1)} \quad \forall x \in X
    \]
    e si ottiene \(\mathtt{Toph} := \mathtt{Top} / \sim_h \) 
\end{example}

\begin{example}[Gruppo quoziente]
    Sia \(G\) un gruppo (visto come monoide, ossia categoria di un oggetto) e
    sia \(H \vartriangleleft G\) e \(\sim \) su \(G\) data da \(a \sim b \iff aH
    = bH\). Allora \(G/N\) è la categoria quoziente \(G / \sim \).
    Viceversa ogni \(\sim \) congruenza su \(G\) si può scrivere in tal
    modo prendendo \(H = \{a \in G : a \sim 1\} \vartriangleleft G \) (esercizio).
\end{example}

\begin{definition}{morfismo invertibile}
    Sia \(f : X\to Y\) un morfismo in una categoria \(C\). Allora esso è
    invertibile a sinistra (destra) se \(\exists  f' : Y \to X\) tale che
    \(f' \circ f = 1_X\) (\(f \circ f' = 1_Y\)).
\end{definition}
\begin{remark}{}
    \(f\) è invertibile a sinistra (destra) in \(C\), allora \(f\) è invertibile
    a destra (sinistra) in \(C^{op}\) 
\end{remark}

\begin{definition}{Isomorfismo}
    \(f : X \to Y\) è un \textbf{isomorfismo} se \(\exists f' : Y \to X\) tale che \(f'
    \circ f = 1_X\) e \(f \circ f' = 1_Y\) 
\end{definition}
\begin{remark}{}
    \(f\) è isomorfismo se e solo se \(f\) è invertibile a destra e a sinistra. 
\end{remark}
\begin{proof}\( \)
\begin{itemize}
    \item[\(\implies \)] ovvio
    \item[\(\impliedby \)] \(\exists f', f''\) tale che \(f' \circ f = 1_X\)  e
        \(f \circ f'' = 1_Y\), allora
        \[
          f' \circ {( f \circ f'')} = f' = f'' = {(f' \circ f)} \circ f''
        \]
        e dunque \(f\) è invertibile.
\end{itemize}
    In particolare dunque la \(f'\) della definizione di isomorfismo è unica e
    viene denotata \(f^{-1}\) 
\end{proof}

\begin{definition}{}
    Siano \(X, Y \in C\). Allora \(X\) e \(Y\) sono isomorfe (\(X \cong Y\)) se
    esiste un \(f: X \to Y\) isomorfismo.
\end{definition}

\begin{remark}{}
    \(1_X\) è isomorfismo e \(1_X^{-1} = 1_X\). Se \(f\) isomorfismo
    allora \(f^{-1}\) isomorfismo e \({(f^{-1})}^{-1} = f\). Se \(f, g\)
    isomorfismi componibili, allora \(g \circ f\) è isomorfismo e \({(g \circ f)}^{-1} = f^{-1} \circ g^{-1}\) 

    Ne segue che \(\cong\) è una relazione di equivalenza su \(\mathrm{Ob}{(C)}\) 
\end{remark}

\begin{definition}{}
    Un morfismo \(f : X \to Y\) in \(C\) è detto \textbf{monomorfismo} se \(\forall Z \in C\) la funzione 
    \begin{align*}
        f_*: C{(Z, X)} &\longrightarrow C{(Z, Y)} \\
        g &\longmapsto f_*(g) = f \circ g
    \end{align*}
    è iniettiva
\end{definition}
\begin{definition}{Epimorfismo}
    \(f\) è un \textbf{epimorfismo} in \(C\) se è monomorfismo in \(C^{op}\),
    ossia \(\forall Z \in C\) la funzione
    \begin{align*}
        f^*: C{(Y, Z)} &\longrightarrow C{(X, Z)} \\
        g &\longmapsto f^*(g) = g \circ f
    \end{align*}
    è iniettiva.
\end{definition}

\begin{proposition}{}
    \(f\) è invertibile a sinistra (destra), allora \(f\) è monomorfismo
    (epimorfismo)
\end{proposition}
\begin{proof}{}
    Basta dimostrare che se \(f\) è invertibile a sinistra, allora è mono.

    Sappiamo che \(\exists f' : Y\to X\) tale che \(f' \circ f = 1_X\). Dobbiamo
    dimostrare che \(f_*\) è iniettiva. Siano \(g, h \in C{(Z, X)}\) tali che \(f_*{(g)} = f_*{(h)}\). Allora \(f \circ g = f \circ g\), da cui \(f' \circ f \circ g = f' \circ f \circ h\) e dunque \(g = h\) 
\end{proof}

\begin{proposition}{}
    Sia \(C\) concreta. Allora
    \[
      f \text{ invertibile a \emph{sinistra}/\emph{destra}} \implies f \text{ \emph{iniettiva}/\emph{suriettiva}} \implies f \text{ \emph{mono}/\emph{epi}}
    \]
\end{proposition}
\begin{proof}{}
    Non possiamo usare il trick della categoria opposta, perché non è detto che
    \(C^{op}\) sia ancora concreta.

    Sia \(f'\) tale che \(f' \circ f = 1_X\) (\(f \circ f' = 1_Y\)), allora
    chiaramente \(f\) iniettiva (suriettiva) perché le composizioni \(1_X\) e \(1_Y\) sono biunivoche.

    Se \(f\) è iniettiva, allora siano \(g_{1}, g_{2}: Z \to X\). Dunque \(\forall x \in X\) 
    \[
      f{(g_{1}{(x)})} = f{(g_{2}{(x)})} \overset{f \text{ inj.}}{\implies } g_{1}{(x)} = g_{2}{(x)}
    \]
    ossia \(f_*\) è iniettiva, ossia \(f\) è monomorfismo.

    se \(f\) è suriettiva, allora siano \(
    g_{1}, g_{2} : Y \to Z\). Sappiamo che \(\forall y \in Y\) esiste \(x_y \in X\) tale che \(f{(x_y)} = y\). Allora abbiamo che, assumendo che \(g_{1}\circ f = g_{2}\circ f\) 
    \[
        g_{1}{(y)} = g_{1}{(f{(x_y)})} = g_{2}{(f{(x_y)})} = g_{2}{(y)} \quad \forall y \in Y
    \]
    ossia \(f^{*}\) è iniettiva e dunque \(f\) è epimorfismo
\end{proof}

In generale non vale nessuna delle \(\impliedby\).
\begin{example}{}
    In \(\mathtt{Set}\) se \(f : X \to Y\) è suriettiva, allora \(f\) è invertibile a
    sinistra. Infatti basta scegliere (AOC) \(f'{(y)} \in f^{-1}\{y\} \) per
    ogni \(y \in Y\). Inoltre se \(X \neq \varnothing\) e \(f : X \to Y\) è
    iniettiva, allora \(f\) è invertibile a sinistra.
\end{example}

\begin{eser}{}
    In \(\mathtt{A-Mod}\), mostrare che \(f : M\to N\) iniettiva è invertibile a
    sinistra se e solo se \(\mathrm{Im}{(f)} \subseteq N \) è addendo diretto.

    Mostrare che \(f : M \to N\) suriettiva è invertibile a destra se e solo se
    \(\mathrm{Ker}{(f)} \subseteq M \) è addendo diretto

    Concludere che valgono sempre entrambe le implicazioni se e solo se \(A\) è
    semisemplice.
\end{eser}

\begin{example}{}
    In \(\mathtt{Set}\), se \(f\) è mono (epi), allora \(f\) è iniettiva
    (suriettiva).

    Infatti, poniamo per assurdo \(f : X \to Y\) non iniettiva, dunque siano \(x, y \in X\) tali che \(f{(x)} = f{(y)}\). Allora preso \(Z = \{z\} \) e \(g, h : Z\to X\) tali che \(g
    {(z)} = x\) e \(h{(z)} = y\) abbiamo che \(f \circ g = f \circ h\) da cui \(g = h\) e dunque \(x = y\) 

    Supponiamo \(f\) non suriettiva, mostrare \emph{per esercizio} \(\exists g, h : Y \to Z\) tali che
    \(g \neq h\) ma \(g \circ f = h \circ f\) 
\end{example}

\begin{example}{}
    In \(\mathtt{A-Mod}\) \(f : M \to N\) è mono (epi), allora \(f\) è iniettiva (suriettiva).

    Infatti \(i : \mathrm{Ker}f \to M\) inclusione tale che \(f \circ i = 0\) e
    anche \(0 : \mathrm{Ker}f \to M\) è tale che \(f \circ 0 = 0\). Concludiamo
    che \(i = 0\) e dunque \(\mathrm{Ker}f = 0\).

    Similmente \(\pi : N \to \mathrm{coKer}f\) è tale che \(\pi \circ f = 0\) e
    se \(f\) è epi allora \(0 = \pi \) e dunque \(\mathrm{coKer} f= 0\) e dunque
    \(f\) è suriettiva.
\end{example}

\begin{example}{}
    In \(\mathtt{Grp}\) \(f\) mono (epi), allora \(f\) iniettiva (suriettiva)

    Per mono \(\implies \) iniettiva si può usare la stessa dei moduli, mentre
    per l'altra è un po' più complicato, ma si dimostra che è vero lo stesso
\end{example}

\begin{example}{}
    In \(\mathtt{Rng}\) \(f : A\to B\) mono, allora \(f\) iniettiva. % TODO eser

    Tuttavia \(f\) epi \textbf{non implica} f suriettiva. Ad esempio preso \(i :
    \mathbb{Z} \hookrightarrow \mathbb{Q}\) è epi, infatti \(\forall A\) anello
    esiste al più un omomorfismo \(\mathbb{Q} \to A\) (\(f : \mathbb{Q}\to A\)
    sia omomorfismo, allora \(f|_{\mathbb{Z}} \) è l'unico omomorfismo e \(f{(\frac{a}{b})} = f{(a)}f{(b)}^{-1}\)). Chiaramente però non è suriettiva.
\end{example}

\begin{definition}{Funtore}
    Un funtore \(F : C \to D\) tra 2 categorie è dato da una funzione
    \(F : \mathrm{Ob}{(C)} \to \mathrm{Ob}{(D)}\) e \(\forall X, X' \in C\) una
    funzione \(F = F_{X, X'} : C{(X, X')} \to D{(F{(X)}, F{(X')})}\) tale che
    \[
      F{(g\circ f)} = F{(g)}\circ F{(f)} 
    \]
    (se \(f\) e \(g\) sono componibili in \(C\)) e \(F{(1_X)} = 1_{F{(X)}} \)
    per ogni \(X \in C\) 
    
\end{definition}
\begin{proposition}{}
    Sia \(F\) un funtore e \(f\) invertibile a sinistra (destra). Allora \(F{(f)}\) è invertibile a sinistra (destra)
\end{proposition}
\begin{proof}{}
    \(\exists f'\) tale che \(f' \circ f = 1_X\), allora \(F{(f')} \circ F{(f)} = F{(f'\circ f)} = F{(1_X)} = 1_{F{(X)}} \).
\end{proof}
\begin{remark}{}
    Segue che \(f\) iso, allora \(F{(f)}\) iso e \(F{(f)}^{-1} = F{(f^{-1})}\) 
\end{remark}

\begin{example}{}
    Sia \(C' \subseteq C \) sottocategoria. Allora \(C' \to C\), \(X \mapsto X\) e \(f \mapsto f\) è un funtore
\end{example}

\begin{example}{}
    Se \(\sim \) è una congruenza, allora \(C \to C /\sim\) è un funtore, con
    \(X \mapsto X\) e \(f \mapsto \overline{f}\) 
\end{example}

\begin{example}[Funtore dimenticante]
    \(C \to \mathtt{Set}\) con \(C\) categoria discreta e \(X \mapsto X\), \(f
    \mapsto f\) è un funtore, che ``dimentica'' la struttura aggiunta.

    Analogamente anche \(\mathtt{Rng} \to \mathtt{Ab}\), con \({(A, +, \cdot )} \to {(A, +)}\) è un funtore dimenticante.
\end{example}
\begin{remark}{}
    Notare che il secondo funtore dimenticante non preserva gli epimorfismi.
    Sarebbe infatti \(i : \mathbb{Z} \to \mathbb{Q}\) l'inclusione è
    un'epimorfismo in \(\mathtt{Rng}\) ma non in \(\mathtt{Ab}\) 
\end{remark}


\begin{example}{}
    Sia \(A \to B\) un omomorfismo di anelli. Allora la restrizione degli
    scalare è un funtore \(\mathtt{B-Mod} \to \mathtt{A-Mod}\) 
\end{example}


\begin{example}{}
    Funtore tra 2 categorie discrete \(C\) e \(D\) è una funzione \(\mathrm{Ob}{(C)} \to \mathrm{Ob}{(D)}\) 
\end{example}

\begin{example}{}
    Un funtore tra 2 preordini \(C\) e \(D\) è una funzione \(\mathrm{Ob}{(C)} \to \mathrm{Ob}{(D)}\)  che preserva la relazione di preordine.
\end{example}

\begin{example}{}
    Un funtore tra 2 monoidi è un omomorfismo di monoidi.

    Più in generale dato \(G\)  un monoide e una categoria \(C\) , un funtore \(G \to C\) è dato da \(X \in C\) e da un omomorfismo di monoidi \(G \to \mathrm{End}_C{(X)}\) 

    Se \(G\) è un gruppo un funtore \(G \to C\) è dato da \(X \in C\) e un
    omomorfismo di gruppi \(G \to \mathrm{Aut}_C{(X)}\). Ad esempio se \(C = \mathtt{Set}\) il funtore dà un'azione di un gruppo su un insieme.
    Similmente se \(C = \mathbb{K}\)-spazi vettoriali ho una rappresentazione di
    \(G\).
\end{example}

\begin{example}[Funtore costante]
    Date \(C, D\) categorie preso \(Y \subseteq D \) si può considerare il
    funtore costante di valore \(Y\), \(C \to D\), \(X \mapsto Y\) e \(f \mapsto 1_Y\) 
\end{example}

\begin{example}{}
    Presa \(\mathtt{Top}_*\) la categoria degli spazi topologici puntati, il
    gruppo fondamentale 
    \[
        \pi_{1} : \mathtt{Top}_* \to \mathtt{Grp}
    \]
    è un funtore
\end{example}

\begin{example}{}
    \(\forall n \in \mathbb{N}\) ci sono funtori di omologia (singolare)
    \[
      H_{n} : \mathtt{Top} \to \mathtt{Ab}
    \]
\end{example}
\begin{theorem}{Omomorfismo}
    Sia \(\sim \) una congruenza su \(C\) e \(F : C \to D\) un funtore tale che
    se \(f \sim f'\) in \(C\) allora \(F{(f)} = F{(f')}\).
    Allora esiste un unico funtore \(\overline{F} : C /_\sim \to D\) tale
    che \(\overline{F}{(\overline{f})} = F{(f)}\) per ogni \(f\) morfismo di \(C\) 
\end{theorem}
\begin{example}{}
    Negli esempi precedenti se \(f\) e \(f'\) sono omotope, allora \(\pi_{1}{(f)} = \pi_{1}{(f')}\) e \(H_{n}{(f)} = H_{n}{(f')}\), dunque inducono funtori
    \[
      \pi_{1} : \mathtt{Toph}_* \to \mathtt{Grp} \quad H_{n} : \mathtt{Toph} \to \mathtt{Ab}
    \]
\end{example}

\begin{note}{}
    I funtori che abbiamo definito si dicono anche funtori covarianti
\end{note}
\begin{definition}{funtore controvariante}
    Un funtore \textbf{controvariante} \(C \to D\) è un funtore (covariante) \(C^{op} \to D\).
\end{definition}
\begin{example}{}
    \(\forall n \in \mathbb{N}\) i funtori di coomologia (singolare) sono funtori
    controvarianti \(H^{n} : \mathtt{Top(h)}^{op} \to \mathtt{Ab}\) 
\end{example}
\begin{example}{}
    Sia \(C\) una categoria, \(X \in C\)
    \begin{align*}
        C{(X, \--)} : C &\to \mathtt{Set} \\
        Y \mapsto C{(X, Y)} \quad {(f : Y \to Y')} &\mapsto {(f_{*} : C{(X, Y)} \to C{(X, Y')})} \\
            & \quad \quad \quad \quad\quad\quad\quad  g \mapsto f \circ g
    \end{align*}
    è un funtore perché \({(f' \circ f)}_* = f'_* \circ f_*\) 

    Analogamente 
    \begin{align*}
        C{(\--, Y)} : C^{op} &\to \mathtt{Set} \\
        X \mapsto C{(X, Y)} \quad {(f : X \to X')} &\mapsto {(f^{*} : C{(X', Y)} \to C{(X, Y)})} \\
            & \quad \quad \quad \quad\quad\quad\quad  g \mapsto g \circ f
    \end{align*}
\end{example}
\begin{remark}{}
    C'è anche un funtore 
    \begin{align*}
        C{(\--, =)} : C^{op} \times C &\to \mathtt{Set} \\
        {(X, Y)} &\mapsto C{(X, Y)} \\
        {(f : X \to X', g : Y \to Y')} &\mapsto {(f^* : C{(X', Y)} \to  C{(X,Y)}, g_* : C{(X, Y)} \to C{(X, Y')})}
    \end{align*}
\end{remark}
\begin{example}{}
    Per ogni gruppo \(G\), preso il sottogruppo dei commutatori \([G, G]\),
    allora per ogni \(f : G \to H\) omomorfismo di gruppi, \(f{([G, G])} \subseteq [H, H] \) 
    quindi si ottiene un funtore
    \begin{align*}{}
        \mathtt{Grp} &\to \mathtt{Grp} \\
        G &\mapsto [G, G] \\
        {(f : G \to H)} &\to {(f |_{[G, G]} : [G, G] \to [H, H] )}
    \end{align*}
    e anche 
    \begin{align*}
        \mathrm{Abel} : \mathtt{Grp} &\to \mathtt{Ab} \\
        G &\mapsto \frac{G}{[G, G]} \text{ (abelianizzato di \(G\) ) } \\
        {(f : G \to H)} &\mapsto {\left(\overline{f} : \frac{G}{[G,G]} \to \frac{H}{[H,H]}\right)}
    \end{align*}
\[\begin{tikzcd}
	{G} & {H} \\
    {\frac{G}{[G,G]}} & {\frac{H}{[H,H]}}
	\arrow["f", from=1-1, to=1-2]
	\arrow["{p}", from=1-1, to=2-1]
	\arrow["{q}", from=1-2, to=2-2]
	\arrow["\overline{f}"', from=2-1, to=2-2]
\end{tikzcd}\]
\end{example}

\begin{eser}{}
    Indicando con \(Z{(X)}\) il centro di \(X\),

\begin{enumerate}[label = \alph*.]
    \item Mostrare che non esiste un funtore \(F : \mathtt{Rng} \to \mathtt{Rng}\) tale
    che \(\forall A \in \mathtt{Rng}\) \(F{(A)} = Z{(A)}\).
    \item Mostrare che non esiste un funtore \(F : \mathtt{Grp} \to \mathtt{Ab}\) tale
    che \(\forall G \in \mathtt{Grp}\) \(F{(G)} = Z{(G)}\).
\end{enumerate}
    \tcblower
    Supponiamo l'esistenza di \(F\).
\begin{enumerate}[label = \alph*.]
    \item Se prendo \(i : \mathbb{C} \hookrightarrow \mathbb{H}\),
        allora \(F{(\mathbb{C})} = \mathbb{C} \) e \(F{(\mathbb{H})} = \mathbb{R}\). A tal punto però \(
        F{(i)} : \mathbb{C} \to \mathbb{R}\) che non esiste perché altrimenti \[-1 = F{(i)}{(-1)} = F{(i)}{(i^2)} = F{(i)}{(i)}^2\]
    \item Consideriamo
        \[
          \{(1), (12)\} \overset{i}{\hookrightarrow } S_{3} \overset{\varepsilon}{\to } \{\pm 1\} 
        \]
        Allora \(\varepsilon \circ i = \mathrm{Id}_{C_{2}} \). Allora avremmo
        \[
          0_{\mathrm{End}{(C_{2})}}  = F{(\varepsilon)} \circ F{(i)} = F{(\varepsilon \circ i)} = F{(\mathrm{id}_{C_{2}})} = \mathrm{id}_{C_{2}}
        \]
\end{enumerate}

\end{eser}

    L'identità
    \[
      \mathrm{id}_C : C \to C \quad X \mapsto X \quad f \mapsto f
    \]
    è un funtore
    Si possono comporre i funtori. Dati ad esempio 
    \[
      C \overset{F}{\to } D \overset{G}{\to } E
    \]
    funtori, possiamo definire \(G \circ F : C \to E\) come \(X \mapsto G{(F{(X)})}\)  e \(f \mapsto G{(F{(f)})}\) è un funtore.

    La composizione è associativa  e \(F \circ \mathrm{id}_C = F = \mathrm{id}_C \circ F\) 

In tal modo otteniamo una categoria \(\mathtt{Cat}\) delle categorie (piccole\footnote{si potrebbe anche fare di tutte le categorie, ma per motivi
insiemistici/logici dovremmo introdurre gli universi di Grothendieck e fare le
cose per bene. Al fine di evitare questo inutile sforzo, ci limitiamo a
considerare le categorie piccole.})

\begin{definition}{}
    Un funtore \(F : C \to D\) è un isomorfismo se lo è in \(\mathtt{Cat}\),
    cioè se \(\exists G : D \to C\)  funtore tale che \(G \circ F = \mathrm{id}_C = F \circ G\) 
\end{definition}

\begin{definition}{iniettivo e suriettivo}
    Un funtore \(F : C \to D\) è \emph{iniettivo}/\emph{suriettivo} se \(F : \mathrm{Ob}{(C)} \to \mathrm{Ob}{(D)}\) è \emph{iniettivo}/\emph{suriettivo}.

    Nel caso in cui \(F\) sia sia iniettivo che suriettivo, è \textbf{biunivoco}.
\end{definition}

\begin{definition}{Fedele e pieno}
    \(F\) è detto \textbf{fedele} (\textbf{pieno}) se \(\forall X, Y \in C\), \(F : C{(X, Y)} \to D{(F{(X)}, F{(Y)})}\) è iniettivo (suriettivo).

    Nel caso in cui \(F\) sia sia fedele che pieno, si dice che è
    \textbf{pienamente fedele}
\end{definition}
\begin{eser}{}
    \(F\) funtore è isomorfismo se e solo se \(F\) è pienamente fedele e
    biunivoco.
\end{eser}


\begin{example}{}
    Se \(C' \subseteq C \) è una sottocategoria, allora il funtore di inclusione
    \(i : C' \to C\) è iniettivo e fedele ed è pieno se e solo se \(C' \subseteq C\) è piena.

    Ad esempio se \(\sim \) è una congruenza in \(C\) , allora il funtore
    quoziente \(C \to C / \sim \) è biunivoco e pieno.
\end{example}

\begin{example}{}
    Un omomorfismo di monoidi (categorie di un oggetto) è iniettivo (suriettivo)
    se e solo se come funtore è fedele (pieno). In ogni caso è biunivoco.
\end{example}

\begin{example}{}
    I funtori dimenticanti \(\mathtt{\mathbb{Z}-Mod} \to \mathtt{Ab}\) e \(\mathtt{\mathbb{Z}-Alg} \to \mathtt{Rng}\) sono isomorfismi.
\end{example}

\begin{example}{}
    Anche \(\mathtt{Mod-A}\cong \mathtt{A^{op}-Mod}\) ed esiste un isomorfismo
    (anche se non sono categorie piccole).
\end{example}

\begin{definition}{}
    Un funtore \(F: C \to D\) è \textbf{essenzialmente
    \emph{iniettivo}/\emph{suriettivo}} se la funzione ridotta 
    \[
        \mathrm{Ob}{(C)}/_{\cong} \to \mathrm{Ob}{(D)}/_{\cong}
    \]
    è \emph{iniettivo}/\emph{suriettivo}
\end{definition}

\begin{remark}{}
    Se \(F\) è suriettivo allora \(F\) è essenzialmente suriettivo. L'altra
    implicazione non vale. Ad esempio
\[\begin{tikzcd}
	{(\bullet)} & {(\bullet} & {\bullet)}
	\arrow[from=1-1, to=1-2]
	\arrow[shift left, from=1-2, to=1-3]
	\arrow[shift left, from=1-3, to=1-2]
\end{tikzcd}\]
    Nessuna delle implicazioni tra iniettiva e essenzialmente iniettiva è vera.
    Basti considerare
\[\begin{tikzcd}
	{(\bullet)} & {(\bullet} & {\bullet)}
	\arrow[from=1-2, to=1-1]
	\arrow[shift left, from=1-2, to=1-3]
	\arrow[shift left, from=1-3, to=1-2]
\end{tikzcd}\]
    per essenzialmente iniettiva \(\not\implies\) iniettiva
    e
\[\begin{tikzcd}
	{(\bullet} & {\bullet)} & {(\bullet} & {\bullet)}
	\arrow[from=1-2, to=1-3]
	\arrow[shift left, from=1-3, to=1-4]
	\arrow[shift left, from=1-4, to=1-3]
\end{tikzcd}\]
    per iniettiva \(\not\implies \) essenzialmente iniettiva.
\end{remark}

\begin{proposition}{}
    Sia \(F: C \to D\) un funtore pienamente fedele. Allora \(F\) è
    essenzialmente iniettivo
\end{proposition}
\begin{proof}{}
    Siano \(X, Y \in C\) tali che \(F{(X)} \cong F{(Y)}\) in \(D\). Devo
    dimostrare che \(X \cong Y\) in \(C\).

    Sappiamo che esiste \(g : F{(X)} \to F{(Y)}\) isomorfismo in \(D\). Poiché
    \(F\) è pieno esiste \(f \in C{(X, Y)}\) tale che \(F{(f)} = g\).
    Analogamente \(\exists f' \in C{(Y, X)}\) tale che \(F{(f')} = g^{-1}\).
    \[
      F{(f' \circ f)} = F{(f')} \circ F{(f)} = g^{-1} \circ g = 1_{F{(X)} = F{(1_X)}} 
    \]
    Se \(F\) è fedele, allora \(f' \circ f = 1_X\)  e analogamente \(f \circ f' = 1_Y\) da cui \(f\) è isomorfismo e duque \(X \cong Y\) 
\end{proof}

\begin{definition}{Trasformazione naturale}
    Siano \(F, F' : C \to D\) funtori.

    Una \textbf{trasformazione naturale} \(\alpha : F \to F'\) (si può anche
    scrivere \(\alpha : F \implies F'\)) è il dato di un morfismo
    \[
      \alpha_X : F{(X)} \to F'{(X)} \text{ in \(D\) } \forall X \in C
    \]
    tale che \(\forall f : X \to Y\) morfismo di \(C\) il diagramma
\[\begin{tikzcd}
	{F(X)} & {F(Y)} \\
	{F'(X)} & {F'(Y)}
	\arrow["{F(f)}", from=1-1, to=1-2]
	\arrow["{\alpha_X}", from=1-1, to=2-1]
	\arrow["{\alpha_Y}", from=1-2, to=2-2]
	\arrow["{F'(f)}"', from=2-1, to=2-2]
\end{tikzcd}\]
    commuta in \(D\), cioè \(\alpha_Y \circ F{(f)} = F'{(f)} \circ \alpha_X\) 
\end{definition}

\begin{example}{}
    Consideriamo i due funtori \(\mathrm{Abel} : \mathtt{Grp} \to \mathtt{Grp}\) e \(\mathrm{id} : \mathtt{Grp} \to \mathtt{Grp}\). 
    C'è una trasformazione naturale \(\alpha : \mathrm{id} \to \mathrm{Abel}\)
    definita per ogni \(G \in \mathtt{Grp}\) da 
    \begin{align*}
        \alpha_G: G &\longrightarrow \frac{G}{[G, G]} \\
        a &\longmapsto \alpha_G(a) = a[G, G]
    \end{align*}
    è naturale perché \(\forall f : G \to H\) in \(\mathtt{Grp}\) il diagramma
\[\begin{tikzcd}
	{G} & {H} \\
    {\frac{G}{[G,G]}} & {\frac{H}{[H,H]}}
	\arrow["f", from=1-1, to=1-2]
	\arrow["{\alpha_G}", from=1-1, to=2-1]
	\arrow["{\alpha_H}", from=1-2, to=2-2]
	\arrow["\overline{f}"', from=2-1, to=2-2]
\end{tikzcd}\]
\end{example}

\begin{example}{}
    Supponendo di avere \(F, F' : G \to \mathtt{Set}\) funtori (\(G\) gruppo
    visto come categoria con un oggetto), cioè \(G\)-insiemi (azioni di \(G\) su
    insiemi).
    Allora una trasformazione naturale \(\alpha : F \to F'\) è un morfismo di \(G\)-insiemi cioè una
    funzione \(\alpha : F{(G)} \to F'{(G)}\) tale che \(\alpha{(gx)} = g \alpha{(x)}\) per ogni \(g \in G\) e per ogni \(x \in F{(G)}\).
\end{example}
\begin{remark}{}
    \(\forall F : C\to D\), \(\mathrm{id}_F : F \to F\) data da \({(\mathrm{id}_F)}_X = \mathrm{id}_{F{(X)}} \) per ogni \(X \in C\) è una trasformazione naturale.
\end{remark}

\begin{eser}{}
    Dati \(F, F', F'' : C \to D\) funtori, \(\alpha : F \to F'\) e \(\beta : F'
    \to F''\) trasformazioni naturali, allora la composizione \(\beta \circ \alpha : F \to F''\) è definita da
    \[
      \beta_X \circ \alpha_X =: {(\beta \circ \alpha)}_X : F{(X)} \to F''{(X)}
    \]
    Mostrare che \(\alpha \circ \beta\) è una trasformazione naturale
\end{eser}

La composizione dell'esercizio precedente è anche detta composizione verticale
di trasformazioni naturali, per via di questo disegno esplicativo:
\[\begin{tikzcd}
	C & D
	\arrow[""{name=0, anchor=center, inner sep=0}, "F", curve={height=-20pt}, from=1-1, to=1-2]
	\arrow[""{name=1, anchor=center, inner sep=0}, "{F''}"', curve={height=20pt}, from=1-1, to=1-2]
	\arrow[""{name=2, anchor=center, inner sep=0}, "F"{description}, from=1-1, to=1-2]
	\arrow["\alpha", between={0.2}{0.8}, Rightarrow, from=0, to=2]
	\arrow["\beta", between={0.2}{0.8}, Rightarrow, from=2, to=1]
\end{tikzcd}\]

Considerando funtori e trasformazioni naturali, si ottiene (assumiamo sempre \(C\) piccola) la categoria \(\mathtt{Fun}{(C, D)}\) (anche denotata \(D^{C}\) ) con oggetti i funtori \(C \to D\), morfismi le trasformazioni naturali e composizione la composizione verticale.

\begin{definition}{}
    Data una categoria \(C\), la categoria dei morfismi di \(C\) è
    \[
      \mathtt{Mor}{(C)} := \mathtt{Fun}{(\cdot \to \cdot , C)}
    \]
    che ha come oggetti esattamente \(\{f : \to X -Y : f \text{ morfismo di }C\} \) e trasformazioni naturali date da \({(X \overset{f}{\to } Y )} \to {(X' \overset{f'}{\to } Y' )}\) è data da \({(g : X \to  X' , h : Y \to  Y')}\) tale che 
\[\begin{tikzcd}
	{X} & {Y} \\
    {X'} & {Y'}
	\arrow["f", from=1-1, to=1-2]
	\arrow["g", from=1-1, to=2-1]
	\arrow["h", from=1-2, to=2-2]
	\arrow["f'", from=2-1, to=2-2]
\end{tikzcd}\]
\end{definition}

\begin{definition}{}
    Date \(F, G : C \to D\) funtori, \(\alpha : F \to G\) trasformazione
    naturale, allora \(\alpha\) è \emph{isomorfismo} (\emph{naturale} o di funtori) se è isomorfismo in \(\mathtt{Fun}{(C, D)}\) cioè se 
    \(\exists \beta : G \to F\) trasformazione naturale tale che \(\beta \circ
    \alpha = \mathrm{id}_F\), \(\alpha \circ \beta = \mathrm{id}_G\).

    In tal caso \(F\) e \(G\) si dicono \emph{isomorfi} (denotato \(F \cong
    G\)).
\end{definition}
\begin{remark}{}
    \(\cong\) di funtori è una relazione di equivalenza 
\end{remark}
\begin{example}{}

Il primo gruppo di omologia si può vedere come l'abelianizzato del gruppo
fondamentale. In linguaggio categorico abbiamo 
\begin{align*}
  \mathtt{Top_*} \overset{\pi_{1}}{\to } &\mathtt{Grp} \overset{\mathrm{Abel}}{\to } \mathtt{Ab} \\ 
  &\text{ e } \\
  \mathrm{Top_*} \to &\mathrm{Top} \overset{H_{1}}{\to } \mathtt{Ab} \\
  {(X, x_{0})} \mapsto &\text{ comp. c.p.a. a \(x_{0}\)}
\end{align*}
sono funtori isomorfi
\end{example}

\begin{remark}{}
    \(F \cong F'\) allora \(F\) e \(F'\) inducono la stessa funzione \(\mathrm{Ob}{(C)} /_{\cong} \to \mathrm{Ob}{(D)}/_{\cong}\) quindi \(F\) è essenzialmente 
   \emph{iniettiva} / \emph{suriettiva} se e solo se \(F'\) lo è.
\end{remark}

\begin{eser}{}
    Mostrare che non necessariamente la precedente osservazione vale per le
    proprietà di iniettività e suriettività.
\end{eser}

\begin{proposition}{}
    Se \(F \cong F'\) allora \(F\) è \emph{fedele}/\emph{pieno} se e solo se \(F'\) è \emph{fedele}/\emph{pieno}.
\end{proposition}
\begin{proof}{}
    Sia \(\alpha : F \to F'\) l'isomorfismo. Definiamo \\\(\overline{\alpha} = {(g \mapsto \alpha_Y \circ F{(F)} \circ \alpha_X^{-1})}\) 
    Per ogni \(X, Y \in C\),
\[\begin{tikzcd}
	& {D(F(X),F(Y))} \\
	{C(X,Y)} \\
	& {D(F'(X),F'(Y))} \\
	\arrow["{\overline{\alpha}}", from=1-2, to=3-2]
	\arrow["F", from=2-1, to=1-2]
	\arrow["{F'}", from=2-1, to=3-2]
\end{tikzcd}\]
% TODO: Finire
\end{proof}

\begin{proposition}{}
\(\alpha, \beta\) trasformazioni naturali inducono una trasformazione naturale
\(\beta * \alpha : G \circ F \to G' \circ F'\) 
\[\begin{tikzcd}
	{G{(F{(X)})}} & {G{(F'{(X)})}} \\
    {G'{(F{(X)})}} & {G'{(F'{(X)})}}
	\arrow["G{(\alpha_X)}", from=1-1, to=1-2]
	\arrow["\beta_{F{(X)}}  ", from=1-1, to=2-1]
	\arrow["\beta_{F'{(X)}} ", from=1-2, to=2-2]
	\arrow["G'{(\alpha_X)}", from=2-1, to=2-2]
\end{tikzcd}\]
dunque \((\beta * \alpha)_X := \beta_{F'{(X)}} \circ G{(\alpha_X)} = G'{(\alpha_x)} \circ \beta_{F{(X)}} \).
\end{proposition}
\begin{proof}[Dimostrazione che è una trasformazione naturale]
Vogliamo mostrare che \(b * a\) è naturale, cioè \(\forall f : X \to Y\) in \(C\) il diagramma
\[\begin{tikzcd}
	{G{(F{(X)})}} & {G{(F{(X)})}} \\
    {G'{(F'{(X)})}} & {G'{(F'{(X)})}}
	\arrow["G{(F{(f)})}", from=1-1, to=1-2]
	\arrow["{(\beta * \alpha)}_X", from=1-1, to=2-1]
	\arrow["{(\beta * \alpha)}_Y", from=1-2, to=2-2]
	\arrow["G'{(F'{(f)})}", from=2-1, to=2-2]
\end{tikzcd}\]
commuta. Ma questo è vero perché
\begin{align*}
    G'{(F'{(f)})} \circ {(\beta * \alpha)}_X &= G'{(F'{(f)})} \circ G'{(\alpha_x)}
  \circ \beta_{F{(X)}} = G'{(F'{(f)} \circ \alpha_X)} \circ \beta_{F{(X)}} = \\
    &\overset{\alpha \text{ nat.}}{=} G'{(\alpha_Y \circ F{(f)})} \circ
    \beta_{F{(X)}} = G'{(\alpha_Y)} \circ G'{(F{(f)} \circ \beta_{F{(X)}})} = \\
    &\overset{\beta \text{ nat.}}{=} G'{(\alpha_Y)} \circ \beta_{F{(Y)}}
    \circ G{(F{(f)})} = {(\beta * \alpha)}_Y \circ G{(F{(f)})}
\end{align*}
\end{proof}

Ovviamente è chiaro che si potrebbe definire allora la categoria delle
trasformazioni naturali eccetera e andare avanti all'infinito. Per
assiomatizzare queste cose in realtà bisognerebbe esplicitare che abbiamo
definito le ``2-frecce'' e che quindi siamo in una \emph{2-categoria}

\begin{note}[zione]
    Se \(\beta = \mathrm{id}_G \) invece di \(\mathrm{id_G} * \alpha\) si scrive
    \(G \circ \alpha\) (dunque con \({(G \circ \alpha)}_X = G{(\alpha_X)}\)). Se
    \(\alpha = \mathrm{id}_F\) invece di \(\beta * \mathrm{id}_F\) si scrive \(\beta \circ F\) (con \({(\beta \circ F)}_X = \beta_{F{(X)}} \)). In generale 
    \[
      \beta * \alpha = {(\beta \circ F')} \circ {(G \circ \alpha)} = {(G' \circ
      \alpha)} \circ {(\beta \circ F)}
    \]
\end{note}
\begin{remark}{}
    Se \(\alpha, \beta\) sono isomorfismi, allora \(\beta * \alpha\) è
    isomorfismo. Questo significa che se 
    \[
      F \cong F', \quad G \cong G' \implies G \circ F \cong G' \circ F'
    \]
    cioè l'isomorfismo di funtori è una congruenza su \(
    \mathtt{Cat}\) e quindi si ottiene la categoria \(\mathtt{Cat} /_{\cong}\)
\end{remark}
\begin{definition}{Equivalenza}
    Un funtore \(F: C \to D\) è un'equivalenza se \(\exists G : D \to C\)
    funtore tale che \(G \circ F \cong \mathrm{id}_G\) e \(F \circ G \cong \mathrm{id}_D\).

    Un tale \(G\) si dice un \emph{quasi-inverso} di \(F\).
\end{definition}
\begin{remark}{}
    \(F\) è un'equivalenza se e solo se \(\overline{F} \text{ in } \mathtt{Cat}/_{\cong} \) è un isomorfismo.
\end{remark}

Segue che se \(F \cong F'\), allora \(F\) è un'equivalenza se e solo se \(F'\) è
un'equivalenza e un quasi-inverso di \(F\) è unico a meno di isomorfismo e
l'equivalente di categorie è una relazione di equivalenza su \(\mathtt{Cat}\) 

\begin{definition}{Scheletro}
    Una sottocategoria piena \(C' \subseteq C \) è detta \emph{scheletro} se \(\forall X \in C\), \(\exists ! X' \in C'\) tale che \(X \cong X'\) 
\end{definition}

\begin{lemma}{}\label{lem:ext-scheletro}
    Sia \(F: C\to D\) un funtore, e si supponga che \(\forall X \in C\), \(\alpha_X : F{(X)} \to F'{(X)}\) sia un isomorfismo in \(D\). Allora \(F' : \mathrm{Ob}{(C)} \to \mathrm{Ob}{(D)}\) Si estende in modo unico a un funtore \(F' : C \to D\) tale che \(\alpha : F \to F'\) è isomorfismo.
\end{lemma}

\begin{theorem}{Finalmente un teorema}
    Un funtore \(F : C \to D\) è un'\emph{equivalenza} se e solo se \(F\) è \emph{pienamente fedele} e \emph{essenzialmente suriettivo}
\end{theorem}
\begin{remark}{}
    Non è necessario aggiungere l'ipotesi che \(F\) sia essenzialmente iniettivo
    perché come mostrato prima pienamente fedele implica essenzialmente
    iniettivo (ma non essenzialmente suriettivo).
\end{remark}
\begin{example}{}
    Supponiamo che \(C' \subseteq C \) sia una sottocategoria piena. Allora il
    funtore di inclusione \(C' \hookrightarrow C\) è pienamente fedele ed è
    essenzialmente suriettivo (quindi è un'equivalenza) se e solo se \(\forall X \in C\) esiste \(X' \in C'\) tale che \(X \cong X'\).
\end{example}

\begin{proof}\( \)
\begin{itemize}
    \item[\(\implies \)] Sia \(G : D \to C\) un quasi-inverso di \(F\). Allora
        \(F \circ G \cong \mathrm{id}_G\) che è essenzialmente suriettivo, e
        dunque \(F\) è essenzialmente suriettivo. D'altra parte lo stesso \(F
        \circ G\) è fedele, e dunque \(G\) è fedele.
        
        Ora, per ogni \(X, Y \in C\),
        \[
          C{(X, Y)} \overset{F_{X,Y} }{\to } D{(F{(X)}, F{(Y)})} \overset{G_{F{(X)}, F{(Y)}}  \text{ inj}}{\to} C{(G{(F{(X)})}, G{(F{(Y)})})}
        \]
        poiché la composizione è biunivoca e \(G\) è fedele, allora entrambi
        devono essere biunivoci, ossia in particolare \(F\) è pienamente fedele.
    \item[\(\impliedby \)] Consideriamo prima il caso di un'inclusione \(C' \subseteq C \) sottocategoria piena tale che \(\forall X \in C\) esista \(X' \in C\) tale che \(X \cong X'\). 
        Sia \(I : C' \to C\) il funtore di inclusione (pienamente fedele e
        essenzialmente suriettivo).

        Allora \(\forall X \in C\) scelto (AoC) un
        isomorfismo \(\alpha_X : X \to \tilde{P}{(X)} \in C'\) e se \(X \in 
        C'\) in particolare prendiamo \(\alpha_X = 1_X\). Applico ora il
        lemma~\ref{lem:ext-scheletro} con \(F = \mathrm{id}_C\) e dunque \(\exists !\) estensione di \(\tilde{P}\) a un funtore \(\tilde{P} : C \to C\) tale che \(\alpha : \mathrm{id}_C \to \tilde{P}\) è isomorfismo. Allora \(\exists ! P : C \to C'\) funtore tale che \(\tilde{P} = I \circ P\) e \(P\) è un quasi-inverso di \(I\) poiché \(I \circ P = \tilde{P} \cong \mathrm{id}_C\) e \(P \circ I = \mathrm{id}_{C'}\).

        In generale, dato \(F : C \to D\) pienamente fedele. Siano allora \(
        I : C' \to C\) e \(J : D' \to D\) due scheletri. Per il caso qui fatto
        \(I, J\) sono equivalenze e siano \(P : C\to C'\) quasi-inverso di \(I\) e \(Q : D \to D'\) quasi-inverso di \(J\).
\[\begin{tikzcd}
	C & D \\
	{C'} & {D'}
	\arrow["F"{description}, from=1-1, to=1-2]
	\arrow["P"', shift right, from=1-1, to=2-1]
	\arrow["Q", shift left, from=1-2, to=2-2]
	\arrow["I"', shift right, from=2-1, to=1-1]
	\arrow["{F'}"{description}, from=2-1, to=2-2]
	\arrow["J", shift left, from=2-2, to=1-2]
\end{tikzcd}\]
    Sia \(F' := Q \circ F \circ I : C' \to D'\) come nel diagramma. Allora \(
    I, F, Q\) sono pienamente fedeli e essenzialmente suriettivi (\(I\) per
    definizione, \(F\) per ipotesi e \(Q\) perché è un'equivalenza e vale il
    punto \({(\implies )}\)) dunque \(
    F'\) è pienamente fedele e essenzialmente suriettivo.

    \(F'\) è essenzialmente biunivoco, \(C'\) e \(D'\) sono scheletri, dunque \(F'\) è biunivoco, quindi isomorfismo e quindi equivalenza.
    \[
        F = \mathrm{id}_D \circ F \circ \mathrm{id}_c \cong J \circ Q \circ F
        \circ I \circ P = J \circ F' \circ P
    \]
    equivalenza perché lo sono \(J, F'\) e \(P\) 
\end{itemize}
\end{proof}

\begin{example}{}
    Sia \(\sim \) una relazione d'equivalenza su un insieme \(X\) che vedo come
    categoria \(C\) con \(\mathrm{Ob}{(C)} = X\)  e \(C{(x,y)} \neq \varnothing \iff x \sim y\).

    Il funtore \(C \to X/_\sim\) (categoria discreta) definito da \(x \mapsto
    \overline{x}\) è un'equivalenza poiché pienamente fedele e essenzialmente
    suriettiva.
\end{example}

\begin{eser}{}
    Mostrare che ogni categoria equivalente a una categoria discreta è una relazione di
    equivalenza, ossia una categoria dove \(\forall X, Y \in C, \,\,
    C{(X,Y)} \neq O \iff x \sim y \) per una qualche \(\sim \) relazione di
    equivalenza.
\end{eser}

\section{Categorie preadditive}
\begin{definition}{Categoria preadditiva}
    Una categoria \emph{preadditiva} è una categoria \(\mathcal{A}\) con una
    struttura di gruppo abeliano (notazione: additivo) su \(\mathcal{A}{(X, Y)}\) per ogni \(X, Y \in \mathcal{A}\) ed è tale che la composizione di morfismi sia \(\mathbb{Z}\)-bilineare, ossia
    \[
      g \circ {(f + f')} = g \circ f + g \circ f' \quad \text{ e } \quad {(g +
      g')} \circ f = g \circ f + g' \circ f 
    \]
    per ogni \(
    X, Y, Z \in \mathcal{A}\), \(f, f' \in \mathcal{A}{(X, Y)}\) e \(g, g' \in \mathcal{A}{(Y, Z)}\).
\end{definition}
\begin{remark}{}
    Si dice anche che \(\mathcal{A}\) è una \(\mathtt{Ab}\)-Categoria. Si può
    studiare quando si può generare una categoria simile partendo da altre
    categorie invece di \(\mathtt{Ab}\) ma non è argomento di questo corso.

    Si può anche dire che \(\mathcal{A}\) è \(\mathbb{Z}\)-lineare. Più in
    generale \(\forall\)\footnote{normalmente in mezzo alla frase così avrei scritto esplicitamente ``per ogni'' ma trovavo divertente la quantità di \(\mathcal{A}\) e di \(A\) nella frase quindi ho valutato simpatico aggiungere anche un \(\forall\)} anello commutativo \(A\) una categoria \(A\)-lineare è
    una categoria \(\mathcal{A}\) con una struttura di \(A\)-modulo su \(\mathcal{A}{(X,Y)}\) tale che la composizione sia \(A\)-bilineare.
\end{remark}
\begin{proposition}{}
    Se \(A\) è non commutativo, allora \(\forall a, b \in A\) e \(\forall f :
    X\to Y\) morfismo di \(\mathcal{A}\),
    \[
      {(ab)}f = {(ba)}f
    \]
\end{proposition}
\begin{proof}{}
    \[
      {(ab)}f = a{(bf)} = a{({(bf)}\circ 1_X)} = {(bf)} \circ {(a1_x)} = f \circ
      {(b{(a 1_X)})} = f \circ {(ba)}1_X = {(ba)}f
    \]
\end{proof}

\begin{example}{}
    Sia \(A\) un anello, allora \(A\mathtt{-Mod}\) è preadditiva. Infatti per
    ogni \(M, N \in {(A\mathtt{-Mod})}\), \(A\mathtt{-Mod}{(M, N)} = \mathrm{Hom}_A{(M, N)}\) è un gruppo abeliano e \(\circ\) è \(\mathbb{Z}\)-bilineare. Se \(A\) è commutativo, allora \(A\mathtt{-Mod}\) è anche \(A\)-lineare.
    Più in generale se \(B\) è una \(A\)-algebra allora \(B\mathtt{-Mod}\) è \(A\)-lineare.
\end{example}

\begin{remark}{}
    Sia \(X \in \mathcal{A}\) categoria \(A\)-lineare (quindi \(A\)
    commutativo). Allora \(\mathrm{End}_\mathcal{A}{(X)}\) è una \(A\)-algebra.
    Infatti \({(\mathrm{End}_{\mathcal{A}} {(X)}, \circ)}\) è un monoide e \(\mathrm{End}_\mathcal{A}\) è \(A\)-modulo e \(\circ\) è \(A\)-lineare.

    Quindi le categorie \(A\)-lineari con un solo oggetto sono \(A\)-algebre. In
    particolare \textbf{\emph{le categorie preadditive con un solo oggetto sono anelli}}.
\end{remark}

\begin{remark}{}
Sia \(\mathcal{A}\) preadditiva, allora \(\mathcal{A}^{op}\) è preadditiva con
la stessa struttura di gruppo abeliano su \(\mathcal{A}^{op}{(X, Y)} = \mathcal{A}{(Y, X)}\) per ogni \(X, Y \in \mathcal{A}\).
\end{remark}

\begin{remark}{}
Se \(\mathcal{A}\) è preadditiva, allora \(\mathcal{A}' \subseteq \mathcal{A}\) sottocategoria tale che \(\mathcal{A}'{(X, Y)} < \mathcal{A}{(X,Y)}\) per ogni \(X, Y \in \mathcal{A}'\), allora \(\mathcal{A}'\) è preadditiva. In particolare la condizione è sempre verificata per le categorie piene.
\end{remark}

    Sia \(\mathcal{A}\) preadditiva, \(\sim \) una congruenza tale che \(\forall X, Y \in \mathcal{A}\), \(\forall f, f', g \in \mathcal{A}{(X, Y)}\) allora \(f \sim f' \implies f + g \sim f' + g\). In tale ipotesi \(\mathcal{A}/_\sim \) è preadditiva con \(\overline{f} + \overline{g} = \overline{f+g}\).

    Data una tale congruenza, sia \(\forall X, Y \in \mathcal{A}\) 
    \[
      \mathfrak{I}{(X,Y)} = \{f \in \mathcal{A}{(X, Y)} : f \sim 0\} 
    \]
    e indichiamo con \(\mathfrak{I} \subseteq \mathcal{A}\) la collezione di
    tutti gli \(\mathfrak{I}{(X, Y)}\). Allora vale la proprietà di ideale, gioè
    dati \(f, g\) morfismi di \(\mathcal{A}\) componibili, 
    \[
      f \circ \mathfrak{I} \text{ o } g \in \mathfrak{I} \implies g \circ f \in \mathfrak{I}
    \]
    Se per esempio \(f \in \mathfrak{I}\) ossia \(f \sim 0\), allora \(g \circ f
    \sim g \circ 0 = 0 \) e dunque \(g \circ f \in \mathfrak{I}\).

    Arriviamo dunque alla seguente definizione

\begin{definition}{}
    Definiamo un ideale \(\mathfrak{I}\) in una categoria preadditiva \(\mathcal{A}\) come \(\mathfrak{I}{(X,Y)} < \mathcal{A}{(X,Y)}\) per ogni \(X, Y \in \mathcal{A}\) tale che
    \[
      f \in \mathfrak{I} \text{ o } g \in \mathfrak{I} \implies g \circ f \in
      \mathfrak{I}
    \]
\end{definition}
Viceversa, dato \(\mathfrak{I} \subseteq \mathcal{A} \) ideale, si ottiene una
congruenza \(\sim \) su \(\mathcal{A}\) definito da 
\[
  f \sim f' \iff f' - f \in \mathfrak{I}{(X, Y)} \quad \forall X, Y \in \mathcal{A} \quad \forall f, f' \in  \mathcal{A}{(X, Y)}
\]
ed è tale che \(f \sim f' \implies f + g \sim f'+ g\).

In tali ipotesi si può anche denotare \(\mathcal{A} / \mathfrak{I}\) invece di
\(\mathcal{A}/_\sim\).

Una categoria \(\mathcal{C}\) può non avere nessuna struttura di categoria
    preadditiva (ad esempio se \(\exists X, Y \in \mathcal{C}\)) tale che \(\mathcal{C}{(X, Y)} = \varnothing\) o averne più di una.

\begin{example}{G}
    Possiamo pensare ad anelli \(A\) e \(B\) tali che \({(A, \cdot )} \cong {(B, \cdot )}\) e \({(A, +)} \not\cong {(B, +)}\).

    Ad esempio possiamo prendere \(A = \mathbb{Z} / 4 \mathbb{Z}\) e \(B = \mathbb{Z} /_{2 \mathbb{Z}} [X] / {(X^2)}\). Allora evidentemente 
    \[
      {(A, +)} \cong C_4 \not\cong C_{2} \times C_{2} \cong B
    \]
    ma gli elementi diversi da \(0\) e \(1\) di \(A\) sono \(\overline{2}\) e \(\overline{3}\) e sono tali che \(\overline{2}^2 = \overline{0}\), \(\overline{3}^2 = \overline{1}\) e \(\overline{2} \cdot \overline{3} = \overline{2}\). Similmente in \(B\) abbiamo che
    \(
      \overline{X}^2 = \overline{0}
    \), \(\overline{1+X}^2 = \overline{1}\) e \(\overline{X} \cdot \overline{1+X} = \overline{X}\)
\end{example}

\begin{definition}{}
    Un funtore \(F : \mathcal{A} \to \mathcal{B}\) tra categorie preadditive è
    additivo se 
    \[
      F_{X,Y}  : \mathcal{A}{(X, Y)} \to \mathcal{B}{(F{(X)}, F{(Y)})}
    \]
    è omomorfismo di gruppi \(\forall X, Y \in \mathcal{A}\).

    Più in generale \(F: \mathcal{A} \to \mathcal{B}\) tra categorie
    \(A\)-lineari è detto \(A\)-lineare se \(F_{X, Y} \) è \(A\)-lineare \(\forall X, Y \in \mathcal{A}\).
\end{definition}

\begin{example}{}
    Sia \(\mathcal{A}' \subseteq \mathcal{A} \) sottocategoria tale che \(\mathcal{A}'{(X, Y)} < \mathcal{A}{(X,Y)}\) per ogni \(X,Y \in \mathcal{A}'\). Allora l'inclusione \(\mathcal{A}' \to \mathcal{A}\) è addditivo.
\end{example}

\begin{example}{}
    Se \(\mathcal{A}\) è preadditiva e \(\mathfrak{I} \subseteq \mathcal{A} \)
    ideale, allora il funtore \(f : \mathcal{A} \to \mathcal{A}/\mathfrak{I}\) 
    definito da \(X \mapsto X\) e \(f \mapsto \overline{f}\) è additivo.
\end{example}

\begin{eser}{}
    Sia \(F : \mathcal{A} \to \mathcal{B}\) additivo tale che ``\(\mathfrak{I} = \ker F\)'' cioè \(F{(f)} = 0 \), \(\forall f \in \mathfrak{I}\), allora mostrare che esiste un unico \(\overline{F} : \mathcal{A}/\mathfrak{I} \to \mathcal{B}\) funtore additivo tale che \(F = \overline{F} \circ P\) 
\end{eser}

\begin{example}{}
    Siano \(A, B\) anelli (categorie preadditive con un solo oggetto), allora un
    funtore additivo \(A\to B\) è un omomorfismo di anelli.

    Più in generale per ogni anello \(A\) e per ogni \(\mathcal{A}\) categoria
    preadditiva un funtore additivo \( A \to \mathcal{A}\) è dato da un oggetto
    \(X \in \mathcal{A}\) e un omomorfismo di anelli \(A \to \mathrm{End}_\mathcal{A}{(X)}\).
    Quindi un \(A\)-modulo è un funtore additivo \(A \to \mathtt{Ab}\) 
\end{example}

\begin{example}{}
    Sia \(A\to B\) un omomorfismo di anelli. Allora il funtore di restrizione
    degli scalari \(B\mathtt{-Mod} \to A\mathtt{-Mod}\) è additivo.
\end{example}

\begin{example}{}
    Se \(\mathcal{A}\) preadditiva, allora \(\forall X \in \mathcal{A}\) ci sono
    funtori additivi
    \[
      \mathcal{A}{(X, \--)} : \mathcal{A} \to \mathtt{Ab} \quad , \quad \mathcal{A}{(\-- , X)} : \mathcal{A}^{op} \to \mathtt{Ab}
    \]
    e in generale se \(
    \mathcal{A}\) è \(A\)-lineare, allora i due funtori hanno codominio \(A\mathtt{-Mod}\) 
    e sono \(A\)-lineari
\end{example}

Notare che se \(\mathcal{A} \overset{F}{\to } \mathcal{B} \overset{G}{\to } \mathcal{C}\) 
sono funtori additivi, allora \(G \circ F\) è additivo.
Inoltre \(\mathrm{id}_\mathcal{A}\) è additivo. 
Dunque si ottiene una categoria contenente le \textbf{categorie preadditive}
(piccole) e morfismi i funtori additivi.

    Sia \(\mathcal{C}\) una categoria (piccola) e \(\mathcal{A}\) una categoria
    preadditiva. Allora \(\mathrm{Fun}{(\mathcal{C}, \mathcal{A})}\) è
    preadditiva (in modo naturale) con la seguente struttura

    \(\forall F, G \in \mathtt{Fun}{(\mathcal{C}, \mathcal{A})}\) e \(\forall \alpha, \beta : F \to G\) trasformazioni
    naturali allora \(\alpha + \beta : F \to G\) trasformazione naturale
    definita \(\forall X \in \mathcal{C}\) da
    \[
      {(\alpha + \beta)}_X := \alpha_X + \beta_X : F{(X)} \to G{(X)} \text{ in }
      \mathcal{A}
    \]
    è naturale perché \(
    \forall f : X \to C\), 
\[\begin{tikzcd}
	{F(X)} & {F(Y)} \\
	{G{(X)}} & {G{(Y)}}
	\arrow["{F(f)}", from=1-1, to=1-2]
	\arrow["{\alpha_X + \beta_X}", from=1-1, to=2-1]
	\arrow["{\alpha_Y + \beta_Y}", from=1-2, to=2-2]
	\arrow["{G{(f)}}"', from=2-1, to=2-2]
\end{tikzcd}\]
commuta

Se anche \(\mathcal{C}\) è preadditiva sia 
\[
  \mathtt{AddFun}{(\mathcal{C}, \mathcal{A})}
\]
la sottocategoria piena di \(\mathtt{Fun}{(\mathcal{C}, \mathcal{A})}\) con 
oggetti i funtori additivi. Allora tale categoria è preadditiva.

\begin{example}{}
    Se \(A\) è anello, allora gli \(A\)-moduli sono gli oggetti di \(\mathtt{AddFun}{(A, \mathtt{Ab})}\) (già visto)
    e in effetti \(A\mathtt{-Mod} \cong \mathtt{AddFun}{(A, \mathtt{Ab})}\),
    poiché dati \(M, N : A \to \mathtt{Ab}\) funtori (cioè \(A \to \mathrm{End}{(M)}\) e \(A \to \mathrm{End}{(N)}\) omomorfismi di anelli) la trasformazione naturale \(\alpha : M \to N\) è data da
    \(\alpha : M \to N\) in \(\mathtt{Ab}\) tale che \(\forall a \in A\) 
\[\begin{tikzcd}
	{M} & {N} \\
	{N} & {N}
	\arrow["{a}", from=1-1, to=1-2]
	\arrow["{\alpha}", from=1-1, to=2-1]
	\arrow["{\alpha}", from=1-2, to=2-2]
	\arrow["{a}"', from=2-1, to=2-2]
\end{tikzcd}\]
commuta, cioè \(a \alpha {(x)} = \alpha {(a x)}\) per ogni \(x \in M\), ossia \(
\alpha \) è omomorfismo di \(A\)-moduli.
\end{example}
\begin{remark}{}
    \(\forall \mathcal{A}\) preadditiva (piccola) si può definire la categoria
    (preadditiva) \(\mathcal{A-}\mathtt{Mod} := \mathtt{AddFun}{(\mathcal{A}, \mathtt{Ab})}\).
\end{remark}
\begin{proposition}{}
    Siano \(\mathcal{A}, \mathcal{B}\) preadditive, \(F, G : \mathcal{A} \to
    \mathcal{B}\) funtori tali che \(F \cong G\) e \(F\) additivo. Allora \(G\)
    è additivo
\end{proposition}
\begin{proof}{}
    Sia \(\alpha : F \to G\) isomorfismo. Allora \(\forall  f : X \to Y\) in \(\mathcal{A}\), 
    \(G{(f)} = \alpha_Y \circ F{(f)} \circ \alpha_X^{-1}\).
    Inoltre \(\forall f, f' : X \to Y\),
    \begin{align*}G{(f + f')} &= \alpha_Y \circ F{(f +
    f')} \circ \alpha_X^{-1} = \alpha_Y \circ {(F{(f)} + F{(f')})} \circ \alpha_X^{-1} = \\
    &= \alpha_Y \circ F{(f)} \circ \alpha_X^{-1} + \alpha_Y \circ F{(f')} \circ
    \alpha_X^{-1} = G{(f)} + G{(f')}
\end{align*}
\end{proof}

\begin{remark}{}
    Sia \(F : \mathcal{A} \to \mathcal{B}\) un funtore pienamente fedele con \(\mathcal{B}\) preadditiva.
    Allora esiste un'unica struttura preadditiva su \(\mathcal{A}\) tale che \(F\) 
    sia additivo
\end{remark}
\begin{proof}{}
    \(\forall X, Y \in \mathcal{A}\) voglio che \(F_{X,Y}  : \mathcal{A}{(X, Y)} \to \mathcal{B}{(F{(X)}, F{(Y)})}\) sia isomorfismo di gruppi, che è vero se e solo se \(\forall f, f' \in \mathcal{A}{(X,Y)}\) 
    \[
      f + f' = F^{-1}{(F{(f)} + F{(f')})} 
    \]
    (verifica lasciata in esercizio) % TODO esercizio
\end{proof}

\subsection{Prodotti, coprodotti, proprietà universali}

\begin{definition}{Prodotto}
Sia \(\mathcal{C}\) una categoria, siano \(
X_\lambda \in \mathcal{C}\) con \(\lambda \in \Lambda\) insieme. Un prodotto
degli \(
X_\lambda\) è dato da \(X \in \mathcal{C}\) con morfismi morfismi \(p_\lambda
\in C{(X, X_\lambda)}\) per ogni \(\lambda \in \Lambda\) tale che
vale la seguente proprietà universale:
\[
  \forall Y \in \mathcal{C}, \,\, \forall \lambda \in \Lambda, \,\, \forall f_{\lambda} \in \mathcal{C}{(Y, X_\lambda)} \quad \exists ! f \in \mathcal{C}{(X, Y)} : f_\lambda = p_\lambda \circ f
\]
o equivalentemente il seguente diagramma commuta:
\[\begin{tikzcd}
	Y \\
	{X_\mu} & X
	\arrow["{f_\mu}"', from=1-1, to=2-1]
	\arrow["{\exists ! f}", dashed, from=1-1, to=2-2]
	\arrow["{p_\mu}", from=2-2, to=2-1]
\end{tikzcd}\]
\end{definition}

\begin{proposition}{}
    Siano \({(X, \{p_\lambda\}_{\lambda \in \Lambda} )}\) e \({(X', \{p'_{\lambda} \}_{\lambda \in \Lambda} )}\).
    due prodotti in \(C\) degli \(X_{\lambda} \). Allora \(\exists ! f \in \mathcal{C}{(X', X)}\) 
    tale che \(p'_\lambda = p_\lambda \circ f\) per ogni \(\lambda\) e \(f\) è
    isomorfismo. 

    Viceversa se \({(X, \{p_\lambda\}_{\lambda \in \Lambda} )}\) è un prodotto
    degli \(X_\lambda\) e \(f : X' \to X\) è isomorfismo, anche \({(X', \{p_\lambda \circ f\}_{\lambda \in \Lambda} )}\) è un prodotto degli \(X_\lambda\) 
\end{proposition}
\begin{remark}{}
    Si dice che il prodotto è \emph{unico a meno di unico isomorfismo}
\end{remark}
\begin{proof}
    \emph{(Prima parte)}
    Esiste unico \(f\) per la proprietà universale, analogamente \(\exists ! f'
    \in \mathcal{C}{(X, X')}\) tale che \(p_\lambda = p'_\lambda \circ f\), \(\forall \lambda \in \Lambda\). Dunque \(p_\lambda = p'_\lambda \circ f' = p_\lambda \circ f \circ f' = p_\lambda
     \circ 1_X\). Ne consegue che \(1_X = f \circ f'\) per la proprietà
     universale e analogamente \(1_Y = f'\circ f\).

     \emph{(Seconda parte)} Dati \(Y \in \mathcal{C}\) e \(g_\lambda : Y \to
     X'_\lambda\) devo dimostrare che \(\exists ! g : Y \to X\) tale che \(
     g_\lambda = f_\lambda \circ f \circ g\) per ogni \(\lambda \in \Lambda\).
     Ora per la proposizione universale di \(X\) \(\exists ! g : Y \to X\) tale
     che \(g_\lambda = p_\lambda \circ g\). Voglio \(g = f \circ g'\) cioè \(g'
     = f^{-1} \circ g\) 
\end{proof}

\begin{note}[zione]
    L'oggetto prodotto \(X\) si indica con
    \[
      X =: \prod_{\lambda \in \Lambda} X_\lambda
    \]
\end{note}

\begin{example}{}
    In \(\mathtt{Set}\) il prodotto di insiemi \(X_\lambda\) per \(\lambda \in \Lambda\) 
    è dato dall'usuale prodotto cartesiano con le proiezioni.

    In categorie concrete come \(\mathtt{Grp}, \mathtt{Rng}, A\mathtt{-Mod}\) un
    prodotto si ottiene dal prodotto in \(\mathtt{Set}\) mettendo la struttura
    disgiuntiva componente per componente.
\end{example}

\begin{example}{}
    In \(\mathtt{FinSet}\) (la sottocategoria piena di \(\mathtt{Set}\)) con
    oggetti insiemi finiti, non esiste \(\prod_{\lambda \in \Lambda} X_\lambda\) se \(\# \Lambda = \infty\) e \(\# X_\lambda > 1\) per ogni \(\lambda \in \Lambda\) 

    Infatti se per assurdo supponiamo il prodotto essere \(X\) per la proprietà
    universale \(\forall Y \in \mathtt{FinSet}\), \( \infty > 
    \# \mathtt{Set}{(Y, X)} = \prod_{\lambda \in \Lambda} \# \mathtt{Set}(Y,
    X_\lambda) = \infty\) 
\end{example}

\begin{remark}{}
    Se \(\# \Lambda = 1\) allora un prodotto di \(X_1\) è \(p_1 : X \to X_{1}\)
    in \(\mathcal{C}\) tale che \(\forall Y \in \mathcal{C}\) e \(\forall f_{1} : Y \to X_{1}\) 
    \(\exists ! f : Y \to X\) tale che \(f_{1} = p_{1} \circ f\) 

    Questo è vero se \(p_{1}\) è isomorfismo (\(f = p_{1}^{-1} \circ f_{1}\)).
    D'altra parte se \(p_{1}\) non è isomorfismo non fattorizza unicamente ogni
    altro morfismo. Quindi un prodotto di \(
    X \in \mathcal{C}\) è qualunque isomorfismo \(X' \to X\) (in particolare \(1_X\)).
\end{remark}

\begin{definition}{Oggetto terminale}
    Un oggetto terminale di una categoria \(\mathcal{C}\) è un prodotto vuoto in
    \(\mathcal{C}\), cioè \(X \in \mathcal{C}\) tale che \(\forall Y \in \mathcal{C}\) esiste un unico \(Y \to X\) morfismo di \(C\), cioè \(\#\mathcal{C}{(Y, X)} = 1\) 
\end{definition}

\begin{example}{}
    In \(\mathtt{Set}\) \(X\) è terminale se e solo se \(\# X = 1\).
    Analogamente in \(\mathtt{Grp}, \mathtt{Rng}, A\mathtt{-Mod}\) è ogni gruppo
    anello o \(A\mathtt{-Mod}\) banale.
\end{example}

\begin{example}{}
    Se \(G\) è un monoide non banale, allora \(G\) (come categoria con un solo
    oggetto) non ha oggetto terminale.
\end{example}

\begin{proposition}{}
    Una categoria \(\mathcal{C}\) ha tutti i prodotti finiti se e solo se ha
    oggetto terminale e i prodotti di coppie di oggetti.
\end{proposition}
\begin{proof}{}
    Dimostro solo per induzione l'implicazione non ovvia.

    Il passo base è dato dalla presenza dell'oggetto terminale.
    Per induzione supponiamo che esista
    il prodotto di \(n-1\) oggetti \(X'=\prod_{i = 1}^{n-1} X_i\) con \(p'_i :
    X' \to X_{i}\) per ogni \(i\). Sia ora un elemento \(X_n\) e per ipotesi
    esiste \(X = X' \times X_{n}\) con \(p_{n} : X \to X_{n}, p' : X \to X'\).
    Allora \(X\) è prodotto di tutti gli \(\{X_i\}_{i=1}^{n}\) con \(p_{i} :=
    p'_i \circ p'\) per ogni \(i < n\). 
\end{proof}

\begin{definition}{Coprodotto}
    Un coprodotto di \(X_\lambda\) \((\lambda \in \Lambda)\) in una categoria \(\mathcal{C}\) è un prodotto degli \(X_\lambda\) in \(\mathcal{C}^{op}\), cioè è dato da \(
    X \in \mathcal{C}\) e da morfismi \(i_\lambda \in \mathcal{C}{(X_\lambda, X)}\) tali che vale la proprietà universale (duale di quella del prodotto)

\[
  \forall Y \in C, \,\, \forall \lambda \in \Lambda,\,\, \forall f_\lambda \in \mathcal{C}{(X_\lambda, Y)}, \quad \exists ! \,f \in \mathcal{C}{(X, Y)} : f_\lambda = f \circ i_\lambda
\]
e viene denotato 
\[
  X =: \coprod_{\lambda \in \Lambda} X_\lambda
\]
Diagrammaticamente, il diagramma
\[\begin{tikzcd}
	Y \\
	{X_\lambda} & \coprod_{\lambda \in \Lambda} X_\lambda
	\arrow["{f_\lambda}", from=2-1, to=1-1]
	\arrow["{i_\lambda}"', from=2-1, to=2-2]
	\arrow["{\exists! f}"', dashed, from=2-2, to=1-1]
\end{tikzcd}\]
commuta
\end{definition}

Nelle categorie preadditive si può parlare di \textbf{somma diretta} invece di
coprodotto e usare \(\bigoplus\) invece di \(\coprod\).

\begin{example}{}
    In \(\mathtt{Set}\) il coprodotto è l'unione disgiunta. In \(A\mathtt{-Mod}\) è la somma diretta usuale. In \(\mathtt{Grp}\) i coprodotti sono i \textbf{prodotti liberi}.
\end{example}

\begin{definition}{Oggetto iniziale}
    Un \emph{oggetto iniziale} di \(\mathcal{C}\) è un coprodotto vuoto in \(C\), ossia \(X \in \mathcal{C}\) tale che
    \(\forall Y \in \mathcal{C}\), \(\# \mathcal{C}{(X, Y)} = 1\) 
\end{definition}

Può succedere che uno stesso oggetto sia terminale che iniziale. In tal caso per entrambe le definizioni esiste un solo morfismo da uno all'altro. Se tale morfismo è un isomorfismo
allora l'oggetto è sia iniziale che terminale.

\begin{definition}{Oggetto nullo}
    Un oggetto sia iniziale che terminale si dice \emph{nullo}
\end{definition}

\begin{example}{}
    In \(\mathtt{Set}\), \(\varnothing\) è iniziale (non nullo).
    In \(\mathtt{Grp}/A\mathtt{-Mod}\) ogni \emph{gruppo}/\emph{modulo} banale è
    nullo.
    In \(\mathtt{Rng}\), \(\mathbb{Z}\) è iniziale (non nullo)
\end{example}

    Se \(X \in \mathcal{C}\) è nullo allora \(\forall Y, Z \in \mathcal{C}\)
    esiste il morfismo \(0 \in \mathcal{C}{(Y, Z)}\) dato dalla composizione
    \(Y \overset{\exists !}{\to } X \overset{\exists !}{\to } Z\). In tal caso
    abbiamo che effettivamente \(f \circ 0 = 0\) e \(0 \circ g = 0\)  per ogni
    \(f, g\) componibili con \(0\).
\begin{example}{}
    In \(\mathcal{A}\) preadditiva (in cui esiste un oggetto nullo) il morfismo
    0 di cui sopra coincide con 0 dello struttura preadditiva.
\end{example}

\begin{definition}{Preservazione del prodotto}
    Un funtore \(F : \mathcal{C} \to \mathcal{D}\) \emph{preserva un prodotto}
    \({(X, \{p_\lambda\} _{\lambda \in \Lambda} )}\) di \(\{X_\lambda\}_{\lambda \in \Lambda} \subseteq \mathcal{C} \) se \({(F{(X)}, \{F{(p_\lambda)}\}_{\lambda \in \Lambda}  )}\) è un prodotto degli \(F{(X_{\lambda} )}\) in \(\mathcal{D}\).

    Diremo inoltre che \(F\) \emph{preserva i prodotti (o prodotti finiti)} se preserva
    tutti i prodotti (o prodotti finiti) che esistono in \(\mathcal{C}\) 
\end{definition}
\begin{remark}{}
    Se \(F\) preserva un prodotto degli \(X_\lambda\), allora li preserva tutti.
\end{remark}

\begin{definition}{Preservazione del coprodotto}
    \(F: \mathcal{C} \to \mathcal{D}\) preserva un coprodotto di \(\mathcal{C}\) se \(F^{op}\) preserva il
    corrispondente prodotto di \(C^{op}\) 
\end{definition}

\begin{example}{}
    I funtori dimenticanti \(\mathtt{Grp}/\mathtt{Rng}/A\mathtt{-Mod} \to \mathtt{Set}\) preservano i prodotti ma non i coprodotti.
\end{example}





