\chapter{Algebra Omologica}

Sia \(\mathcal{A}\) una categoria preadditiva, e \(C{(\mathcal{A})}\) la
categoria dei complessi. Siano \(X^{\cdot }\) e \(Y^{\cdot } \in
C{(\mathcal{A})}\). Allora dati \(\kappa^{i} \in \mathcal{A}{(X^{i}, Y^{i-1})}\)
per ogni \(i \in \mathbb{Z}\) 
\begin{equation}{}\label{eq:omotopia-complessi}
    sono arrivato in ritardo
\end{equation}


\begin{definition}{Omotopia}
    Sia \(f^\bullet \in C{(\mathcal{A})}{(X^{\cdot }, Y^{\cdot})}\) è
    \emph{omotopo a 0} se \(\exists k^{i} \in \mathcal{A}{(X^{i}, Y^{i-1})}\)
    tale che vale~\eqref{eq:omotopia-complessi}.

    \(f^{\bullet}\) e \(g^\bullet \in C{(\mathcal{A})}{(X^\bullet, Y^\bullet)}\)
    sono \emph{omotopi} (denotato con \(f^\bullet \sim g^\bullet\)) se
    \(f^\bullet - g^\bullet \) è omotopo a 0.
\end{definition}

\begin{remark}{} 
    \(\mathtt{Htp}{(X^\bullet, Y^\bullet)} := \{f^\bullet \in C{(\mathcal{A})}
    {(X^\bullet, Y^\bullet)} | f^\bullet \sim 0\} \) % TODO: cos'è
\end{remark}

Come conseguenza si ottiene una categoria preadditiva \(K{(\mathcal{A})} :=
C{(\mathcal{A})} / \mathtt{Htp}\) (categoria \emph{omotopa} di \(\mathcal{A}\))
con funtore additivo \(Q = Q_{\mathcal{A}} : C{(\mathcal{A})} \to
K{(\mathcal{A})}\) tale che \(X^\bullet \mapsto X^\bullet\) e \(f^\bullet
\mapsto [f^\bullet]\) 

\begin{remark}{}
    Si dimostra che se \(\mathcal{A}\) è abeliana, allora 
    \[
      \mathcal{A} \text{ è semisemplice } \iff K{(\mathcal{A})}\text{ è abeliana
      } \iff K{(\mathcal{A})}\text{ è preabeliana }
    \]
\end{remark}

\begin{remark}{}
    Sia \(F: \mathcal{A} \to \mathcal{B}\) un funtore additivo tra categorie
    preadditive. Allora \(F\) induce funtori \(C{(F)} : C{(\mathcal{A})} \to C
    {(\mathcal{B})}\) e \(K{(F)} : K{(\mathcal{A})} \to K{(\mathcal{B})}\) tali
    che, per ogni \(X \in C{(\mathcal{A})}\), \(C{(F)} {(X^\bullet)} =
    K{(F)}{(X^\bullet)} = \) 
    \[
       = {\left( \dots \to F{(X^{i-1})} \overset{F{(d^{i-1})}}{\to} F{(X^{i})}
       \overset{F{(d^{i})}}{\to} F{(X^{i+1})} \right)} 
    \]
    e per funtorialità \(F{(d^{i})} \circ F{(d^{i-1})} = F{(d^{i} \circ
    F^{i-1})} = F{(0)} = 0\) perché è additivo.

    E esplicitamente \(\forall f^\bullet : X^\bullet \to Y^\bullet\),
    \(C{(F)}{(f^\bullet)}^{i} := F{(f^{i})}\) e \(K{(F)} {([f^\bullet])} :=
    [C{(F)}{(f^\bullet)}]\) (mostrare per esercizio che è sempre ben
    definito). Di solito si scrive direttamente \(F\) invece di \(C{(F)}\) o
    \(K{(F)}\) 
\end{remark}

\begin{lemma}{}
    Sia \(\mathcal{A}\) una categoria abeliana. Sia \(f^\bullet \in
    \mathtt{Htp}{(X^\bullet, Y^\bullet)}\). Allora \(H^{i}{(f^\bullet)} = 0\)
    per ogni \(i \in Z\) 
\end{lemma}
\begin{proof}{}
    \(f^{i} = d_{Y^\bullet} ^{i-1} \circ k^{i} + k^{i+1} \circ
    d_{X^\bullet}^{i}\). Osservando il diagramma a righe esatte
\[ \begin{tikzcd}
	0 & \mathrm{Im}{(d_{X^\bullet} ^{i-1})} & \mathrm{Ker}{(d_{X^\bullet} ^{i})}
      & H^{i}{(X^\bullet)} & 0 \\
	0 & \mathrm{Im}{(d_{Y^\bullet} ^{i-1})} & \mathrm{Ker}{(d_{Y^\bullet} ^{i})}
      & H^{i}{(Y^\bullet)} & 0
	\arrow[from=1-1, to=1-2]
	\arrow[from=1-2, to=1-3]
	\arrow["I^{i}{(f^\bullet)}"', from=1-2, to=2-2]
	\arrow[from=1-3, to=1-4]
	\arrow["K^{i}{(f^\bullet)}", from=1-3, to=2-3]
	\arrow[from=1-4, to=1-5]
	\arrow["{H^i(f^\bullet)}", dashed, from=1-4, to=2-4]
	\arrow[from=2-1, to=2-2]
	\arrow[from=2-2, to=2-3]
	\arrow[from=2-3, to=2-4]
	\arrow[from=2-4, to=2-5]
\end{tikzcd} \]

Ma allora \(H^{i}{(f^\bullet)} = 0 \iff K^{i}{(f^\bullet)}\) fattorizza
attraverso \(I^{i}{(Y^\bullet)} \to K^{i}{(Y^\bullet)}\) se e solo se
\(\mathrm{Im}{(K^{i}{(f^\bullet)})} \subseteq \mathrm{Im}{(d_{Y^\bullet}
^{i-1})} \). Allora \(K^{i}{(f^\bullet)}\) è indotto da \(f^{i}\), quindi da
\(d_{Y^\bullet} ^{i-1} \circ K^{i}\) (perché \(d_{X^\bullet} ^{i}\) è 0 su
\(K^{i}{(f^\bullet)}\)).
\end{proof} 

\begin{corollary}{}
    Sia \(\mathcal{A}\) abeliana. Allora \(\forall i \in \mathbb{Z}\) il funtore
    \(H^{i} : C{(\mathcal{A})} \to \mathcal{A}\) induce un funtore additivo
    \(H^{i} : K{(\mathcal{A})} \to \mathcal{A}\) 
\end{corollary}

\begin{definition}{Risoluzione}
    Sia \(\mathcal{A}\) una categoria abeliana. Una \emph{risoluzione} a
    \emph{sinistra}/\emph{destra} di un oggetto \(X \in \mathcal{A}\) è una
    successione esatta 
    \[
     \dots \to X^{-2} \to X^{-1} \to X^{0} \to X \to 0 / 0 \to X \to X^{0} \to
     X_{1} \to X_{2} \to \dots
    \] 
    (di solito si scrive \(\dots\to X_2 \to X_{1} \to X_{0} \to X \to
    0\) nel caso di risoluzione sinistra).

    Inoltre una tale soluzione ha \emph{lunghezza} \(n\) se (ad esempio nel caso
    di risoluzione destra)
    \[
      X^{n} \neq 0 \text{ e } {(X^{i} = 0)} \quad \forall i > n
    \]
    Se tale \(n\) non esiste si dice che la lunghezza è \(\infty\) 
\end{definition}

Esistono diversi tipo di risoluzione a seconda del tipo di problemi che si sta
affrontando. Per i nostri scopi vedremo di studiare gli oggetti iniettivi e
suriettivi.

\begin{definition}{}
    Una risoluzione \emph{proiettiva}/\emph{iniettiva} di \(X \in \mathcal{A}\)
    è una risoluzione \emph{sinistra}/\emph{destra} \emph{con} \(X_{i}\)
    \emph{proiettivo} / \emph{con }\(X^{i}\) \emph{iniettivo} per ogni \(i \in
    \mathbb{N}\).
\end{definition}

\begin{proposition}{}
    Ogni \(X \in \mathcal{A}\) ha una risoluzione
    \emph{proiettiva}/\emph{iniettiva} se e solo se \(\mathcal{A}\) ha
    abbastanza \emph{proiettivi}/\emph{iniettivi}.
\end{proposition}
\begin{proof}\( \)
\begin{itemize}
    \item[\(\implies \)] ovvio.
    \item[\(\impliedby \)] \(\exists d^{0} : X^{0} \to X\) epimorfismo, con
        \(X^{0}\) proiettivo. Allora \(\exists \tilde{d}^{1} : X^{1} \to
        \mathrm{Ker}{(d^{0})}\) epi con \(X^{1}\) proiettivo. Allora \(d^{1} :=
        \mathrm{Ker}{(d^{0})} \circ \tilde{d}^{1} : X^{1} \to X^{0}\) e si può
        proseguire per induzione.
\end{itemize}
\end{proof}

\begin{proposition}{}\label{prop:non-so-niente}
    Sia \(\mathcal{A}\) una categoria abeliana e sia \(f : X \to Y\) un morfismo
    di \(\mathcal{A}\). E siano
    \[ \begin{tikzcd}
	0 & X & X^{0} & X^{1} & \dots \text{ risoluzione di \(X\) }\\
    0 & Y & I^{0} & I^{1} & \dots \text{ complesso con ogni \(I^{i}\) iniettivo}
	\arrow[from=1-1, to=1-2]
	\arrow["s", from=1-2, to=1-3]
	\arrow["f"', from=1-2, to=2-2]
	\arrow["d^{0}", from=1-3, to=1-4]
	\arrow["f^{0}", dashed, from=1-3, to=2-3]
	\arrow["d^{1}", from=1-4, to=1-5]
	\arrow["f^{1}", dashed, from=1-4, to=2-4]
	\arrow[from=2-1, to=2-2]
	\arrow["t",from=2-2, to=2-3]
	\arrow["\ell^{0}", from=2-3, to=2-4]
	\arrow["\ell^{1}", from=2-4, to=2-5]
\end{tikzcd} \]
    Si denoti inoltre \(X^\bullet := {\left( 0 \to X^{0} \overset{d^{0}}{\to}
    X^{1} \to \dots \right)} \) e \(I^\bullet := {\left( 0 \to I^{0}
\overset{\ell^{0}}{\to} I^{1} \to \dots \right)} \).
    Allora
\begin{enumerate}[label = \arabic*)]
    \item \(\exists \, f : X^\bullet \to I^\bullet\) in \(C{(\mathcal{A})}\) che
        estende \(f\), cioè \(f^0 \circ s = t \circ f\) 
    \item \([f^\bullet]\) è unico in \(K{(\mathcal{A})}\) 
\end{enumerate}
\end{proposition}
\begin{proof}{}
    Siano \(\tilde{X}^\bullet\) e \(\tilde{I}^\bullet\) i complessi delle righe
    del diagramma. In particolare \(\tilde{X}^{-1} = X\), \(\tilde{X}^{i} =
    X^{i}\) per ogni \(i \in \mathbb{N}\) e \(\tilde{X}^{i} = 0\) per ogni \(i <
    -1\). Similmente per \(\tilde{I}^{\bullet}\). Inoltre si denoti anche \(s =
    d^{-1}\) e \(t = \ell^{-1}\).

\begin{enumerate}[label = \arabic*)]
    \item Cerco \(\tilde{f}^\bullet : \tilde{X}^\bullet \to \tilde{I}^\bullet\)
        in \(C{(\mathcal{A})}\) tale che \(\tilde{f}^{-1} = f\). Dato \(n \in
        \mathbb{N}\) e supponendo di avere già definito \(\tilde{f}^{i} \) per
    ogni \(i < n\) tale che \(\tilde{f}^{i} \circ d^{i-1} = \ell^{i-1} \circ
    \tilde{f}^{i-1}\) per ogni \(i < n\).

    Cerco ora \(f^{n} = \tilde{f}^{n} : X^{n} \to I^{n}\) tale che \(f^{n} \circ
    d^{n-1} = \ell^{n-1} \circ \tilde{f}^{n-1}\).
    \[ \begin{tikzcd}
	\tilde{X}^{n-2} & \tilde{X}^{n-1} & \tilde{X}^{n} = X^{n} \\
	& \tilde{I}^{n-1} & I^{n}
	\arrow["{d^{n-2}}", from=1-1, to=1-2]
	\arrow["{d^{n-1}}", from=1-2, to=1-3]
	\arrow["{\tilde{f}^{n-1}}"', from=1-2, to=2-2]
	\arrow[from=1-2, to=2-3]
	\arrow["{f^n}", dashed, from=1-3, to=2-3]
	\arrow["{\ell^{n-1}}"', from=2-2, to=2-3]
\end{tikzcd} \]
    è esatto, dunque \(I^{n}\) è iniettivo. Allora
    \[
      \mathcal{A}{(X^{n}, I^{n})} \overset{d^{n-1 *}}{\to}
      \mathcal{A}{(\tilde{X}^{n-1}, I^{n})} \overset{d^{n-2*}}{\to}
      \mathcal{A}{(\tilde{X}^{n-2}, I^{n})}
    \] è esatta, e quindi esiste \(f^{n} \in \mathcal{A}{(X^{n}, I^{n})}\) tale
    che 
    \begin{align*}
      &d^{n-1*}{(f^{n})} = f^{n} \circ d^{n-1} = \ell_{n-1}  \circ
      \tilde{f}^{n-1} \iff \\ \iff &0 = d^{n-2*}{\left( \ell^{n-1} \circ \tilde{f}^{n-1}
      \right)} = \ell^{n-1} \circ \tilde{f}^{n-1} \circ d^{n-2} =
      \underbrace{\ell^{n-1} \circ \ell^{n-2}}_{0} \circ \tilde{f}^{n-2}
    \end{align*}

    \item Dati \(f^\bullet, g^\bullet : X^\bullet \to I\) che estendono \(f\) ,
        devo dimostrare che \(f^\bullet \sim g^\bullet\). Basta allora
        dimostarre che dato \(h^\bullet : X^\bullet \to I^\bullet\) che estede
        \(0 : X \to Y\), allora \(h^\bullet \sim 0\).
        A tale scopo, sia \(\tilde{h}^\bullet : \tilde{X}^\bullet \to
        \tilde{I}^\bullet\) dato da \(\tilde{h}^{i} := h^{i}\), \(\forall i \in
        \mathbb{N}\) e \(\tilde{h}^{i} := 0\), \(\forall i < 0\). Allora
        \(\tilde{h}^\bullet\) è un morfismo in \(C{(\mathcal{A})}\). Voglio
        dunque dimostrare che \(\exists k^{i} : \tilde{X}^{i} \to
        \tilde{I}^{i-1}\) tale che 
        \begin{equation}\label{eq:helpext1}
          \tilde{h}^{i} = \ell^{i-1} \circ k^{i} + k^{i+1} \circ d^{i} \quad
          \forall i \in \mathbb{Z}\text{ , con \(k^{0} = 0\)}
        \end{equation}
        (infatti da questo seguirebbe che \(\tilde{h}^\bullet \sim 0 \) e
        \(h^\bullet \sim 0\)).

        Dato \(n \in \mathbb{N}\) e assumendo di avere già definito i \(k^{i}\)
        per ogni \(i \le n\) tale che~\eqref{eq:helpext1} valga per ogni \(i <
        n\). Cerco allora \(k^{n+1} : X^{n+1} \to I^{n}\) tale che
        valga~\eqref{eq:helpext1} per \(i = n\). Allora il seguente diagramma
        con prima riga esatta commuta:
    \[ \begin{tikzcd}
	\tilde{X}^{n-1} & {X}^{n} & {X}^{n+1} \\
	& I^{n}
	\arrow["{d^{n-1}}", from=1-1, to=1-2]
	\arrow["{d^{n}}", from=1-2, to=1-3]
	\arrow["{h^{n} - \ell^{n-1} \circ k^{n}}"', from=1-2, to=2-2]
	\arrow["{k^{n+1}}", dashed, from=1-3, to=2-2]
\end{tikzcd} \]
    e come prima dunque \(\exists \, k^{n+1}\) tale che
    valga~\eqref{eq:helpext1} per \(i = n\) se e solo se 
    \begin{align*}
        0 &= \ell^{n-1} \circ k^{n} \circ d^{n-1} = \ell^{n-1} \circ {\left(
        \tilde{h}^{n-1} - \ell^{n-2} \circ k^{n-1} \right)} = \\ &= \tilde{h}^{n} \circ
      d^{n-1} - \ell^{n-1} \circ \tilde{h}^{n-1} = 0
    \end{align*}
\end{enumerate}
\end{proof}

\begin{corollary}{}
    Una risoluzione iniettiva di \(X \in \mathcal{A}\) categoria abeliana (se
    esiste) è unica a meno di isomorfismo in \(K{(\mathcal{A})}\).
\end{corollary}
\begin{proof}{}
    Siano \(0 \to X \to I^\bullet\) e \(0 \to X \to J^\bullet\) risoluzioni
    iniettive di \(X\). Allora applico la proposizione~\ref{prop:non-so-niente}
    con \(f = 1_X\), dunque \(\exists \, f^\bullet : I^\bullet \to J^\bullet\)
    che estende \(1_X\). Analogamente \(\exists \, g^\bullet : J^\bullet \to
    I^\bullet\) che estende \(1_X\). Segue che anche \(g^\bullet \circ f^\bullet
    : I^\bullet \to I^\bullet\) estende \(1_X\), ma anche \(1_{i^\bullet} \)
    estende \(1_X\), da cui \(g^\bullet \circ f^\bullet \sim 1_{I^\bullet} \) e
    analogamente \(f^\bullet \circ g^\bullet \sim 1_{J^\bullet} \) da cui
    \([f^\bullet]\) è isomorfismo.
\end{proof}

\begin{corollary}{}
    Sia \(\mathcal{A}\) una categoria abeliana con abbastanza iniettivi. Allora
    c'è un funtore ``risoluzione iniettiva'' \(I_\mathcal{A} : \mathcal{A} \to
    K(\mathcal{A})\) tale che \(I_{\mathcal{A}} {(X)} := I^\bullet\) e \(0 \to X
    \to I^\bullet\) è una risoluzione iniettiva (AoC per poterla scegliere per
    ogni \(X \in \mathcal{A}\)).

    Inoltre \(\forall f : X \to Y\) in \(\mathcal{A}\), \(I_{\mathcal{A}} {(f)}
    := [f^\bullet]\) con \(f^\bullet\) che estende \(f\). Si ottiene dunque un
    funtore (esercizio) la cui classe di isomorfismo non dipende dalle scelte.
\end{corollary}




