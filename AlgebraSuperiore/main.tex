%! TEX program = lualatex
\input{../preamble_appunti_report.tex}

\title{Appunti di Algebra Superiore}
\author{Github Repository:
\href{https://github.com/Oxke/appunti/tree/main/AlgebraSuperiore}{\texttt{Oxke/appunti/AlgebraSuperiore}}}

\date{Primo semestre, 2025 \-- 2026, prof. Alberto Canonaco}

\begin{document}
\maketitle

In topologia: sia \(X\) uno spazio topologico, allora omologia / coomologia (a
coeff in \(A\)) 
\[
  H_{n}{(X, A)} / H^{n}{(X, A)}
\]
Ottenuti a partire da \textbf{complessi} di catene o cocatene, cioè successioni
\[
    C_{n}{(X, A)} \overset{d_{n}}{\to } C_{n-1} {(X, A)} \overset{d_{n-1} }{\to } \dots
\]
e \(d_{n-1} \cdot d_{n} = 0\) (cioè \(\mathrm {Im} {(d_{n})} \subseteq \mathrm{Ker}{(d_{n-1} )} \))  e \(H_{n} {(X, A)} := \frac{\mathrm{Ker}{(d_{n-1} )}}{\mathrm{Im}{(d_{n})}}\) 
In astratto introdurremo le categorie \emph{abeliane} (che generalzzano le
strutture dei moduli su un anello) e studieremo i ``fattori derivati'' di
funtori (additivi) tra categorie abeliane.

I funtori derivati misureranno la mancanza di \textbf{esattezza} (ossia mandare
successioni esatte in successioni esatte) del funtore di
partenza.

\paragraph{Libri utili}
\begin{itemize}[label = --]
    \item Per la parte di algebra omologica Hilton-Stammbach, Osborne e
Weibel.
    \item Dispense sui moduli (su KIRO) utili
    \item Aluffi, \emph{Algebra Chapter 0}
\end{itemize}
Il corso è di 60 ore, non perché sia più pesante ma perché dovrebbero esserci
ore di esercitazioni (non sarà necessariamente vero ma Canonaco cercherà di
andare un po' nel dettaglio, fornire esempi e controesempi per quanto possibile)


\section{Richiami sugli Anelli}

Per convenzione, parlando di \textbf{anelli} si parlerà sempre di \textbf{anelli
con unità}
\begin{definition}{Anello}
    Un \textbf{anello} \(A, +, \cdot \) è un gruppo abeliano \(A, +\) (con 0
    elemento neutro) e
    contemporaneamente un monoide \(A, \cdot \)  (cn 1 elemento neutro).
    Inoltre le due operazioni sono legate dalle proprietà distributive 
    \[
      a{(b + c)} = ab + ac \quad ; \quad {(b + c)}a = ba + ca
    \]
    Diremo che l'anello è \textbf{commutativo} se l'operazione \(\cdot \) è
    commutativa
\end{definition}
Per quasi tutto ciò che si vedrà in questo corso non è necessario andare a
disturbare anelli non commutativi, dunque si useranno quasi sempre anelli
commutativi.
\begin{example}
    \(\mathbb{Z}, \mathbb{Q}, \mathbb{R}, \mathbb{C}, \mathbb{H}, \mathbb{Z} / n \mathbb{Z}\)
\end{example}
\begin{example}
    Se \(A\) è un anello (commutativo), allora i polinomi a coefficienti in A e
    con variabili in \(\Lambda\) costituiscono l'anello \(A[x_{\lambda} \mid
    \lambda \in  \Lambda]\) 
\end{example}
\begin{example}[Anello Banale]
    L'anello composto da un solo elemento \(\{0 = 1\} \)  
\end{example}
\begin{example}[Non comm.]
    \(A\) anello, allora l'anello \(M_{n}{(A)}\) delle matrici \(n \times n\) a
    coefficienti in \(A\) non è commutativo se \(n > 1\) (e se non è l'anello
    banale ma dai l'anello banale non esiste davvero)
\end{example}
\begin{example}{Endomorfismi}
    Se \({(G, +)}\) è un gruppo abeliano, allora \(\mathrm{End}{(G)}\) è anello
    con \(+\) determinato da \({(f + g)}{(a)} = f{(a)} + g{(a)}\) e \(\cdot \)
    dato dalla composizione \({(f \circ g)}{(a)} = f{(g{(a)})}\) 

    In generale se \(G, G'\) sono gruppi con \({(G, +)}\) abeliano, allora l'insieme
    \(\mathrm{Hom {(G', G)}}\) degli omomorfismi da \(G'\) a \(G\)  è un sottogruppo
    di \(G^{G'}\) il gruppo delle funzioni da \(G'\) a \(G\).

    Infatti se \(X\) è un insieme allora \(G^{X}\) è un gruppo con \({(f + g)}{(a)} = f{(a)} + g{(a)}\) 
\end{example}

\begin{definition}{Invertibile}
    \(a \in A\) è invertibile a sinistra (destra) se \(\exists a' \in A\) tale
    che \(a' a = 1\) (\(a a' = 1\)).

    \(a\) viene detto \textbf{invertibile} se \(\exists a' \in A\) tale che \(a' a = aa' = 1\)
\end{definition}
\begin{remark}[invertibile \(\iff\) invertibile a destra e sinistra]
    solo una implicazione non è ovvia. Se \(a', a'' \in A\) sono tali che \(a' a
    = a a'' = 1\) allora
    \begin{align*}
        {(a' a)}a'' &= a' (a a'') 
        1 a'' = a'' &= a' = a' 1
    \end{align*}
    quindi \(a\) è invertibile e \(a^{-1} = a' = a''\) 
\end{remark}


\begin{remark}[Gruppo degli invertibili]
    L'insieme degli elementi invertibili forma un gruppo con l'operazione di
    prodotto e si indica con \(A^{*}\) 
\end{remark}

In generale, se \(1 \neq 0\), allora \(A^{*} \subseteq A \sminus \{0\}  \) 

\begin{definition}{Anello con Divisione}
    \(A\) si dice \textbf{anello con divisione} se \(A^{*} = A \sminus \{0\} \).
    Un campo è un anello con divisione commutativo.
\end{definition}

\begin{definition}{Divisore di zero}
    \(a \in A\) è detto \textbf{divisore di zero} a sinistra (destra) se \(\exists  a' \in A \sminus \{0\} \) tale che \(a a' = 0\) (\(a' a = 0\))
\end{definition}

\begin{definition}{Dominio}
    \(A\) viene detto \textbf{dominio} se \(A \neq 0\) e \(A\) non ha divisori
    di zero. Viene inoltre chiamato \textbf{dominio di integrità} se è
    commutativo.
\end{definition}
\begin{example}
    I campi, \(\mathbb{Z}\), se \(A\) dominio d'integrità, allora anche \(A[x_{\lambda} \mid \lambda \in \Lambda ]\) è dominio d'integrità.
\end{example}

\begin{remark}
    \(A \neq 0\) tale che \(\forall \, 0 \neq a \in A\) è invertibile a sinistra,
    allora \(A\) è un anello con divisione.
\end{remark}
\begin{proof}
    \(\exists a' \in A\) tale che \(a' a = 1\) ma anche \(\exists a'' \in A :
    a''a' = 1\). Allora \(a'\) è invertibile a sinistra e a destra, infatti
    \[
      a'^{-1} = a = a'' \implies a \in A^{*}
    \]
\end{proof}
\begin{definition}{Sottoanello}
    \(A' \subseteq A \) è \textbf{sottoanello} di \(A\) se \({(A', +)} < {(A, +)} \), \(ab \in A' \) per ogni \(a, b \in A'\) e \(1 \in A'\) 
\end{definition}
\begin{example}{}
    \(\mathbb{Z} \subseteq \mathbb{Q} \subseteq \mathbb{R} \subseteq \mathbb{C} \subseteq \mathbb{H}    \) sono tutti sottoanelli
\end{example}
\begin{example}{}
    \(A \subseteq A[X] \) sottoanello
\end{example}

\begin{definition}{Ideale}
    \(I \subseteq A \) è un'ideale sinistro (destro) se \({(I, +)} < {(A, +)}\)
    e \(ab \in I\) (\(ba \in I\)), \(\forall a \in A\) e \(\forall b \in I\).

    Un \textbf{ideale} bilatero è un ideale sia sinistro che destro.
\end{definition}

\begin{example}{}
    Gli ideali in \(\mathbb{Z}\) sono tutti e soli della forma \(n \mathbb{Z}\),
    con \(n \in \mathbb{N}\) 
\end{example}
\begin{remark}{}
    Se \(I\) è un ideale sinistro o destro allora
    \[
      I = A \iff I \cap A^{*} \neq \varnothing
    \]
    quindi \(A\) con divisione \(\implies \) gli unici ideali sinistri o destri
    sono \(\{0\} \) e \(A\) 
\end{remark}
\begin{definition}{Anello opposto}
    L'\textbf{anello opposto} di un anello \(A\) è \(A^{op}\), con \({(A^{op}, +)} := {(A, +)}\) e con prodotto \(ab\) in \(A^{op}\) definito come \(ba\) in \(A\) 
\end{definition}
\begin{remark}{}
    \({(A^{op})}^{op} = A\) e \(A^{op} = A \iff A\) commutativo
\end{remark}

\begin{proposition}[Anello Quoziente]
    Se \(I \subseteq A \) ideale, allora il gruppo abeliano \(A / I , +\)  è un
    anello con prodotto \(\overline{a} \overline{b} := \overline{ab}\), dove \(\overline{a} := a + I \in A / I\) 
\end{proposition}

\begin{definition}{omomorfismo di anelli}
    Siano \(A, B\) anelli. \(f: A\to B\) è \textbf{omomorfismo} di anelli se, \(\forall a, a' \in A\) 
\begin{enumerate}[label = \roman*)]
    \item \(f{(a + a')} = f{(a)} + f{(a')}\) 
    \item \(f{(a a')} = f{(a)}f{(a')}\) 
    \item \(f{(1_A)} = 1_B\) 
\end{enumerate}
ed è \textbf{isomorfismo} se è un omomorfismo biunivoco
\end{definition}
\begin{remark}{}
    \(f\) omomorfismo è isomorfismo \(\iff\) \(\exists f' : B\to A\) omomorfismo
    tale che \(f' \circ f = \mathrm{id}_A\) e \(f \circ f' = \mathrm{id}_B\) 
\end{remark}
Indicheremo \(A \cong B\) se esiste un isomorfismo tra \(A\) e \(B\) 
\begin{proposition}{}
    Se \(f : A\to B\) è un omomorfismo allora
\begin{enumerate}[label = \arabic*.]
    \item \(A' \subseteq A \) è sottoanello \(\implies f{(A')} \subseteq B  \) è
        sottoanello.
    \item \(B' \subseteq B \) sottoanello \(\implies f^{-1}{(B')} \subseteq A \) è sottoanello
    \item \(J \subseteq B \) è ideale (sinistro / destro) \(\implies f^{-1}{(J)} \subseteq A \) è ideale (sinistro / destro). In particolare \(\mathrm{Ker}f := f^{-1}{(0_B)} \subseteq A \) è ideale
    \item \(f\) \textbf{suriettivo} e \(I \subseteq A \) ideale \(\implies f{(I)} \subseteq B \) è ideale
\end{enumerate}
\end{proposition}

\begin{remark}{}
    \(f : A\to B\) è iniettivo \(\iff\) \(\mathrm{Ker} f = \{0_A\} \) e in tal
    caso \(A \cong \mathrm{Im}f := f{(A)}\) che dunque è sottoanello di \(B\) 
\end{remark}

\begin{theorem}[Omomorfismo]
    \(f : A\to B\) è omomorfismo di anelli, \(I \subseteq A \) ideale tale che
    \(I \subseteq \mathrm{Ker}f \). Allora
    \[
      \exists ! \overline{f} : A / I \to  B \text{ omomorfismo tale che } \overline{f}{(\overline{a})} = f{(a)} \quad \forall  a \in A
    \]
\[\begin{tikzcd}
	A & B \\
	{A / I}
	\arrow["f", from=1-1, to=1-2]
	\arrow["\pi"', from=1-1, to=2-1]
	\arrow["{\overline{f}}"', dotted, from=2-1, to=1-2]
\end{tikzcd}\]

Inoltre \(\mathrm{im}\overline{f} = \mathrm{im}f\) e \(\mathrm{Ker}\overline{f} = \mathrm{Ker} f / I\) 
\end{theorem}
\begin{proposition}{}
    Gli ideali di \(A / I\) sono tutti e soli della forma \(J / I\) con \(J \subseteq A \) ideale tale che \(I \subseteq J \) 
\end{proposition}


\begin{theorem}[Primo teorema di isomorfismo]
    \(f : A \to B\) è omomorfismo di anelli, allora \(\mathrm{im}f \cong A / \mathrm{Ker}f\)  
\end{theorem}

\begin{definition}{}
    L'ideale generato da \(U \subseteq A \) è il più piccolo ideale di \(A\) che
    contiene \(U = \bigcap_{U \subseteq I \subseteq A \text{ideale}} I \) ed
    esplicitamente è 
    \[
      A U A := \left\{\sum_{i=1}^{n} a_{i} u_{i} b_{i} : n \in N, a_{i}, b_{i} \in A, u_{i} \in U \right\} 
    \]
\end{definition}
\begin{remark}{}
    Se \(A\) è commutativo e \(U = \{u\} \) allora \(A \{ u \} A = Au = \{au : a
    \in A\} \) (ideale principale)
\end{remark}
\begin{definition}{PID}
    \(A\) è un dominio (d'integrità) a ideali principali (PID) se ogni ideale di
    \(A\) è a ideali principali.
\end{definition}
\begin{example}{}
    Campi (non ci sono ideali propri)
\end{example}
\begin{example}{}
    \(\mathbb{Z}\) (con ideali \(nZ = (n)\) )
\end{example}
\begin{example}{}
    \(K[X]\) con \(K\) campo
\end{example}

\end{document}
