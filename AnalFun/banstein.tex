\section{Banach \-- Steinhaus}
Sia \(X = c_{00}\). Consideriamolo dunque con la norma \(\|\cdot \|_{\infty}\).
Ci chiediamo se \(X\) è Banach. Ovviamente no perché ad esempio la successione 
\[
  x_k{(n)} = \begin{cases}{}
      \frac{1}{n} & n < k \\
      0 & n \ge k
  \end{cases}
\]
è convergente a \(\overline{x}{(n)} = \frac{1}{n}\) in \(\ell_{\infty} \), che
però non è in \(c_{00}\). Poiché \(x_k\) è convergente in \(\ell_{\infty}\), in
particolare è di Cauchy, e dunque \(c_{00} \) non è Banach.

Definiamo ora il funzionale
\begin{align*}
    T_{n}: X &\longrightarrow \mathbb{R} \\
    x &\longmapsto T_{n}(x) = n x{(n)}
\end{align*}

Ora fissato \(x \in c_{00}\), \(\lim_{n \to \infty} T_n x = 0\). Inoltre \(T_{n}\) è lineare ed è continuo poiché \(|T_n x | = | n x {(n)} | \le n \|x\|_{\infty} \). Dunque per \(
n\) fissato, \(T_n\) è continuo.

Inoltre \(\|T_{n}\|_{X'} = n \), dunque \(\lim_{n \to \infty} \|T_{n}\|_{X'} = +
\infty\).


Per introdurre il teorema di Banach \-- Steinhaus, abbiamo bisogno del seguente lemma:

\begin{lemmao}[Baire]
    Sia \(X\) uno spazio metrico completo. Data \(\{X_{n}\}_{n \in \mathbb{N}}\) una successione di sottoinsiemi chiusi magri, ossia con interno vuoto, si ha che \(
    \bigcup_{n \in \mathbb{N}} X_{n}\) è magro.

    Equivalentemente data \(\{X_{n}\} \) una successione di sottoinsiemi chiusi
    tale che \(\bigcup_{n \in \mathbb{N}} X_{n} = X\), allora \(\exists n_{0}
    \in \mathbb{N} : \mathring{X_{n_{0}}} \neq \varnothing\) 
\end{lemmao}
\begin{proof}
  Dimostriamo una versione equivalente: 
  \[
    \{A_n\} \subset X \text{ aperti densi} \implies \bigcap_{n} A_n \text{ denso}
  \]
  Fissiamo \(U \subset X\) aperto non vuoto. Sia \(x_0 \in U\), per le ipotesi su \(U\) possiamo trovare \(r_0 > 0\) tale che
  \[
    \overline {B_{r_0}(x_0)} \subset U.
  \]
  Per densità di \(A_1\) possiamo trovare un certo \(x_1\) tale che
  \[
    x_1 \in A_1 \cap B_{r_0}(x_0), 
  \]
  e un \(r_1 > 0\) tale che 
  \[
    \begin{cases}
      \overline{B_{r_1}(x_1)} \subset B_{r_0}(x_0) \cap A_1, \\
      0 < r_1 < \frac{r_0}{2}.
    \end{cases}
  \]
  Procedendo induttivamente possiamo costruire due successioni, \(\{x_n\} \subset X\) e \(\{r_n\} \subset \mathbb R^+\), tali che per ogni \(n\) valga che
  \[
    \begin{dcases}
      \overline{B_{r_n}(x_n)} \subset B_{r_{n-1}}(x_{n-1}) \cap A_n, \\
      0 < r_n < \frac{r_{n-1}}{2}.
    \end{dcases}
  \]
  Chiaramente la successione \(r_n\) converge a \(0\). Inoltre la successione \(\{x_n\}\) è di Cauchy, infatti per \(m > n\) vale che
  \[
  \begin{aligned}
      d(x_n, x_m) &\le d(x_n, x_{n+1}) + d(x_{n+1}, x_{n+2}) + \ldots + d(x_{m-1}, x_m) \\
      &< r_n + r_{n+1} + \ldots + r_{m-1} < r_n + \frac{r_n}{2} + \frac{r_n}{4} + \ldots = 2 r_n,
  \end{aligned}
  \]
  che è infinitesima. Per completezza dello spazio \(X\) esiste \(\overline x \in X\) tale che \(x_n \to \overline x\). Vorremmo mostrare che 
  \[
    \overline x \in U \cap \bigcap_{n} A_n.
  \]
  Siccome, per ogni \(n, p \geq 0\) naturali vale che 
  \[
    x_{n + p} \in B_{r_n}(x_n) 
  \]
  si può affermare, passando al limite per \(p \to +\infty\), che
  \[
    \overline x \in \overline{B_{r_n}(x_n)} \quad \forall n \geq 0.
  \]
  Quindi, per la costruzione delle palle, si ha che
  \[
    \overline x \in A_n \quad \forall n \geq 0.
  \]
  Inoltre, per la costruzione iniziale, si ha che
  \[
    \overline x \in \overline{B_{r_0}(x_0)} \subset U.
  \]  
  Questo conclude la dimostrazione. 
\end{proof}

\begin{theorem}[Banach \-- Steinhaus]\label{thm:banach-steinhaus}
    Sia \(X\) uno spazio di Banach, \( Y\) uno spazio normato. Sia \(\{T_{i}\}_{i \in I} \) 
    una famiglia di operatori lineari e continui \(T_{i} : X \to Y\).

    Allora se \(\forall x \in X\) esiste \(m {(x)}\) tale che \(\|T_{i}{(x)}\|_Y \le m {(x)}\) per ogni \(i \in I\) vale che 
    \[
      \forall x \in X \,\, \exists m {(x)} : \sup_{i \in I}\|T_{i} x \|_Y \le  m {(x)} \implies \sup_{i \in I} \|T_{i}\|_{\mathcal{L}{(X, Y)}} < +\infty
    \]
    in altre parole \emph{puntualmente limitato} implica \emph{uniformemente
    limitato}
\end{theorem}
\begin{proof}
    Costruiamo 
    \[
      X_{n} = \{x \in X : \|T_{i}x\| \le n \,\, \forall i \in I\}
    \]
    Allora
\begin{enumerate}[label = \arabic*.]
    \item \(\forall n\), \(X_{n}\) è chiuso (controimmagine di chiuso tramite la
        composizione delle funzioni continue \(T_{i}\) e \(\|\cdot \|\))
    \item \(X = \bigcup_{n} X_{n}\)

        Vogliamo mostrare che data \(x \in X\), \(\exists \overline{n} : x \in X_{\overline{n}}\). Sappiamo che \(\exists m {(x)} : \|T_{i}{(x)}\| \le m {(x)}\) per ogni \(i \in I\), e allora basta prendere \(\overline{n} \ge m {(x)}\).

    Ora abbiamo le condizioni del lemma di Baire, e possiamo dunque dire che
    esiste \(n_{0} : \mathring{X}_{n_{0}} \neq \varnothing\). Prendo dunque \(x_{0} \in \mathring{X}_{n_0}\).
    Esiste dunque \(\delta > 0 : B_{\delta} {(x_{0})} \subseteq X_{n_0}\). In
    particolare \(x_{0} + \frac{x}{\|x\|}\delta \in X_{n_{0}} \) per ogni \(x \neq 0\). Allora dalla definizione degli \(X_{n}\) segue che
    \[
      \left\|T_{i} {\left(x_{0} + \delta \frac{x}{\|x\|}\right)}\right\| \le n_{0}
    \]
    ora per linearità e proprietà della norma abbiamo che
    \[
      \delta \left\| T_{i} \frac{x}{\|x\|}\right\| \le \|T_{i} x_{0}\| + \left\|T_i {\left( x_{0} + \delta \frac{x}{\|x\|} \right)} \right\| \le n_{0} + n_{0} = 2n_{0}
    \]

    Finalmente concludiamo che
    \[
        \|T_{i}x\|\le {\left( \frac{2n_{0}}{\delta} \right)} \|x\| \implies \sup_{x \in X \sminus \{0\} } \frac{\|T_{i}x\|}{\|x\|} \le \frac{2n_{0}}{\delta}
    \]
\end{enumerate}

\end{proof}

\begin{corollary}
  Sia \(X\) uno spazio di Banach, e sia \(Y\) uno spazio normato. Si consideri una successione di operatori lineari e continui \(\{T_n\} \subset \mathcal L(X, Y)\). Si supponga che, per ogni \(x \in X\), esista \(y \in Y\) tale che 
  \[
    \lim_{n \to +\infty} \|T_n x - y\| = 0.
  \]
  Detta \(T : X \to Y\) l'applicazione che associa, ad ogni \(x \in X\), il rispettivo limite in \(Y\), si può dire che:
  \begin{enumerate}
    \item \(T \in \mathcal L(X, Y)\); 
    \item \(\sup_{n \in \mathbb N} \|T_n\|_{\mathcal L(X, Y)} < +\infty\);
    \item \(\|T\|_{\mathcal L(X, Y)} \le \liminf_{n \to +\infty} \|T_n\|_{\mathcal L(X, Y)}\). 
  \end{enumerate}
\end{corollary}

\begin{proof}
  La linearità di \(T\) segue direttamente dalla linearità degli operatori \(T_n\) e dalle proprietà del limite.

  Mostriamo il secondo punto. Per applicare il Teorema \ref{thm:banach-steinhaus}, dobbiamo verificare che la famiglia di operatori \(\{T_n\}\) sia puntualmente limitata. Fissiamo un arbitrario \(x \in X\). Per ipotesi, la successione \(\{T_n x\}_{n \in \mathbb{N}}\) converge in \(Y\). Ogni successione convergente in uno spazio normato è limitata. Pertanto, esiste una costante \(m(x) > 0\) tale che
  \[
    \|T_n x \|_Y \leq m(x), \quad \forall n \in \mathbb N.
  \]
  Dato che questo vale per ogni \(x \in X\), le ipotesi del Teorema \ref{thm:banach-steinhaus} sono soddisfatte. Ne consegue che la successione delle norme degli operatori è uniformemente limitata, ossia
  \[
    M := \sup_{n \in \mathbb N} \|T_n\|_{\mathcal L(X, Y)} < +\infty.
  \]

  Mostriamo il primo punto. La linearità è già stata mostrata. Per la continuità, dobbiamo mostrare che \(T\) è un operatore limitato. Usando il risultato del punto precedente, sappiamo che \(\|T_n\|_{\mathcal L(X, Y)} \le M\) per ogni \(n\). Dunque, per ogni \(x \in X\), vale
  \[
    \|T_n x\|_Y \le \|T_n\|_{\mathcal L(X, Y)} \|x\|_X \le M \|x\|_X.
  \]
  Poiché \(T_n x \to T x\) e la norma è una funzione continua, si ha \(\|T_n x\|_Y \to \|T x\|_Y\). Passando al limite per \(n \to \infty\) nella disuguaglianza precedente, otteniamo:
  \[
    \|T x\|_Y = \lim_{n \to \infty} \|T_n x\|_Y \le M \|x\|_X.
  \]
  Questo dimostra che \(T\) è limitato (con \(\|T\| \le M\)), e quindi continuo.

  Mostriamo il terzo punto. Partiamo dalla disuguaglianza \(\|T_n x\|_Y \le \|T_n\|_{\mathcal L(X, Y)} \|x\|_X\), valida per ogni \(n \in \mathbb{N}\) e ogni \(x \in X\). Prendendo il limite inferiore per \(n \to +\infty\) in entrambi i membri, otteniamo:
  \[
      \liminf_{n \to +\infty} \|T_n x\|_Y \le \liminf_{n \to +\infty} \left( \|T_n\|_{\mathcal L(X, Y)} \|x\|_X \right) = \left(\liminf_{n \to +\infty} \|T_n\|_{\mathcal L(X, Y)}\right) \|x\|_X.
  \]
  Dato che la successione \(\|T_n x\|_Y\) converge a \(\|T x\|_Y\), il suo limite inferiore coincide con il limite. Sostituendo nel membro di sinistra, si ha:
  \[
      \|T x\|_Y \le \left(\liminf_{n \to +\infty} \|T_n\|_{\mathcal L(X, Y)}\right) \|x\|_X.
  \]
  Poiché questa disuguaglianza vale per ogni \(x \in X\), per definizione di norma di un operatore si conclude che
  \[
      \|T\|_{\mathcal L(X, Y)} \le \liminf_{n \to +\infty} \|T_n\|_{\mathcal L(X, Y)}.
  \]
\end{proof}
\begin{example}[Disugaglianza stretta nel punto 3]
  Si consideri \(X = Y = \ell^2\). Si definiscano gli operatori
  \[
    T_n : \ell^2 \to \ell^2, \quad (T_n x)(k) = \begin{cases}
      x(k), & k = n, \\
      0, & k \neq n. \end{cases}
  \]
  Gli operatori sono lineari, e si osserva che sono anche limitati di norma unitaria. L'operatore limite, \(T\), manda ogni \(x \in \ell^2\) nell'elemento nullo di \(\ell^2\). Quindi vale la disugaglianza stretta
  \[
      \|T\|_{\mathcal L(\ell^2, \ell^2)} = 0 < \liminf_{n \to +\infty} \|T_n\|_{\mathcal L(\ell^2, \ell^2)} = 1.
  \]
\end{example}
