\begin{eser}{}
    Sia \(Y \subseteq X \) un sottospazio di \(X\) spazio normato. Mostrare che
    \(\overline{Y}\) è un sottospazio di \(X\).
\end{eser}
\begin{lemmao}{}
    Sia \(X\) uno spazio normato e \(Y \subseteq X \) un sottospazio tale che
    \(\overline{Y} \subset X \). Allora \(\exists f : X \to \mathbb{K}\) con \(f\) lineare continua, ossia \(f \in X'\) tale che:
\begin{enumerate}[label = \arabic*.]
    \item \(f \neq 0\) 
    \item \(\Span{f, x} = 0\) per ogni \(x \in Y\) 
\end{enumerate}
\end{lemmao}
\begin{remark}{}
    Sia \(X\) uno spazio normato, \(Y \subseteq X \) un sottospazio si supponga
    che se un funzionale \(f \in X'\) tale che \(\Span{f, x} = 0\) per ogni \(x \in Y\) allora necessariamente \(f = 0\). Segue che \(\overline{Y} = X\) 
\end{remark}
\begin{proof}{}
    \(\exists x_{0} \in X - \overline{Y}\), allora \(X_{0} = Y \oplus \mathbb{K} x_{0}\). A questo punto prendiamo il funzionale \(g : X_{0} \to \mathbb{K}\) definito da \(g{(y + \alpha x_{0})} = \alpha\). Mostriamo ora che \(g \in X_{0}'\) e che è vero che \(g|_Y = 0\). La seconda è banalmente vera perché se \(y \in Y\) allora \(g{(y)} = g{(y + 0*x_{0})} = 0\). Mostriamo che \(g\) è lineare e continuo. 
    Supponiamo \(x_{1} = y_{1} + \alpha_{1} x_{0}\) e \(x_{2} = y_{2} + \alpha_{2} x_{0}\). Allora
    \begin{align*}
        g{(\lambda x_{1} + \mu x_{2})} &= g{(\lambda y_{1} + \lambda \alpha_{1} x_{0} + \mu y_{2} + \mu \alpha_{2} x_{0})} = \\
        &= g{({(\lambda y_{1} + \mu y_{2})} + {(\lambda \alpha_{1} + \mu \alpha_{2})} x_{0})} = \lambda \alpha_{1} + \mu \alpha_{2} = \\
        &= \lambda g{(x_{1})} + \mu g{(x_{2})}
    \end{align*}
    Per la continuità, prendiamo \(\alpha \neq 0\) e abbiamo che
    \[
      \|x\| = \| y + \alpha x_{0}\| = \left\|{(-\alpha)}{\left(  \frac{y}{-\alpha} - x_{0} \right)} \right\| = | \alpha | \left\| \frac{y}{-\alpha} - x_{0}\right\|
    \]
    necessariamente \(\frac{y}{-\alpha} \in Y\) e dunque
    possiamo proseguire la precedente equazione con 
    \[
      \|x\| = | \alpha | \left\| \frac{y}{-\alpha} - x_{0}\right\| \ge
      |\alpha|\, d{(x_{0}, Y)} = |g{(x)}| \,d(x_{0}, Y)
    \]
    per cui concludiamo che \(g\) è continua con norma \(\|g\|\le 1 / {d{(x_{0},
    Y)}}\). Questa disuguglianza è in realtà un'uguaglianza, infatti
    poiché \(d{(x_{0}, {Y})} = \inf_{y \in {Y}} \|x_{0} - y\|\) abbiamo che
    \[
      \exists  y_{n} \in {Y} : \|y_{n} - x_{0}\| < \frac{n+1}{n} d{(x_{0}, {Y})}
    \]
    e ora abbiamo che
    \[
      \frac{n}{n+1} \frac{\|y_{n} - x_{0}\|}{d{(x_{0}, Y)}} < 1 = | g{(x_{0} -
      y_{n})} | \le \|g\| \|x_{0} - y_{n}\|
    \]
    da cui per \(n \to \infty\) otteniamo \(\|g\|_{X_{0}'} \ge 1 / d{(x_{0}, Y)}\).

    Ora estendo \(g\) a tutto \(X\) con Hahn-Banach ottenendo \(f \in X'\) tale
    che \(f|_{X_{0}}  = g \) e dunque \(f|_Y = 0\). Inoltre l'estensione poiché
    Hahn-Banach conserva la norma, abbiamo che
    \[
      \|f\|_{X'} = \frac{1}{d{(x_{0}, Y)}}
    \]
\end{proof}
\begin{corollary}{}
    Sia \(X\) uno spazio normato reale. Allora per ogni \(x_{0} \in X\) esiste una \(f \in X'\) tale che \(\Span{f, x_{0}} = \|x_{0}\|^2\) e \(\|f\|_{X'} = \|x_{0}\|\) 
\end{corollary}
\begin{proof}{}
    Sia \(X_{0} = \mathbb{R}x_{0}\). Sia \(x = tx_{0} \in X_{0}\), allora
    definiamo \(g{(x)} = g{(tx_{0})} = t\|x_{0}\|^2\). Verifichiamo che la norma
    sia corretta: \(|g{(tx_{0})}| = |t| \|x_{0}\|^2\) dunque \(\|g\|_{X_{0}'} = \|x_{0}\| \).

    Per Hahn-Banach possiamo estendere \(g\) a tutto \(X'\) ottenendo \(f : X \to \mathbb{R}\) tale che \(\Span{f, x} = \Span{g, x} \) per ogni \(x \in X_{0}\) e \(\|f\|_{X'}  = \|g\|_{X_{0}'} =\|x_{0}\| \). In particolare anche \(\Span{f, x_{0}}  = \Span{g, x_{0}}  = \|x_{0}\|^2\) 
\end{proof}

Il corollario precedente motiva la seguente definizione:
\begin{definition}{Mappa di dualità}
    Chiamiamo la \textbf{mappa di dualità} la seguente funzione
    \begin{align*}
        \mathcal{F}: X &\longrightarrow 2^{X'} \\
        x &\longmapsto \mathcal{F}(x) = \{f \in X' : \Span{f, x} =
        \|x\|^2 \,\, ; \,\, \|f\|_{X'} = \|x\|\} 
    \end{align*}
    che associa a ogni elemento di \(X\) l'insieme degli elementi ``a lui duali''.
\end{definition}

\begin{eser}{}
    Consideriamo 
    \[
      \mathcal{F}'{(x)} = \{f \in X' : \Span{f, x} = \|x \|^2 \,\, ; \,\,
      \|f\|_{X'} \le \|x\|\} '
    \]
    Mostrare che \(\mathcal{F}' = \mathcal{F}\) 
    \tcblower
    Fissato \(x \in X\), è evidente che \(\mathcal{F}{(x)} \subseteq \mathcal{F}'{(x)} \). Supponiamo ora che \(f \in \mathcal{F}'{(x)}\), ossia \(\|f\|_{X'} \le \|x\|\). Da \(|\Span{f, x}| = \|x\| \|x\| \) concludiamo che \(\|f\|_{X'} = \|x\|\) e dunque \(f \in \mathcal{F}{(x)}\) 
\end{eser}

\begin{eser}{}
    Consideriamo
    \[
      \mathcal{I}{(x)} = \left\{f \in X' : \frac{1}{2}\|y\|^2 - \frac{1}{2}\|x\|^2 \ge \Span{f, y-x} \,\, \forall y \in X \right\} 
    \]
    Mostrare che \(\mathcal{I} = \mathcal{F}\) 
    \tcblower
    Fissiamo \(x \in X\) 
    \begin{itemize}
        \item[\(\subseteq \)] Sia \(f \in \mathcal{I}{(x)}\). Iniziamo mostrando
            che \(\Span{f, x} = \|x\|^2\). Scegliamo \(y = \alpha x \) per \(\alpha \in \mathbb{R}\). Segue che
            \[
              \frac{1}{2}\alpha^2 \|x\|^2 - \frac{1}{2}\|x\|^2 \ge \Span{f, x}
              {(\alpha - 1)}
            \]
            con questa uguaglianza, dividendo i casi per \(
            \alpha > 0 \) e \(\alpha < 1\), prendiamo il limite di \(\alpha \to 1^{+}\) e \(\alpha \to 1^{-}\), ottenendo le due disuguaglianza \(\Span{f, x} \le \|x\|^2\) e \(\Span{f, x} \ge \|x\|^2\).

            Rimane da controllare che \(\|f\|_{X'} \le \|x\|\). Scegliamo \(y \in X\) tale che \(\|y\| = \|x\|\). Otteniamo che
            \[
              \Span{f, y}  \le  \Span{f, x}  = \|x\|^2 \implies |\Span{f, y}| \le
              \|y\|\|x\| \implies \|f\|_{X'} \le \|x\|
            \]
        \item[\(\supseteq \)] Sia \(f \in \mathcal{F}{(x)}\) e \(y \in X\). Allora 
            \begin{align*}
                \Span{f, y - x} &= \Span{f, y}  - \Span{f, x} \le \|f\|\|y\| - \|x\|^2 \le \frac{1}{2} \|f\|^2 + \frac{1}{2}\|y\| ^2 - \|x\|^2 \\ &\le  \frac{1}{2}\|y\|^2 - \frac{1}{2}\|x\|^2
            \end{align*}
            da cui \(f \in \mathcal{I}{(x)}\) 
    \end{itemize}
\end{eser}
\begin{remark}{}
    Il precedente esercizio suggerisce che \(f \in \mathcal{F}{(x)}\) svolge in
    un certo senso il ruolo della derivata di \(\varphi {(x)} = \frac{1}{2}\|x\|^2\)  valutata in \(x\). Vedremo più avanti il significato di questa analogia.
\end{remark}
\begin{eser}
    Mostrare che
    \begin{align*}
        c_{0} &= \{x : \mathbb{N} \to \mathbb{R} : \lim_{n \to \infty} x{(n)} = 0\} \\ 
        c &= \{x \in \ell^{\infty} : \lim_{n \to \infty} x{(n)} \text{ esiste }\} 
    \end{align*}
    sono sottospazi chiusi di \(\ell^{\infty}\).
    \tcblower
    Le dimostrazioni contose esplicite sono lasciate davvero come esercizio,
    riporto dimostrazioni più sintetiche.

    Utilizzando la \(f\) definita come il limite come poco più avanti (dopo il
    teorema), abbiamo che \(c_{0}\) è chiuso in quanto \(c_{0} = f^{-1}{(\{0\} )}\)
    controimmagine continua di chiuso.
\end{eser}


\begin{theorem}{}
    Sia \(p \in [1, +\infty)\), sia \(f \in {(\ell^{p})}'\). Sia \(q \in \mathbb{R} \) tale che \(\frac{1}{p} + \frac{1}{q} = 1\). Allora
    \[
      \exists ! y \in \ell^{q} : \Span{f, x} = \sum_{n=1}^{\infty} x{(n)}y{(n)} 
    \]
    e inoltre \(\|f\|_{{(\ell^{p})}'} = \|y\|_{\ell^{q}}  \) 
\end{theorem}
Notare che il precedente teorema non vale per \(p = \infty\). Costruiamo infatti
un funzionale lineare e continuo su \(\ell^{\infty}\) che non si rappresenta con
\(y \in \ell^{1}\). Consideriamo infatti
    \(g : c \to \mathbb{R}\) la funzione definita da \(\Span{g, x} = \lim_{n \to \infty} x{(n)}\). Allora \(g\) è lineare ed è continuo perché \(|\Span{g,x} | = | \lim_{n \to \infty} x{(n)} | \le \|x\|_{\infty} \) per cui \(g \in c'\) e in particolare \(\|g\|_{c'} = 1\) ad esempio prendendo la successione \(x \in c\) definita da \(x{(n)} = 1\).

    Estendo ora \(g\) a tutto \(\ell^{\infty}\) con Hahn-Banach, ottenendo \(f :
    \ell^{\infty} \to \mathbb{R}\) lineare continuo con \(\|f\|_{(\ell^{\infty})'} = 1 \).

    Supponiamo ora per assurdo che esista \(y \in \ell^{1}\) tale che
    \[
      \Span{f, x} = \sum_{n=1}^{\infty} y{(n)}x{(n)} \quad \forall x \in \ell^{\infty} 
    \]
    e consideriamo ora gli \(x_k\) definiti come\footnote{concedetemi questa notazione da informatico} \(x_k = {(n == k)}\). Allora abbiamo \(\Span{f, x_k} = \lim_{n \to \infty} x_k{(n)} = 0\) ma per ogni \(k\) allora avremmo che \\ \(0 = \Span{f, x_k} = \sum_{n=1}^{\infty} y{(n)}x_k{(n)} = y{(k)} \) per cui \(y = 0\) che è impossibile perché sappiamo che \(f\) ha norma 1.



\begin{eser}{}
    Mostrare che
    \[
        c_{00} = \{x : \mathbb{N}\to \mathbb{R} \text{ definitivamente nulle}\} 
    \]
    è denso in \(\ell^{p}\), per ogni \(p \in [1, \infty)\) 
\end{eser}

\begin{eser}{}
Mostrare che \(T : \ell^2 \to \ell^2\) dato da 
\[
  {(Tx)}{(n)} = \frac{x{(n)}}{n}
\]
è ben definito, lineare, continuo e \(T{(\ell^2)}\) non è chiuso in \(\ell^2\)
ed è denso in \(\ell^2\) 
    
\end{eser}\label{eser:Txn}

\begin{definition}{Immersione Compatta}
    Siano \(X, Y\) spazi di Banach. Dico che \(Y\) è immerso con compattezza in
    \(X\) (indicato \(Y \subset \subset  X\) ) se 
\begin{enumerate}[label = \arabic*.]
    \item \(\exists C > 0 : \|x\|_X \le C\|x\|Y\) per ogni \(x \in Y\) (dunque
        l'immersione \(Y \hookrightarrow X\) è continua.
    \item Ogni successione limitata in \(Y\) ha un'estratta convergente in \(X\) 
\end{enumerate}
\end{definition}

\begin{eser}{}
    Sia \(X = \ell^2\) e consideriamo 
    \[
      Y = \left\{x : \mathbb{N}\to \mathbb{R} : \sum_{n=1}^{\infty} n^2 |x{(n)}|^2 < +\infty \right\} 
    \]
    Mostrare che
\begin{enumerate}[label = \arabic*.]
    \item \(Y\) è sottospazio vettoriale di \(\ell^2\)
    \item Posto \(\|x\|^2_Y := \sum_{n=1}^{\infty} n^2 |x{(n)}|^2 \), questa è una norma indotta da un prodotto scalare
    \item L'inclusione \(Y \hookrightarrow X\) è continua
    \item \(Y\) è completo
    \item \(Y \subset \subset X  \) 
\end{enumerate}
\tcblower
\begin{enumerate}[label = \arabic*.]
    \item preso \(x \in Y\), \(|x{(n)}|^2 \le n^2 |x{(n)}|^2\) e poiché \(\{nx{(n)}\}_{n \in \mathbb{N}}  \in \ell^2\) allora anche \(x \in \ell^2\). Facile verificare che \(Y\) è sottospazio
    \item Tutte facili verifiche, con prodotto scalare \(\Span{x,y}_Y = \sum_{n=1}^{\infty} n^2 x{(n)}y{(n)} \) 
    \item Come visto nel punto 1., \(\|x\|_{\ell^2} \le \|\{nx{(n)}\}_{n \in \mathbb{N}} \|_{\ell^2} = \|x\|_Y \) 
    \item Sia \(\{x_k\}_{k \in \mathbb{N}}  \) di Cauchy in \(\|\cdot \|_Y\).
        Vogliamo mostrare che \(x_k \to \overline{x}\) in \(Y\).
        Poiché \(\|x_n - x_m\|_{\ell^2} \le \|x_n - x_{m}\|_Y\) allora esiste \(\overline{x} \in \ell^2\) tale che \(x_k \to \overline{x}\) in \(\ell^2\) per completezza di \(\ell^2\).
        Poste \(y_k{(n)} = nx_k{(n)} \), evidentemente \(y_k\) è di Cauchy in \(\ell^2\), dunque esiste \(
        \overline{y} \in \ell^2\) tale che \(y_k \to \overline{y}\). Vogliamo
        ora mostrare che \(\overline{y}{(n)} = n \overline{x}{(n)}\). Questo si
        può dire perché la convergenza in \(\ell^2\) implica la convergenza
        puntuale, e per ogni \(k \in \mathbb{N}\) si ha che \(y_k{(n)} = n x_k{(n)}\)
    \item Sia \(\{x_k\} \) limitata in \(Y\). Allora \(\exists M > 0:
        \|x_k\|_Y^2 \le M\). Vogliamo trovare una sottosuccessione \(\{ x_{k_{j}}\} \subseteq \{x_k\} \) tale che \(x_{k_{j}} \overset{\ell^2}{\to } \overline{x} \in \ell^2\). Ora usando un risultato che ancora non abbiamo dimostrato, la \textbf{compattezza debole}, diciamo che \(\exists \overline{x} \in \ell^2\) e una sottosuccessione tale che
        \[
          \Span{y, x_{k_{j}} } \to \Span{y, \overline{x}} \quad \forall y \in
          \ell^2
        \]
    (reindicizziamo per comodità i \(k\) a indicare \(k_j\), per alleggerire la
    notazione)
    Fisso \(n \in \mathbb{N}\) e prendo \(y{(i)} = {(i == n)}\). Allora
    otteniamo dalla precedente che \(\Span{y, x_k } = x_k{(n)} \to \overline{x}{(n)} = \Span{y, \overline{x}} \). Vogliamo ora mostrare che la convergenza è in \(\ell^2\) 
    \begin{align*}
        \|x_k - \overline{x}\|^2_{\ell^2} &= \sum_{n=1}^{\infty} |x_k{(n)} - \overline{x}{(n)}|^2 = \sum_{n=1}^{m} |x_k{(n)} - \overline{x}{(n)}|^2 + \\ &+ \sum_{n=m+1}^{\infty} \frac{1}{n^2} n^2 | x_k{(n)} - \overline{x}{(n)}|^2
    \end{align*}
    osservo ora che \(n^2 | x_k{(n)} - \overline{x}{(n)}|^2 \le  n^2 {( | x_k{(n)}| + |\overline{x}{(n)}| )}^2 \le  2 n^2 |x_k{(n)}|^2 + 2n^2 |\overline{x}{(n)}|^2\) 

    Prima di proseguire vogliamo dire che \(\overline{x} \in Y\). Abbiamo che
    \[
      nx_k{(n)} \to n\overline{x}{(n)} \implies n^2 |x_k{(n)}|^2 \to n^2 |\overline{x}{(n)}|^2 \quad \forall n \in \mathbb{N}
    \]
    per il lemma di Fatou, abbiamo che
    \[
      \sum_{n=1}^{\infty} n^2 | \overline{x}{(n)}|^2 \le \liminf_{k \to \infty}
      \sum_{n=1}^{\infty} n^2 | x_k{(n)}|^2 \le M
    \]
    possiamo ora proseguire la disuguaglianza precedente
    \begin{align*}
        \|x_k - \overline{x}\|^2_{\ell^2} &= \sum_{n=1}^{m} |x_k{(n)} - \overline{x}{(n)}|^2 +  \sum_{n=m+1}^{\infty} \frac{1}{n^2} n^2 | x_k{(n)} - \overline{x}{(n)}|^2 \\ &\le \sum_{n=1}^{m} |x_k{(n)} - \overline{x}{(n)}|^2 + \frac{4}{{(m+1)}^2 } M 
    \end{align*}

    Infine fisso \(\varepsilon > 0\) e prendo \(m \in \mathbb{N} : \frac{4M}{{(m+1)}^2} < \frac{\varepsilon}{2}\) e \(\overline{k} = \overline{k}{(\varepsilon, m)}\) tale che anche la somma troncata del primo addendo sia minore di \(\frac{\varepsilon}{2}\). Concludiamo che
    \[
      \|x_k - \overline{x}\|^2_{\ell^2} < \varepsilon
    \]
    e dunque \(x_k \to \overline{x}\) in \(\ell^2\) 
\end{enumerate}
\label{eser:Y}
\end{eser}
Nell'esercizio precedente abbiamo che similmente si comporterebbe anche
\[
  Y_{\alpha}  = \left\{x: \mathbb{N}\to \mathbb{R} : \sum_{n=1}^{\infty} n^{2 \alpha} |x{(n)}|^2 < +\infty  \right\} \text{ con } \alpha \in (0, 1)
\]

Riprendendo l'operatore \(T\) definito nell'esercizio~\ref{eser:Txn}, abbiamo
che \(T{(\ell^2)} \neq \ell^2\). Poiché \(T\) è iniettivo, definiamo \(A = T^{-1}\) come \({(Ax)}{(n)} = nx{(n)}\). Ovviamente il dominio di \(A\) non è tutto \(\ell^2\), ma
\begin{align*}
    A&: D{(A)} \longrightarrow \ell^2 \\
    D{(A)} &= \left\{ x \in \ell^2 : \sum_{n=1}^{\infty} n^2 | x{(n)}|^2 < +\infty  \right\} 
\end{align*}
ossia \(Y\) dell'esercizio~\ref{eser:Y}. Ma allora \(A\) è lineare ma non
limitato, infatti
\[
  \|Ax\|^2_{\ell^2} = \|x\|^2_Y \not \le  C \|x\|_{\ell^2} 
\]
\begin{corollary}{}
    Sia \(X\) uno spazio normato. Allora
    \[
      \|x\| = \sup_{f \in X' ; \|f\|_{X'} \le 1 } |\Span{f, x}|
    \]
    che è in realtà un massimo
\end{corollary}
\begin{proof}{}
    Prendo \(x\neq 0\), \(|\Span{f, x} | \le  \|x\|\), \(\forall f \in X'\) e
    quindi anche
    \[
      \sup_{f : \|f\|_{X'} \le 1 } |\Span{f, x} | \le  \|x\|
    \]
    preso ora \(
    f \in \mathcal{F}{(x)}\), abbiamo che \(\Span{f, x} = \|x\|^2\) e \(\|f\|_{X'} = \|x\|\). Prendiamo ora \(f_{1} = \frac{f}{\|x\|}\) e dunque \(\|f_{1}\|_{X'} = 1\) e \(\Span{f_{1}, x} = \|x\| \) 
    ne consegue che il \(\sup\) è un \(\max\) ed è raggiunto da \(f_{1}\) 
\end{proof}

\begin{definition}{Stretta convessità}
    Sia \({(X, \|\cdot \|)}\) uno spazio normato. Allora \({(X, \|\cdot \|)}\) è
    \textbf{strettamente convesso} se, dati \(
    x, y \in X\) 
    \[
      x\neq y \text{ e }\|x\| = \|y\| = 1 \implies \left\|\frac{x+y}{2}\right\| < 1
    \]
\end{definition}
\begin{example}{}
    \(\mathbb{R}^2\) è strettamente convesso in norma \(p\) per \(p \in (1, \infty)\) 
\end{example}
\begin{example}[Spoiler]
    Gli spazi di Hilbert sono strettamente convessi
\end{example}
\begin{proposition}{Unicità in Hahn-Banach}
    Sia \(X\) uno spazio normato tale che \(X'\) sia strettamente convesso.
    Allora dato \(X_{0} \subseteq X \) sottospazio e \(g \in X_{0}'\), 
    \[
      \exists ! f \in X' : f \text{ estende } g \text{ ; } \|f\|_{X'}  = \|g\|_{X_{0}'} 
    \]
\end{proposition}
\begin{proof}{}
    Siano \(f_{1}\) e \(f_{2}\) due estensioni di \(g\). Se \(g \equiv 0\)
    allora necessariamente \(f_{1} = f_{2} \equiv 0\).

    Assumo che \(\|g\|_{X_{0}'} = \|f_{1}\|_{X'}  = \|f_{2}\|_{X'} = 1\). Allora
    \[
      \frac{f_{1}+f_{2}}{2} \big|_{X_{0}'} = g \implies \left\|
      \frac{f_{1} + f_{2}}{2}\right\| \ge \|g\|_{X_{0}'} = 1
    \]
    allora dalla contropositiva della stretta convessità, concludiamo che \(f_{1} = f_{2}\) 
\end{proof}

\begin{definition}{Spazio Separabile}
    \(X\) spazio metrico è detto \textbf{separabile} se esiste \(D \mathbb{C} X\) tale che
\begin{enumerate}[label = \arabic*.]
    \item \(D\) è numerabile
    \item \(D\) è denso in \(X\) 
\end{enumerate}
\end{definition}
\begin{proposition}{}
    Se \(X\) è separabile e \(M_{0} \subseteq X \) allora \(M_{0}\) è
    separabile.
\end{proposition}
\begin{proof}{}
    \(M_{0} \cap D\) è numerabile e denso in \(M_{0}\) 
\end{proof}
\begin{theorem}{}
    Sia \(X\) uno spazio normato tale che \(X'\) è separabile. Allora \(X\) è
    separabile.
\end{theorem}
\begin{proof}{}
    Sia \(\{f_{n}\}_{n \in \mathbb{N}} \) numerabile e chiuso in \(X'\). Allora
    \[
    \exists \{x_{n}\}_{n \in \mathbb{N}}, \,\, x_{n} \in X : |\Span{f_{n}, x_{n}} | \ge \frac{1}{2}\|f_{n}\|_{X'} 
    \]
    per la definizione di norma duale.

    Assumo momentaneamente che \(\mathbb{K} = \mathbb{R}\), allora
    \[
      D = \left\{x \in X : x=\sum_{k=1}^{n} \alpha_k x_k , \quad n \in \mathbb{N}, \quad \alpha_k \in \mathbb{Q}\right\} 
    \]
    che è numerabile in quanto unione numerabile di numerabili (insiemi \\\(\{\sum_{i=1}^{n} \alpha_k x_k , \,\, \alpha_k \in \mathbb{Q}  \} \hookrightarrow \mathbb{Q}^{n} \) per \(n\) fissato).

    Mostriamo ora la densità. Consideriamo l'insieme 
    \[
      D = \left\{x \in X : x=\sum_{k=1}^{n} \alpha_k x_k , \quad n \in \mathbb{N}, \quad \alpha_k \in \mathbb{R}\right\} 
    \]
    e chiaramente \(\overline{D} = \overline{Y}\) dunque dobbiamo solo
    dimostrare \(\overline{Y} = X\). Mostriamo la condizione equivalente che se \(f \in X'\) e
    \(f \equiv 0 \) su \(Y\), allora \(f \equiv 0\) su tutto \(X\). A tal scopo
    fissiamo \(\varepsilon > 0\) e troviamo \(f_{n} \in X'\) tale che (con \(\|x_{n}\| = 1\))
    \[
    \|f_{n} - f\|_{X'} \le \varepsilon \implies \frac{1}{2}\|f_{n}\|_{X'}  \le |
    \Span{f_{n}, x} | = | \Span{f_{n}-f,x_{n}} + \Span{f, x_{n}} | \le \|f_{n} -
    f\|_{X'} \le \varepsilon
    \]
    ma allora
    \[
      \|f\|_{X'}  \le \|f_{n}-f\|_{X'} + \|f_{n}\|_{X'} \le 3\varepsilon
    \]
    e per arbitrarietà di \(\varepsilon\) concludiamo che \(f \equiv 0\) su \(X\) 

    Il caso complesso è analogo, ma prendendo \(\alpha_k \in \mathbb{Q} \oplus i \mathbb{Q} \subseteq \mathbb{C}\).
\end{proof}

\begin{example}{}
    \(c_{00} \) è denso in \(\ell^{p}\) per \(p \in [1, +\infty)\). Allora preso
    \[
        D = \{x \in c_{00} : x{(n)} \in \mathbb{Q} \quad \forall n \in \mathbb{N}\} 
    \]
    si ha che \(\overline{D} = \overline{c_{00}} = \ell^{p}\) 

\end{example}

\begin{eser}{}
    Mostrare che \(c_{0}\) e \(c\) sono separabili con la \(\|\cdot \|_{\infty}  \) 
\end{eser}

\begin{proposition}{}
    \(\ell^{\infty}\) \textbf{non} è separabile
\end{proposition}
\begin{proof}{}
    Vogliamo mostrare che se \(D\) è un sottoinsieme numerabile di \(\ell^{\infty}\) allora non può essere denso. Sia \(D = \{y_{n}\}_{n \in \mathbb{N}} \). Ora consideriamo \(
    x : \mathbb{N} \to \mathbb{R}\) dato da
    \[
      x{(n)} = \begin{cases}{}
          1 + y_n{(n)} & |y_n{(n)}| \le 1 \\
          0 & \text{ altrimenti }
      \end{cases}
    \]
    e dunque chiaramente \(\|x\|_{\infty}  \le 2\) e in particolare \(x \in \ell^{\infty}\) ma allora
    \[
      \|x - y_k\|_{\infty} = \sup_{n \in \mathbb{N}} |x{(n)} - y_k{(n)}| \ge
      |x{(k)} - y_k{(k)}| \ge 1
    \]
    dove l'ultima disuguaglianza è perché se
    \(|y_k{(k)}| \le 1 \implies |x{(k)} - y_k{(k)} | = 1\) e se
    \(|y_k{(k)}| > 1\) allora \(|x{(k)} - y_k{(k)}| = |y_k{(k)}|\).

    Concludiamo che \(D\) non può essere denso in \(\ell^{\infty}\) e dunque \(
    \ell^{\infty}\) non è separabile.
\end{proof}

\section{Forme geometriche di Hahn-Banach}
Ora supponiamo \( \mathbb{K} = \mathbb{R}\).

Sia \(X\) uno spazio normato e \(f \in X'\). Allora un iperpiano è definito come
\(\ker f\). Se vogliamo generalizzare a iperpiani non sottospazi vettoriali,
possiamo prendere, preso un \(\alpha \in \mathbb{R}\), lo spazio
\[
    H = \{x \in X : \Span{f, x} = \alpha\} =: [f = \alpha]
\]

Allora due insiemi \(A, B \in X\) sono separati in senso largo da \([f = \alpha]\) se
\begin{align*}
    f{(x)} \le \alpha & \forall x \in A \\
    f{(x)} \ge \alpha & \forall x \in B
\end{align*}
e dico che l'iperpiano \([f = \alpha]\) separa in senso stretto \(A\) e \(B\) se
esiste \(\varepsilon > 0\) tale che
\begin{align*}
    f{(x)} \le \alpha - \varepsilon & \forall x \in A \\
    f{(x)} \ge \alpha + \varepsilon & \forall x \in B
\end{align*}
\begin{theorem}[Hahn-Banach, prima forma geometrica]
    Sia \(X\) spazio normato, \(A, B \subseteq X \) convessi, non vuoti e
    disgiunti. Allora se \(A\) è aperto esiste \(f \in X'\) e \(\alpha \in \mathbb{R}\)  tale che \([f = \alpha]\) separa \(A\) e \(B\) in senso largo
\end{theorem}
\begin{remark}{}
    Non è migliorabile (avere senso stretto) neanche nel caso a dimensione
    finita. Ad esempio su \(\mathbb{R}\) posso avere \(A=\{x >0\} \) e \(B=\{ x \le 0\}\) che sono separati da \( \{0\} \) ma solo in senso largo.
\end{remark}
\begin{definition}{Funzionale di Minkowski}
    Sia \(X\) uno spazio normato. \(C\) aperto convesso che contiene lo 0. Sia
    \[
      p{(x)} = \inf \left\{ r > 0 : \frac{x}{r} \in C\right\}  \text{ il funzionale di
      Minkowski}
    \]
    viene anche detto \emph{gauge di \(C\)}.
\end{definition}
\begin{proof}[Buona definizione]
    \(p : X \to \mathbb{R}\) è ben definito. Poiché \(C\) è aperto e \(0 \in C\) , esiste \(\varepsilon > 0\) tale che \(B_\varepsilon {(0)} \subseteq C \), ossia \(\|x\| \le \varepsilon \implies x \in C\).
    Fissato ora \(x \in X\), allora preso \(r = \frac{\varepsilon}{\|x\|}\)
    abbiamo che \(\frac{\|x\|}{r} = \varepsilon \) e dunque \(\frac{x}{r} \in C\).

    Dunque l'insieme di cui si fa l'inf è non vuoto.
\end{proof}

\begin{lemmao}[Proprietà del funzionale di Minkowski] Il funzionale di Minkowski \(p\) ha diverse proprietà
\begin{enumerate}[label = \arabic*.]
    \item \(p {(\lambda x )} = \lambda p {(x)}\) per ogni \(x \in X\)  e per ogni
        \(\lambda > 0\) 
    \item \(p{(x + y)} \le p {(x)} + p {(y)}\) per ogni \(x, y \in X\) 
    \item \(\exists  m > 0 : 0 \le p {(x)} \le m \|x\|\) per ogni \(x \in X\) 
    \item \(C = \{ x \in X : p {( x)} <1\} \) 
\end{enumerate}
\end{lemmao}
\begin{proof}{}
\begin{enumerate}[label = \arabic*.]
    \item ovvio
    \item[3.] \[
          \exists R > 0 : \overline{B}_R {(0)} \subseteq C \text{, cioè } \forall x \in X \sminus \{0\}, \quad R \frac{x}{\|x\|} \in C
        \]
        dunque \(p {(x)} \le \|x\|/R\) per ogni \(x \in X\) 
    \item[4.] Sia \(x \in C\), trovo \(\varepsilon > 0\) tale che \({(1+\varepsilon)}x \in C\) poiché \(C\) è aperto. Allora \(p{(x)} \le \frac{1}{1+\varepsilon} < 1\).
        Viceversa se \(p {(x)} < 1\), allora esiste \(\alpha \in (0,1)\) tale
        che \(\frac{x}{\alpha} \in C\). Ma allora per la convessità di \(C\) e
        poiché \(0 \in C\), anche \(\alpha \frac{x}{\alpha} + {(1-\alpha)}x \in C\) e dunque \(x \in C\) 
    \item[2.] Prendo \(x,y \in X\) e \(\varepsilon > 0\). Allora
        \[
          \frac{x}{p{(x)} + \varepsilon}, \frac{y}{p{(y)} + \varepsilon} \in C
        \]
        (infatti \(p{(x)} + \varepsilon > p {(x)}\)). Ora poiché \(
        C\) è convesso, \(\forall t \in (0, 1)\) 
        \[
          t \frac{x}{p{(x)} + \varepsilon} + {(1-t)} \frac{y}{p{(y)} +\varepsilon } \in C
        \]
        preso ora \(t = \frac{p{(x)} + \varepsilon}{p {(x)} + p {(y)} + 2\varepsilon}\), quindi \(
        1-t = \frac{p {(y)} + \varepsilon}{p {(x)} + p {(y)} + 2\varepsilon}\)
        abbiamo che
        \[
            \frac{x + y}{p {(x)} + p {(y)} + 2\varepsilon} \in C
        \]
        Ne consegue (da 4. o dalla definizione) che
        \[
          p {( x + y)} < p {(x)} + p {(y)} + 2\varepsilon \quad \forall \varepsilon > 0
        \]
        e per arbitrarietà di \(\varepsilon\) segue la sottolinearità
        \(
          p {(x+y)} \le p {(x)} + p {(y)}
        \)
        
\end{enumerate}
\end{proof}
    Abbiamo detto dunque che \(p\) è una seminorma. Cosa dovremmo aggiungere per
    renderla una norma? Serve che sia omogenea anche per \(\lambda \le 0\),
    dunque vogliamo \(p {( \lambda x)} = |\lambda| p {(x)}\) e inoltre vorremmo
    che \(p {(x)} = 0 \iff x =0\).

    Per la prima abbiamo che se \(\lambda < 0\) allora \(p {(\lambda x)} = p {( -\lambda {(-x)})} = |\lambda| p {(-x)}\). Vogliamo dunque che \(p {(-x)} = p {(x)}\). Abbiamo però che
    \[
      p {(-x )} = \inf \{r > 0 : -\frac{x}{r} \in C\} = \inf \{ r > 0 : -x \in rC\} 
    \]
    Una proprietà dunque che renderebbe l'uguaglianza vera (p ``pari'' diciamo)
    sarebbe avere che se \(x \in C\), allora \(-x \in C\). Per poter avere
    inoltre che \(p {(x)} = 0 \iff x = 0\) dobbiamo anche richiedere che \(C\)
    sia limitato. In tali ipotesi in realtà \(p\) non è solo una norma ma è
    equivalente a \(\|\cdot \|_X\).

\begin{proposition}{}
    Se \(C\) aperto convesso non vuoto e limitato è tale che \(x \in C \iff -x \in C\), allora \(\exists m_{2} > 0\) tale che \(\|x\| \le m_{2} p {(x)}\) per ogni \(x \in X\)
\end{proposition}
\begin{proof}{}
    Sappiamo che \(\exists R > 0 : C \subseteq B_{R} {(0)} \). Prendo ora \(x
    \in X \sminus \{0\} \). Sia ora \(0 < \overline{r} = \|x\| / R\). Allora per
    \(r < \overline{r}\), 
    \[
      r < \frac{\|x\|}{R} \implies \frac{\|x\|}{r} > R \implies \frac{x}{r}
      \not\in 
    \]
    Sappiamo che \(p {(x)} \ge  \|x\| / R\) per ogni \(x \in X\). Infatti se per
    assurdo \( p {( x)} < \|x\| /R\) allora esisterebbe \(\tilde{r}\) tale che
    \(\frac{x}{\tilde{r}} \in C\) e \(\tilde{r} < \|x\| / R \implies \frac{x}{\tilde{r}} \not\in C\).

    Abbiamo dunque che \(R =: m_{2}\) 
\end{proof}

In particolare, se \(C = B_{1}{(0)}\), allora \(p {(x)} = \|x\|\). Infatti
ovviamente \(p {(x)} \ge \|x\|\) perché \(B_{1}{(0)} \supseteq C\).

Sia ora \(X = \ell^2\). Sia \(C = \{x : \mathbb{N} \to \mathbb{R}: |x{(n)}| \le \frac{1}{n} \,\, \forall n \in \mathbb{N}\}\). Allora
\begin{enumerate}[label = \arabic*.]
    \item \(C \neq \varnothing\), \(C\) è chiuso in \(\ell^2\) e \(C\) è
        convesso.

        Infatti \(0 \in C\). Siano \(x_{1} \in C\) e \(x_{2} \in C\), allora
        \[
          | tx_{1}{(n)} + {(1-t)}x_{2}{(n)}| \le t | x_{1}{(n)}| + {(1-t)} | x_{2}{(n)} | \le  \frac{t}{n} + (1-t)/n = \frac{1}{n}
        \]
        dunque \(C\) è convesso.

        Infine \(C\) è chiuso perché ogni \([-\frac{1}{n}, \frac{1}{n}] \subseteq \mathbb{R} \) è chiuso.
    \item \(\mathring{C} = \varnothing\) 

        Mi chiedo se \(\exists \varepsilon > 0\) tale che, se \(x \in \ell^2\) e
        \(\|x\|_2 < \varepsilon \), allora \(x \in C\). Questo non è vero.
        Infatti prendendo
        \[
          \overline{x}{(n)} = \begin{cases}{}
              \varepsilon & n = \overline{n} \\
              0 & \text{altrimenti}
          \end{cases}
        \]
        abbiamo che \(\|\overline{x}\| = \varepsilon\) ma \(\overline{x} \not\in C\) 
    \item \(C\) \textbf{non} è compatto in \(\ell^2\) 
    \item \(\exists m > 0\) tale che \(\|x\|_2 \le m p {(x)}\)

        Basta dimostrare che \(C\) è limitato. Ma questo è banalmente vero
        perché per ogni \(x \in X\) 
        \[
          \|x\|_2^2 = \sum_{n=1}^{\infty} |x{(n)}|^2 \le \sum_{n=1}^{\infty} \left|\frac{1}{n}\right|^2 = \frac{\pi^2}{6}  
        \]
\end{enumerate}





\begin{lemma}[Separazione di un convesso non vuoto da un punto esterno]
    Sia \(C \subseteq X \) convesso aperto non vuoto e sia \(x_{0} \not\in C\).
    Allora \(\exists f \in X'\) tale che 
    \[
      f{(x)} < f{(x_{0})} \quad \forall x \in C
    \]
\end{lemma}
\begin{proof}{}
Sia \(X_{0} = \mathbb{R}x_{0}\). Allora \(X_{0}\) è sottospazio di \(X\). Sia
\(g : X_{0} \to \mathbb{R}\) definita da \(g{(tx_{0})} = t\), per \(t \in
\mathbb{R}\). Sia \(p\) il funzionale di Minkowski di \(C\), vogliamo dire che
\(g{(tx_{0})} \le p {(tx_{0})}\) per ogni \(t > 0\). Poiché \(x_{0} \not\in C\),
abbiamo che \(p {(x_{0})} \ge 1\). Effettivamente allora
\[
  g {(tx_{0})} = t \cdot  1 \le t p {(x_{0})} = p {(t x_{0})}
\]
Se invece \(t < 0\) banalmente \(g {(tx_{0})} \le  0 \le  p {(t x_{0})}\).

Ora possiamo applicare Hahn-Banach dicendo che esiste \(f : X \to \mathbb{R}\)
lineare tale che \(f = g\) su \(X_{0}\) e \(f{(x)} \le  p {(x)}\) per ogni \(x
\in X\). Per le proprietà di Minkowski, abbiamo che \(f {(x)} \le  p {(x)} \le  m \|x\|\) per ogni \(x \in X\). Inoltre per linearità \(-f{(x)} = f{(-x)} \le m \|-x\| = m \|x\|\). Dalle due otteniamo che
\[|f{(x)}| \le m \|x\| \quad \forall x \in X\] 
e dunque \(f \in X'\).

Ora se \(x \in C\), allora
\[
  f{(x)} \le p {(x)} < 1 = g{(x_{0})} = f{(x_{0})}
\]
\end{proof}
\begin{theorem}[Hahn-Banach \--- prima forma geometrica]
    Sia \(X\) uno spazio normato. Siano \(A, B\) sottoinsiemi non vuoti,
    disgiunti e convessi. Allora se \(A\) è aperto esiste un iperpiano chiuso
    che separa \(A\) e \(B\), cioè \(\exists f \in X'\) e \(\exists \alpha \in \mathbb{R}\) tale che 
    \[
      f{(x)} = \alpha \quad \forall x \in A \quad e \quad f{(x)}\ge \alpha \quad
      \forall x \in B
    \]
\end{theorem}

\begin{proof}{}
    Sia \(C = A - B = \{x \in X : x = a-b, \,\, a \in A, \,\, b \in B\} \).
    Dobbiamo ora mostrare che \(C\) è convesso, aperto e non contiene lo 0. È
    aperto in quanto
    \[
      C = \bigcup_{b \in B} {(A - b)} 
    \]
    unione di aperti. È convesso in quanto, se \(a_{i} - b_{i} = x_{i} \in C\)
    per \(i = 1,2\), allora
    \[
      {(1-t)}{(a_{1} - b_{1})} + t {(a_{2} -b_{2})} = ({(1-t)}a_{1} + ta_{2}) - {({(1-t)}b_{1} + tb_{2})} \in C
    \]
    Infine chiaramente \(0 \not\in C\) poiché \(A\) e \(B\) sono disgiunti.

    Allora \(\exists f \in X'\) tale che \(0 = f{(0)} > f{(z)}\) per ogni \(z
    \in C\). Se \(z = x-y\), con \(x \in A\) e \(y \in B\) abbiamo dunque per
    linearità che
    \[
      f{(x)} < f{(y)} \quad \forall x \in A, \quad \forall y \in B
    \]
    concludiamo l'esistenza di \(\alpha\) della tesi.
\end{proof}

\begin{theorem}[Hahn-Banach \--- seconda forma geometrica]
    Sia \(X\) uno spazio normato, \(A \subseteq X \), \(B \subseteq X \). Siano
    \(A, B\) convessi non vuoti e disgiunti. Allora, se \(A\) è chiuso e \(B\) è
    compatto, esiste un iperpiano chiuso che separa \(A\) e \(B\) strettamente,
    cioè
    \[
      \exists f \in X', \exists \alpha \in \mathbb{R}, \exists \varepsilon > 0 : f{(x)}
      \le \alpha - \varepsilon \,\, \forall x \in A \text{ e } f{(x)} \ge \alpha
      + \varepsilon \,\, \forall x \in B
    \]
\end{theorem}
\begin{proof}{}
    \(\forall \varepsilon > 0\), siano \(A_\varepsilon = A + B_\varepsilon {(0)}\) e
    \(B_\varepsilon = B + B_\varepsilon {(0)}\). Dimostro ora che \(A_\varepsilon\) e \(B_\varepsilon\) sono convessi, disgiunti e aperti. Sono convessi in quanto somma di due convessi.
    Sono aperti in quanto 
    \[
      A_\varepsilon = \bigcup_{a \in A} B_{\varepsilon}{(a)} \quad \text{ e }
      \quad B_\varepsilon = \bigcup_{b \in B} B_\varepsilon {(b)}
    \]
    Dimostro ra che \(\exists \overline{\varepsilon} > 0 : \forall \varepsilon < \overline{\varepsilon}\), \(A_\varepsilon \cap B_\varepsilon = \varnothing\). Per assurdo supponiamo esista una successione \(\varepsilon_n \to 0\) decrescente 
    e \(x_{n}, y_{n}, w_{n}, z_{n}\), con \(x_{n} \in A\), \(y_{n} \in B\), \(w_{n} \in B_{\varepsilon_n}{(0)}\) e \(z_{n} \in B_{\varepsilon_n} {(0)}\) tali che \(x_{n} + w_{n} = y_{n} + z_{n}\).
    Allora 
    \[
      \|x_{n} - y_{n}\| = \|z_{n} - w_{n}\| \le 2 \varepsilon_n
    \]
    poiché \(y_{n} \in B\) compatto, esiste \(n_k\) sottosuccessione con \(y_{n_k} \to \overline{y} \in B\). Allora

    \[
      \|x_{n_k} - \overline{y}\| \le \|x_{n_k} - y_{n_k} \| + \|y_{n_k} - \overline{y}\| \overset{k \to \infty}{\longrightarrow} 0
    \]
    Da cui \(x_{n_k} \to \overline{y}\), ma \(x_{n_k} \in A\) chiuso, dunque \(\overline{y} \in A\). Risulterebbe che \(\overline{y} \in A \cap B\) che è assurdo.

    Abbiamo dunque che \(\exists f \in X'\), ed \(\exists \alpha > 0\) tale che 
    \begin{align*}
      f{(z)} \le \alpha \quad \forall z \in A_\varepsilon \\
      f{(z)} \ge \alpha \quad \forall z \in B_\varepsilon
    \end{align*}
    Ossia \(f{(x + \varepsilon w)} \le \alpha\), per ogni \(x \in A\) e \(w \in B_1{(0)}\) da cui
    \(f{(x)} + \varepsilon f{(w)} \le \alpha\) e poiché vale per ogni \(w \in B_1{(0)}\) abbiamo 
    \[
     f{(x)} \le \alpha - \varepsilon \|f\|_{X'} \text{ e analogamente }
      f{(x)} \ge \alpha + \varepsilon \|f\|_{X'} 
    \]
\end{proof}

\section{Funzioni convesse}
Sia \(X\) uno spazio vettoriale e sia \(\varphi : X \to \mathbb{R} + \{+ \infty\} \). Dico che \(\varphi \) è \emph{propria} se \(D{(\varphi )} = \{x \in X : \varphi {(x)} \neq +\infty\} \neq \varnothing\). Dico che \(\varphi \) è \emph{convessa} se, \(\forall x, y \in X\) e \(\forall t \in [0, 1]\), 
\[
  \varphi {(tx + {(1-t)}y)} \le t \varphi {(x)} + {(1-t)}\varphi {(y)}
\]
Sia \(C \subseteq X\) convesso non vuoto. Allora sia 
\[
  \varphi{(x)}= \begin{cases}{}
      0 & x \in C \\
      \infty & x \not\in  C
  \end{cases}
\]
è una funzione convessa detta \emph{indicatrice} di \(C\) 
\begin{definition}{Epigrafico}
    Data \(\varphi\) una funzione, il suo \emph{epigrafico} è
    \[
      \mathrm{epi}\varphi = \{{(x, \lambda)} \in X \times  \mathbb{R} : \varphi {(x)} \le \lambda\} 
    \]
\end{definition}
\begin{proposition}{}
    Si ha che
\begin{enumerate}[label = (\arabic*)]
    \item \(\varphi : X \to \mathbb{R}\) è convessa se e solo se \(\mathrm{epi}\varphi \) è convesso
    \item Se \(\varphi : X \to \mathbb{R} \cup \{+\infty\} \) è convessa, allora
        \(\{x \in X : \varphi {(x)} \le \lambda\} \) è convesso \(\forall \lambda \in \mathbb{R}\) 
    \item Se \(\varphi_{1}, \varphi_{2}\) sono convesse, allora
        \(\varphi_{1}+\varphi_{2}\) è convessa
    \item Se \(\{\varphi_{i}\}_{i \in I } \) sono convesse, allora \(\varphi{(x)} = \sup_{i \in I} \varphi_i {(x)}\) è convessa.
\end{enumerate}
\end{proposition}
\begin{example}{}
    Sia \(X\) uno spazio normato. Allora \(\varphi : X \to \mathbb{R}\) data da
    \(\varphi{(x)} = \|x\|\) è convessa. Infatti
\[
  \|tx + {(1-t)}y\| \le t \|x\| + {(1-t)}\|y\| \quad \forall x, y \in X, \quad
  \forall t \in (0, 1)
\]
\end{example}

\section{semicontinuità}
\begin{definition}{Funzione semicontinua inferiormente}
    Sia \(X\) uno spazio normato.
    \(\varphi : X \to \mathbb{R}\) è \emph{semicontinua inferiormente} se
    \[
      \varphi {(x)} \le \liminf_{y \to x} \varphi {(y)}
    \]
    viene anche abbreviato s.c.i.
\end{definition}
\begin{proposition}[Caratterizzazioni equivalenti]
\begin{enumerate}[label = (\roman*)]
    \item \(\varphi \) è s.c.i. se e solo se \(\mathrm{epi}\varphi \) è chiuso in \(X \times \mathbb{R}\) 
    \item \(\varphi \) è s.c.i. se e solo se \(\{x \in X : \varphi {(x)} \le \lambda\} \) è chiuso per ogni \(\lambda \in \mathbb{R}\) s.c.i.
    \item Se \(\varphi_{1}, \varphi_{2}\) sono s.c.i.
        allora anche \(\varphi_{1} + \varphi_{2}\) lo è.

    \item Se \(\{\varphi_i\}_{i \in I}\) sono s.c.i., allora \(\varphi{(x)} = \sup_{i \in I} \varphi_i{(x)}\) è s.c.i..
    \item Sia \(X\) compatto, \(\varphi : X \to \mathbb{R}\) s.c.i., allora \(
        \varphi \) ha minimo.
\end{enumerate}
\end{proposition}
\begin{proof}[Dimostrazione di (v)]
    S.c.i. se e solo se è continua secondo la topologia della semicontinuità
    inferiore, con aperti della base della forma \((a, +\infty)\), con \(a \in \mathbb{R}
    \)  come base. Allora se \(X\) è compatto, anche \(\varphi{(X)}\) è compatto. I compatti di \(\mathbb{R}_\text{s.c.i.}\) sono limitati inferiormente, poiché il ricoprimento con gli aperti della base ammmette sottoricoprimento finito.
\end{proof}
\begin{theorem}{}
    Sia \(X\) uno spazio normato, \(\varphi  : X \to \mathbb{R} \cup \{+\infty\} \) convessa, propria e s.c.i. Allora esiste \(f \in X'\) e \(\exists c \in \mathbb{R}\) tale che
    \[
      \varphi  {(x)} \ge \Span{f, x} + c \quad \forall x \in X
    \]
\end{theorem}




