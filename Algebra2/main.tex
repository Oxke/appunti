\documentclass{article}
\usepackage{layout}
\usepackage[a4paper, total={5in,9in}]{geometry}
\usepackage[T1]{fontenc}
\usepackage[italian]{babel}
\usepackage{mathtools}
\usepackage{amsthm}
\usepackage[framemethod=TikZ]{mdframed}
\usepackage{amsmath}
\usepackage{amssymb}
\usepackage{cancel}
\usepackage[dvipsnames]{xcolor}
\usepackage{tikz}
\usepackage{tikz-cd}
\usepackage{pgfplots}
\pgfplotsset{compat=1.18}
\usepackage[many]{tcolorbox}
\usepackage{import}
\usepackage{pdfpages}
\usepackage{transparent}
\usepackage{enumitem}
\usepackage[colorlinks]{hyperref}

\newcommand*{\sminus}{\raisebox{1.3pt}{$\smallsetminus$}}

\newcommand*{\transp}[2][-3mu]{\ensuremath{\mskip1mu\prescript{\smash{\mathrm t\mkern#1}}{}{\mathstrut#2}}}%

% newcommand for span with langle and rangle around
\newcommand{\Span}[1]{{\left\langle#1\right\rangle}}

\newcommand{\incfig}[2][1]{%
    \def\svgwidth{#1\columnwidth}
    \import{./figures/}{#2.pdf_tex}
}

\pdfsuppresswarningpagegroup=1

\newcounter{theo}[section]\setcounter{theo}{0}
\renewcommand{\thetheo}{\arabic{section}.\arabic{theo}}

\newcounter{excounter}[section]\setcounter{excounter}{0}
\renewcommand{\theexcounter}{\arabic{section}.\arabic{excounter}}

\numberwithin{equation}{section}

\newenvironment{theorem}[1][]{
    \refstepcounter{theo}
     \ifstrempty{#1}
    {\mdfsetup{
        frametitle={
            \tikz[baseline=(current bounding box.east),outer sep=0pt]
            \node[anchor=east,rectangle,fill=blue!20,rounded corners=5pt]
            {\strut Teorema~\thetheo};}
        }
    }{\mdfsetup{
        frametitle={
            \tikz[baseline=(current bounding box.east),outer sep=0pt]
            \node[anchor=east,rectangle,fill=blue!20,rounded corners=5pt]
            {\strut Teorema~\thetheo:~#1};}
        }
    }
    \mdfsetup{
        roundcorner=10pt,
        innertopmargin=10pt,linecolor=blue!20,
        linewidth=2pt,topline=true,
        frametitleaboveskip=\dimexpr-\ht\strutbox\relax,
        % nobreak=false
    }
\begin{mdframed}[]\relax}{
\end{mdframed}}

% \newenvironment{definition}[1][]{
%     \refstepcounter{theo}
%      \ifstrempty{#1}
%     {\mdfsetup{
%         frametitle={
%             \tikz[baseline=(current bounding box.east),outer sep=0pt]
%             \node[anchor=east,rectangle,fill=violet!20,rounded corners=5pt]
%             {\strut Definizione~\thetheo};}
%         }
%     }{\mdfsetup{
%         frametitle={
%             \tikz[baseline=(current bounding box.east),outer sep=0pt]
%             \node[anchor=east,rectangle,fill=violet!20,rounded corners=5pt]
%             {\strut Definizione~\thetheo:~#1};}
%         }
%     }
%     \mdfsetup{
%         roundcorner=10pt,
%         innertopmargin=10pt,linecolor=violet!20,
%         linewidth=2pt,topline=true,
%         frametitleaboveskip=\dimexpr-\ht\strutbox\relax,
%         nobreak=true
%     }
% \begin{mdframed}[]\relax}{
% \end{mdframed}}

\newtcolorbox[auto counter, number within=section]{definition}[2][]{
    colframe=violet!0,
    coltitle=violet, % Title text color
    fonttitle=\bfseries, % Title font
    title={Definizione~\thetcbcounter\ifstrempty{#2}{}{:~#2}}, % Title format
    sharp corners, % Less rounded corners
    boxrule=0pt, % Line width of the box frame
    toptitle=1mm, % Distance from top to title
    bottomtitle=1mm, % Distance from title to box content
    colbacktitle=violet!5, % Background color of the title bar
    left=0mm, right=0mm, top=1mm, bottom=1mm, % Padding around content
    enhanced, % Enable advanced options
    before skip=10pt, % Space before the box
    after skip=10pt, % Space after the box
    breakable, % Allow box to split across pages
    colback=violet!0,
    borderline west={2pt}{-5pt}{violet!40},
    #1
}

\newenvironment{lemmao}[1][]{
    \refstepcounter{theo}
     \ifstrempty{#1}
    {\mdfsetup{
        frametitle={
            \tikz[baseline=(current bounding box.east),outer sep=0pt]
            \node[anchor=east,rectangle,fill=green!20,rounded corners=5pt]
            {\strut Lemma~\thetheo};}
        }
    }{\mdfsetup{
        frametitle={
            \tikz[baseline=(current bounding box.east),outer sep=0pt]
            \node[anchor=east,rectangle,fill=green!20,rounded corners=5pt]
            {\strut Lemma~\thetheo:~#1};}
        }
    }
    \mdfsetup{
        roundcorner=10pt,
        innertopmargin=10pt,linecolor=green!20,
        linewidth=2pt,topline=true,
        frametitleaboveskip=\dimexpr-\ht\strutbox\relax,
        % nobreak=true
    }
\begin{mdframed}[]\relax}{
\end{mdframed}}

\theoremstyle{plain}
\newtheorem{lemma}[theo]{Lemma}
\newtheorem{corollary}{Corollario}[theo]
\newtheorem{proposition}[theo]{Proposizione}

\theoremstyle{definition}
\newtheorem{example}[excounter]{Esempio}

\theoremstyle{remark}
\newtheorem*{note}{Nota}
\newtheorem*{remark}{Osservazione}

\newtcolorbox{notebox}{
  colback=gray!10,
  colframe=black,
  arc=5pt,
  boxrule=1pt,
  left=15pt,
  right=15pt,
  top=15pt,
  bottom=15pt,
}

\DeclareRobustCommand{\rchi}{{\mathpalette\irchi\relax}} % beautiful chi
\newcommand{\irchi}[2]{\raisebox{\depth}{$#1\chi$}} % inner command, used by \rchi

\newtcolorbox[auto counter, number within=section]{eser}[1][]{
    colframe=black!0,
    coltitle=black!70, % Title text color
    fonttitle=\bfseries\sffamily, % Title font
    title={Esercizio~\thetcbcounter~#1}, % Title format
    sharp corners, % Less rounded corners
    boxrule=0mm, % Line width of the box frame
    toptitle=1mm, % Distance from top to title
    bottomtitle=1mm, % Distance from title to box content
    colbacktitle=black!5, % Background color of the title bar
    left=0mm, right=0mm, top=1mm, bottom=1mm, % Padding around content
    enhanced, % Enable advanced options
    before skip=10pt, % Space before the box
    after skip=10pt, % Space after the box
    breakable, % Allow box to split across pages
    colback=black!0,
    borderline west={1pt}{-5pt}{black!70}, 
    segmentation style={dashed, draw=black!40, line width=1pt} % Dashed dividing line
}
\newcommand{\seminorm}[1]{\left\lvert\hspace{-1 pt}\left\lvert\hspace{-1 pt}\left\lvert#1\right\lvert\hspace{-1 pt}\right\lvert\hspace{-1 pt}\right\lvert}


\title{Appunti di Algebra 2}
\author{Github Repository:
\href{https://github.com/Oxke/appunti/tree/main/Algebra2}{\texttt{Oxke/appunti/Algebra2}}}

\date{Secondo semestre, 2024 \-- 2025, prof. Paola Frediani}

\begin{document}
\maketitle

I testi preferiti sono
\begin{itemize}[label = --]
    \item \textit{Algebra}, di Michael Artin
    \item \emph{Algebra}, di Herstein
\end{itemize}

\section{Azioni di gruppi su insiemi}
Chiameremo \(G\) un gruppo e \(S\) un insieme
\begin{definition}
    Un'azione (sinistra) di \(G\) su \(S\) e un'applicazione
    \[
      F : G\times S \to S
    \]
    tale che
\begin{enumerate}[label = \roman*)]
    \item \(F(e, s) = s\) per ogni \(s \in S\)
    \item \(\forall g, h \in G\) e \(\forall s \in S\) vale \(F{(g, F{(h, s)})}
        = F{(gh, s)}\) 
\end{enumerate}
\end{definition}
Si usa anche la notazione \(F{(g, s)} =: g{(s)}\) che permette la scrittura più
concisa
\[
  e{(s)} = s \quad \text{ e } \quad g{(h{(s)})} = {(gh)}{(s)} \quad \forall s
  \in S, \quad \forall g, h \in  G
\]
\begin{proposition}\label{prop:azione_biiezione}
    Per ogni \(g \in G\), l'applicazione \(F_g : S \to S\) definita da
    \(F_g{(s)} = F{(g, s)} = g{(s)}\) è una biiezione
\end{proposition}
\begin{proof}
    \({(Fg)}^{-1} = F_{g^{-1}}\) infatti 
    \[
      F_g \circ F_{g^{-1}} {(s)} = g{(g^{-1}{(s)})} \overset{(ii)}{=} e{(s)}
      \overset{(i)}{=} s
    \]
    e analogamente per l'altra composizione
\end{proof}

\begin{proposition}

L'applicazione \(\psi : G \to S(S) = \{f : S\to S \text{ biunivoche}\}\) dove
\(S{(S)}\) il gruppo delle permutazioni di \(S\) è un omomorfismo di gruppi.
\end{proposition}
\begin{proof}
\[
    \psi{(gh)}= F_{gh} \overset{(ii)}{=} F_g \circ F_h = \psi{(g)} \circ
    \psi{(h)}
\]
\end{proof}
\begin{definition}{Azione fedele}
    Un'azione \(F : G \times S \to S\) si dice \textbf{fedele} se \(\psi\) è
    iniettivo
\end{definition}
\begin{remark}
    Ovvero se e solo se \(\text{Ker}\psi = \{e\} \iff (\psi{(g)} = \text{Id}_S
    \iff g = e)\) 
\end{remark}

\begin{example}
    Se \(S = G\) il gruppo stesso e sia
    \[
        m : G \times G \to G \quad \text{ con } \quad m{(g, h)} = gh
    \]
    la moltiplicazione a sinistra. Allora \(m\) è un'azione sinistra, infatti
\begin{enumerate}[label = \roman*)]
    \item \(m {(e, h)} = eh = h\) per ogni \(h \in G\) 
    \item \(m {(gg', h)} = {(g g')}h = g{(g'h)} = m {(g, g'h)}\) per ogni \(g,
        g', h \in G\) 
\end{enumerate}
    Inoltre \(m\) è un'azione fedele, infatti 
    \[
        \psi{(g)}{(h)} = h \forall h \in G \iff gh = h \implies g = e
    \]
    \begin{remark}
        Se \(G\) è un gruppo finito, con \(\# G = n\) allora \(S{(G)} \cong S_n\)
        e poiché \(\psi\) è iniettivo, \(G \cong \psi{(G)} < S{(G)} \cong S_n\)
        il teorema di Cayley
    \end{remark}
\end{example}
\begin{example}
    Sempre con \(G = S\) possiamo considerare l'azione di coniugio
    \[
        \varphi : G \times G \to G \quad \text{ con } \quad \varphi{(g, h)} =
        ghg^{-1}
    \]
    \begin{enumerate}[label = \roman*)]
        \item \(\varphi{(e, h)} = ehe^{-1} = h\) per ogni \(h \in G\)
        \item \(\varphi {(g g', h)} = {(g g')}h {( g g' )}^{-1} = g g' h g'^{-1}
            g^{-1} = g{(\varphi {(g', h)})} g ^{-1} = \varphi {(g, \varphi {(g',
            h)})}\) 
    \end{enumerate}
    \(\psi : G \to S(G)\) e \(\text{Im} \psi = \text{Inn} {(G)} <
    \text{Aut}{(G)}\). Non è necessariamente un'azione fedele, infatti
    \[
        \text{Ker}{(\psi)} = \{g \in G : \forall h \in G \quad ghg^{-1} = h\} =
        Z(G)
    \]
    da cui per il primo teorema di isomorfismo
    \[
      G / Z{(G)} = \text{Inn}{(G)}
    \]
\end{example}
\begin{example}
    Con \(G = S_n\) e \(S = \{1, \dots, n\} \) allora la funzione
    \[
        (\sigma, i) \mapsto \sigma{(i)}
    \]
    è ovviamente un'azione
\end{example}
\begin{example}
    Preso \(G \cong \mathbb{Z} / 2 \mathbb{Z} \cong \{1, \sigma\} \) con
    \(\sigma^2 = 1\)  e \(S = \mathbb{C}\) allora la funzione
    \[
        F : G \times \mathbb{C} \to \mathbb{C} \quad \text{ con } \quad F{(1,
        z)} = z \quad \text{ e } \quad F{(\sigma, z)} = \overline{z} \quad
        \forall z \in \mathbb{C}
    \]
    è un'azione.
\end{example}
\begin{definition}{Orbita e Stabilizzatore}
    Sia \(F : G \times  S \to S\) un'azione di un gruppo \(G\) su \(S\). Allora
    per ogni \(s \in S\) si definisce \textbf{orbita} di \(s\) l'insieme
    \[
        O_s = \{g{(s)} : g \in G\}
    \]
    e si definisce \textbf{stabilizzatore} di \(s\) l'insieme
    \[
        \text{stab}{(s)} = \{g \in G : g{(s)} = s\}
    \]
\end{definition}
\begin{example}
    Nell'esempio dell'azione di coniugio lo stabilizzatore di \(h\) è
    \[
        \text{stab}_h = \{g \in G : ghg^{-1} = h\} = \{g \in G : gh = hg\} =
        C_G{(h)}
    \]
\end{example}

\begin{proposition}

Le orbite \(O_s\) per un'azione di \(G\) sono classi di equivalenza per la
relazione di equivalenza su \(S\) seguente:
\[
  S \sim S' \iff \exists g \in G : s' = g{(s)} = F{(g, s)}
\]
\end{proposition}
\begin{proof}
\(\sim\) è in effetti una relazione di equivalenza, infatti:
\begin{itemize}[label = --]
    \item \emph{riflessiva}: \(s = e{(s)}\)
    \item \emph{simmetrica}: se \(s' = g{(s)}\) allora \(s = g^{-1}{(s')}\)
        perché \({(F_g)}^{-1} = F_{g^{-1}} \) 
    \item \emph{transitiva}: se \(s' = g{(s)}\) e \(s'' = h{(s')}\) allora
        \(s'' = h{(s')} = h{(g{(s)})} \overset{(ii)}{=} {(hg)}{(s)}\)
\end{itemize}

Ne segue chiaramente che \(O_s = {[s]}_\sim \) e allora \(\displaystyle S = \coprod_{s \in S}
O_s\) 
\end{proof}
% TODO separa in a parte corollario e reffalo in dopo
\begin{proposition}
    \(\text{stab}_s < G\) 
\end{proposition}
\begin{proof}
    Supponiamo \(g, h \in \text{stab}_s\). Allora \(g{(s)} = h{(s)} = s\), ne
    consegue che
    \[
      F{(gh, s)} = F{(g, F{(h, s)})} 
    \]
\end{proof}

\begin{definition}{Azione transitiva}
    Un'azione \(F : G \times S \to S \) si dice \textbf{transitiva} se per ogni
    \(s, s' \in S\) esiste \(g \in G\) tale che \(s' = g{(s)}\)
\end{definition}

\begin{proposition}
    Sia \(F : G \times  S \to S\) un'azione di gruppo. Allora fissato un \(s \in
    S\), consideriamo \(O_s \subseteq   S\) e \(H := \mathrm{stab}_s < G\). Allora esiste
    un'applicazione naturale biettiva
    \begin{align*}
        \Phi: G / H &\longrightarrow O_s \\
        gH &\longmapsto \Phi(gH) = g{(s)} = F(g, s)
    \end{align*}
    Inoltre per ogni \(C \in G / H\), \(g{(\Phi{(C)})} = \Phi{(g{(C)})}\) dove
    la prima azione è quella di \(G\) su \(O_s\) e la seconda è quella di \(G\)
    su \(G / H\)
\end{proposition}
\begin{proof} \(\) 
    \begin{itemize}[label = --]
        \item \emph{Ben definita}: se \(aH = bH\) allora \(b^{-1}a \in H\) e
            quindi esiste un \(h \in H\) tale che \(b^{-1}a = h\) e quindi \(a =
            bh\). Allora \(F{(a, s)} = F{(bh, s)} = F{(b, F{(h, s)})}\) =
            \(F{(b, s)}\) 
        \item \emph{Iniettiva}: supponiamo che esistano \(a, b \in G\) tali che
            \(\Phi{(aH)} = \Phi{(bH)}\), allora \(F{(a, s)} = F{(b, s)}\) ma
            allora
            \[
                F{(b^{-1}a, s)} = F{(b^{-1}, F{(a, s)})} = F{(b^{-1}, F{(b,
                s)})} = F{(b^{-1}b, s)} = F{(e, s)} = s
            \]
            e quindi \(b^{-1}a \in H \iff aH = bH\) 
        \item \emph{Suriettiva}: per ogni \(s' \in O_s\) esiste \(g \in G\) tale
            che \(s' = g{(s)}\) e quindi \(s' = g{(s)} = \Phi(gH)\)
    \end{itemize}
\end{proof}
\begin{corollary}
    Se \(G\) è un gruppo finito e ho un'azione \(F : G \times S \to S\), allora
    per ogni \(s \in S\) vale \(\# O_s = [G : \text{stab}_s]\) o
    equivalentemente
    \[
        \# G = \# O_s \cdot \# \mathrm{stab}_s
    \]
    e inoltre
    \[
        \# G = \sum_{[s] \in S} \# O_s
    \]
\end{corollary}
\begin{corollary}\label{cor:classi_coniugio}
    Sia \(F : G \times G \to G\) l'azione di coniugio \({(g, h)} \mapsto
    ghg^{-1}\). Ricordiamo che \(\mathrm{stab}_a = C{(a)}\) e la formula delle
    classi si traduce in 
    \[
        \# G = \# C{(a)} \cdot \# O_a = \sum_{[g] \in G} \# O_g = \sum_{[g] \in
        G} \frac{\# G}{\# C{(g)}}
    \]
    inoltre se \(g \in Z{(G)}\) allora \(C{(g)} = G\) e dunque
    \[
        \# G = \# Z + \sum_{[g] \in G\sminus Z} \# O_g
    \]
\end{corollary}
\begin{theorem}
    Sia \(G\) un gruppo tale che \(\# G = p^{n}\) con \(p\) primo. Allora
    \(Z{(G)} \neq \{e\} \) 
\end{theorem}
\begin{proof}
    Se \(a \not\in Z\) allora \(C{(a)} = p^{n_a}\) con \(n_a < n\) e quindi da
    \[
        p^{n} = \# G = \# Z + \sum_{[g] \in G\sminus Z} \# \frac{p^{n}}{p^{n_a}}
    \]
    ne deduciamo che \(p | \#Z\) 
\end{proof}
\begin{corollary}
    Sia \(G\) un gruppo di cardinalità \(p^2\), con \(p\) primo. Allora \(G\) è
    abeliano.
\end{corollary}
\begin{proof}
    Per il teorema sappiamo che \(Z \neq \{e\} \) e quindi \(\# Z = p\) oppure
    \(\# Z = p^2\). Nel secondo caso \(G = Z\) e quindi è abeliano. Nel primo
    caso invece esiste un \(a \in G \sminus Z\) e dunque \(C{(a)} \neq G\). Ma
    \[
        \{e\} < Z < C{(a)} < G
    \]
    e quindi \(C{(a)} = Z\) per cardinalità che è assurdo perché \(a \in
    C{(a)}\) e \(a \not\in Z\).
\end{proof}
\begin{example}
    Riprendendo l'esempio della moltiplicazione a sinistra \(m : G \times G \to
    G\). Allora \(m\) è un'azione transitiva. Infatti per ogni \(g', g'' \in G\) se
    prendo \(h = {(g')}^{-1} g''\) allora \(m {(g', h)} = g'{(g'^{-1} g'')} =
    g''\) 
\end{example}
\begin{example}
    Se prendo \(GL{(V)}\) il gruppo lineare delle trasformazioni invertibili su
    uno spazio vettoriale \(V\), allora l'azione \({(T, v)} \mapsto Tv\) è
    transitiva su \(V \sminus \{0\}\)  
\end{example}
\begin{theorem}[Cauchy per gruppi abeliani]
    Sia \(G\) un gruppo abeliano finito e \(p\) un primo tale che \(p | \# G\).
    Allora 
    \[
      \exists e\neq a \in G \text{ tale che } a^{p} = e
    \]
\end{theorem}
\begin{proof}
    Procediamo per induzione su \(n = \# G\).
    Se \(2 = \# G\) allora \(G = \{e, a\} \) e dunque \(a^2 = e\). Supponiamo
    ora \(\# G \ge 3\).

    Se \(G\) non ha sottogruppi \(e \neq H \neq G\) allora \(G\) è ciclico di
    ordine primo. Infatti se \(G\) non è ciclico allora esistono due elementi
    \(e\neq g_{1}, g_{2}\) e \(g_{2} \not\in \Span{g_{1}} \). Ma allora
    \(\{e\} \neq \Span{g_{1}} \neq G\) è un sottogruppo. Dunque \(G\) è
    ciclico, inoltre è di ordine primo perché se così non fosse (ad esempio \(n
    = ab\)) allora \(\{e\} \neq \Span{g^{a}} \neq G\) è un sottogruppo, con
    \(g\) tale che \(<g> = G\).

    Allora se \(G\) non ha sottogruppi propri esistono \(p-1\) elementi in \(G\)
    di ordine \(p\).

    Supponiamo ora che \(G\) abbia qualche sottogruppo non banale. Sia \(N < G\)
    con \(\{e\} \neq N \neq G\). Allora se \(p | \# N\) per ipotesi induttiva si
    conclude. Se invece \(p\not| \# N\) allora \(G / N\) è un gruppo abeliano
    con \(\# G / N < \# G\) e quindi per ipotesi induttiva (infatti \(G/N\) ha
    ordine multiplo di \(p\) poiché \(N\) non lo è) esiste \(bN \in G/N\), \(b
    \not\in N\) e tale che \(b^{p} \in N\). Allora \(b^{p\#N} = e\) e ci resta
    solo da dimostrare che \(c := b^{\#N} \neq e\).

    Supponiamo che \(c = b^{\#N} = e\). Sappiamo che \(MCD(p, \#N) = 1\) e
    dunque per il teorema di Bézout esistono \(\alpha, \beta \in \mathbb{Z}\)
    tali che \(\alpha p + \beta \#N = 1\).
    Allora 
    \[
      bN = {(bN)}^{\alpha p + \beta\# N} = {(bN)}^{\alpha p} \cdot {(bN)}^{\beta
      \#N}
    \]
    e poiché \(b^{p} \in N\) e \(b^{\#N} = e\) otteniamo che \(bN = N\) che è
    assurdo perché \(b \not\in N\).
\end{proof}
\begin{theorem}[Cauchy]
    Sia \(G\) è un gruppo finito e \(p\) è un primo tale che \(p | \# G\).
    Allora
    \[
        \exists a \in G \text{ tale che } \# \langle a \rangle = p
    \]
\end{theorem}
\begin{proof}
    Vogliamo procedere per induzione su \(\#G\). Se \(\#G = 2\) è già
    dimostrato. Se esiste \(H < G\) tale che \(p | \# H\) concludo per ipotesi
    induttiva.

    Supponiamo dunque che \(p\) non divide l'ordine di nessun sottogruppo
    proprio di \(G\). Dalla formula delle classi
    \[
        \# G = \# Z{(G)} + \sum_{[a] \in G\sminus Z{(G)}} \frac{\# G}{\# C{(a)}}
    \]
    tutti i termini della serie sono divisibili per \(p\), infatti se \(a
    \not\in Z\), \(C{(a)}\neq G\) è un sottogruppo proprio e quindi \(p \not{|}
    \# C{(a)}\). Allora \(p | \# Z\) ma quindi \(Z = G\) e quindi \(G\) è
    abeliano. Concludiamo con il teorema di Cauchy per gruppi abeliani.
\end{proof}
\begin{theorem}[Sylow (prima parte)]
    Sia \(G\) un gruppo finito e \(p\) un primo tale che esiste \(\mathbb{Z} \ni
    m \ge 1\)  tale che \(p^{m} | \# G\) ma \(p^{m+1} \not{|} G\). Allora
    \[
        \exists H < G \text{ tale che } \# H = p^{m}
    \]
\end{theorem}
\begin{proof}
    Procediamo per induzione su \(\# G\) e su \(m\). Se \(\# G = 2\) allora \(H = G\).
    Supponiamo ora \(\#G > 2\) e il risultato vero per ogni gruppo di
    cardinalità minore di \(G\). Se \(m=1\) allora ci si riduce al teorema di
    Cauchy. Supponiamo ora che \(m \ge 2\) e \(\#G \ge 3\).

    Se esiste \(H < G\) proprio con \(p^{m} | \# G\) allora concludo per ipotesi
    induttiva.

    Supponiamo dunque che \(p^{m} \not{|} \# K\) per ogni \(K < G\) proprio.

    \[
        \# G = \# Z{(G)} + \sum_{[a] \in G\sminus Z{(G)}} \frac{\# G}{\# C{(a)}}
    \]
    nuovamente come nel teorema di Cauchy, otteniamo che \(p | \# Z\) e quindi
    per il teorema di Cauchy abbiamo che esiste \(e \neq b \in Z\) tale che
    \(b^{p} = e\). Allora \(\Span{b} \trianglelefteq G\). 

    Ma allora poiché \(\# (G / \Span{b} ) = \# G / p\) abbiamo che \(p^{m-1} |
    \# {(G / \Span{b} )}\) e dunque per ipotesi induttiva esiste \(\overline{S}
    < G / \Span{b} \) tale che \(\# \overline{S} = p^{m-1}\). Allora \(S = \pi
    ^{-1}{(\overline{S})}\) è un sottogruppo di \(G\) e \(\overline{S} = S /
    B\). Ne consegue infine che \(\# S = p^{m}\).
\end{proof}
\begin{remark}
    Il sottogruppo \(H\) viene detto \(p\)-sottogruppo di Sylow di \(G\).
\end{remark}


\end{document}
