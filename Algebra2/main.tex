%! TeX program = lualatex
\documentclass{article}
\usepackage{layout}
\usepackage[a4paper, total={5in,9in}]{geometry}
\usepackage[T1]{fontenc}
\usepackage[italian]{babel}
\usepackage{mathtools}
\usepackage{amsthm}
\usepackage[framemethod=TikZ]{mdframed}
\usepackage{amsmath}
\usepackage{amssymb}
\usepackage{cancel}
\usepackage[dvipsnames]{xcolor}
\usepackage{tikz}
\usepackage{tikz-cd}
\usepackage{pgfplots}
\pgfplotsset{compat=1.18}
\usepackage[many]{tcolorbox}
\usepackage{import}
\usepackage{pdfpages}
\usepackage{transparent}
\usepackage{enumitem}
\usepackage[colorlinks]{hyperref}

\newcommand*{\sminus}{\raisebox{1.3pt}{$\smallsetminus$}}

\newcommand*{\transp}[2][-3mu]{\ensuremath{\mskip1mu\prescript{\smash{\mathrm t\mkern#1}}{}{\mathstrut#2}}}%

% newcommand for span with langle and rangle around
\newcommand{\Span}[1]{{\left\langle#1\right\rangle}}

\newcommand{\incfig}[2][1]{%
    \def\svgwidth{#1\columnwidth}
    \import{./figures/}{#2.pdf_tex}
}

\pdfsuppresswarningpagegroup=1

\newcounter{theo}[section]\setcounter{theo}{0}
\renewcommand{\thetheo}{\arabic{section}.\arabic{theo}}

\newcounter{excounter}[section]\setcounter{excounter}{0}
\renewcommand{\theexcounter}{\arabic{section}.\arabic{excounter}}

\numberwithin{equation}{section}

\newenvironment{theorem}[1][]{
    \refstepcounter{theo}
     \ifstrempty{#1}
    {\mdfsetup{
        frametitle={
            \tikz[baseline=(current bounding box.east),outer sep=0pt]
            \node[anchor=east,rectangle,fill=blue!20,rounded corners=5pt]
            {\strut Teorema~\thetheo};}
        }
    }{\mdfsetup{
        frametitle={
            \tikz[baseline=(current bounding box.east),outer sep=0pt]
            \node[anchor=east,rectangle,fill=blue!20,rounded corners=5pt]
            {\strut Teorema~\thetheo:~#1};}
        }
    }
    \mdfsetup{
        roundcorner=10pt,
        innertopmargin=10pt,linecolor=blue!20,
        linewidth=2pt,topline=true,
        frametitleaboveskip=\dimexpr-\ht\strutbox\relax,
        % nobreak=false
    }
\begin{mdframed}[]\relax}{
\end{mdframed}}

% \newenvironment{definition}[1][]{
%     \refstepcounter{theo}
%      \ifstrempty{#1}
%     {\mdfsetup{
%         frametitle={
%             \tikz[baseline=(current bounding box.east),outer sep=0pt]
%             \node[anchor=east,rectangle,fill=violet!20,rounded corners=5pt]
%             {\strut Definizione~\thetheo};}
%         }
%     }{\mdfsetup{
%         frametitle={
%             \tikz[baseline=(current bounding box.east),outer sep=0pt]
%             \node[anchor=east,rectangle,fill=violet!20,rounded corners=5pt]
%             {\strut Definizione~\thetheo:~#1};}
%         }
%     }
%     \mdfsetup{
%         roundcorner=10pt,
%         innertopmargin=10pt,linecolor=violet!20,
%         linewidth=2pt,topline=true,
%         frametitleaboveskip=\dimexpr-\ht\strutbox\relax,
%         nobreak=true
%     }
% \begin{mdframed}[]\relax}{
% \end{mdframed}}

\newtcolorbox[auto counter, number within=section]{definition}[2][]{
    colframe=violet!0,
    coltitle=violet, % Title text color
    fonttitle=\bfseries, % Title font
    title={Definizione~\thetcbcounter\ifstrempty{#2}{}{:~#2}}, % Title format
    sharp corners, % Less rounded corners
    boxrule=0pt, % Line width of the box frame
    toptitle=1mm, % Distance from top to title
    bottomtitle=1mm, % Distance from title to box content
    colbacktitle=violet!5, % Background color of the title bar
    left=0mm, right=0mm, top=1mm, bottom=1mm, % Padding around content
    enhanced, % Enable advanced options
    before skip=10pt, % Space before the box
    after skip=10pt, % Space after the box
    breakable, % Allow box to split across pages
    colback=violet!0,
    borderline west={2pt}{-5pt}{violet!40},
    #1
}

\newenvironment{lemmao}[1][]{
    \refstepcounter{theo}
     \ifstrempty{#1}
    {\mdfsetup{
        frametitle={
            \tikz[baseline=(current bounding box.east),outer sep=0pt]
            \node[anchor=east,rectangle,fill=green!20,rounded corners=5pt]
            {\strut Lemma~\thetheo};}
        }
    }{\mdfsetup{
        frametitle={
            \tikz[baseline=(current bounding box.east),outer sep=0pt]
            \node[anchor=east,rectangle,fill=green!20,rounded corners=5pt]
            {\strut Lemma~\thetheo:~#1};}
        }
    }
    \mdfsetup{
        roundcorner=10pt,
        innertopmargin=10pt,linecolor=green!20,
        linewidth=2pt,topline=true,
        frametitleaboveskip=\dimexpr-\ht\strutbox\relax,
        % nobreak=true
    }
\begin{mdframed}[]\relax}{
\end{mdframed}}

\theoremstyle{plain}
\newtheorem{lemma}[theo]{Lemma}
\newtheorem{corollary}{Corollario}[theo]
\newtheorem{proposition}[theo]{Proposizione}

\theoremstyle{definition}
\newtheorem{example}[excounter]{Esempio}

\theoremstyle{remark}
\newtheorem*{note}{Nota}
\newtheorem*{remark}{Osservazione}

\newtcolorbox{notebox}{
  colback=gray!10,
  colframe=black,
  arc=5pt,
  boxrule=1pt,
  left=15pt,
  right=15pt,
  top=15pt,
  bottom=15pt,
}

\DeclareRobustCommand{\rchi}{{\mathpalette\irchi\relax}} % beautiful chi
\newcommand{\irchi}[2]{\raisebox{\depth}{$#1\chi$}} % inner command, used by \rchi

\newtcolorbox[auto counter, number within=section]{eser}[1][]{
    colframe=black!0,
    coltitle=black!70, % Title text color
    fonttitle=\bfseries\sffamily, % Title font
    title={Esercizio~\thetcbcounter~#1}, % Title format
    sharp corners, % Less rounded corners
    boxrule=0mm, % Line width of the box frame
    toptitle=1mm, % Distance from top to title
    bottomtitle=1mm, % Distance from title to box content
    colbacktitle=black!5, % Background color of the title bar
    left=0mm, right=0mm, top=1mm, bottom=1mm, % Padding around content
    enhanced, % Enable advanced options
    before skip=10pt, % Space before the box
    after skip=10pt, % Space after the box
    breakable, % Allow box to split across pages
    colback=black!0,
    borderline west={1pt}{-5pt}{black!70}, 
    segmentation style={dashed, draw=black!40, line width=1pt} % Dashed dividing line
}
\newcommand{\seminorm}[1]{\left\lvert\hspace{-1 pt}\left\lvert\hspace{-1 pt}\left\lvert#1\right\lvert\hspace{-1 pt}\right\lvert\hspace{-1 pt}\right\lvert}


\title{Appunti di Algebra 2}
\author{Github Repository:
\href{https://github.com/Oxke/appunti/tree/main/Algebra2}{\texttt{Oxke/appunti/Algebra2}}}

\date{Secondo semestre, 2024 \-- 2025, prof. Paola Frediani}

\begin{document}
\maketitle

I testi preferiti sono
\begin{itemize}[label = --]
    \item \textit{Algebra}, di Michael Artin
    \item \emph{Algebra}, di Herstein
\end{itemize}

\section{Azioni di gruppi su insiemi}
Chiameremo \(G\) un gruppo e \(S\) un insieme
\begin{definition}{Azione di gruppo}
    Un'azione (sinistra) di \(G\) su \(S\) e un'applicazione
    \[
      F : G\times S \to S
    \]
    tale che
\begin{enumerate}[label = \roman*)]
    \item \(F(e, s) = s\) per ogni \(s \in S\)
    \item \(\forall g, h \in G\) e \(\forall s \in S\) vale \(F{(g, F{(h, s)})}
        = F{(gh, s)}\) 
\end{enumerate}
\end{definition}
Si usa anche la notazione \(F{(g, s)} =: g{(s)}\) che permette la scrittura più
concisa
\[
  e{(s)} = s \quad \text{ e } \quad g{(h{(s)})} = {(gh)}{(s)} \quad \forall s
  \in S, \quad \forall g, h \in  G
\]
\begin{proposition}\label{prop:azione_biiezione}
    Per ogni \(g \in G\), l'applicazione \(F_g : S \to S\) definita da
    \(F_g{(s)} = F{(g, s)} = g{(s)}\) è una biiezione e in particolare
    \begin{equation}\label{eq:azione_inversa}
        F_g^{-1} = F_{g^{-1}}
    \end{equation}
\end{proposition}
\begin{proof}
    \[
      F_g \circ F_{g^{-1}} {(s)} = g{(g^{-1}{(s)})} \overset{(ii)}{=} e{(s)}
      \overset{(i)}{=} s
    \]
    e analogamente per l'altra composizione
\end{proof}

\begin{proposition}

L'applicazione \(\psi : G \to S(S) = \{f : S\to S \text{ biunivoche}\}\) dove
\(S{(S)}\) il gruppo delle permutazioni di \(S\) è un omomorfismo di gruppi.
\end{proposition}
\begin{proof}
\[
    \psi{(gh)}= F_{gh} \overset{(ii)}{=} F_g \circ F_h = \psi{(g)} \circ
    \psi{(h)}
\]
\end{proof}
\begin{definition}{Azione fedele}
    Un'azione \(F : G \times S \to S\) si dice \textbf{fedele} se \(\psi\) è
    iniettivo
\end{definition}
\begin{remark}
    Ovvero se e solo se \(\text{Ker}\psi = \{e\} \iff (\psi{(g)} = \text{Id}_S
    \iff g = e)\) 
\end{remark}

\begin{example}
    Se \(S = G\) il gruppo stesso e sia
    \[
        m : G \times G \to G \quad \text{ con } \quad m{(g, h)} = gh
    \]
    la moltiplicazione a sinistra. Allora \(m\) è un'azione sinistra, infatti
\begin{enumerate}[label = \roman*)]
    \item \(m {(e, h)} = eh = h\) per ogni \(h \in G\) 
    \item \(m {(gg', h)} = {(g g')}h = g{(g'h)} = m {(g, g'h)}\) per ogni \(g,
        g', h \in G\) 
\end{enumerate}
    Inoltre \(m\) è un'azione fedele, infatti 
    \[
        \psi{(g)}{(h)} = h \quad\forall h \in G \iff gh = h \implies g = e
    \]
    \begin{remark}
        Se \(G\) è un gruppo finito, con \(\# G = n\) allora \(S{(G)} \cong S_n\)
        e poiché \(\psi\) è iniettivo, \(G \cong \psi{(G)} <  S{(G)} \cong S_n\)
        il teorema di Cayley.
    \end{remark}
\end{example}
\begin{example}
    Sempre con \(G = S\) possiamo considerare l'azione di coniugio
    \[
        \varphi : G \times G \to G \quad \text{ con } \quad \varphi{(g, h)} =
        ghg^{-1}
    \]
    \begin{enumerate}[label = \roman*)]
        \item \(\varphi{(e, h)} = ehe^{-1} = h\) per ogni \(h \in G\)
        \item \(\varphi {(g g', h)} = {(g g')}h {( g g' )}^{-1} = g g' h g'^{-1}
            g^{-1} = g{(\varphi {(g', h)})} g ^{-1} = \varphi {(g, \varphi {(g',
            h)})}\) 
    \end{enumerate}
    \(\psi : G \to S(G)\) e \(\text{Im} \psi = \text{Inn} {(G)} <
    \text{Aut}{(G)}\). Non è necessariamente un'azione fedele, infatti
    \[
        \text{Ker}{(\psi)} = \{g \in G : \forall h \in G \quad ghg^{-1} = h\} =
        Z(G)
    \]
    da cui per il primo teorema di isomorfismo
    \[
      G / Z{(G)} = \text{Inn}{(G)}
    \]
\end{example}
\begin{example}
    Con \(G = S_n\) e \(S = \{1, \dots, n\} \) allora la funzione
    \[
        (\sigma, i) \mapsto \sigma{(i)}
    \]
    è ovviamente un'azione
\end{example}
\begin{example}
    Preso \(G \cong \mathbb{Z} / 2 \mathbb{Z} \cong \{1, \sigma\} \) con
    \(\sigma^2 = 1\)  e \(S = \mathbb{C}\) allora la funzione
    \[
        F : G \times \mathbb{C} \to \mathbb{C} \quad \text{ con } \quad F{(1,
        z)} = z \quad \text{ e } \quad F{(\sigma, z)} = \overline{z} \quad
        \forall z \in \mathbb{C}
    \]
    è un'azione.
\end{example}
\begin{definition}{Orbita e Stabilizzatore}
    Sia \(F : G \times  S \to S\) un'azione di un gruppo \(G\) su \(S\). Allora
    per ogni \(s \in S\) si definisce \textbf{orbita} di \(s\) l'insieme
    \[
        O_s = \{g{(s)} : g \in G\}
    \]
    e si definisce \textbf{stabilizzatore} di \(s\) l'insieme
    \[
        \text{stab}_s = \{g \in G : g{(s)} = s\}
    \]
\end{definition}
\begin{example}
    Nell'esempio dell'azione di coniugio lo stabilizzatore di \(h\) è
    \[
        \text{stab}_h = \{g \in G : ghg^{-1} = h\} = \{g \in G : gh = hg\} =
        C_G{(h)}
    \]
\end{example}

\begin{proposition}

Le orbite \(O_s\) per un'azione di \(G\) sono classi di equivalenza per la
relazione di equivalenza su \(S\) seguente:
\[
  S \sim S' \iff \exists g \in G : s' = g{(s)} = F{(g, s)}
\]
\end{proposition}
\begin{proof}
\(\sim\) è in effetti una relazione di equivalenza, infatti:
\begin{itemize}[label = --]
    \item \emph{riflessiva}: \(s = e{(s)}\)
    \item \emph{simmetrica}: se \(s' = g{(s)}\) allora \(s = g^{-1}{(s')}\)
        per la proposizione~\ref{prop:azione_biiezione}
    \item \emph{transitiva}: se \(s' = g{(s)}\) e \(s'' = h{(s')}\) allora
        \(s'' = h{(s')} = h{(g{(s)})} \overset{(ii)}{=} {(hg)}{(s)}\)
\end{itemize}

Ne segue chiaramente che \(O_s = {[s]}_\sim \) e allora \(\displaystyle S = \coprod_{s \in S}
O_s\) 
\end{proof}
\begin{proposition}
    \(\text{stab}_s < G\) 
\end{proposition}
\begin{proof}
    Supponiamo \(g, h \in \text{stab}_s\). Allora \(g{(s)} = h{(s)} = s\), ne
    consegue che
    \[
        F_{gh^{-1}} {(s)} = F_{g} {(F_{h^{-1}} {(s)})}
        \overset{\eqref{eq:azione_inversa}}{=} F_g {(F^{-1}_h{(s)})} =
        F_g {(s)} = s
    \]
\end{proof}



\end{document}

