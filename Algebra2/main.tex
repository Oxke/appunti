%! TEX program = lualatex
\input{../preamble_appunti.tex}

\title{Appunti di Algebra 2}
\author{Github Repository:
\href{https://github.com/Oxke/appunti/tree/main/Algebra2}{\texttt{Oxke/appunti/Algebra2}}}

\date{Secondo semestre, 2024 \-- 2025, prof. Paola Frediani}

\begin{document}
\maketitle

I testi preferiti sono
\begin{itemize}[label = --]
    \item \textit{Algebra}, di Michael Artin
    \item \emph{Algebra}, di Herstein
\end{itemize}

\section{Azioni di gruppi su insiemi}
Chiameremo \(G\) un gruppo e \(S\) un insieme
\begin{definition}{Azione di gruppo}
    Un'azione (sinistra) di \(G\) su \(S\) e un'applicazione
    \[
      F : G\times S \to S
    \]
    tale che
\begin{enumerate}[label = \roman*)]
    \item \(F(e, s) = s\) per ogni \(s \in S\)
    \item \(\forall g, h \in G\) e \(\forall s \in S\) vale \(F{(g, F{(h, s)})}
        = F{(gh, s)}\) 
\end{enumerate}
\end{definition}
Si usa anche la notazione \(F{(g, s)} =: g{(s)}\) che permette la scrittura più
concisa
\[
  e{(s)} = s \quad \text{ e } \quad g{(h{(s)})} = {(gh)}{(s)} \quad \forall s
  \in S, \quad \forall g, h \in  G
\]
\begin{proposition}\label{prop:azione_biiezione}
    Per ogni \(g \in G\), l'applicazione \(F_g : S \to S\) definita da
    \(F_g{(s)} = F{(g, s)} = g{(s)}\) è una biiezione e in particolare
    \begin{equation}\label{eq:azione_inversa}
        F_g^{-1} = F_{g^{-1}}
    \end{equation}
\end{proposition}
\begin{proof}
    \[
      F_g \circ F_{g^{-1}} {(s)} = g{(g^{-1}{(s)})} \overset{(ii)}{=} e{(s)}
      \overset{(i)}{=} s
    \]
    e analogamente per l'altra composizione
\end{proof}

\begin{proposition}

L'applicazione \(\psi : G \to S(S) = \{f : S\to S \text{ biunivoche}\}\) dove
\(S{(S)}\) il gruppo delle permutazioni di \(S\) è un omomorfismo di gruppi.
\end{proposition}
\begin{proof}
\[
    \psi{(gh)}= F_{gh} \overset{(ii)}{=} F_g \circ F_h = \psi{(g)} \circ
    \psi{(h)}
\]
\end{proof}
\begin{definition}{Azione fedele}
    Un'azione \(F : G \times S \to S\) si dice \textbf{fedele} se \(\psi\) è
    iniettivo
\end{definition}
\begin{remark}
    Ovvero se e solo se \(\mathrm{Ker}\psi = \{e\} \iff (\psi{(g)} = \mathrm{Id}_S
    \iff g = e)\) 
\end{remark}

\begin{example}
    Se \(S = G\) il gruppo stesso e sia
    \[
        m : G \times G \to G \quad \text{ con } \quad m{(g, h)} = gh
    \]
    la moltiplicazione a sinistra. Allora \(m\) è un'azione sinistra, infatti
\begin{enumerate}[label = \roman*)]
    \item \(m {(e, h)} = eh = h\) per ogni \(h \in G\) 
    \item \(m {(gg', h)} = {(g g')}h = g{(g'h)} = m {(g, g'h)}\) per ogni \(g,
        g', h \in G\) 
\end{enumerate}
    Inoltre \(m\) è un'azione fedele, infatti 
    \[
        \psi{(g)}{(h)} = h \quad\forall h \in G \iff gh = h \implies g = e
    \]
    \begin{remark}
        Se \(G\) è un gruppo finito, con \(\# G = n\) allora \(S{(G)} \cong S_n\)
        e poiché \(\psi\) è iniettivo, \(G \cong \psi{(G)} <  S{(G)} \cong S_n\)
        il teorema di Cayley.
    \end{remark}
\end{example}
\begin{example}
    Sempre con \(G = S\) possiamo considerare l'azione di coniugio
    \[
        \varphi : G \times G \to G \quad \text{ con } \quad \varphi{(g, h)} =
        ghg^{-1}
    \]
    \begin{enumerate}[label = \roman*)]
        \item \(\varphi{(e, h)} = ehe^{-1} = h\) per ogni \(h \in G\)
        \item \(\varphi {(g g', h)} = {(g g')}h {( g g' )}^{-1} = g g' h g'^{-1}
            g^{-1} = g{(\varphi {(g', h)})} g ^{-1} = \varphi {(g, \varphi {(g',
            h)})}\) 
    \end{enumerate}
    \(\psi : G \to S(G)\) e \(\mathrm{Im} \psi = \mathrm{Inn} {(G)} <
    \mathrm{Aut}{(G)}\). Non è necessariamente un'azione fedele, infatti
    \[
        \mathrm{Ker}{(\psi)} = \{g \in G : \forall h \in G \quad ghg^{-1} = h\} =
        Z(G)
    \]
    da cui per il primo teorema di isomorfismo
    \[
      G / Z{(G)} = \mathrm{Inn}{(G)}
    \]
\end{example}
\begin{example}
    Con \(G = S_n\) e \(S = \{1, \dots, n\} \) allora la funzione
    \[
        (\sigma, i) \mapsto \sigma{(i)}
    \]
    è ovviamente un'azione
\end{example}
\begin{example}
    Preso \(G \cong \mathbb{Z} / 2 \mathbb{Z} \cong \{1, \sigma\} \) con
    \(\sigma^2 = 1\)  e \(S = \mathbb{C}\) allora la funzione
    \[
        F : G \times \mathbb{C} \to \mathbb{C} \quad \text{ con } \quad F{(1,
        z)} = z \quad \text{ e } \quad F{(\sigma, z)} = \overline{z} \quad
        \forall z \in \mathbb{C}
    \]
    è un'azione.
\end{example}
\begin{definition}{Orbita e Stabilizzatore}
    Sia \(F : G \times  S \to S\) un'azione di un gruppo \(G\) su \(S\). Allora
    per ogni \(s \in S\) si definisce \textbf{orbita} di \(s\) l'insieme
    \[
        O_s = \{g{(s)} : g \in G\}
    \]
    e si definisce \textbf{stabilizzatore} di \(s\) l'insieme
    \[
        \mathrm{stab}_s = \{g \in G : g{(s)} = s\}
    \]
\end{definition}
\begin{example}
    Nell'esempio dell'azione di coniugio lo stabilizzatore di \(h\) è
    \[
        \mathrm{stab}_h = \{g \in G : ghg^{-1} = h\} = \{g \in G : gh = hg\} =
        C_G{(h)}
    \]
\end{example}

\begin{proposition}

Le orbite \(O_s\) per un'azione di \(G\) sono classi di equivalenza per la
relazione di equivalenza su \(S\) seguente:
\[
  S \sim S' \iff \exists g \in G : s' = g{(s)} = F{(g, s)}
\]
\end{proposition}
\begin{proof}
\(\sim\) è in effetti una relazione di equivalenza, infatti:
\begin{itemize}[label = --]
    \item \emph{riflessiva}: \(s = e{(s)}\)
    \item \emph{simmetrica}: se \(s' = g{(s)}\) allora \(s = g^{-1}{(s')}\)
        per la proposizione~\ref{prop:azione_biiezione}
    \item \emph{transitiva}: se \(s' = g{(s)}\) e \(s'' = h{(s')}\) allora
        \(s'' = h{(s')} = h{(g{(s)})} \overset{(ii)}{=} {(hg)}{(s)}\)
\end{itemize}

Ne segue chiaramente che \(O_s = {[s]}_\sim \) e allora \(\displaystyle S = \coprod_{s \in S}
O_s\) 
\end{proof}
% TODO separa in a parte corollario e reffalo in dopo
\begin{proposition}
    \(\mathrm{stab}_s < G\) 
\end{proposition}
\begin{proof}
    Supponiamo \(g, h \in \mathrm{stab}_s\). Allora \(g{(s)} = h{(s)} = s\), ne
    consegue che
    \[
        F_{gh^{-1}} {(s)} = F_{g} {(F_{h^{-1}} {(s)})}
        \overset{\eqref{eq:azione_inversa}}{=} F_g {(F^{-1}_h{(s)})} =
        F_g {(s)} = s
    \]
\end{proof}

\begin{definition}{Azione transitiva}
    Un'azione \(F : G \times S \to S \) si dice \textbf{transitiva} se per ogni
    \(s, s' \in S\) esiste \(g \in G\) tale che \(s' = g{(s)}\)
\end{definition}

\begin{proposition}
    Sia \(F : G \times  S \to S\) un'azione di gruppo. Allora fissato un \(s \in
    S\), consideriamo \(O_s \subseteq   S\) e \(H := \mathrm{stab}_s < G\). Allora esiste
    un'applicazione naturale biettiva
    \begin{align*}
        \Phi: G / H &\longrightarrow O_s \\
        gH &\longmapsto \Phi(gH) = g{(s)} = F(g, s)
    \end{align*}
    Inoltre per ogni \(C \in G / H\), \(g{(\Phi{(C)})} = \Phi{(g{(C)})}\) dove
    la prima azione è quella di \(G\) su \(O_s\) e la seconda è quella di \(G\)
    su \(G / H\)
\end{proposition}
\begin{proof} \(\) 
    \begin{itemize}[label = --]
        \item \emph{Ben definita}: se \(aH = bH\) allora \(b^{-1}a \in H\) e
            quindi esiste un \(h \in H\) tale che \(b^{-1}a = h\) e quindi \(a =
            bh\). Allora \(F{(a, s)} = F{(bh, s)} = F{(b, F{(h, s)})}\) =
            \(F{(b, s)}\) 
        \item \emph{Iniettiva}: supponiamo che esistano \(a, b \in G\) tali che
            \(\Phi{(aH)} = \Phi{(bH)}\), allora \(F{(a, s)} = F{(b, s)}\) ma
            allora
            \[
                F{(b^{-1}a, s)} = F{(b^{-1}, F{(a, s)})} = F{(b^{-1}, F{(b,
                s)})} = F{(b^{-1}b, s)} = F{(e, s)} = s
            \]
            e quindi \(b^{-1}a \in H \iff aH = bH\) 
        \item \emph{Suriettiva}: per ogni \(s' \in O_s\) esiste \(g \in G\) tale
            che \(s' = g{(s)}\) e quindi \(s' = g{(s)} = \Phi(gH)\)
    \end{itemize}
\end{proof}
\begin{corollary}
    Se \(G\) è un gruppo finito e ho un'azione \(F : G \times S \to S\), allora
    per ogni \(s \in S\) vale \(\# O_s = [G : \text{stab}_s]\) o
    equivalentemente
    \[
        \# G = \# O_s \cdot \# \mathrm{stab}_s
    \]
    e inoltre
    \[
        \# G = \sum_{[s] \in S} \# O_s
    \]
\end{corollary}
\begin{corollary}\label{cor:classi_coniugio}
    Sia \(F : G \times G \to G\) l'azione di coniugio \({(g, h)} \mapsto
    ghg^{-1}\). Ricordiamo che \(\mathrm{stab}_a = C{(a)}\) e la formula delle
    classi si traduce in 
    \[
        \# G = \# C{(a)} \cdot \# O_a = \sum_{[g] \in G} \# O_g = \sum_{[g] \in
        G} \frac{\# G}{\# C{(g)}}
    \]
    inoltre se \(g \in Z{(G)}\) allora \(C{(g)} = G\) e dunque
    \begin{equation}\label{eq:formula_classi}
        \# G = \# Z + \sum_{[g] \in G\sminus Z} \# O_g = \# Z + \sum_{[g] \in G
        \sminus Z} \frac{\# G}{\# C{(g)}}
    \end{equation}
\end{corollary}
\begin{theorem}
    Sia \(G\) un gruppo tale che \(\# G = p^{n}\) con \(p\) primo. Allora
    \(Z{(G)} \neq \{e\} \) 
\end{theorem}
\begin{proof}
    Se \(a \not\in Z\) allora \(C{(a)} = p^{n_a}\) con \(n_a < n\) e quindi da
    \[
        p^{n} = \# G = \# Z + \sum_{[g] \in G\sminus Z} \# \frac{p^{n}}{p^{n_a}}
    \]
    ne deduciamo che \(p \mid \#Z\) 
\end{proof}
\begin{corollary}
    Sia \(G\) un gruppo di cardinalità \(p^2\), con \(p\) primo. Allora \(G\) è
    abeliano.
\end{corollary}
\begin{proof}
    Per il teorema sappiamo che \(Z \neq \{e\} \) e quindi \(\# Z = p\) oppure
    \(\# Z = p^2\). Nel secondo caso \(G = Z\) e quindi è abeliano. Nel primo
    caso invece esiste un \(a \in G \sminus Z\) e dunque \(C{(a)} \neq G\). Ma
    \[
        \{e\} < Z < C{(a)} < G
    \]
    e quindi \(C{(a)} = Z\) per cardinalità che è assurdo perché \(a \in
    C{(a)}\) e \(a \not\in Z\).
\end{proof}
\begin{example}
    Riprendendo l'esempio della moltiplicazione a sinistra \(m : G \times G \to
    G\). Allora \(m\) è un'azione transitiva. Infatti per ogni \(g', g'' \in G\) se
    prendo \(h = {(g')}^{-1} g''\) allora \(m {(g', h)} = g'{(g'^{-1} g'')} =
    g''\) 
\end{example}
\begin{example}
    Se prendo \(GL{(V)}\) il gruppo lineare delle trasformazioni invertibili su
    uno spazio vettoriale \(V\), allora l'azione \({(T, v)} \mapsto Tv\) è
    transitiva su \(V \sminus \{0\}\)  
\end{example}
\begin{theorem}[Cauchy per gruppi abeliani]
    Sia \(G\) un gruppo abeliano finito e \(p\) un primo tale che \(p \mid \# G\).
    Allora 
    \[
      \exists e\neq a \in G \text{ tale che } a^{p} = e
    \]
\end{theorem}
\begin{proof}
    Procediamo per induzione su \(n = \# G\).
    Se \(2 = \# G\) allora \(G = \{e, a\} \) e dunque \(a^2 = e\). Supponiamo
    ora \(\# G \ge 3\).

    Se \(G\) non ha sottogruppi \(e \neq H \neq G\) allora \(G\) è ciclico di
    ordine primo. Infatti se \(G\) non è ciclico allora esistono due elementi
    \(e\neq g_{1}, g_{2}\) e \(g_{2} \not\in \Span{g_{1}} \). Ma allora
    \(\{e\} \neq \Span{g_{1}} \neq G\) è un sottogruppo. Dunque \(G\) è
    ciclico, inoltre è di ordine primo perché se così non fosse (ad esempio \(n
    = ab\)) allora \(\{e\} \neq \Span{g^{a}} \neq G\) è un sottogruppo, con
    \(g\) tale che \(<g> = G\).

    Allora se \(G\) non ha sottogruppi propri esistono \(p-1\) elementi in \(G\)
    di ordine \(p\).

    Supponiamo ora che \(G\) abbia qualche sottogruppo non banale. Sia \(N < G\)
    con \(\{e\} \neq N \neq G\). Allora se \(p \mid \# N\) per ipotesi induttiva si
    conclude. Se invece \(p\nmid \# N\) allora \(G / N\) è un gruppo abeliano
    con \(\# G / N < \# G\) e quindi per ipotesi induttiva (infatti \(G/N\) ha
    ordine multiplo di \(p\) poiché \(N\) non lo è) esiste \(bN \in G/N\), \(b
    \not\in N\) e tale che \(b^{p} \in N\). Allora \(b^{p\#N} = e\) e ci resta
    solo da dimostrare che \(c := b^{\#N} \neq e\).

    Supponiamo che \(c = b^{\#N} = e\). Sappiamo che \(MCD(p, \#N) = 1\) e
    dunque per il teorema di Bézout esistono \(\alpha, \beta \in \mathbb{Z}\)
    tali che \(\alpha p + \beta \#N = 1\).
    Allora 
    \[
      bN = {(bN)}^{\alpha p + \beta\# N} = {(bN)}^{\alpha p} \cdot {(bN)}^{\beta
      \#N}
    \]
    e poiché \(b^{p} \in N\) e \(b^{\#N} = e\) otteniamo che \(bN = N\) che è
    assurdo perché \(b \not\in N\).
\end{proof}
\begin{theorem}[Cauchy]
    Sia \(G\) è un gruppo finito e \(p\) è un primo tale che \(p \mid \# G\).
    Allora
    \[
        \exists a \in G \text{ tale che } \# \langle a \rangle = p
    \]
\end{theorem}
\begin{proof}
    Vogliamo procedere per induzione su \(\#G\). Se \(\#G = 2\) è già
    dimostrato. Se esiste \(H < G\) tale che \(p \mid \# H\) concludo per ipotesi
    induttiva.

    Supponiamo dunque che \(p\) non divide l'ordine di nessun sottogruppo
    proprio di \(G\). Nella formula delle classi~\eqref{eq:formula_classi},
    tutti i termini della serie sono divisibili per \(p\), infatti se \(a
    \not\in Z\), \(C{(a)}\neq G\) è un sottogruppo proprio e quindi \(p \nmid
    \# C{(a)}\). Allora \(p \mid \# Z\) ma quindi \(Z = G\) e quindi \(G\) è
    abeliano. Concludiamo con il teorema di Cauchy per gruppi abeliani.
\end{proof}
\begin{theorem}[Sylow (prima parte)]
    Sia \(G\) un gruppo finito e \(p\) un primo tale che esiste \(\mathbb{Z} \ni
    m \ge 1\)  tale che \(p^{m} \mid \# G\) ma \(p^{m+1} \nmid G\). Allora
    \[
        \exists H < G \text{ tale che } \# H = p^{m}
    \]
\end{theorem}
\begin{proof}
    Procediamo per induzione su \(\# G\) e su \(m\). Se \(\# G = 2\) allora \(H = G\).
    Supponiamo ora \(\#G > 2\) e il risultato vero per ogni gruppo di
    cardinalità minore di \(G\). Se \(m=1\) allora ci si riduce al teorema di
    Cauchy. Supponiamo ora che \(m \ge 2\) e \(\#G \ge 3\).

    Se esiste \(H < G\) proprio con \(p^{m} \mid \# G\) allora concludo per ipotesi
    induttiva.

    Supponiamo dunque che \(p^{m} \nmid \# K\) per ogni \(K < G\) proprio.

    Come nel teorema di Cauchy usiamo~\eqref{eq:formula_classi}, otteniamo che
    \(p \mid \# Z\) e quindi per il teorema di Cauchy abbiamo che esiste \(e \neq b
    \in Z\) tale che \(b^{p} = e\). Allora \(\Span{b} \trianglelefteq G\). 

    Ma allora poiché \(\# (G / \Span{b} ) = \# G / p\) abbiamo che \(p^{m-1} \mid
    \# {(G / \Span{b} )}\) e dunque per ipotesi induttiva esiste \(\overline{S}
    < G / \Span{b} \) tale che \(\# \overline{S} = p^{m-1}\). Allora \(S = \pi
    ^{-1}{(\overline{S})}\) è un sottogruppo di \(G\) e \(\overline{S} = S /
    B\). Ne consegue infine che \(\# S = p^{m}\).
\end{proof}
\begin{remark}
    Il sottogruppo \(H\) viene detto \(p\)-sottogruppo di Sylow di \(G\).
\end{remark}

% TODO: aggiungere due lezioni mancanti

\section{Gruppi di permutazioni}
\begin{definition}{Definizioni riguardo alle permutazioni}
    Una permutazione \(\sigma \in S_n\) si dice \textbf{pari}se si può scrivere
    come prodotto di un numero pari di trasposizioni. Questo è ben definito
    perché esiste un unico omomorfismo \(\varepsilon : S_n \to \mathbb{Z} / 2\)
    \(\varepsilon(\sigma) = 1\) se \(\sigma\) è una trasposizione. Tale
    omomorfismo è detto \textbf{segno} della permutazione.

    Il \textbf{gruppo alterno} \(A_n\) è il nucleo di \(\varepsilon\).
\end{definition}
Poiché \(A_{n}\) ha indice 2 in \(S_{n}\), \(A_{n} \trianglelefteq S_{n} \).

\begin{remark}[Ogni \(\sigma \in A_{n}\) si può esprimere come prodotto di
    3-cicli] Basta mostrare che ogni prodotto di 2 trasposizioni è scrivibile
    come prodotto di 3-cicli. Se \(\sigma = (a\, b)(b\, c) = (a\, b\, c)\) è
    facile. Se invece \(\sigma = (a\, b)(c\, d)\) disgiunti allora \(\sigma =
    (c\,b\,a)(a\,c\,d)\). Notando che i 3-cicli sono pari si conclude anche che il
    sottogruppo generato dai 3-cicli è esattamente \(A_{n}\) 
\end{remark}

\begin{example}[Per \(n \ge 5\) tutti i 3-cicli sono tra loro coniugati in
    \(A_{n}\)] Prendiamo \(\sigma  = (i\,j\,k)\) e \(\sigma' = (i'\,j'\,k')\) due
    3-cicli. Allora esiste \(\gamma \in S_{n}\) tale che \(\gamma \sigma
    \gamma^{-1} = \sigma'\). Supponiamo \(\gamma \not\in A_{n}\). Siccome \(n
    \ge 5\) allora esistono \(r, s \in \{1, \dots, n\} \sminus \{i, j, k\}  \).
    Ma allora \(\theta = \gamma\cdot {(r\, s)} \in A\cap \) e \(\theta \sigma
    \theta^{-1} = \sigma'\)
\end{example}
\begin{eser}[2 elementi coniugati in \(S_{5}\) ma non in \(A_{5}\)] 
    Mostrare che \({(1\,2\,3\,4\,5)}\) e \({(2\,1\,3\,4\,5)}\) non sono
    coniugati in \(A_{5}\) (ma lo sono in \(S_{5}\)).
\end{eser}

\begin{definition}{Gruppo semplice}
    \(G\) si dice \textbf{semplice} se gli unici sottogruppi normali sono
    \(\{e\} \) e \(G\) stesso.
\end{definition}

\begin{theorem}
    \(\forall n \ge 5\), \(A_{n}\) è semplice
\end{theorem}
\begin{proof}
Sia \(N \trianglelefteq A_{n}\) con \(N \neq \{e\} \). Dimostriamo che \(N\)
contiene un 3-ciclo \(\sigma\) allora \(N\) contiene tutti i coniugati di
\(\sigma\) perché è normale. 

Sia \(\sigma \in N\) con \(\sigma \neq 1\) con il massimo numero di punti
fissi. Vogliamo mostrare che \(\sigma\) è un 3-ciclo.

Consideriamo l'azione \(\Span{\sigma} \times \{1, \dots, n\} \to \{1, \dots, n\}
\) data da \({(\sigma^{k}, i)} \mapsto \sigma^{k}{(i)}\). Siccome \(\sigma \neq
e\) esiste un \(i\) tale che \(\sigma{(i)} = j \neq i\). Dunque \(O_i \ni \{i,
j\} \) e quindi \(\# O_i \ge  2\).

\begin{enumerate}[label = \arabic*.]
    \item Supponiamo che tutte le orbite di elementi non fissati da \(\sigma\)
        abbiano cardinalità 2. Poiché \(\sigma\) è pari ci devono essere almeno
        2 orbite \(O_i, O_r\) di cardinalità 2 (porta ad assurdo) % TODO assurdo
\end{enumerate}
Quindi esiste almeno un'orbita di \(\Span{\sigma} \) con 3 elementi \(i, j, k\).
Se \(\sigma = {(i\,j\,k)}\) ho finito. Altrimenti, \(\sigma\) necessariamente
deve muovere altri due elementi, essendo pari. Siano essi \(r\) e \(s\). Allora
prendiamo \(\tau = {(k\,r\,s)}\). A questo punto
\[
    \sigma' = \underbrace{\tau \sigma \tau^{-1}}_{\in N} \,\, \sigma^{-1} \in N
\] 
Allora tutti i punti fissi di \(\sigma\) lo sono anche per \(\sigma'\) e inoltre
\(\sigma'{(j)} = j\) che non è punto fisso di \(\sigma\), dunque \(\sigma'\) ha
più punti fissi di \sigma, assurdo.
\end{proof}
\begin{example}[\(A_{4}\) non è semplice]
    \(A_{4} = \{e, (1\,2\,3), (1\,3\,2), (1\,2\,4), (1\,4\,2),\\ (2\,3\,4),
    (2\,4\,3), (1\,3\,4), (1\,4\,3), (2\,3\,4), (2\,4\,3), (3\,4\,2),
    (3\,2\,4)\}\) e \(H = \{e, (1\,2)(3\,4), \\(1\,3)(2\,4), (1\,4)(2\,3)\}\) è
    normale in \(A_{4}\) e non è \(\{e\} \) o \(A_{4}\). Infatti è l'unione
    dell'intera classe di coniugio del tipo di due trasposizioni disgiunte (e
    l'identità).
\end{example}

\begin{definition}{Gruppo risolubile}
    Un gruppo \(G\) si dice \textbf{risolubile} se esiste una catena di
    sottogruppi 
    \[
      \{e\}  = G_{0} \trianglelefteq G_{1} \trianglelefteq G_{2} <~\dots
      \trianglelefteq G_{n} = G
    \]
    dove tutti i quozienti \(G_{i+1} / G_{i} \) sono abeliani per ogni \( i = 0,
    \dots, n-1\) 
\end{definition}
\begin{remark}
    Non è detto che \(G_{i} \trianglelefteq G\) per ogni \(i\).  
\end{remark}
\begin{example}[Gruppi abeliani]
    Ogni gruppo abeliano è risolubile, con la catena \(\{1\} \trianglelefteq G\) 
\end{example}
\begin{example}[\(S_3\) ]
    \(S_{3}\) è risolubile con la catena
    \[
        \{1\} \trianglelefteq A_{3} \trianglelefteq S_{3} 
    \]
    Infatti \(S_{3} / A_{3} \cong \mathbb{Z} / 2\mathbb{Z}\) è abeliano e
    \(A_{3} / \{1\} \cong \mathbb{Z} / 3\mathbb{Z}\) è abeliano.
\end{example}
\begin{example}
    \(D_{n}\) è risolubile con la catena
    \[
        \{1\}  \trianglelefteq \langle r \rangle \trianglelefteq D_{n}
    \]
    Infatti \(D_{n} = \Span{r, s \mid r^{n} = 1, s ^2 = 2, rs = sr^{-1} } =
    \Span{r} \rtimes_\varphi \Span{s}  \)
\end{example}
\begin{example}
    \(S_{4}\) è risolubile con la catena
    \[
        \{1\} \trianglelefteq V_4 \trianglelefteq A_{4} \trianglelefteq S_{4}
    \]
    con \(V_{4}\) il gruppo di Klein (in particolare anche \(A_{4}\) è
    risolubile). Notare l'osservazione precedente: Il sottogruppo
    \(\Span{{(1\,2)}{(3\,4)}} \) è normale in \(V_{4}\) ma non in \(S_{4}\) 
\end{example}

\paragraph{Gruppi di ordine 12}
Vogliamo mostrare che esistono esattamente 5 classi di isomorfismo di gruppi di
ordine 12:
\begin{itemize}[label = --]
    \item \(\mathbb{Z} / 12\mathbb{Z}\)
    \item \(\mathbb{Z} / 2\mathbb{Z} \times \mathbb{Z} / 6\mathbb{Z}\)
    \item \(D_{6}\) 
    \item \(A_{4}\) 
    \item \(\Span{x, y \mid x^4 = 1, y^3=1, xyx^{-1} = y^2} = \mathbb{Z} / 3
        \mathbb{Z} \rtimes \mathbb{Z}/ 4 \mathbb{Z}\) 
\end{itemize}
\begin{proof}
    \(12 = 2^2 \cdot 3\) quindi per il teorema di Sylow esiste un 2-Sylow e un
    3-Sylow. Sia \(H\) un 2-Sylow e \(K\) un 3-Sylow. Allora \(H \cong
    \mathbb{Z}_2 \times \mathbb{Z}_2\) oppure \(H \cong \mathbb{Z}_4\), \(K
    \cong \mathbb{Z}_3\). Per il teorema di Sylow il numero \(r\) di 2-Sylow
    deve essere 1 oppure 3. % TODO: controlla
    Inoltre \(s\) il numero di 3-Sylow deve essere 1 oppure 4. % TODO: controlla

    Ora necessariamente \(H \trianglelefteq G\) oppure \(K \trianglelefteq G\).
    Infatti supponiamo che \(K\) non sia normale in \(G\), allora esistono 4
    3-Sylow tra loro coniugati. % TODO: Integrare
    Li chiamiamo \(K_{1}, K_{2}, K_{3}, K_{4}\). Allora poiché 3 è primo,
    \(K_{i} \cap K_{j} = \{1\} \) per ogni \(i\neq j\). Allora
    \(\# \bigcup_{i=1}^{4} K_{i} = 9 \)  ma quindi gli unici elementi che non
    prende sono 3 elementi. Allora \(H\), un 2-Sylow, deve contenere tali 3
    elementi. Essendocene uno solo, è normale. Ora possiamo procedere nei casi
    seguenti:
\begin{enumerate}[label = \arabic*.]
    \item \(H \trianglelefteq G\) e \(K \trianglelefteq G\).

        Allora \(H \cap K = \{e\} \). \(HK < G\) e \(\# HK = \frac{\# H \cdot \#
        K}{\# H \cap K} = 12\). Allora \(HK = G\) e \(G \cong H \times K\)
        tramite l'isomorfismo \({(h, k)} \mapsto hk\) 

        Allora ne consegue che \(G \cong \mathbb{Z}_4 \times \mathbb{Z}_3 =
        \mathbb{Z}_{12} \) oppure \(G \cong \mathbb{Z}_2 \times \mathbb{Z}_2
        \times \mathbb{Z}_3 \cong \mathbb{Z}_2 \times \mathbb{Z}_6\) 

    \item \(K\) non è normale e \(H \triangleleft G\).

    Sia \(\psi : G \to A{(S)} \cong S_{4}\).
    Allora \(\mathrm{Stab}_{K_{i}}  = \{g \in G : gK_{i}g^{-1} = K_{i}\} =
    N{(K_{i})}\). Ma allora \(\# N{(K_{i})} = 3\) (formula delle classi) e
    poiché \(K_{i} \subseteq N{(K_{i})} \) e ha la stessa cardinalità, essi
    coincidono. 
     Ma allora \(\mathrm{Ker}\psi = \{e\} \) e \(\psi\) è iniettivo.

     \(G\) ha 4 3-Sylow e dunque ha \(8\) elementi di ordine 3. Ma questi sono
     anche tutti i 3-cicli di \(S_{4}\), e dunque per cardinalità \(G \cong A_{4}\)  

     \item \(K \triangleleft G\) e \(H\) non è normale.
        
        
\end{enumerate}
% TODO: completare + lezione mancante
\end{proof}

\begin{definition}{}
     Sia \(G\) un gruppo abeliano di ordine \(p^{n}\) con \(p\) primo. Allora
     \(G \cong A_{1}\times \dots A_k\) con \(\Span{a_{i}} = A_{i}\) e
     \(o{(a_{i})} = p^{n_{i}}\) e \( n_{1} \ge n_{2} \ge \dots \ge n_k > 0\).

     Allora gli \(n_{i}\) si dicono invarianti di \(G\) e 
     \[
       \# G = p^{n} = p^{n_{1}} \cdot \dots \cdot p^{n_{k}}
     \]
     gli invarianti danno una partizione di \(n\).
\end{definition}
\begin{remark}
\(\forall  m \in \mathbb{Z}\quad G{(m)} = \{x \in G: x^{m} = e\} \le  G\) 
\end{remark}

\begin{lemma}\label{lem:2}
    Sia \(G\) un gruppo abeliano come nella definizione precedente. Sia \(m \in
    \mathbb{Z}\) tale che \(n_t > m \ge n_{t+1} \). Allora
    \[
      G{(p^{m})} \cong B_{1} \times \dots \times B_{t} \times A_{t+1} \times
      \dots \times A_{k}
    \]
    dove \(B_{i}\) è ciclico di ordine \(p^{m}\) e 
    \[
      \# G{(p^{m})} = p^{u} \quad u = tn + \sum_{j=t+1}^{k} n_{j} 
    \]
\end{lemma}
\begin{proof}
    %TODO fare tutta la dimostrazione
\end{proof}
\begin{corollary}
    Nelle ipotesi del lemma~\ref{lem:2}, \(\# G{(p)} = p^{k}\) 
\end{corollary}

\begin{theorem}[Classificazione]
    Due gruppi abeliani di ordine \(p^{n}\), con \(p\) primo sono isomorfi se e
    solo se hanno gli stessi invarianti.

    Equivalentemente se \(G\) e \(G'\) sono due gruppi abeliani con \(\#G = \#G'
    = p^{n}\) e
    \begin{align*}
        G &\cong A_{1} \times \dots \times A_{k} \quad A_{i} = \Span{a_{i}}
        \quad {(o{(a_{i})} = p^{n_{i}})} \quad n_{1} \ge n_{2} \ge \dots \ge n_k
        > 0\\ 
        G' &\cong A'_{1} \times \dots \times A'_{s} \quad A'_{i} = \Span{a'_{i}}
        \quad o{(a_{i}')} = p^{n_{i}'} \quad n_{1}' \ge n_{2}' \ge \dots \ge n_s' > 0
    \end{align*}
\end{theorem}
\begin{proof}\( \)
\begin{itemize}
    \item[\(\implies \)] Se \(G \cong G'\) allora \(G{(p)} \cong G'{(p)}\) e
        dunque \(\# G{(p)} = \# G'(p)\) e quindi \(k = s\).

        Supponiamo ora che esista un \(i\)  tale che \(n_{i} \neq h_{i}\). Sia
        \(t\) il primo tale intero, supponendo che \(m := n_t > h_t\) 
        Sia \(H = \{x^{p^{m}} : x \in G\} \le G\) perché \(G\) è abeliano e \(H'
        = \{x'^{p^{m}} : x' \in G\} \le G\) perché \(G'\) è abeliano.
    \item[\(\impliedby \)] se \(k = s\) e \(n_{i} = n_{i}'\) \(\forall i\)
        allora \begin{align*}
            \Phi: G &\longrightarrow G' \\
        {(a_{1}^{\alpha_{1}} \dots a_k^{\alpha_k})} &\longmapsto
        \Phi({(a_{1}^{\alpha_{1}} \dots a_k^{\alpha_kk})}) =
            {(a_{1}'^{\alpha_{1}} \dots a_{k}'^{\alpha_{k}})} 
        \end{align*}
\end{itemize}
\end{proof}
\begin{corollary}
    Il numero di classi di isomorfismo di gruppi abeliani di ordine \(p^{n}\) è
    il numero di partizioni di \(n\)
\end{corollary}
\begin{corollary}
    Il numero di classi di isomorfismo di gruppi abeliani di ordine
    \(p_{1}^{\alpha_{1}} \cdot \dots \cdot p_r^{\alpha_r}\) con \(p_{i}\) primi
    distinti è uguale a \(p{(\alpha_{i})}\cdot \dots \cdot p^{\alpha_r}\) dove
    \(p^{\alpha_{i}}\) è il numero di partizioni di \(\alpha_{i}\) 
\end{corollary}
\begin{example}
    I gruppi abeliani di ordine \(8\) sono \(3\), infatti \(8 = 2^{3}\) e \(3\)
    ha 3 partizioni:
\begin{itemize}[label = --]
    \item \(3 = 3 \implies C_8\)
    \item \(3 = 2+1 \implies C_4 \times C_2 \)
    \item \(3 = 1 + 1 + 1 \implies C_{2} \times C_{2} \times C_{2}\) 
\end{itemize}
che per il teorema sono sicuramente non isomorfi e tutti.
\end{example}
\begin{example}
    I gruppi abeliani di ordine \(12\) sono 2, infatti \(12 = 2^{2}\cdot 3\) e
    le partizioni di 2 sono 2: \(2 = 2\) e \(2 = 1 + 1\), e i gruppi sono
    \(C_{2}\times C_{2}\times C_{3}\) e \(C_{4} \times C_{3}\) 
\end{example}
\begin{eser}
    Elencare i gruppi abeliani di ordine 36
\end{eser}

\begin{eser}
    Supponiamo di avere un gruppo \(G\) di ordine \(1125 = 3^2 \cdot
    5^3\)a tale che un suo 5-Sylow sia ciclico. Allora:
\begin{itemize}[label = --]
    \item Mostrare che \(G\) è abeliano
    \item Determinare tutte le classi di isomorfismo possibili per \(G\)  
\end{itemize}
\tcblower
Esiste solo un 5-Sylow, infatti \(1+5k \mid 3^2 \iff k = 0\) e quindi è normale.
Per questioni di ordine deve essere quindi che \(G \cong H \rtimes_\varphi  K\) con
\(H\) il 5-Sylow e \(K\) un 3-Sylow.
Allora sapendo che \(H\) è ciclico, \(\mathrm{Aut}{(G)} = {C_{125}}^{\star}\)
che ha ordine \(\varphi{(125)} = 100\). Allora \(\mathrm{Im}{(\varphi)} <
\mathrm{Aut}{(H)}\) ma inoltre \(\mathrm{Im}{(\varphi )}< K\) e dunque
necessariamente \(\# \mathrm{Im}{(\varphi )} = 1\) e quindi \(G \cong H \times
K\) è abeliano.

Sapendo che \(K\) è abeliano di ordine 9 possiamo trovare le due possibili
classi di isomorfismo, ossia \(C_{1125} \) e \(C_{375} \times C_{3}\)
\end{eser}

\begin{eser}
    Siano \(A, B\) gruppi abeliani. Consideriamo 
    \[
        \mathrm{Hom}{(A, B)} = \{f : A \to B : f \text{ è un omomorfismo}\}
    \]
    allora \(\mathrm{Hom}{(A, B)}\) è un gruppo abeliano rispetto alla somma
    data da
    \[
      {(f + g)}(a) = f(a) + g(a)
    \]
    Consideriamo l'omomorfismo \(\Phi : \mathrm{Hom}{(\mathbb{Z}_m,
    \mathbb{Z}_n)} \to \mathbb{Z}_n \) dato da \(\Phi{(f)} = f{([1]_m)}\) 

    Dimostrare che \(\Phi\) è un omomorfismo iniettivo di gruppi.

    \tcbline

    % TODO risolvere esercizio
    \tcbline
\end{eser}


\end{document}
