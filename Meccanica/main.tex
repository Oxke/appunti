%! TEX program = lualatex
\documentclass{article}
\usepackage{layout}
\usepackage[a4paper, total={5in,9in}]{geometry}
\usepackage[T1]{fontenc}
\usepackage[italian]{babel}
\usepackage{mathtools}
\usepackage{amsthm}
\usepackage[framemethod=TikZ]{mdframed}
\usepackage{amsmath}
\usepackage{amssymb}
\usepackage{cancel}
\usepackage[dvipsnames]{xcolor}
\usepackage{tikz}
\usepackage{tikz-cd}
\usepackage{pgfplots}
\pgfplotsset{compat=1.18}
\usepackage[many]{tcolorbox}
\usepackage{import}
\usepackage{pdfpages}
\usepackage{transparent}
\usepackage{enumitem}
\usepackage[colorlinks]{hyperref}

\newcommand*{\sminus}{\raisebox{1.3pt}{$\smallsetminus$}}

\newcommand*{\transp}[2][-3mu]{\ensuremath{\mskip1mu\prescript{\smash{\mathrm t\mkern#1}}{}{\mathstrut#2}}}%

% newcommand for span with langle and rangle around
\newcommand{\Span}[1]{{\left\langle#1\right\rangle}}

\newcommand{\incfig}[2][1]{%
    \def\svgwidth{#1\columnwidth}
    \import{./figures/}{#2.pdf_tex}
}

\pdfsuppresswarningpagegroup=1

\newcounter{theo}[section]\setcounter{theo}{0}
\renewcommand{\thetheo}{\arabic{section}.\arabic{theo}}

\newcounter{excounter}[section]\setcounter{excounter}{0}
\renewcommand{\theexcounter}{\arabic{section}.\arabic{excounter}}

\numberwithin{equation}{section}

\newenvironment{theorem}[1][]{
    \refstepcounter{theo}
     \ifstrempty{#1}
    {\mdfsetup{
        frametitle={
            \tikz[baseline=(current bounding box.east),outer sep=0pt]
            \node[anchor=east,rectangle,fill=blue!20,rounded corners=5pt]
            {\strut Teorema~\thetheo};}
        }
    }{\mdfsetup{
        frametitle={
            \tikz[baseline=(current bounding box.east),outer sep=0pt]
            \node[anchor=east,rectangle,fill=blue!20,rounded corners=5pt]
            {\strut Teorema~\thetheo:~#1};}
        }
    }
    \mdfsetup{
        roundcorner=10pt,
        innertopmargin=10pt,linecolor=blue!20,
        linewidth=2pt,topline=true,
        frametitleaboveskip=\dimexpr-\ht\strutbox\relax,
        % nobreak=false
    }
\begin{mdframed}[]\relax}{
\end{mdframed}}

% \newenvironment{definition}[1][]{
%     \refstepcounter{theo}
%      \ifstrempty{#1}
%     {\mdfsetup{
%         frametitle={
%             \tikz[baseline=(current bounding box.east),outer sep=0pt]
%             \node[anchor=east,rectangle,fill=violet!20,rounded corners=5pt]
%             {\strut Definizione~\thetheo};}
%         }
%     }{\mdfsetup{
%         frametitle={
%             \tikz[baseline=(current bounding box.east),outer sep=0pt]
%             \node[anchor=east,rectangle,fill=violet!20,rounded corners=5pt]
%             {\strut Definizione~\thetheo:~#1};}
%         }
%     }
%     \mdfsetup{
%         roundcorner=10pt,
%         innertopmargin=10pt,linecolor=violet!20,
%         linewidth=2pt,topline=true,
%         frametitleaboveskip=\dimexpr-\ht\strutbox\relax,
%         nobreak=true
%     }
% \begin{mdframed}[]\relax}{
% \end{mdframed}}

\newtcolorbox[auto counter, number within=section]{definition}[2][]{
    colframe=violet!0,
    coltitle=violet, % Title text color
    fonttitle=\bfseries, % Title font
    title={Definizione~\thetcbcounter\ifstrempty{#2}{}{:~#2}}, % Title format
    sharp corners, % Less rounded corners
    boxrule=0pt, % Line width of the box frame
    toptitle=1mm, % Distance from top to title
    bottomtitle=1mm, % Distance from title to box content
    colbacktitle=violet!5, % Background color of the title bar
    left=0mm, right=0mm, top=1mm, bottom=1mm, % Padding around content
    enhanced, % Enable advanced options
    before skip=10pt, % Space before the box
    after skip=10pt, % Space after the box
    breakable, % Allow box to split across pages
    colback=violet!0,
    borderline west={2pt}{-5pt}{violet!40},
    #1
}

\newenvironment{lemmao}[1][]{
    \refstepcounter{theo}
     \ifstrempty{#1}
    {\mdfsetup{
        frametitle={
            \tikz[baseline=(current bounding box.east),outer sep=0pt]
            \node[anchor=east,rectangle,fill=green!20,rounded corners=5pt]
            {\strut Lemma~\thetheo};}
        }
    }{\mdfsetup{
        frametitle={
            \tikz[baseline=(current bounding box.east),outer sep=0pt]
            \node[anchor=east,rectangle,fill=green!20,rounded corners=5pt]
            {\strut Lemma~\thetheo:~#1};}
        }
    }
    \mdfsetup{
        roundcorner=10pt,
        innertopmargin=10pt,linecolor=green!20,
        linewidth=2pt,topline=true,
        frametitleaboveskip=\dimexpr-\ht\strutbox\relax,
        % nobreak=true
    }
\begin{mdframed}[]\relax}{
\end{mdframed}}

\theoremstyle{plain}
\newtheorem{lemma}[theo]{Lemma}
\newtheorem{corollary}{Corollario}[theo]
\newtheorem{proposition}[theo]{Proposizione}

\theoremstyle{definition}
\newtheorem{example}[excounter]{Esempio}

\theoremstyle{remark}
\newtheorem*{note}{Nota}
\newtheorem*{remark}{Osservazione}

\newtcolorbox{notebox}{
  colback=gray!10,
  colframe=black,
  arc=5pt,
  boxrule=1pt,
  left=15pt,
  right=15pt,
  top=15pt,
  bottom=15pt,
}

\DeclareRobustCommand{\rchi}{{\mathpalette\irchi\relax}} % beautiful chi
\newcommand{\irchi}[2]{\raisebox{\depth}{$#1\chi$}} % inner command, used by \rchi

\newtcolorbox[auto counter, number within=section]{eser}[1][]{
    colframe=black!0,
    coltitle=black!70, % Title text color
    fonttitle=\bfseries\sffamily, % Title font
    title={Esercizio~\thetcbcounter~#1}, % Title format
    sharp corners, % Less rounded corners
    boxrule=0mm, % Line width of the box frame
    toptitle=1mm, % Distance from top to title
    bottomtitle=1mm, % Distance from title to box content
    colbacktitle=black!5, % Background color of the title bar
    left=0mm, right=0mm, top=1mm, bottom=1mm, % Padding around content
    enhanced, % Enable advanced options
    before skip=10pt, % Space before the box
    after skip=10pt, % Space after the box
    breakable, % Allow box to split across pages
    colback=black!0,
    borderline west={1pt}{-5pt}{black!70}, 
    segmentation style={dashed, draw=black!40, line width=1pt} % Dashed dividing line
}
\newcommand{\seminorm}[1]{\left\lvert\hspace{-1 pt}\left\lvert\hspace{-1 pt}\left\lvert#1\right\lvert\hspace{-1 pt}\right\lvert\hspace{-1 pt}\right\lvert}


\title{Appunti di Meccanica Razionale}
\author{Osea}
\date{Secondo semestre, 2024 \-- 2025, prof. Ada Pulvirenti}

\begin{document}

\maketitle

Testo di riferimento: \emph{Meccanica Analitica} di Fasano Marmi, più lungo e
preciso. Mentre il testo \emph{Meccanica Classica} di Goldstein è il testo
classico dei fisici. 

\section{Spazio-Tempo-Moto}
\subsection{Moto}
Studiare il moto in meccanica significa studiare la funzione \(I = [0, T) \to
\mathcal{E}\), con \(t \mapsto P(t)\), dove \(\mathcal{E}\) e \(I\) sono
rispettivamente lo spazio della meccanica classica e un intervallo incluso
nell'asse dei tempi, e entrambi sono spazi affini euclidei.
\begin{notebox}
    Nella meccanica classica il \textbf{tempo} è \textbf{assoluto}, ovvero è
    indipendente dallo stato di moto dell'osservatore.
\end{notebox}
\begin{definition}{Spazio affine}
    Uno spazio affine reale di dimensione \(n\) è un insieme \(\mathbb{A} ^n\) i
    cui elementi sono detti punti, dotato delle seguenti strutture:
\begin{enumerate}[label = \arabic*.]
    \item Uno spazio vettoriale reale di dimensione \(n\), \(V\), detto spazio
        delle traslazioni (o dei vettori liberi)
\item Un'applicazione \(\varphi : \mathbb{A}^{n} \times  \mathbb{A}^{n} \to
        V\); \(P, Q \mapsto P - Q\)  con le seguenti proprietà
        \begin{enumerate}[label = \alph*.]
            \item \(\forall (P, v) \in  A^{n} \times  V\) esiste un unico punto
                \(Q\) tale che \(Q - P = \mathbf{v} \) 
            \item \((P - Q) + (Q - R) = P - R\) per ogni \(P, Q, R \in
                \mathbb{A}^{n}\) 
        \end{enumerate}
\end{enumerate}
\end{definition}

\begin{definition}{Retta}
    Una \textbf{retta} in \(\mathbb{A}^{n}\) passante per un dato punto \(P\) e
    con direzione \(\mathbf{v} \) è il sottospazio affine \(P + \Span{\mathbf{v} } \) ed è
    parametrizzata da \(t \mapsto P + t \mathbf{v} \) 
\end{definition}

\begin{definition}{Vettore applicato}
    Una coppia ordinata \({(P, \mathbf{u} )} \in \mathbb{A}^{n} \times V\) si dice vettore
    applicato a \(P\).
\end{definition}

\begin{definition}{Sistema di riferimento}
    Indicheremo con sistema di riferimento affine in \(\mathbb{A}^{n}\) un
    insieme 
    \[
      \Sigma = \{O \in \mathbb{A}^{n} ; \mathbf{v} _{1}, \mathbf{v} _{2}, \dots,
      \mathbf{v} _{n}\} \quad ;
      \quad {\{\mathbf{v} _{i}\}}_{i = 1, \dots, n} \text{ base di } V
    \]
\end{definition}
\begin{remark}
    Allora ogni punto \(P\) rispetto a \(\Sigma\) è individuato da \\\(P - O =
    x_{1}\mathbf{v} _{1} +~\dots + x_{n}\mathbf{v} _{n}\) 
\end{remark}

\begin{definition}{Spazio euclideo}
    Uno spazio affine reale di dimensione \(n\), \textbf{dotato di prodotto
    scalare} su \(V\) si dice spazio \textbf{affine euclideo}
\end{definition}

Dal prodotto scalare possiamo definire la distanza tra due punti di
\(\mathbb{A}^{n}\) come
\[
d(P, Q) = \|P - Q\| = \sqrt{\langle P - Q, P - Q \rangle} = \sqrt{{(P -
    Q)} \cdot {(P - Q)}}
\]
e la nozione di angolo tra due vettori \(\mathbf{u} , \mathbf{v}  \neq \mathbf{0} \) come il valore \(\alpha
\in [0, \pi]\) tale che
\[
  \cos \alpha = \frac{\mathbf{u}  \cdot \mathbf{v} }{\|\mathbf{u} \| \|\mathbf{v} \|}
\]
(notare che per Schwarz risulta che effettivamente tale valore esiste)

Comunemente ci ricondurremo a utilizzare un sistema di riferimento
\textbf{ortonormale}, ossia che ha come base una base ortonormale dello spazio.
In particolare \(\mathcal{E}\) è uno spazio affine reale euclideo di dimensione
3, e dunque useremo il sistema di riferimento
\[
    \Sigma = \{O, \mathbf{e} _{1}, \mathbf{e} _{2}, \mathbf{e} _{3}\} \quad ; \quad \mathbf{e} _{i} \cdot \mathbf{e} _{j} =
    \delta_{ij} \quad \forall i, j = 1, 2, 3
\]
Studiare il moto significherà studiare la relazione \(t \mapsto P(t)\) funzione
\([0, T) \to \mathcal{E}\). Introdotto \(\Sigma\), possiamo scrivere allora che
il problema è ricondotto a studiare la funzione \([0, T) \to \mathbb{R}^{3}\)
tale che \(t \mapsto (x_{1}, x_{2}, x_{3})\) e tale che \(P(t) = x_{1} \mathbf{e} _{1} +
x_{2} \mathbf{e} _{2} + x_{3} \mathbf{e} _{3}\).

\begin{definition}{Coordinate Curvilinee}
Sia \(Q \subseteq \mathbb{R}^{n} \). Consideriamo un'applicazione \(\mathbf{x} :
Q \to D \subseteq \mathbb{R}^{n} \) tale che 
\[
  \mathbf{q} = \begin{pmatrix}
      q_{1} \\
      \vdots  \\
      q_n
  \end{pmatrix} \mapsto \mathbf{x} {(\mathbf{q} )} = \begin{pmatrix}
      x_{1}{(q_{1}, \dots, q_{n})} \\
      \vdots \\
      x_{n}{(q_{1}, \dots, q_{n})}
  \end{pmatrix}
\]
che gode delle seguenti proprietà:
\begin{enumerate}[label = \arabic*.]
    \item \(\mathbf{x}  \in C^{1}{(Q)}\) 
    \item La matrice Jacobiana \(J\mathbf{x} \) abbia rango massimo
\end{enumerate}
Allora \(\mathbf{x} \) rappresenta un sistema di coordinate su \(D\) che sono
dette \emph{coordinate curvilinee}
\end{definition}
I vettori colonna della matrice \(J\mathbf{x} \), ossia
\[
  \mathbf{u}_i = \frac{\partial \mathbf{x} }{\partial q_{i}} \quad \forall i =
  1, \dots, n
\]
sono una base per \(V\) che viene chiamata \textbf{base locale}.

La base locale è ortogonale se \(\mathbf{u}_i \cdot \mathbf{u}_j = \mathbf{0} \) per ogni
\(i \neq j\). Vogliamo ora operare Gram-Schmidt per ottenere una base
ortonormale \(\{\tilde{\mathbf{u}}_i\} \) a partire da una base generica
\(\{\mathbf{u}_i\}\). 

\begin{remark}
    I vettori \(\mathbf{u}_i\) della base locale sono in ogni punto \(P\)
    tangenti alla rispettiva linea coordinata \(\mathbf{x} {(q_{i})} = \mathbf{x}(q_{1}, \dots, q_{i},
    \dots, q_{n})\), dove tutte le coordinate sono fissate tranne \(q_{i}\).
    Allora \(\mathbf{u}_i = \frac{\partial \mathbf{x} }{\partial q_{i}}\) è
    tangente alla linea coordinata \(\mathbf{x} {(q_{i})}\).
\end{remark}

\begin{example}[Coordinate polari nel piano]
    Prendiamo \(Q = {(0, +\infty)} \times  [0, 2\pi)\) e \(\mathbf{q} = {(q_{1},
    q_{2})} = {(r, \theta)}\). Allora 
    \[
      \mathbf{x} {(r, \theta)} = \begin{pmatrix}
          x_{1}{(r, \theta)}=r \cos \theta \\
          x_{1}{(r, \theta)}=r \sin \theta
      \end{pmatrix} \in C^{1}
    \]
    E la base locale è 
    \[
        \mathbf{u}_1 = \frac{\partial \mathbf{x} }{\partial r} = \begin{pmatrix}
            \cos \theta \\
            \sin \theta
        \end{pmatrix} \quad ; \quad \mathbf{u}_2 = \frac{\partial \mathbf{x}}{\partial \theta} = \begin{pmatrix}
            -r \sin \theta \\
            r \cos \theta
        \end{pmatrix}
    \]
    ed evidentemente \(\mathbf{u}_1 \cdot \mathbf{u}_2 = \mathbf{0} \) dunque la base
    locale è ortogonale. Per ottenere una base ortonormale dunque prendiamo
    \[
      \mathbf{e}_r = \frac{\mathbf{u}_1}{\|\mathbf{u}_1\|} = \mathbf{u}_1 \quad
      ; \quad \mathbf{e}_\theta = \frac{\mathbf{u}_2}{\|\mathbf{u}_2\|} =
      \frac{1}{r} \mathbf{u}_2 
    \]
    Osserviamo che effettivamente \(\mathbf{e}_r\) è tangente alla linea
    coordinata corrispondente (fisso \(\theta\)) e \(\mathbf{e}_\theta\) è
    tangente alla linea coordinata corrispondente (fisso \(r\)).
\end{example}
\begin{example}[Coordinate sferiche]
    \[
        Q = {(0, +\infty)} \times [0, 2\pi) \times (0, \pi) \subseteq
        \mathbb{R}^3 \quad ; \quad \mathbf{q} = {(q_1, q_2, q_3)} = {(r, \theta,
        \varphi)}
    \]
    \[
      \mathbf{x} {(\mathbf{q} )} = \begin{pmatrix}
          x_{1}{(r, \theta, \varphi)} = r \sin \theta \cos \varphi  \\
          x_{2}{(r, \theta, \varphi)} = r \sin \theta \sin \varphi  \\
          x_{3}{(r, \theta, \varphi)} = r \cos \theta
      \end{pmatrix} : Q \to \mathcal{E} \sminus \{\text{ asse \(z\)}\} 
    \]
    che è chiaramente di classe \(C^{1}\) e ha Jacobiano di rango massimo. La
    base locale è
    \begin{align*}
      \mathbf{u}_1 = \frac{\partial \mathbf{x} }{\partial r} = \begin{pmatrix}
          \sin \theta \cos \varphi \\
          \sin \theta \sin \varphi \\
          \cos \theta
          \end{pmatrix} \quad &; \quad \mathbf{u}_2 = \frac{\partial \mathbf{x}
          }{\partial \varphi } = \begin{pmatrix}
          - r \sin \theta \sin \varphi \\
          r \sin \theta \cos \varphi \\
          0
      \end{pmatrix} \quad ; \\ \mathbf{u}_3 = \frac{\partial \mathbf{x}
      }{\partial \theta} &= \begin{pmatrix}
          r \cos \theta \cos \varphi \\
          r \cos \theta \sin \varphi \\
          -r\sin \theta
      \end{pmatrix}
    \end{align*}

    e dunque la base locale è ortogonale. Per ottenere una base ortonormale
    prendiamo, per ogni \(\mathbf{u} _{i}\), il vettore \(\mathbf{u}_i /
    \|\mathbf{u} _i\|\), ossia
    \[
      \mathbf{e}_r = \mathbf{u}_1 \quad ; \quad \mathbf{e}_\varphi =
      \frac{\mathbf{u}_2}{r \sin\theta} \quad ;  \quad \mathbf{e}_\theta =
      \frac{\mathbf{u}_3}{r}
    \]
\end{example}
\paragraph{Orientazione dello spazio}
Lo spazio vettoriale \(V\) soggiacente lo spazio affine \(\mathcal{E}\) è
orientabile.

Sia \(\mathcal{B}\) l'insieme di tutte le basi di \(V\). Date \(A, B \in
\mathcal{B}\), sia \(\mathcal{M}_A^{B}\) la matrice di cambio base da \(A\) a
\(B\). Allora \(\det \mathcal{M}_A^{B} \neq 0\). Allora la relazione 
\[
    A \sim B \iff \det \mathcal{M}_A^{B} > 0
\]
è una relazione di equivalenza su \(\mathcal{B}\). Tale relazione partiziona
\(\mathcal{E}\) in due classi, dette \textbf{classi di orientazione}: la base
destrorsa e la base sinistrorsa.
\paragraph{Prodotto vettoriale} Il prodotto vettoriale è un'operazione binaria 
\begin{align*}
    \times : V \times V &\longrightarrow V \\
    \mathbf{u} , \mathbf{v}  &\longmapsto \mathbf{u} \times  \mathbf{v}=
    \mathbf{w} 
\end{align*}
tale che, se l'angolo tra \(\mathbf{v} \) e \(\mathbf{u}\) è \(\alpha\), allora
\begin{itemize}[label = --]
    \item \(\mathbf{w} \) ha modulo pari a \(|\mathbf{u} | |\mathbf{v} | \sin \alpha\) 
    \item \(\mathbf{w} \) ha direzione perpendicolare al piano di \(\mathbf{v}
        \) e \(\mathbf{u} \) 
    \item \(\mathbf{w} \) ha verso tale che \(\mathbf{u} \), \(\mathbf{v} \),
        \(\mathbf{w} \) formano una base destrorsa
\end{itemize}
e ha le seguenti proprietà:
\begin{enumerate}[label = \arabic*.]
    \item \emph{antisimmetria}: \(\mathbf{u} \times \mathbf{v} = - \mathbf{v}
        \times \mathbf{u}\)
    \item \emph{linearità}: \((\lambda \mathbf{u} ) \times \mathbf{v} = \lambda
        (\mathbf{u} \times \mathbf{v} )\) e \((\mathbf{u} + \mathbf{v} ) \times
        \mathbf{w} = \mathbf{u} \times \mathbf{w} + \mathbf{v} \times \mathbf{w}\)
    \item \textbf{non} è associativo
\end{enumerate}
\paragraph{Prodotto misto} Il prodotto misto tra \(\mathbf{u} \), \(\mathbf{v}
\) e \(\mathbf{w} \) è definito da \(\mathbf{u} \times \mathbf{v} \cdot \mathbf{w} \) 
\paragraph{Doppio prodotto vettoriale} Dati \(\mathbf{u} , \mathbf{v} ,
\mathbf{w}  \in V\), si chiama doppio prodotto vettoriale il vettore
\({(\mathbf{u}  \times \mathbf{v} )} \times \mathbf{w} \) e vale la proprietà
\[
  {(\mathbf{u} \times \mathbf{v} )} \times \mathbf{w}  = {(\mathbf{u} \cdot
  \mathbf{w} )}\mathbf{v} - {(\mathbf{v} \cdot \mathbf{w} )}\mathbf{u} 
\]
\paragraph{Equazione vettoriale}
Dati \(\mathbf{a} , \mathbf{b}  \in V\), determinare i vettori \(\mathbf{x} \)
tali che
\begin{equation}\label{eq:equazione_vettoriale}
  \mathbf{x}  \times  \mathbf{a} = \mathbf{b} 
\end{equation}
Nel caso degenere in cui \(\mathbf{a} = \mathbf{b} = \mathbf{0}\), l'equazione è
un'identità. Se \(\mathbf{a} = \mathbf{0} \) e \(\mathbf{b} \neq \mathbf{0}  \) allora non esiste
una soluzione. Infine se \(\mathbf{a} \neq \mathbf{0}\) e \(\mathbf{b} = \mathbf{0}  \) allora
\(\mathbf{x}  = \lambda \mathbf{a} \) per ogni \(\lambda \in \mathbb{R}\) 

\begin{proposition}
    Siano \(\mathbf{a} , \mathbf{b} \in V \) con \(\mathbf{a} \neq \mathbf{0} \) e \(b
    \neq \mathbf{0} \). Allora l'equazione vettoriale~\eqref{eq:equazione_vettoriale} ha
    soluzione se e solo se \(\mathbf{b} \cdot \mathbf{a}  = \mathbf{0} \) 
\end{proposition}
\begin{proof}\( \)
\begin{itemize}
    \item[\(\implies \)] Sia \(\mathbf{x} \) soluzione di \(\mathbf{x} \times
        \mathbf{a} = \mathbf{b}\). Moltiplicando scalarmente per \(\mathbf{a} \)
        ambo i membri otteniamo
        \[
          \mathbf{0}  = \mathbf{x}  \times \mathbf{a} \cdot \mathbf{a} = \mathbf{b} * \mathbf{a} 
        \]
    \item[\(\impliedby \)] Sia \(\mathbf{b}\cdot \mathbf{a} = \mathbf{0} \). Allora
        \[
          {(\mathbf{a} \times \mathbf{b})}\times \mathbf{a}  = a^2 \mathbf{b} -
          \cancel{{(\mathbf{b}\cdot \mathbf{a} )}}\mathbf{a} 
        \]
        da cui otteniamo
        \[
          \mathbf{b} = \frac{1}{a^2} {\left( {(\mathbf{a} \times
          \mathbf{b})}\times \mathbf{a}  \right)} 
        \]
        ma allora
        \[
          \mathbf{x}  \times \mathbf{a} = \frac{1}{a^2}  {\left( {(\mathbf{a} \times
          \mathbf{b})}\times \mathbf{a}  \right)} \iff {\left( \mathbf{x} -
  \frac{1}{a^2}{(\mathbf{a} \times \mathbf{b})} \right)} \times \mathbf{a} = \mathbf{0} 
        \]
        che ha soluzione 
        \[
            \mathbf{x}  = \frac{1}{a^2}{(a \times b)} + \lambda \mathbf{a} \quad
            \forall \lambda \in \mathbb{R}
        \]
\end{itemize}
\end{proof}

\section{Cinematica del punto}
Sia \(\mathcal{E}\) uno spazio euclideo, \(P\) un punto in modo nello spazio
\(\mathcal{E}\). Sia \(I \subseteq \mathbb{R} \) e studiare il moto significa
studiare la relazione \(I \ni t \mapsto P{(t)}\) e in particolare fissando un
sistema di riferimento \(\Sigma = \{\mathbf{O} , \mathbf{e}_1, \mathbf{e}_2,
\mathbf{e}_3\}\) studiamo la relazione \(t \mapsto P{(t)} - \mathbf{O} =:
\mathbf{x} {(t)} =: \mathbf{r} {(t)}\).

L'immagine in \(\mathcal{E}\) dell'applicazione \(t \mapsto P{(t)}\) si dice
\textbf{traiettoria} del punto \(P\).

\begin{definition}{Velocità}
    Scelto \(\mathbf{x}{(t)}\) il raggio descrivente il moto di un punto \(P\)
    nel sistema di riferimento \(\Sigma\), la \textbf{velocità} del punto \(P\)
    è
    \[
        \mathbf{v}{(t)}|_\Sigma := \mathbf{\dot{x}}{(t)}|_\Sigma = \frac{d}{dt} \mathbf{x}
        {(t)}|_\Sigma = \dot{x_{1}} \mathbf{e}_1 + \dot{x_{2}} \mathbf{e}_2 +
        \dot{x_{3}} \mathbf{e}_3
    \]
    Similmente si definisce l'accelerazione 
    \[
        \mathbf{a} |_\Sigma = \mathbf{\ddot{x}}|_\Sigma = \frac{d}{dt}
        \mathbf{v}|_\Sigma
    \]
\end{definition}
Se si effettua il cambio di coordinate \(\mathbf{x} = \mathbf{x} {(\mathbf{q}
{(t)})}\) allora il moto è descritto da 
\[
    \mathbf{v} = \sum_{i=1}^{n} \frac{\partial \mathbf{x} }{\partial q_{i}}
    \dot{q_{i}} = \mathbf{u} \cdot \mathbf{\dot{q}} 
\]
\begin{example}[Coordinate polari nel piano \(\mathcal{E}_2\) ]
    \(\mathbf{x}  = \mathbf{x} {(r, \theta)}\) e dunque
    \begin{align*}
        \mathbf{v} &= \dot{r} \mathbf{u}_r + \dot{\theta} \mathbf{u}_\theta =
        \dot{r} \mathbf{e}_r + r \dot{\theta} \mathbf{e}_\theta\\
        \mathbf{a} &= \ddot{r} \mathbf{e}_r + \dot{r} {\left(
        \frac{d}{dt}\mathbf{e}_r  \right)} + \dot{r} \dot{\theta}
        \mathbf{e}_\theta + r \ddot{\theta} \mathbf{e}_\theta + r \ddot{\theta}
        \frac{d}{dt}\mathbf{e}_\theta = \\
                   &= \ddot{r} \mathbf{e}_r = \dot{r} \mathbf{e}_r +
                   \dot{r}\dot{\theta} \mathbf{e}_\theta + \dot{r} \dot{\theta}
                   \mathbf{e}_\theta + r\ddot{\theta}\mathbf{e}_\theta -
                   r\dot{\theta}^2 \mathbf{e}_r = \\
                   &= {\left( \ddot{r} - r \theta^2 \right)} \mathbf{e}_r +
                   {\left( r\ddot{\theta} + 2 \dot{r}\dot{\theta} \right)}
                   \mathbf{e} _\theta
    \end{align*}
    che è la scomposizione in accelerazione radiale e trasversa
\end{example}

\begin{remark}
    Se \(\mathbf{b} \) è un versore allora \(\frac{d\mathbf{b} }{dt} \cdot
    \mathbf{b} = 0\) (perché \(\mathbf{b} \) ha modulo costante, quindi come fa
    a cambiare in modulo? Deve cambiare solo in verso)
\end{remark}
\begin{eser}
    Calcolare la velocità e l'accelerazione in coordinate sferiche (esprimendola
    in base locale) e in coordinate cilindriche
\end{eser}

\subsection{Curve parametrizzate}
Vogliamo studiare il moto del punto \(P\) su una curva assegnata. Sappiamo
dunque la traiettoria del punto assegnato. Supponiamo di essere nello spazio
affine euclideo \(\mathcal{E}_3\). Supponiamo dunque di avere una curva
\(\gamma: {(a, b)} \to \mathbb{R}^3\) e dunque \(\gamma{(t)} = \mathbf{x} {(t)}\) 
\begin{definition}{Parametrizzazione in lunghezza d'arco}
    È possibile parametrizzare la curva \(\gamma\) nel modo
    \[
        s{(t)} = \int_{a}^{t} \|\mathbf{\dot{x}}{(\tau)}\| d\tau
    \]
    che descrive la posizione in una dimensione, descrivendo la lunghezza
    percorsa sulla curva.
    e dunque \(\mathbf{x} = \mathbf{x} {(s)}\) 
\end{definition}
\begin{remark}
    \(|s'{(t)}| = |\mathbf{x}'{(t)}|\) 
\end{remark}
\begin{definition}{Versore tangente}
    Il versore tangente alla curva \(\gamma\) è
    \[
        \mathbf{t} {(t)} =
        \frac{\mathbf{\dot{x}}{(t)}}{\|\mathbf{\dot{x}}{(t)}\|} =
        \frac{d\mathbf{x} {(s)}}{ds}
    \]
    dove ovviamente la prima uguaglianza vale se \(\mathbf{\dot{x}}{(t)} \neq
    0\)
\end{definition}

\begin{definition}{Versore normale}
    Il versore normale alla curva \(\gamma\) è definito, nei punti in cui
    \(\frac{d\mathbf{t} }{ds} \neq 0\) come 
    \[
        \mathbf{n} {(t)} := \frac{\frac{d\mathbf{t} }{ds}}{\left\|\frac{d\mathbf{t}
        }{ds}\right\|} = \frac{\frac{d^2 \mathbf{x} }{ds^2}}{\left\|\frac{d^2 \mathbf{x}
    }{ds}\right\|}
    \]

\end{definition}
\begin{definition}{Curvatura}
    La quantità \(\kappa{(s)} = \left\|\frac{d^2 \mathbf{x} }{ds^2}\right\|\) si dice
    \emph{curvatura} e il suo inverso moltiplicativo \(R\) si dice \emph{raggio di
    curvatura}. 
\end{definition}
\begin{definition}{Piano Osculatore}
    Il piano individuato dai versori \(\mathbf{t} \) e \(\mathbf{n} \) si dice
    \emph{piano osculatore}
\end{definition}
\begin{definition}{Versore binormale}
    Il versore \(\mathbf{b} = \mathbf{t}  \times  \mathbf{n} \) viene detto
    \emph{versore della binomiale}
\end{definition}

\begin{definition}{Terna intrinseca}
    La terna \(\{\mathbf{t} , \mathbf{n} , \mathbf{b} \}\) è detta terna
    intrinseca alla curva \(\gamma\) se
    \begin{itemize}
        \item \(\mathbf{t} \) è il versore tangente
        \item \(\mathbf{n} \) è il versore normale
        \item \(\mathbf{b} \) è il versore binormale
    \end{itemize}
    e viene anche chiamato \emph{riferimento di Frenet}

\end{definition}

Possiamo osservare che \(\frac{d\mathbf{b} }{ds}\) è normale a
\(\mathbf{b}{(s)}\) e inoltre
\begin{align*}
    \frac{d\mathbf{b} }{ds} &= \frac{d}{ds}(\mathbf{t}  \times  \mathbf{n} ) =
    \frac{d\mathbf{t} }{ds} \times  \mathbf{n}  + \mathbf{t}  \times
    \frac{d\mathbf{n} }{ds} = -\kappa \mathbf{n}  \times  \mathbf{n}  + \mathbf{t}
    \times \frac{d\mathbf{n} }{ds}=\\
    &= \frac{d\mathbf{n} }{ds}
\end{align*}
E dunque dato che è normale sia a \(\mathbf{b}\) che a \(\mathbf{t} \) allora
deve essere proporzionale a \(\mathbf{n} \) e in particolare la seguente è una
buona definizione
\begin{definition}{Torsione}
    La funzione \(\tau = \tau{(s)}\) tale che
    \[
        \frac{d\mathbf{b}{(s)} }{ds} = \tau{(s)} \mathbf{n}{(s)}
    \]
    è detta \emph{torsione} della curva \(\gamma\)
\end{definition}

Riassumendo, le seguenti vengono dette \emph{formule di Frenet}:
\begin{align*}
    \frac{d\mathbf{t} }{ds} &= \kappa \mathbf{n}  \\
    \frac{d\mathbf{n} }{ds} &= -\kappa \mathbf{t}  - \tau \mathbf{b}  \\
    \frac{d\mathbf{b} }{ds} &= \tau \mathbf{n}
\end{align*}
dove la seconda è trovata dalle altre due e da \(\mathbf{n} = \mathbf{b} \times
\mathbf{t} \) 

Dal precedente marchingegno possiamo riscrivere la velocità e l'accelerazione
nella terna intrinseca come, se \(P = P{(s{(t)})}\) 
\begin{align}\label{eq:velocita_accelerazione}
    \mathbf{v}  &= \frac{d\mathbf{x} }{dt} = \frac{d\mathbf{x} }{ds}
    \frac{ds}{dt} = \dot{s} \mathbf{t}\\
    \mathbf{a} &= \frac{d\mathbf{v} }{dt} = \frac{d}{dt} (\dot{s} \mathbf{t}) =
    \ddot{s} \mathbf{t} + \dot{s} \frac{d\mathbf{t} }{ds} \frac{ds}{dt} =
    \ddot{s} \mathbf{t} + \dot{s}^2 \kappa \mathbf{n}
\end{align}

\begin{example}[Curvatura rispetto a un parametro \(\lambda\)]
    Supponiamo \(P = P{(\lambda)}\). Dimostriamo che vale la seguente formula
    \[
        \kappa = \frac{|P' \times P''|}{|P'|^3}
    \]
    \begin{proof}
        Ci vogliamo ricondurre all'ascissa curvilinea \(s\). Se
        \(\mathbf{r} = \mathbf{r} {(s)}\) otteniamo
        \begin{equation}\label{eq:curvatura_s}
          \left\|\frac{d\mathbf{r} }{ds} \times  \frac{d^2\mathbf{r}
          }{ds^2}\right\| =
          \|\mathbf{t} {(s)} \times \kappa{(s)}\mathbf{n} {(s)}\| =
          |\kappa{(s)}|
        \end{equation}
        se ora consideriamo \(s = s{(\lambda)}\) abbiamo 
        \begin{align*}
            \frac{d\mathbf{r} }{ds} &= \frac{d\mathbf{r} }{d\lambda}
          \frac{d\lambda}{ds} = \frac{d\mathbf{r} }{d\lambda} \frac{1}{s'} =
          \frac{\mathbf{\dot{r}} }{s'}\\
            \frac{d^2 \mathbf{r} }{ds^2} &= \frac{d}{ds} {\left(
            \frac{d\mathbf{r} }{ds} \right)} = \frac{1}{s'} \frac{d}{d\lambda}
            {\left( \frac{1}{s'} \frac{d\mathbf{r} }{d\lambda} \right)} =
            \frac{1}{s'} {\left( \frac{1}{s'} \frac{d^2\mathbf{r}
            }{d\lambda^2} + \frac{d\mathbf{r} }{d\lambda} \frac{-s''}{{(s')}^2}
    \right)} = \frac{s' \mathbf{\ddot{r}}  - s'' \mathbf{\dot{r}}}{s'^3}
        \end{align*}
        E dunque l'equazione~\eqref{eq:curvatura_s} diventa 
        \begin{align*}
            \kappa{(s)} &= \left\|\frac{d\mathbf{r} }{ds} \times  \frac{d^2 \mathbf{r}
          }{ds^2}\right\| = \left\| \frac{1}{s'^{4}} {(\mathbf{\dot{r}} \times
          {(s' \mathbf{\ddot{r}} - s'' \mathbf{\dot{r}} )})}  \right\|
          = \left\| \frac{1}{s'^{4}} {(s' \mathbf{\dot{r}} \times
              \mathbf{\ddot{r}} - s'' \cancel{\mathbf{\dot{r}} \times
              \mathbf{\dot{r}}})}\right\| =
          \\ &= \left\| \frac{\mathbf{\dot{r}} \times  \mathbf{\ddot{r}}
          }{s'^3}\right\|
        \end{align*}
    \end{proof}
\end{example}

\begin{example}[\(\gamma\) grafico di una funzione \(y = f{(x)}\) ]
    Risulta naturale parametrizzare la curva \(\gamma\) come
    \[
        P - O = \mathbf{r} {(x)} = x \mathbf{\hat{i}} + f{(x)} \mathbf{\hat{j}}
    \]
    e dunque
    \[
      \kappa{(x)} = \frac{|f''{(x)}|}{{\left( \sqrt{1 + {(f'{(x)})}^2} \right)} }
    \]
\end{example}

\begin{eser}
    Sia \(P\) descrivente con velocità costante in modulo la curva che in un
    sistema di coordinate cilindriche \(\rho, \theta, z\) ha equazione
    \[
      \begin{cases}
          \rho = k \theta \\
          z = h \theta
      \end{cases}
    \]
    Determinare \(|\mathbf{a} {(p)}|\) 
    \tcblower

    Suggerimento: usare la~\eqref{eq:velocita_accelerazione} osservando che \(\ddot{s} = 0\).  
\end{eser}

\begin{eser}[Triedrio di Frenet per un'elica cilindrica]
Sia \(\gamma\) l'elica cilindrica di equazione
\[
  \begin{cases}
      x = R \cos\phi \\
      y = R\sin\varphi \\
      z = a \varphi 
  \end{cases}
\]
Calcolare \(s, \mathbf{t}, \mathbf{n} , \mathbf{b} \). Similmente calcolare \(s,
\mathbf{t} , \mathbf{n} , \mathbf{b}\) per il grafico della curva \(y = x^2\) 
\end{eser}

\subsection{Cinematica dei moti centrali}
\begin{definition}{Moto centrale}
    Un punto \(P\) si muove di moto centrale rspetto a un polo fisso \(O\)  se
    durante il moto, risulta \(\mathbf{a} {(P{(t)})}\) è parallelo a \(P{(t)} -
    O\) 
\end{definition}
Il moto centrale ha diverse proprietà, di seguito ne elencheremo alcune
\begin{enumerate}[label = \arabic*.]
    \item Il moto centrale è un \textbf{moto piano}. Infatti se \(\mathbf{a} \)
        è parallelo a \(\mathbf{r} \) allora 
        \[
          \mathbf{a}  \times  \mathbf{r}  = \mathbf{0} \implies \frac{d}{dt}
          {\left(  \mathbf{v}  \times \mathbf{r}  \right)} = \frac{d\mathbf{v}
          }{dt} \times  \mathbf{r} + \mathbf{v} \times \frac{d\mathbf{r} }{dt} =
          \mathbf{0} 
        \]
        infatti \(\frac{d\mathbf{r} }{dt} = \mathbf{v} \). La precedente
        condizione ha la forte condizione che \(\mathbf{v}  \times \mathbf{r} =
        \mathbf{c}\). In particolare deduciamo che
        \[
            \mathbf{v} \times \mathbf{r} \cdot \mathbf{r}  = \mathbf{c} \cdot
            \mathbf{r} = 0
        \]
        e dunque il moto è piano. Se poi \(\mathbf{c} = \mathbf{0}\) il moto è
        rettilineo. 
    \item Il moto si svolge con \textbf{velocità areolare costante}. In
        particolare definiamo il vettore velocità areolare
        \[
          \mathbf{A} {(O)} = \frac{1}{2}{(P - O)} \times \mathbf{v}
          \overset{\text{coord. polari}}{=} \frac{1}{2} {\left( {(r
          \mathbf{e}_r)} \times {(\dot{r} \mathbf{e} _r + r \dot{\theta}
\mathbf{e} _\theta)}  \right)} = \frac{1}{2} r^2 \dot{\theta} \hat{\mathbf{k} }
        \]
        la velocità areolare ha la seguente interpretazione geometrica.
        Considerando infatti l'area \(A{(\theta)}\) spazzata dal raggio vettore 
        \[
          A{(\theta)} = \frac{1}{2}\int_0^{\theta} r^2\,d\theta \implies
          \dot{A}{(\theta)} =
          \frac{1}{2}r^2\dot{\theta}
        \]
        Se il moto è centrale allora \(\mathbf{a}  = a_r\mathbf{e}_r\) 
        e dunque \(a_\theta = r\ddot{\theta} + 2\dot{r}\dot{\theta} = 0\).
        Infine
        \[
            \frac{d}{dt}{\left( r^2\dot{\theta} \right)} =
            2r\dot{r}\dot{\theta} + r^2\ddot{\theta} = ra_\theta = 0
        \]
    \item Vale la \textbf{formula di Binet}, che mostra la dipendenza di \(r\)
        solo da \(\theta\). Ricordando \(\mathbf{a} =a_r
        \mathbf{e}_r\) e \(r^2\dot{\theta} = c\) abbiamo
        \begin{align*}
            \dot{r} &= \frac{dr}{dt} = \frac{dr}{d\theta}\dot{\theta} =
          \frac{c}{r^2} \frac{dr}{d\theta} = -c \frac{d}{d\theta} {\left(
          \frac{1}{r} \right)} \\
            \ddot{r} &= \frac{d}{dt}\dot{r} = \frac{d}{dt} {\left( -c
            \frac{d}{d\theta}{\left( \frac{1}{r} \right)}  \right)} =
            \frac{d}{d\theta} {\left( -c \frac{d}{d\theta}{\left( \frac{1}{r}
            \right)}  \right)} \frac{c}{r^2}
        \end{align*}
        da cui finalmente
        \begin{equation}\label{eq:binet}
        a_r = \ddot{r} - r\dot{\theta}^2 = -\frac{c^2}{r^2} {\left(
        \frac{d^2}{d\theta^2}{\left( \frac{1}{r} \right)} + \frac{1}{r} \right)} 
        \end{equation}

\end{enumerate}

\section{Dinamica del punto materiale}
Ogni cambiamento dello stato di moto di un punto \(P\) è dovuto all'interazione
fra \(P\) e altri punti. 

\begin{definition}{Sistema inerziale}
    Si chiama \textbf{inerziale} un riferimento rispetto al quale un punto \(P\)
    isolato ha accelerazione nulla in ogni istante.
\end{definition}



\end{document}

