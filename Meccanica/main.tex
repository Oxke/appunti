\documentclass{article}
\usepackage{layout}
\usepackage[a4paper, total={5in,9in}]{geometry}
\usepackage[T1]{fontenc}
\usepackage[italian]{babel}
\usepackage{mathtools}
\usepackage{amsthm}
\usepackage[framemethod=TikZ]{mdframed}
\usepackage{amsmath}
\usepackage{amssymb}
\usepackage{cancel}
\usepackage[dvipsnames]{xcolor}
\usepackage{tikz}
\usepackage{tikz-cd}
\usepackage{pgfplots}
\pgfplotsset{compat=1.18}
\usepackage[many]{tcolorbox}
\usepackage{import}
\usepackage{pdfpages}
\usepackage{transparent}
\usepackage{enumitem}
\usepackage[colorlinks]{hyperref}

\newcommand*{\sminus}{\raisebox{1.3pt}{$\smallsetminus$}}

\newcommand*{\transp}[2][-3mu]{\ensuremath{\mskip1mu\prescript{\smash{\mathrm t\mkern#1}}{}{\mathstrut#2}}}%

% newcommand for span with langle and rangle around
\newcommand{\Span}[1]{{\left\langle#1\right\rangle}}

\newcommand{\incfig}[2][1]{%
    \def\svgwidth{#1\columnwidth}
    \import{./figures/}{#2.pdf_tex}
}

\pdfsuppresswarningpagegroup=1

\newcounter{theo}[section]\setcounter{theo}{0}
\renewcommand{\thetheo}{\arabic{section}.\arabic{theo}}

\newcounter{excounter}[section]\setcounter{excounter}{0}
\renewcommand{\theexcounter}{\arabic{section}.\arabic{excounter}}

\numberwithin{equation}{section}

\newenvironment{theorem}[1][]{
    \refstepcounter{theo}
     \ifstrempty{#1}
    {\mdfsetup{
        frametitle={
            \tikz[baseline=(current bounding box.east),outer sep=0pt]
            \node[anchor=east,rectangle,fill=blue!20,rounded corners=5pt]
            {\strut Teorema~\thetheo};}
        }
    }{\mdfsetup{
        frametitle={
            \tikz[baseline=(current bounding box.east),outer sep=0pt]
            \node[anchor=east,rectangle,fill=blue!20,rounded corners=5pt]
            {\strut Teorema~\thetheo:~#1};}
        }
    }
    \mdfsetup{
        roundcorner=10pt,
        innertopmargin=10pt,linecolor=blue!20,
        linewidth=2pt,topline=true,
        frametitleaboveskip=\dimexpr-\ht\strutbox\relax,
        % nobreak=false
    }
\begin{mdframed}[]\relax}{
\end{mdframed}}

% \newenvironment{definition}[1][]{
%     \refstepcounter{theo}
%      \ifstrempty{#1}
%     {\mdfsetup{
%         frametitle={
%             \tikz[baseline=(current bounding box.east),outer sep=0pt]
%             \node[anchor=east,rectangle,fill=violet!20,rounded corners=5pt]
%             {\strut Definizione~\thetheo};}
%         }
%     }{\mdfsetup{
%         frametitle={
%             \tikz[baseline=(current bounding box.east),outer sep=0pt]
%             \node[anchor=east,rectangle,fill=violet!20,rounded corners=5pt]
%             {\strut Definizione~\thetheo:~#1};}
%         }
%     }
%     \mdfsetup{
%         roundcorner=10pt,
%         innertopmargin=10pt,linecolor=violet!20,
%         linewidth=2pt,topline=true,
%         frametitleaboveskip=\dimexpr-\ht\strutbox\relax,
%         nobreak=true
%     }
% \begin{mdframed}[]\relax}{
% \end{mdframed}}

\newtcolorbox[auto counter, number within=section]{definition}[2][]{
    colframe=violet!0,
    coltitle=violet, % Title text color
    fonttitle=\bfseries, % Title font
    title={Definizione~\thetcbcounter\ifstrempty{#2}{}{:~#2}}, % Title format
    sharp corners, % Less rounded corners
    boxrule=0pt, % Line width of the box frame
    toptitle=1mm, % Distance from top to title
    bottomtitle=1mm, % Distance from title to box content
    colbacktitle=violet!5, % Background color of the title bar
    left=0mm, right=0mm, top=1mm, bottom=1mm, % Padding around content
    enhanced, % Enable advanced options
    before skip=10pt, % Space before the box
    after skip=10pt, % Space after the box
    breakable, % Allow box to split across pages
    colback=violet!0,
    borderline west={2pt}{-5pt}{violet!40},
    #1
}

\newenvironment{lemmao}[1][]{
    \refstepcounter{theo}
     \ifstrempty{#1}
    {\mdfsetup{
        frametitle={
            \tikz[baseline=(current bounding box.east),outer sep=0pt]
            \node[anchor=east,rectangle,fill=green!20,rounded corners=5pt]
            {\strut Lemma~\thetheo};}
        }
    }{\mdfsetup{
        frametitle={
            \tikz[baseline=(current bounding box.east),outer sep=0pt]
            \node[anchor=east,rectangle,fill=green!20,rounded corners=5pt]
            {\strut Lemma~\thetheo:~#1};}
        }
    }
    \mdfsetup{
        roundcorner=10pt,
        innertopmargin=10pt,linecolor=green!20,
        linewidth=2pt,topline=true,
        frametitleaboveskip=\dimexpr-\ht\strutbox\relax,
        % nobreak=true
    }
\begin{mdframed}[]\relax}{
\end{mdframed}}

\theoremstyle{plain}
\newtheorem{lemma}[theo]{Lemma}
\newtheorem{corollary}{Corollario}[theo]
\newtheorem{proposition}[theo]{Proposizione}

\theoremstyle{definition}
\newtheorem{example}[excounter]{Esempio}

\theoremstyle{remark}
\newtheorem*{note}{Nota}
\newtheorem*{remark}{Osservazione}

\newtcolorbox{notebox}{
  colback=gray!10,
  colframe=black,
  arc=5pt,
  boxrule=1pt,
  left=15pt,
  right=15pt,
  top=15pt,
  bottom=15pt,
}

\DeclareRobustCommand{\rchi}{{\mathpalette\irchi\relax}} % beautiful chi
\newcommand{\irchi}[2]{\raisebox{\depth}{$#1\chi$}} % inner command, used by \rchi

\newtcolorbox[auto counter, number within=section]{eser}[1][]{
    colframe=black!0,
    coltitle=black!70, % Title text color
    fonttitle=\bfseries\sffamily, % Title font
    title={Esercizio~\thetcbcounter~#1}, % Title format
    sharp corners, % Less rounded corners
    boxrule=0mm, % Line width of the box frame
    toptitle=1mm, % Distance from top to title
    bottomtitle=1mm, % Distance from title to box content
    colbacktitle=black!5, % Background color of the title bar
    left=0mm, right=0mm, top=1mm, bottom=1mm, % Padding around content
    enhanced, % Enable advanced options
    before skip=10pt, % Space before the box
    after skip=10pt, % Space after the box
    breakable, % Allow box to split across pages
    colback=black!0,
    borderline west={1pt}{-5pt}{black!70}, 
    segmentation style={dashed, draw=black!40, line width=1pt} % Dashed dividing line
}
\newcommand{\seminorm}[1]{\left\lvert\hspace{-1 pt}\left\lvert\hspace{-1 pt}\left\lvert#1\right\lvert\hspace{-1 pt}\right\lvert\hspace{-1 pt}\right\lvert}


\title{Appunti di Meccanica Razionale}
\author{Osea}
\date{Secondo semestre, 2024 \-- 2025, prof. Ada Pulvirenti}

\begin{document}

\maketitle

Testo di riferimento: \emph{Meccanica Analitica} di Fasano Marmi, più lungo e
preciso. Mentre il testo \emph{Meccanica Classica} di Goldstein è il testo
classico dei fisici. 

\section{Spazio-Tempo-Moto}
\subsection{Moto}
Studiare il moto in meccanica significa studiare la funzione \(I = [0, T) \to
\mathcal{E}\), con \(t \mapsto P(t)\), dove \(\mathcal{E}\) e \(I\) sono
rispettivamente lo spazio della meccanica classica e un intervallo incluso
nell'asse dei tempi, e entrambi sono spazi affini euclidei.
\begin{notebox}
    Nella meccanica classica il \textbf{tempo} è \textbf{assoluto}, ovvero è
    indipendente dallo stato di moto dell'osservatore.
\end{notebox}
\begin{definition}{Spazio affine}
    Uno spazio affine reale di dimensione \(n\) è un insieme \(\mathbb{A} ^n\) i
    cui elementi sono detti punti, dotato delle seguenti strutture:
\begin{enumerate}[label = \arabic*.]
    \item Uno spazio vettoriale reale di dimensione \(n\), \(V\), detto spazio
        delle traslazioni (o dei vettori liberi)
\item Un'applicazione \(\varphi : \mathbb{A}^{n} \times  \mathbb{A}^{n} \to
        V\); \(P, Q \mapsto P - Q\)  con le seguenti proprietà
        \begin{enumerate}[label = \alph*.]
            \item \(\forall (P, v) \in  A^{n} \times  V\) esiste un unico punto
                \(Q\) tale che \(Q - P = \mathbf{v} \) 
            \item \((P - Q) + (Q - R) = P - R\) per ogni \(P, Q, R \in
                \mathbb{A}^{n}\) 
        \end{enumerate}
\end{enumerate}
\end{definition}

\begin{definition}{Retta}
    Una \textbf{retta} in \(\mathbb{A}^{n}\) passante per un dato punto \(P\) e
    con direzione \(\mathbf{v} \) è il sottospazio affine \(P + \Span{\mathbf{v} } \) ed è
    parametrizzata da \(t \mapsto P + t \mathbf{v} \) 
\end{definition}

\begin{definition}{Vettore applicato}
    Una coppia ordinata \({(P, \mathbf{u} )} \in \mathbb{A}^{n} \times V\) si dice vettore
    applicato a \(P\).
\end{definition}

\begin{definition}{Sistema di riferimento}
    Indicheremo con sistema di riferimento affine in \(\mathbb{A}^{n}\) un
    insieme 
    \[
      \Sigma = \{O \in \mathbb{A}^{n} ; \mathbf{v} _{1}, \mathbf{v} _{2}, \dots,
      \mathbf{v} _{n}\} \quad ;
      \quad {\{\mathbf{v} _{i}\}}_{i = 1, \dots, n} \text{ base di } V
    \]
\end{definition}
\begin{remark}
    Allora ogni punto \(P\) rispetto a \(\Sigma\) è individuato da \\\(P - O =
    x_{1}\mathbf{v} _{1} +~\dots + x_{n}\mathbf{v} _{n}\) 
\end{remark}

\begin{definition}{Spazio euclideo}
    Uno spazio affine reale di dimensione \(n\), \textbf{dotato di prodotto
    scalare} su \(V\) si dice spazio \textbf{affine euclideo}
\end{definition}

Dal prodotto scalare possiamo definire la distanza tra due punti di
\(\mathbb{A}^{n}\) come
\[
d(P, Q) = \|P - Q\| = \sqrt{\langle P - Q, P - Q \rangle} = \sqrt{{(P -
    Q)} \cdot {(P - Q)}}
\]
e la nozione di angolo tra due vettori \(\mathbf{u} , \mathbf{v}  \neq 0\) come il valore \(\alpha
\in [0, \pi]\) tale che
\[
  \cos \alpha = \frac{\mathbf{u}  \cdot \mathbf{v} }{\|\mathbf{u} \| \|\mathbf{v} \|}
\]
(notare che per Schwarz risulta che effettivamente tale valore esiste)

Comunemente ci ricondurremo a utilizzare un sistema di riferimento
\textbf{ortonormale}, ossia che ha come base una base ortonormale dello spazio.
In particolare \(\mathcal{E}\) è uno spazio affine reale euclideo di dimensione
3, e dunque useremo il sistema di riferimento
\[
    \Sigma = \{O, \mathbf{e} _{1}, \mathbf{e} _{2}, \mathbf{e} _{3}\} \quad ; \quad \mathbf{e} _{i} \cdot \mathbf{e} _{j} =
    \delta_{ij} \quad \forall i, j = 1, 2, 3
\]
Studiare il moto significherà studiare la relazione \(t \mapsto P(t)\) funzione
\([0, T) \to \mathcal{E}\). Introdotto \(\Sigma\), possiamo scrivere allora che
il problema è ricondotto a studiare la funzione \([0, T) \to \mathbb{R}^{3}\)
tale che \(t \mapsto (x_{1}, x_{2}, x_{3})\) e tale che \(P(t) = x_{1} \mathbf{e} _{1} +
x_{2} \mathbf{e} _{2} + x_{3} \mathbf{e} _{3}\).

\begin{definition}{Coordinate Curvilinee}
Sia \(Q \subseteq \mathbb{R}^{n} \). Consideriamo un'applicazione \(\mathbf{x} :
Q \to D \subseteq \mathbb{R}^{n} \) tale che 
\[
  \mathbf{q} = \begin{pmatrix}
      q_{1} \\
      \vdots  \\
      q_n
  \end{pmatrix} \mapsto \mathbf{x} {(\mathbf{q} )} = \begin{pmatrix}
      x_{1}{(q_{1}, \dots, q_{n})} \\
      \vdots \\
      x_{n}{(q_{1}, \dots, q_{n})}
  \end{pmatrix}
\]
che gode delle seguenti proprietà:
\begin{enumerate}[label = \arabic*.]
    \item \(\mathbf{x}  \in C^{1}{(Q)}\) 
    \item La matrice Jacobiana \(J\mathbf{x} \) abbia rango massimo
\end{enumerate}
Allora \(\mathbf{x} \) rappresenta un sistema di coordinate su \(D\) che sono
dette \emph{coordinate curvilinee}
\end{definition}
I vettori colonna della matrice \(J\mathbf{x} \), ossia
\[
  \mathbf{u}_i = \frac{\partial \mathbf{x} }{\partial q_{i}} \quad \forall i =
  1, \dots, n
\]
sono una base per \(V\) che viene chiamata \textbf{base locale}.

La base locale è ortogonale se \(\mathbf{u}_i \cdot \mathbf{u}_j = 0\) per ogni
\(i \neq j\). Vogliamo ora operare Gram-Schmidt per ottenere una base
ortonormale \(\{\tilde{\mathbf{u}}_i\} \) a partire da una base generica
\(\{\mathbf{u}_i\}\). 

\begin{remark}
    I vettori \(\mathbf{u}_i\) della base locale sono in ogni punto \(P\)
    tangenti alla rispettiva linea coordinata \(\mathbf{x} {(q_{i})} = \mathbf{x}(q_{1}, \dots, q_{i},
    \dots, q_{n})\), dove tutte le coordinate sono fissate tranne \(q_{i}\).
    Allora \(\mathbf{u}_i = \frac{\partial \mathbf{x} }{\partial q_{i}}\) è
    tangente alla linea coordinata \(\mathbf{x} {(q_{i})}\).
\end{remark}

\begin{example}[Coordinate polari nel piano]
    Prendiamo \(Q = {(0, +\infty)} \times  [0, 2\pi)\) e \(\mathbf{q} = {(q_{1},
    q_{2})} = {(r, \theta)}\). Allora 
    \[
      \mathbf{x} {(r, \theta)} = \begin{pmatrix}
          x_{1}{(r, \theta)}=r \cos \theta \\
          x_{1}{(r, \theta)}=r \sin \theta
      \end{pmatrix} \in C^{1}
    \]
    E la base locale è 
    \[
        \mathbf{u}_1 = \frac{\partial \mathbf{x} }{\partial r} = \begin{pmatrix}
            \cos \theta \\
            \sin \theta
        \end{pmatrix} \quad ; \quad \mathbf{u}_2 = \frac{\partial \mathbf{x}}{\partial \theta} = \begin{pmatrix}
            -r \sin \theta \\
            r \cos \theta
        \end{pmatrix}
    \]
    ed evidentemente \(\mathbf{u}_1 \cdot \mathbf{u}_2 = 0\) dunque la base
    locale è ortogonale. Per ottenere una base ortonormale dunque prendiamo
    \[
      \mathbf{e}_r = \frac{\mathbf{u}_1}{\|\mathbf{u}_1\|} = \mathbf{u}_1 \quad
      ; \quad \mathbf{e}_\theta = \frac{\mathbf{u}_2}{\|\mathbf{u}_2\|} =
      \frac{1}{r} \mathbf{u}_2 
    \]
    Osserviamo che effettivamente \(\mathbf{e}_r\) è tangente alla linea
    coordinata corrispondente (fisso \(\theta\)) e \(\mathbf{e}_\theta\) è
    tangente alla linea coordinata corrispondente (fisso \(r\)).
\end{example}
\begin{example}[Coordinate sferiche]
    \[
        Q = {(0, +\infty)} \times [0, 2\pi) \times (0, \pi) \subseteq
        \mathbb{R}^3 \quad ; \quad \mathbf{q} = {(q_1, q_2, q_3)} = {(r, \theta,
        \varphi)}
    \]
    \[
      \mathbf{x} {(\mathbf{q} )} = \begin{pmatrix}
          x_{1}{(r, \theta, \varphi)} = r \sin \theta \cos \varphi  \\
          x_{2}{(r, \theta, \varphi)} = r \sin \theta \sin \varphi  \\
          x_{3}{(r, \theta, \varphi)} = r \cos \theta
      \end{pmatrix} : Q \to \mathcal{E} \sminus \{\text{ asse \(z\)}\} 
    \]
    che è chiaramente di classe \(C^{1}\) e ha Jacobiano di rango massimo. La
    base locale è
    \begin{align*}
      \mathbf{u}_1 = \frac{\partial \mathbf{x} }{\partial r} = \begin{pmatrix}
          \sin \theta \cos \varphi \\
          \sin \theta \sin \varphi \\
          \cos \theta
          \end{pmatrix} \quad &; \quad \mathbf{u}_2 = \frac{\partial \mathbf{x}
          }{\partial \varphi } = \begin{pmatrix}
          - r \sin \theta \sin \varphi \\
          r \sin \theta \cos \varphi \\
          0
      \end{pmatrix} \quad ; \\ \mathbf{u}_3 = \frac{\partial \mathbf{x}
      }{\partial \theta} &= \begin{pmatrix}
          r \cos \theta \cos \varphi \\
          r \cos \theta \sin \varphi \\
          -r\sin \theta
      \end{pmatrix}
    \end{align*}

    e dunque la base locale è ortogonale. Per ottenere una base ortonormale
    prendiamo, per ogni \(\mathbf{u} _{i}\), il vettore \(\mathbf{u}_i /
    \|\mathbf{u} _i\|\), ossia
    \[
      \mathbf{e}_r = \mathbf{u}_1 \quad ; \quad \mathbf{e}_\varphi =
      \frac{\mathbf{u}_2}{r \sin\theta} \quad ;  \quad \mathbf{e}_\theta =
      \frac{\mathbf{u}_3}{r}
    \]
\end{example}
\end{document}

